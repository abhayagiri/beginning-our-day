\mychaptertoc{Being Comfortable Is Not the End of Suffering}
{Being Comfortable Is Not\\the End of Suffering}
{Luang Por Pasanno}{August 2013}

There's an element of the human psyche that is constantly looking for 
comfort, security, and ease. And we can sometimes believe that the end 
of suffering is when we bring about the circumstances in which we don't 
have to extend ourselves or put forth too much effort. But even when 
we've managed to manipulate conditions in a way that allow us to feel 
relatively comfortable and laid back, we inevitably realize that we are 
still suffering.

That's why, as practitioners, we need to be willing to stretch our 
capabilities and constantly look for ways that help us to do that. 
Whether we're on retreat or engaged with each other in the monastery's 
communal routine, we need to keep experimenting and working with 
different practices to stretch ourselves further.

Of course, there should be a balance in our efforts; constantly pushing 
and striving is itself a form of suffering. This is why the Buddha 
pointed to the Middle Way. So, rather than investigating what it is 
that makes us feel relatively comfortable and secure, we can 
investigate what it is that undermines the tendencies of greed, hatred 
and delusion. We need to look very carefully, asking ourselves, 
\emph{What are those underlying roots of delusion? What are those 
habits of selfishness?} By investigating like this, we learn that we 
don't need to be tripped up by our habits or defilements; we learn to 
be okay with whatever conditions arise. And when that happens, we find 
\emph{true} ease and comfort.

So it's important to investigate in these ways and to stretch our 
capabilities. Whether we are developing our formal, solitary practice 
or engaged in our communal duties, we try to let go of that sense of 
comfort seeking and undermine the underlying tendencies that cause us 
complication and difficulty.

