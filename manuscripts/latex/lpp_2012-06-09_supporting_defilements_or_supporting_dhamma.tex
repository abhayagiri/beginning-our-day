\mychapter{Supporting Defilements or Supporting Dhamma}{Luang Por 
Pasanno}{June 2012}

In yesterday's reading, Ajahn Baen emphasized a question that is quite 
commonly asked in the Forest Tradition. It's a very simple question 
that we should consider and contemplate in our practice: \emph{Am I 
supporting the defilements, or am I supporting the development of 
Dhamma?} That very simple contemplation is critical, because our 
preferences and biases don't tend to lead us to question in that way. 
We tend to have thoughts like, \emph{What do I prefer? What do I want? 
What view do I think is right?} Does that support our defilements of 
greed, hatred and delusion, or does it support Dhamma? That's a 
question we don't often ask ourselves. But we need to bring it up, not 
only when we're watching and investigating the mind during formal 
meditation, but also during our day-to-day activities.

And what is Dhamma? It has many facets, but importantly, Dhamma is that 
which is aligned with the path leading to the cessation of suffering. 
That's the whole point of the Buddha's teachings---to give us the tools 
to free ourselves from discontent, dissatisfaction and dissonance. When 
the mind or heart is dissonant, we can see that clearly as we become 
aligned with Dhamma. And when the mind and heart are each in tune, we 
can feel that right away, reflecting to ourselves, \emph{This feels 
peaceful. This feels clear. This brings happiness and well-being. This 
is the opposite of following my attachments and desires, which always 
leads to dissatisfaction.}

Again, it's important to frequently ask ourselves, \emph{Is this 
supporting the defilements or is this supporting Dhamma?} We ask this 
about our actions and our speech. And we ask, \emph{How am I 
establishing mindfulness and inner cultivation?} This is how we develop 
the Dhamma in our hearts and relinquish the defilements of the mind.

