\mychapter{Having Faith in the Training}{Ajahn Yatiko}{September 2012}

As monastics, it's worth keeping in mind where our focus is. It is not 
on worldly skills such as well-honed public speaking. The Buddha said, 
in former times, the monks who were respected and praised were those 
who lived and trained in the forest and put effort into practice; but 
later, respect and praise went to the monks who had good speaking 
skills. Skills, talents, and even an ability to give good Dhamma talks 
are praiseworthy, but what we are here for is the training. We are here 
to train the heart, and training the heart doesn't necessarily result 
in anything remarkable in terms of the external world. Training the 
heart results in sincerity, openness, peace, and mindfulness. We should 
keep in mind that this is why we're here. There's no other reason.

We need to have faith in the training---faith that the training works, 
faith that it will bear fruit. We might come across obstacles which 
could be with us for a long time, maybe for most of our lives. But the 
training will bear fruit---it doesn't make sense that it would not. 
We're putting sincere effort into cultivating mindfulness, restraint, 
and understanding, and we're dedicating ourselves to the principles of 
the holy life. The training has to bear fruit sooner or later, in this 
life or future lives. If those principles don't hold true, then as far 
as I'm concerned, life has no meaning. But because life does have 
meaning---because those principles do hold true---the best thing for us 
to do is give ourselves to the training, to the best of our capacities, 
and bear with the difficulties that arise.

When an obstacle to our practice arises, we need to remember not to 
throw the baby out with the bathwater. We do not have to disrobe or 
leave Buddhist practice forever. Something may have to change, and that 
might be a radical change in our views or in the way we approach 
practice. To deal with obstacles, we need to think creatively, outside 
the box, especially when we come up against a very persistent problem. 
Most importantly, the way through an obstacle is not to give up the 
struggle. Instead, we try to see it from a radically different 
perspective and carry on with the training. It's the training that 
gives meaning to life.

