\mychapter{Don't Fill Up the Void}{Ajahn Jotipālo}{July 2012}

Why did you come to Abhayagiri? What brought you here? Simply bring 
that inquiry into your mind.

After traveling for the last month and being fairly busy, I returned to 
Abhayagiri a few days ago. I've tended to go back to my \emph{kuṭi} 
in the afternoons and it's been really quiet. There have been no 
expectations or demands on me. I've noticed in the past, after a period 
of busyness, when I go back to my \emph{kuṭi} for some quiet time, 
there is a feeling of melancholy that comes up and a feeling that I 
should be doing something. I've been looking at that for the last two 
days, going back to my \emph{kuṭi} and sitting there with that 
feeling. I'm not trying to label it---anxiety, depression, or whatever 
else the mind wants to call it. I'm just looking at it as a physical 
sensation and seeing how the mind is reacting to it, how the mind is 
trying to figure out what it is and why it's there. When I feel this 
way my tendency and desire is to say to myself, \emph{I should go do 
something to help the monastery or read a book or do walking 
meditation}… There is this low-grade desire in me to avoid feeling 
this melancholy. I don't \emph{want} to feel it. Even something as 
wholesome as reading or studying, if the intention behind that is to 
avoid feeling something, then it's good to investigate that.

For at least a period of our afternoon, when we have solitude, I think 
it is good to set aside some time to do nothing in particular. Don't 
think of it as trying to meditate or trying to focus on the mind, but 
just sit there and be willing to feel what you're feeling. See what 
happens. See if it's aversion or pleasure. To me, this really gets back 
to that first question, ``What brought me here?'' We are often trying 
to fill up the void we feel with activities, planning, or whatever, 
merely because we don't want to feel the discomfort of that void, or of 
any other unwanted emotion. When we do this, we are not honoring the 
wholesome intentions that brought us here. So instead, when we find 
ourselves avoiding our experience, we can take the time to be quiet and 
be willing to feel what we are feeling.

