\mychapter{Blown Into Cosmic Dust}{Luang Por Pasanno}{April 2013}

The Buddha encouraged us to contemplate aging, sickness, and death 
every single day. It's essential that we make an effort do that, 
because the mind's nature is to forget about these contemplations. It 
inclines away from them and, instead, inclines toward thoughts of 
eternal youth and health. For the most part, death doesn't exist for 
us. So we have to make it conscious in our practice by realizing the 
limitations of the human condition on the physical level. We're bound 
to this body that's constantly aging. Even when we're young, it's still 
aging. The reason a baby gets born in the world is because it's too old 
to stay in the womb. And that aging process continues on after birth.

This is a supportive reflection for us---it's not intended to make us 
miserable or depressed. It's intended to bring up wholesome mental 
states to counteract the mind's tendency towards \emph{pamāda}---being 
careless and not very circumspect regarding the truths of our 
existence. These three recollections help establish a sense of 
heedfulness and spiritual urgency. We have this opportunity to hear the 
Dhamma, to study the Dhamma, to practice the Dhamma, to live a life in 
accord with the Dhamma. Reflecting on that makes us grateful for this 
opportunity to practice, and that's vital, because it's easy to let the 
opportunity slip by.

So we recollect the nature of the human body---that our lives are 
subject to aging, sickness, and death. For all of our seeming solidity, 
we're left to nature when we die; our bodies dissipate and disappear. 
The Buddha taught the charnel ground contemplations, which encourage us 
to examine the body's process of dissolution after death. It changes 
shape, bloats, disintegrates and dries out until just the sinews and 
bones are left. The bones are then scattered until there's little 
remaining but a small pile of gray dust. Then a wind comes along and 
blows it here and there until there is nothing left. For all of our 
obsessions about ourselves---our worries, fears and anxieties around 
health and physical well-being, and all the uncertainties in our 
lives---it is absolutely certain that, on a physical level, we're all 
going to be blown into cosmic dust.

Now when we contemplate that fact, it's not to be picked up in a 
nihilistic way, but rather with the sense of urgency that's needed to 
establish the heart within skillful spiritual qualities---the sense of 
urgency we need to keep attending to our actions of body, speech and 
mind, so that they're conducive to clarity, wisdom, and wholesome 
states of mind.

Everyday we recite the Five Subjects for Frequent Recollection. When 
reciting the fifth, we recollect: ``I'm the owner of my kamma, heir to 
my kamma, born of my kamma, related to my kamma, abide supported by my 
kamma. Whatever kamma I shall do, for good or for ill, of that I will 
be the heir.'' These recollections help us establish our kamma---our 
actions of body, speech and mind---in what is wholesome, skillful and 
beneficial, in what bears fruit in terms of peace and clarity, 
happiness and well-being. So it behooves us to take responsibility for 
these contemplations, to reflect upon them and develop them in our 
practice.

