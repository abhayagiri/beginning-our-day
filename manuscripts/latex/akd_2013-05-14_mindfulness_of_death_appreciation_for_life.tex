\mychaptertoc{Mindfulness of Death, Appreciation for Life}{Mindfulness of Death,\\Appreciation for Life}{Ajahn Karuṇadhammo}{May 2013}

Luang Por Pasanno mentioned that Debbie's mother is going to be signing 
into the palliative and hospice care programs because her condition is 
deteriorating. I would think Debbie has been practicing with death 
contemplation as this is happening, particularly after she lost her 
sister-in-law a few months ago. The Buddha encouraged us to reflect 
daily and remind ourselves that death can come at any time. It's easy 
to externalize, \emph{It's happening to somebody else, not me.} But at 
some point, it won't be somebody else, it will be us and it would be 
nice to think that we are ready for it when it comes. This isn't meant 
to be a morbid reflection, but more an encouragement to contemplate 
death, bringing us closer to the reality of it and encouraging a sense 
of heedfulness and urgency in the practice.

It's so easy to get lost in the tasks of the day, particularly as we 
are about to launch into a work period. I would imagine that a 
significant number people, right now, are thinking about the tasks they 
need to do. I find myself doing that sometimes as well. But we can pull 
back a bit and remind ourselves, \emph{Hold on a second, life is 
precious. I don't know when or how I'm going to die. If death occurred 
for me right now, would I be ready for it? Would there be remorse? Are 
there things that I have done or left undone that I would regret if I 
died today?} We can take a few moments to contemplate the potential 
immediacy of death and see what this might bring up for us. 
Contemplating death in this way allows us to clarify what is precious 
in our lives which frees us from the tendency to get lost in the 
details. And this in turn helps us focus on what our priorities are. So 
instead of death contemplation producing a negative feeling such as 
fear or bewilderment, we can be moved to a sense of lightness and 
release as we focus on what is most important to us. And a life well 
lived, focused on what is most meaningful for us, has the greater 
potential to be a life free of regret and remorse.

Death is a present-moment experience; it's not in the future. When the 
moment of death arrives, it will be just that moment---everything 
before that moment of death is still life, with all of its projections, 
worries and fears, including the fear of that approaching death. But 
when death actually occurs, it is just one brief moment. So, at the 
moment of death, death is now. Before that moment, it is just a 
projection. With contemplation of death we become more familiar with 
this inevitable ending so that when it finally comes, we are prepared 
for it, neither afraid of or confused with this very ordinary 
present-moment experience.

If we keep that in mind then we do not really have any other option 
than to contemplate what's going on for us right here and right now. 
It's the only place we are going to be ready when it's time for the 
body to move on, for the elements to dissolve. For most of us, one of 
the best ways to do this is by using mindfulness of the body. We can 
notice the position of the body, the posture of the body---standing, 
walking, sitting, and lying down. This is the most basic contemplation 
of the body and we can maintain it when we are doing just about 
anything---constantly coming back and asking ourselves, \emph{What is 
the posture of the body? What is the disposition of the body? If it's 
moving, how is it moving through space?} We can know what's happening 
with the arms, the legs, the head, the torso, and be present with the 
body. We can also incorporate the mood of the mind---\emph{What's the 
mood like right here and right now?} If we keep on attending to right 
here and right now as we go through our daily activities, then when it 
comes time for death to greet us, we will be ready right there and 
right then to be aware of the event as it happens. Bringing mindfulness 
right here and right now and reminding ourselves of the preciousness of 
this human life is a great way of reducing fear and anxiety and 
establishing a sense of purpose along the path.

