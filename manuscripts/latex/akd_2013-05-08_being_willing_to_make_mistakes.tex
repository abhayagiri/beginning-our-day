\mychapter{Being Willing to Make Mistakes}{Ajahn Karuṇadhammo}{May 
2013}

How much are we willing to learn from our mistakes? This is a crucial 
aspect of the training---the willingness to recognize when we've missed 
the mark, being open to making mistakes. It's not always easy to 
practice in this way. I think many of us here come with conditioning 
around how important it is to be right all the time. We can grow up 
with a sense of shame---\emph{Unless I'm doing everything perfectly all 
the time, then something is wrong with me.} As far as I can tell, there 
are only a few lucky people who learned while growing up that it's just 
fine to make mistakes. In the community here, there are many of us who 
are strong-willed in certain ways. We have plenty of leaders here and 
sometimes it can be difficult for us to break that classic paradigm, 
\emph{The way I think we should do it is the right way.}

How we begin to unravel this paradigm is by learning it's okay not to 
be right all the time, and to use honest self-appraisal to look at 
ourselves. This allows us to say \emph{Okay, perhaps since everybody 
else is doing it a different way, I need to consider that or A number 
of people are indicating to me that I may have missed the mark---maybe 
I should think carefully about what happened.} This is a sign of 
internal strength. I also believe highly realized people tend to take 
this approach as well. Those who have penetrated the perception of 
not-self can see there is no ``me'' or ``myself'' that needs defending. 
They know it's not a problem if they're not always right.

This comes down to a matter of skill. Either it's a skill we've 
learned, or one we haven't, or perhaps we've partially taken it 
on---but it's nothing personal. As Luang Por Pasanno was saying, even 
the Buddha after his enlightenment was constantly readjusting. 
Sometimes he'd set down rules, only to realize later they needed to be 
changed. In those cases he would call the monks together, explain the 
need to alter a rule, and adjust it accordingly. Right after his 
enlightenment, the Buddha was inclined not to teach. As the story goes, 
the Brahmā God Sahampati realized this was the Buddha's inclination, 
came down from the brahmā realm, appeared before the Buddha and said, 
``Please reconsider this: There are those who can learn!'' The Buddha 
thought, \emph{Maybe there is a possibility of teaching others how to 
realize what I have realized. So even at that point he readjusted, and 
he certainly didn't take it personally.}

This is a good reflection to bring to mind. Essentially, it's a 
reminder to be honest with ourselves---whether it's relating to 
community life, the monastic discipline, the training, the views and 
opinions about how things should be done, or one's personal meditation 
practice---we can strive for a more open and balanced point of view. We 
do this by honestly looking at our internal experiences, allowing 
ourselves to be present with our fears of being wrong, and saying to 
ourselves, \emph{Okay, I need to make a change; I need to adjust.} And 
then simply let it go and make the adjustment. Once that's done and the 
change has been made, we can move on and not worry about it so much.

