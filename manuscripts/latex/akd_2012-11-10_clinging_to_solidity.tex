\mychapter{Clinging to Solidity}{Ajahn Karuṇadhammo}{November 2012}

Last night Ajahn Amaro shared some reflections around the theme of 
transience and its relation to impermanence. After his talk I was 
contemplating this theme, especially around the sense of permanence and 
how we relate permanence to solidity. It is one of the fundamental 
delusions we carry around with us much of the time and it forms the 
basis for a false sense of self, a solid ``me'' in this body. The 
quality of holding onto the perception of permanence is what leads to 
so much distress and suffering. It's good to get an understanding of 
how to bring that into conscious awareness as much as possible, not 
only in meditation, but when moving through the day.

When we maintain the sense of apparent permanence or solidity, it 
appears to be bound up with a fundamental motion of mind, the quality 
of clinging. In the suttas, the classic approach to having insight into 
this quality of clinging is by contemplating impermanence. This 
contemplation can give rise to a sense of dispassion, disenchantment, 
and can result in the ability to let go and release the clinging.

Alternatively, we can notice the sense of holding on to anything and 
use that as a basis for letting go of the perception of permanence and 
solidity. We then see how much difficulty it causes when we're involved 
with work or interaction with people and there's this sense of holding 
on. I find it useful to go directly to the mind-motion of holding onto 
something and watch how it reinforces the sense of solidity, 
permanence, and a false reality. When we cling or hold tightly to some 
notion or opinion, as a result everything becomes more concrete, more 
focused on the subject and object, more about ``me''---more entangled 
and involved.

The views we cling to all tend to play with, feed on, and support each 
other. We can attack them from different angles, whether it's directly 
contemplating the transience of a particular situation or whether we 
can see the clinging or holding onto something. In that process of 
recognition, it's possible to realize, \emph{I can drop this. I don't 
need to hold onto this. I can let it go.} We might then notice the 
sense of ease in the mind that comes when we are able to do that. We 
may also gain insight into impermanence that appears right there, 
because if we don't hold onto something then it just moves on by 
itself. As quickly as it appears, it passes away, particularly in the 
realm of feelings, perceptions, and moods. When we don't hold onto 
those experiences, we see how quickly they change, morph, disappear, 
reappear, and disappear again.

The perception of transience makes the process of going through life 
much lighter; the burden doesn't have to be so heavy from holding on. 
So we can practice that throughout the day, taking moments to stop, 
seeing where there's holding and clinging, and then relaxing and 
letting go, allowing things to move on in their own natural way. 
Through this process we gain insight into the true nature of 
impermanence.

