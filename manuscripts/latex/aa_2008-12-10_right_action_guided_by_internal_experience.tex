\mychaptertoc{Right Action Guided by Internal Experience}
{Right Action Guided by\\Internal Experience}
{Ajahn Amaro}{December 2008}

\emph{Sammā Kamanto}---right action---is an essential factor of the 
Noble Eight-Fold Path. Often the Eight-Fold Path is summarized as 
\emph{sīla}, \emph{samādhi}, and \emph{paññā}, virtue, 
concentration, and wisdom. Right action is then woven into the section 
on sīla, along with right speech and right livelihood. As a result, 
when we are considering meditation, we may focus on the samādhi 
section of the Path and forget the influence of right action on our 
minds and states of consciousness throughout the day, including our 
time meditating.

Often during the course of our lives in the monastery, because we have 
a prescribed set of rules for conduct and a prescribed routine, we can 
miss the effects that actions have on our minds, our states of 
consciousness. It's something to be alert to. What we do and say during 
the day have an effect, and that effect leads to either our spiritual 
growth or our degeneration. Because action has such an effect, the 
monastery environment is designed to maximize the wholesome qualities 
and to restrain the unwholesome qualities.

But even within the framework of the precepts we live by and the 
routine we have, it's good to be alert, recognizing the effect on the 
mind of the actions we perform and the things we say during the course 
of the day. It's good to consciously notice---when we've acted or 
spoken in an unskillful way---how much that sticks in the mind. When 
we're sitting down to meditate, we can see the difference. At the end 
of the day, when there has been a very good standard of restraint and 
clarity, then during the evening sitting we may notice, \emph{Oh, look, 
there's nothing on my mind. I'm not remembering anything cruel or 
selfish or coarse that I said today. I'm not feeling the effects of 
having been agitated or uncontrolled in my actions. Look at that, 
there's an absence of unpleasant results}. We don't always notice that, 
but it is quite helpful to our practice if we do.

It's easier to notice, at the end of the day, those times when we've 
had some kind of contentious exchange or have used our speech in an 
uncontrolled or obstructive way. On days like those, that's often 
what's on the mind during the evening sitting. We sit down and 
suddenly---\emph{boom}---there it is, that regrettable conversation is 
replaying itself. These unfortunate episodes arise immediately in the 
mind and then, quite often, they can occupy the entire sitting. They 
can keep popping up over and over again.

It's not necessary to create a sense of guilt, self-hatred or 
self-criticism. But it is helpful when we think, \emph{Oh, look---this 
was the action, that is the result. Because this was done, there is 
that effect. This is the cause, and that is the result. }That clear 
observation of the causality of our experience is the doorway to 
wisdom, concentration, and clarity. The more clearly we see the effects 
in meditation of unskillful actions, the more clearly we realize we are 
simply putting obstacles in our own way and sabotaging our own efforts. 
Then we ask ourselves, \emph{Why would I want to do this? Why on earth 
do I want to mess up my own living space like this?} It's as if we're 
grabbing handfuls of rocks and earth and fistfuls of poison oak and 
sprinkling them all around the inside of our kuṭis. \emph{Why would I 
want to do that? What a strange, stupid thing to do}. This becomes 
clear to us.

When we reflect on wholesome and unwholesome mind states and recognize 
their effects on how we act and speak, then that very recognition of 
cause-and-effect guides us. It helps support our efforts to be more 
restrained in the future. And it's not because there's a set of 
external rules telling us that we should be like this, or because we 
fear an authority figure who's going to scold us if we do something 
wrong. We simply see for ourselves that right action supports the 
cultivation of a pleasant, wholesome abiding, and it allows us to 
fulfill the purpose for which we came to live at the monastery. Through 
right action we are setting the conditions in favor of our purpose, 
steering our life in a way that supports it. We're simply doing 
ourselves a favor, which brings blessings into our lives.

It's not that we're never going to make mistakes or lose our way. But 
we can cultivate that sense of attention and, by seeing how action and 
speech affects the mind, we can let that recognition inform the way we 
operate and speak. Then later, when we find ourselves being drawn into 
an unskillful conversation or getting self-centered, aggressive, or 
lazy, we can let that experience of recognizing cause-and-effect guide 
us. \emph{The last time I did that, it took me three days to get over 
it. It seemed like a good idea at the time, but there was a lot of 
wreckage left behind after acting in that way}. Because of having seen 
for ourselves the painful effects of unskillful actions, we find 
ourselves respecting and moving toward restraint, modesty, simplicity, 
and inner quietude. This is really the best kind of training. Training 
in terms of being obedient to an external force has its place. But 
being obedient to and guided by our own internal experience---that's 
the kind of training that brings well-grounded, long-lasting results.

