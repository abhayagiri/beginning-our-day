\mychapter{Making Uncertainty Clear}{Luang Por Pasanno}{June 2013}

I received an e-mail the other day from Ajahn Ñaniko and Tan 
Ṭhitābho, who recently left on their \emph{tudong} from here to 
Oregon. They had arrived safely at Steve's place in Willits and were 
camped out on the floor of his shrine room, sore and aching because it 
was their first day of walking. They weren't used to carrying all their 
gear for such a long distance, but they'll get used to it. They were 
planning to leave Willits yesterday and head up north. Who knows how 
far they might get today, and where they'll be going on alms-round. 
That is part of the adventure and part of the practice of taking on 
uncertainty, \emph{aniccā}.

When we are out on the road walking like that, uncertainty is a 
constant presence. In fact, wherever we are and whatever we are 
doing---whether on tudong, or walking back to our hut, or working on a 
project---uncertainty is a constant companion. But often we don't 
acknowledge its presence. We tend to fall into assumptions of 
certainty, assumptions that this is a sure thing or this is going to be 
a certain way. We assume that everything is laid out clearly, such as 
how our day is going to go, who will be here and what can happen. There 
is, of course, a certain fallacy to those assumptions. It is important 
to remind ourselves of the uncertainty of our existence, the 
uncertainty of what is happening around us and the uncertainty of the 
things we rely on for comfort, security and well-being. As a part of 
practice and training, it is essential to keep bringing up the 
reflection of impermanence.

When the Buddha speaks about the universal characteristics of all 
phenomena, aniccā is the very first quality he points to. In many 
suttas, the Buddha expresses that it is essential to bring awareness to 
the truth of impermanence. When we bring awareness to the truth that 
everything is changing, it doesn't make us more anxious or fearful. 
Rather, it brings a sense of clarity and immediacy. But uncertainty and 
change are often not engaged with, and so they slip into the 
background. When something is in the background, it tends to be fuzzy 
and imperceptible, and we can have all sorts of incorrect assumptions 
about it.

``\emph{Aniccā vatta saṇkhāra}''---all conditioned things, all 
phenomena, are impermanent---that phrase begins one of the suttas in 
the Dhammapada. By making an effort to bring to mind this reflection on 
uncertainty and change, we have an opportunity to brighten and clarify 
the mind with the truth of impermanence, the truth of our conditioned 
experience.

