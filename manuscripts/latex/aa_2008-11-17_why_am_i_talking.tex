\mychapter{Why Am I Talking?}{Ajahn Amaro}{November 2008}

I've been reflecting on the realm of right speech, an area in our lives 
that very swiftly gets carried away on the wind. Just as autumn leaves 
off the oak tree end up all over the landscape, so too, our resolution 
to be more attentive and more mindful of speech gets carried away on 
the winds of circumstance. We might listen to a Dhamma talk on right 
speech and take in the various principles expressed by the teacher, but 
when we encounter each other we find ourselves wanting to comment on 
some event in the day that we've seen, heard, or read about. We 
overhear a conversation between others and find ourselves hanging 
around the edges, eager to chime in and say, ``Oh yeah, I heard about 
that … yada, yada, yada.'' And then off it goes. This is one of the 
perennial issues of community life.

The other day someone was quoting the little signs they have up at Wat 
Mettā. One of them is the acronym WAIT: Why Am I Talking? We can apply 
that sort of inquiry even before we start talking. \emph{Why do I need 
to talk? Is the world going to be a better place if I chime in at this 
point? Why do I want to get involved? Is it merely to engage for the 
sake of engaging? Is it an urge to burn some energy and connect? Is it 
just to fill up space? Is this actually going to be a benefit? Is this 
going to unite or divide? Is this going to bring clarity to others? Can 
I restrain the urge to comment, to speak, to put forth some opinion or 
some perspective?}

We're a very large community these days. The monks' room has many 
people passing through at different times---just before morning 
\emph{pūjā}, at morning tea time, preparing for a morning meeting, 
during bowl set up, preparing for the work period, getting changed 
after the work period, cleaning up after the meal, enjoying evening tea 
time, before and after evening pūjā. It's good to be aware that the 
conversation we think we're having with just one person in the monks' 
room is actually involving other, unseen people, since the Dhamma Hall 
on the other side of the wall is like an echo chamber for the monks' 
room. This is especially apparent if things are quiet in the Hall and 
there are loud conversations going on in the monks' room. Of course, 
sometimes there are things really worth talking about. But often, it's 
simply random chatter and only for the sake of engaging. So I encourage 
us to apply mindfulness and consideration, asking ourselves, \emph{Is 
this really worthwhile? Am I considering that there are probably 
several people in the Dhamma Hall who will have to listen to all this? 
Is this something that's really worth sharing with so many people? Do I 
need to engage in this, or can I put it aside?}

In this way we can notice our accustomed patterns and see what 
situations draw us into pointless engagement and continuous verbal 
proliferation. When we learn to recognize those situations, we can take 
action to put ourselves somewhere else. Go sit on the porch, sit on the 
bench outside, or come into the Dhamma Hall. We can choose solitude 
over putting ourselves in close quarters where talking and pointless 
chitchat tend to launch themselves. This is one of the simplest and 
most direct ways the Buddha encouraged the development of mindfulness 
and the restraint of the \emph{āsavas,} the outflows. We don't put 
ourselves in situations where that out-flowing is going to be 
encouraged. We put ourselves somewhere else. It's very simple. We're 
not creating the conditions whereby a lack of restraint is being 
encouraged. Rather, we're inclining towards containment.

Every so often we need these kinds of encouragements to look at our own 
habits and involvements. When we make the effort to restrain and take 
the opportunity to disengage, then we can see for ourselves the 
results: how much quieter and settled the mind becomes. There's no need 
to remember the pointless chitchat or inquire about information that's 
not really benefiting our lives. We can recognize for ourselves, 
\emph{Look how much more peaceful my mind is while I'm sitting, how 
much more easeful it is on the worksite when my mind isn't muddling 
around with all of this verbal engagement.} To start with, we can learn 
to WAIT.

