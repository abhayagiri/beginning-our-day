\mychapter{Is Rock Climbing Like Meditation?}{Luang Por Pasanno}{June 
2013}

As Dhamma practitioners, we need to continually turn our attention to 
contemplation, reflection, investigation---to consider things 
carefully---and not just in formal meditation, but also as we go about 
our more mundane activities. These days, even in the mainstream media, 
it's popular to talk about ``being in the present moment, being in the 
here and now.'' It sounds very good, but if that's all we practice, we 
can be left without having reflected or investigated to any great 
degree.

We need to apply some discernment. When we experience difficulty, 
conflict or dis-ease, we direct our attention toward that, investigate 
its causes, and examine the process by which it is unfolding. This is 
not to say we should be reaching out intellectually and coming up with 
rational explanations; rather, it is being willing to investigate and 
bring one's attention to the matter at hand.

Once when Ajahn Chah was visiting the U.S., someone asked him a 
question about the need for sitting meditation: ``I have a friend whose 
meditation is rock climbing. He doesn't have to sit in meditation to 
concentrate his mind. Why do we have to sit in meditation? Couldn't we 
do something like rock climbing---anything that puts us in the present 
moment?'' Then Ajahn Chah asked him, ``When your friend is rock 
climbing, does he contemplate the Four Noble Truths?''

We can be in the present moment, we can be clear, but are we developing 
discernment and learning to understand the nature of the mind, the 
nature of conditions? We mustn't be satisfied with merely cultivating 
calm and clarity; rather, that calm and clarity needs to be put to 
work. And its work is developing discernment and understanding. That's 
the crux of our practice. Take the illuminating idiom, 
``truth-discerning awareness.'' It's not just about awareness---it's 
awareness with discernment.

To develop this discernment we can begin by asking ourselves, 
\emph{What is the nature of things?---the nature of conditions?---the 
nature of my own mind?} Then we bring the attention inwards and focus 
our awareness on the various feelings that are present. In particular, 
we attend to the feelings of dis-ease, dissatisfaction, or suffering 
and come to understand that those feelings are merely feelings. And 
with any particular feeling we have we ask ourselves, \emph{What are 
the causal conditions for that feeling? Where is its resolution? How 
can I help bring about that resolution?} In this way, we are 
contemplating the Four Noble Truths exactly as the Buddha intended.

