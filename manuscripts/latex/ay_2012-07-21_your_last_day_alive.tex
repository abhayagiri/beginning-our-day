\mychapter{Your Last Day Alive}{Ajahn Yatiko}{July 2012}

During this morning's meditation, I was reflecting on the sutta, 
Mindfulness of Death (AN 6:19) where one monk tells the Buddha he keeps 
death in mind once a day thinking that he might live that much longer 
to contemplate the Buddha's teachings. Another monk says he keeps death 
in mind several times a day, and yet another monk brings death to mind 
every minute or two. The Buddha says each one of these monks is 
dwelling heedlessly. One of the monks present says he contemplates that 
he might die in the time it takes for an in-breath or an out-breath. 
The Buddha commends him for that. After all, we could die on an 
in-breath, before breathing out. And with every out-breath, we could 
die before breathing in again. This is a powerful reflection. It brings 
the mind into the present moment in a striking way. Taken to a deep 
level, this reflection can cause the hindrances to be in abeyance, 
because the hindrances almost always involve the passage of time. If 
the mind is in the present moment, and if we recognize that death could 
come this very instant---with the snap of a finger---then there's no 
time for the hindrances to arise.

For many of us, this is an old reflection. One way you can give it new 
life is to assume that this is your last day alive and that your moment 
of death will be tonight at the stroke of ten. How will you spend your 
day if this is your last day alive? You might say to yourself, \emph{If 
this is my last day alive, I don't want to spend it on self-centered 
habits. I don't want this last day of my life to be marred by being 
heedless or by taking for granted this human form, this opportunity, 
this community, or by concerns about my personal health.} Behaving in 
those ways on the last day of your life would be tragic. You might also 
acknowledge the ordinary and practical aspects of life, saying to 
yourself, \emph{I still have things I have to do today. I need to do 
chores and go to work. Let me do this in a way that's going to be an 
offering so that I can give of myself.} \emph{I can support the 
monastery or help some people.} And finally, you might say, \emph{Today 
is my last opportunity to free the heart from the defilements. Let me 
do my best.}

That's why we're here---to free the heart from greed, hatred, and 
delusion. Sometimes we can get lost in judging our 
meditation---\emph{My samādhi is not good enough.} Although our 
meditation is very important, we forget that the whole reason we're 
meditating is to free ourselves from greed, hatred, and delusion. Depth 
of samadhi---and even profound insights---are only significant because 
they support that freedom. So we can reaffirm our intention to be free 
of greed, hatred, and delusion, to be content with our experience. No 
matter how difficult our situation is---our health, our position, or 
our mental states---it's not so bad. It could be much worse. It's 
enough to be content with. This is simply where we find ourselves in 
this moment. This is where our \emph{kamma} has put us. And it 
\emph{is} possible for us to be content with our situation and to use 
it to free ourselves from the defilements.

When we contemplate death and think we have to overcome the 
defilements---which we do---remember that one of the defilements we 
have to overcome is discontent with ourselves. It's so common in this 
culture for people to find fault with themselves, to find it difficult 
to accept themselves. This manifests itself in so many ways. Sometimes, 
we can see this trait in people who are conceited. Often that conceit 
is a mask for a lack of self-acceptance.

As this could be our last day alive, we can look at this 
dissatisfaction, discontent, or lack of self-acceptance and let it go. 
There's no time to waste. We don't have time to indulge in things like 
that. To a certain extent we need to be strict and firm with 
ourselves---\emph{Things are okay. I'm okay. Everyone else here is 
okay. I am good enough and they are good enough, and I don't have to 
make a problem of who I am or who other people are.} We can let go of 
all our critical tendencies; they simply don't matter.

What does matter is that we look into our own hearts to see whether we 
find conceit there, a self-centered quality. Is there pride, ambition, 
anger, judgment, or self-righteousness there? That's why we are 
here---to look into our own hearts. We're not here to look at the 
structures and forms, like the monastery or other people's actions. 
None of this matters. The details don't matter. The monastery can be 
beautifully efficient and well structured, or it can be chaotic, 
strange, confusing, and disharmonious. All that, to a certain extent, 
doesn't matter. What is paramount is looking into our own hearts and 
asking ourselves, \emph{Am I experiencing suffering or stress? What can 
I do to understand it? What can I do to encourage wholesome states of 
mind and decrease the unwholesome states?}

This is our work, and we don't have that much time to do this work, 
because we never know how much time we have or when we're going to die. 
So we need to take this reflection seriously, and to allow it into our 
hearts.

