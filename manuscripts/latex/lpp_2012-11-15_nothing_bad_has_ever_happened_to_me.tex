\mychaptertoc{Nothing Bad Has Ever Happened to Me}
{Nothing Bad Has Ever\\Happened to Me}
{Luang Por Pasanno}{November 2012}

Gratitude is an important quality to bring to mind, because it's easy 
for the mind to focus on faults, flaws, difficulties, and obstructions. 
Whether it's external or internal---the things around us, or our own 
particular difficulties---it's easy to obsess on the negative. When we 
cultivate and bring to mind a sense of gratitude, it helps the mind 
direct attention to what is wholesome, skillful, and positive. It's not 
a small thing, because the mind requires a reserve and foundation of 
well-being, and gratitude is a very direct way of being able to 
establish that. We don't need to have everything go our way or get what 
we want in order to be grateful; the fact that we're alive, breathing 
and conscious is a lot to be grateful for already.

Last night, one of the people at Yoga Mendocino mentioned the Jewish 
concept of a \emph{tzadik}–a person who is an adept or a 
well-developed person. One characteristic of tzadiks is that they have 
a sense of gratitude. In the Jewish tradition, this would be gratitude 
to God. This reminded me of an eastern European Jewish story where a 
rabbi went to visit another highly respected elder rabbi in the 
Ukrainian Hasidic tradition. The visiting rabbi says to the elder 
rabbi, ``In the Talmud it says we should be grateful and praise the 
Lord even for the evil that happens, let alone what is good. How are we 
supposed to understand that?'' And the elder rabbi replies, ``Go to my 
student, Zusha, over in the House of Study and ask him.'' So the 
visiting rabbi walks over to the House of Study and finds Zusha there, 
emaciated, filthy and clothed in rags. He asks, ``Zusha, how are we 
supposed to understand the saying from the Talmud that we should be 
grateful and praise the Lord even for what is evil let alone for what 
is good?'' And Zusha answers, ``I can't tell you. I don't really know. 
Nothing bad has ever happened to me.''

When we are established in the quality of gratitude, it buoys us up 
through everything. There are many logical reasons why we could feel 
that something is awful or not up to our standard. But from a different 
perspective, we have the life faculty and the opportunity to come in 
contact with wholesome experiences. We can make a choice with what is 
good and what is bad, what is conducive to happiness and well-being, 
and what is conducive to making the choices that lead to those 
wholesome qualities. That's something to be incredibly grateful for. 
And we can generate that gratitude by paying attention to the 
inclination of our minds, by recognizing what our minds are doing and 
when we have lost this sense of appreciation. In seemingly difficult 
and trying circumstances, we can encourage those qualities of gratitude 
that allow us to understand, just as Zusha suggests, that generally 
speaking, \emph{Nothing bad has ever happened to me}.

