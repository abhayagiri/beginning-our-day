\mychapter{Waiting Patiently Like a Tick}{Luang Por Pasanno}{May 2013}

Recently I read an extensive article on ticks and Lyme disease. 
Entomologists use a technical term for what ticks do when they're 
sitting on leaves of grass, waiting. It's called 
``questing''---questing ticks. A tick waits patiently for an animal to 
come along. That's what they have to do: sit there and wait. I can 
remember often being with Ajahn Chah sitting and waiting. That's what 
we all do for much of our practice. On one level, we might think, 
\emph{I need to do something.} But if we reframe it, we're simply 
sitting and questing---sitting and watching our breath, sitting and 
waiting for the meal to happen, sitting up at our kuṭis on our own. 
We can reframe how we hold these activities so that we are patient with 
the circumstances we're experiencing, rather than being prey to 
impulses and restless feelings that are so much a part of the human 
condition. We're constantly moving to do something, be something, get 
something, become something, accomplish something, or get rid of 
something. Instead, we can come back to a framework of patiently 
questing, deriving meaning and understanding from the experiences we 
have and the circumstances around us, rather than constantly trying to 
shift, move, and rearrange. So learn how to be patient like a sitting 
tick and use investigation so that it can be a real quest.

