\mychapter{Refocusing on the Defilements}{Ajahn Yatiko}{May 2012}

One of the problems we come across is the tendency to forget the goal 
of our practice and life---in other words, the direction where we 
should be aiming. We need to frequently return to the basic intention 
to be free from the defilements of greed, hatred, and delusion, because 
it's so easy to get caught up in the conditions of daily life, wanting 
the conditions to be a certain way, either liking our present 
conditions or disliking them. People who are active, work a lot, 
coordinate a lot, or do a lot of planning can easily spend the whole 
day manipulating conditions, especially if they are skilled at this.

I once knew a German monk in Thailand who candidly said to me that he 
could look at anything, for instance a water fountain, and tell me five 
things that were wrong with it. He said he could look at anything and 
tell me how to improve it, whether it's in the wrong place, designed 
poorly, or not properly cared for. He said this in a sweet, 
non-boasting way. It's easy for people with that ability to believe 
that they can sort out all the conditions. Sometimes things truly 
\emph{aren't} working well or people \emph{aren't} behaving the way 
they should, and in our minds we think we can get it all sorted out. 
But when we try, it doesn't usually work out the way we want. Or if it 
does, that can be even worse, because then we're not likely to have 
been aware of the defilements that came up when we attached to our view 
about the right way of doing things.

It's as if we have a spotlight of awareness focused on external 
conditions, applying strong views and judgments about the way things 
are or should be. This can create a great amount of suffering in some 
cases. Basically, we're looking in the wrong place. We can spend our 
whole lives focusing outwards, trying to get things, situations, or 
people to do what we want, forgetting that our awareness and perception 
should be focused inwardly, on ourselves. In our tradition it's been 
said that, as monastics, 95 percent of our focus should be on our own 
state of mind, our own movements of mind. We focus internally because 
we want to be free from suffering and this is how we accomplish this 
task.

Not long ago we drove up to the Old Gold Mine Hermitage, and in the van 
I listened to a classic talk from Ajahn Jayasāro. It's called 
\emph{Recognizing the Upakilesas}. In this talk, he analyzes various 
defilements of the mind such as anger, ill will, cruelty, envy, 
belittling, and self-righteousness. As I listened, a mood of irritation 
came up, and I started investigating what this irritation could be and 
how I could describe its nature. To really describe a defilement we 
need to study it. For myself, it's not so helpful to study defilements 
intellectually, as in the Visuddhimagga, where each defilement is 
defined as having specified attributes and proximate causes. Instead, 
the way of study I find most helpful in practice is through direct 
experience. So with irritation, for example, we study what's going on 
right now, once irritation has arisen. We investigate what it's all 
about, what it feels like, and what conditions gave rise to it in this 
specific case. Doing this, we can clearly see that the conditions which 
gave rise to irritation will change, and that irritation will arise 
again in another situation if those same conditions are present.

We can use whatever arises as an object of study. Taking an extreme 
hypothetical situation, let's say an anāgārika calls me ``Ajahn 
Fatico,'' and I get upset about this, thinking what he said was 
inappropriate and insulting. Let's say this gives rise to anger, and I 
really want to set him straight. However, in terms of my own practice, 
the fact that he insulted me is not the point at all because even if I 
do set him straight and he apologizes, in the future I may feel hurt or 
angered by a comment somebody else makes---I've done nothing to prevent 
that hurt and anger from arising when the same conditions recur. It's 
an endless cycle for us if we go about things in this way. So we need 
to change our focus from being set on external conditions, such as what 
other people say and do, and instead focus on the way our attachments, 
cravings, and defilements move through experience. We can see that they 
aren't ours---they don't belong to us. We need to recognize and 
appreciate that.

The Pāli word \emph{āgantuka} means newcomer or guest. An āgantuka 
monk is a visiting monk who comes from a different tradition, who may 
have different standards of Vinaya, or who does not know the way we do 
things here. In one sutta passage, the Buddha says that the defilements 
are visitors---that the mind is intrinsically pure, and the defilements 
come into the mind as āgantuka (AN 1.49-52). So the defilements are 
merely visitors, they don't really belong in the mind. We have to see 
how these defilements arise and pass away, how they're not part of us 
and don't define who we are. To see that takes attention, focus, and a 
clear sense of our what priorities are in the practice.

