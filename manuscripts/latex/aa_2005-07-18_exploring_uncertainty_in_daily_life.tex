\mychapter{Exploring Uncertainty in Daily Life}{Ajahn Amaro}{July 2005}

We often think of insight meditation as something that we do as part of 
the formal practice. The usual instructions are to take some time to 
focus and concentrate the mind, and, when it's steady, to begin 
reflecting upon the flow of experience in terms of \emph{anicca}, 
\emph{dukkha}, and \emph{anattā}---impermanence or uncertainty, 
unsatisfactoriness, and selflessness or things not being self. Luang 
Por Chah encouraged us to employ these themes of investigation and 
contemplation, not just on the meditation mat, but also throughout the 
day, during all of our activities. We can use these themes as constant 
companions to explore and process our own experience as we relate to 
the events of the day.

Ajahn Chah particularly emphasized the contemplation of anicca. He 
would often render anicca as ``uncertainty,'' rather than 
``impermanence.'' When we think of things being ``impermanent,'' it can 
have a remote or objective quality, whereas ``uncertainty'' describes 
the feeling of the mind and heart when we experience change and 
transiency. That's what we perceive---the feeling of uncertainty. We 
don't know. It's not certain because things are changing. It's not 
predictable.

This experience is easily accessible in our day-to-day lives, and is 
extremely useful to remember and sustain as a contemplation. We can 
understand this with something as simple as drinking a cup of tea 
prepared by someone else. As we bring the cup up to our mouths, we 
don't know how it's going to taste; there's that moment of almost 
palpable uncertainty. Or when we are on our way to perform some task, 
we might think, \emph{I'm now heading up to the workshop to get this 
particular tool.} But if we reflect that finding the tool there is 
uncertain, then when we get to the workshop and find that in fact the 
tool isn't there, it's not a problem.

Although Ajahn Pasanno is leaving for Sacramento this morning, if we 
keep the uncertainty reflection in mind, then we can be clear that we 
don't really know if he is going to get to there. The van driving him 
might have a burst tire or the drive shaft might snap somewhere along 
the highway, and then he would spend the day contemplating the heat 
element in the Valley. Not that I wish for that to happen, but I can 
say to myself, \emph{Oh, Ajahn Pasanno is going to Sacrameno is he? Is 
that so? Really?} All we can know is that there's a plan for him to go 
to Sacramento. There's a plan to prepare the meal; there's a plan to go 
into town. This much we know, but what will happen is uncertain. Are 
things really the way we judge them to be? It's not certain. This is 
not meant to create doubt and confusion, but the more we perceive each 
moment as uncertain, the more we see clearly that we don't know what 
will happen. We don't know what the outcome will be. We don't know if 
what we're doing will work or will happen the way we expect.

Similarly, we may judge the people around us and think that they are 
being greedy, aggressive, or selfish because of something we perceive 
they are doing. Later we might find out that they weren't being greedy, 
aggressive, or selfish---they were actually doing something that was 
helpful for somebody else. \emph{It's merely my perception, my 
interpretation of a particular act, my presumption, my guesswork, or 
what I read into it.} When we see our assumptions clearly and find out 
that we were wrong about someone's motive, the framework of anicca can 
provide a quality of spaciousness and freedom for ourselves. The 
perception of anicca loosens the boundaries and obstructions we 
continually create through thinking, presumptions, opinions, judgments, 
expectations, and plans. We can learn to hold material objects, 
thoughts, feelings, and actions in the context of uncertainty. 
\emph{This judgment is uncertain}, or \emph{This activity is 
uncertain}. When perceiving in this way, the heart is completely ready 
and open for the changes that can and often do occur.

If things go in a fortunate way, then we feel the pleasure of that. If 
they go in an unfortunate way, then we feel the painfulness of it. 
Ajahn Chah would say that clinging to happiness is just as bad as 
clinging to unhappiness. It's like trying to take hold of the tail of 
the snake rather than the head. If things go in a fortunate way, we 
might clinging to happiness and say to ourselves, \emph{Oh great, now I 
have it, this is excellent}---which means the heart has invested in 
that happiness. This happiness is like the tail of the snake. Even if 
we grasp the harmless tail, it's not very long before the head whips 
around and bites us. For example, if we've been preparing the meal, we 
think, \emph{Oh great, this is the best potato salad I've ever made. 
It's really good, I'm really pleased with this}. Then someone comes 
along, takes a mouthful and says, ``Is it supposed to taste like 
this?'' The anger or misery we feel is exactly proportional to the 
degree we had invested our happiness in the potato salad being just 
right, pleasing, and good. So we can learn to see that judgment as it's 
arising and consider, \emph{Is the salad truly good? It's uncertain.} 
That way, we don't take hold of the snake's tail. If someone says it's 
bad, we can realize that it's simply their judgment, and the snake's 
head doesn't bite us. We might say to ourselves, \emph{That person's a 
fool for thinking that.} But then we can see that this too is our own 
judgment. ``Good'' or ``bad'' salad, or whatever, is just the way we 
perceive things. It was uncertain to begin with. When we see this 
clearly, the causes for conflict, confusion, stress, and living a 
burdened life are no longer generated.

