\mychapter{Mindfulness Develops Right Speech}{Ajahn Jotipālo}{May 2013}

Ajahn Dton came to visit us once, right after one of our winter 
retreats. During a question and answer period, I asked him if there 
were any exercises we could do to improve the wholesomeness of our 
speech or how to approach right speech as a practice. He saw right 
through me. He could see I was hoping there was some sort of technique 
I could apply before speaking that would solve all my problems around 
speech---a technique that would tell me when my speech was appropriate 
and timely. Ajahn Dton shook his head and said, ``No, if you are 
looking for a method or a technique, it's too late. What you really 
need to be developing is all-around, all-encompassing mindfulness. For 
example, right now, where are your sandals? If you were mindful when 
you took them off you would know exactly where they are, what direction 
they are facing, and possibly whose sandals are next to yours.''

After we've taken off our jackets or set our car keys down, if we don't 
know exactly where they are right now, we probably weren't mindful when 
we set them down. We can take that as a practice, a training. With 
everything we do we can try to make a mental note of it. That's one way 
to bring ourselves fully into the present moment. To do that we need to 
carefully slow down and be aware of the action we are doing. If we 
practice being aware of where we place our personal items, then as a 
benefit from that, we have trained ourselves to slow down and be aware 
of our mind states. When speech then comes into play, the mind can be 
more aware of what we are doing and we can start seeing what is 
appropriate or not appropriate---if this is the right time to speak or 
not. We may learn to question whether we need to say anything at all. 
By training in this way, the ability to discern and make judgments 
about speech can become much stronger because we have developed it with 
mindfulness.

