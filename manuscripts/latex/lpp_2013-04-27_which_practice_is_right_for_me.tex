\mychapter{Which Practice Is Right for Me?}{Luang Por Pasanno}{April 
2013}

After the meal today we will be taking formal leave of Ajahn Sucitto. 
Living as we do in an American Buddhist monastery, we're in a distinct 
minority and rather isolated. So when visiting teachers like Ajahn 
Sucitto come here and talk about specific practice experiences they've 
had, it's a precious opportunity for us. Having listened to them 
carefully, we can then reflect on our own experience, asking ourselves: 
\emph{What have I done that works? What has been beneficial? What has 
helped the mind relinquish its attachments and defilements? What has 
helped the mind become more peaceful, settled and clear?} It's not 
something we can learn from a book, reading about some theory and 
trying to make sense of it. Rather, it's about paying attention to this 
element of experience---our own experiences and the experiences of 
others. For instance, when Ajahn Dtun visited here recently, the 
teachings he offered were always conveyed in terms of his own practice: 
``This was my experience. This was the practice I used. This is what 
worked for me.'' After listening to teachings like that, we may be 
inspired to apply some of the practices described. But if we do, it's 
important that we attend to what genuinely works and examine this in 
our daily lives, interactions, and formal practice.

To put it another way, we can ask ourselves, \emph{What is it that 
aligns us with Dhamma?} The Buddha tells us that if something aligns or 
accords with the Dhamma, it's going to increase our happiness, 
well-being, clarity, and understanding. The opposite is true as well. 
When what we're doing increases our \emph{dukkha}---our anxieties, 
confusion, and agitation---it's a pretty sure bet that we're not 
aligned with the Dhamma. Trying to emulate a practice used by a 
respected teacher may seem like a fine idea, but the test is whether 
applying that practice accords with the Dhamma \emph{for you}---in your 
own, personal experience.

To employ this test skillfully takes practice, reflection, and a 
willingness to experiment and try things out. But it's something we 
need to do. Practicing Dhamma in accordance with 
Dhamma---\emph{dhammānudhamma paṭipatti}---is a quality that needs 
our close attention. In fact, it's a key element of stream entry. So as 
we go about our daily lives, as we continue to cultivate the practice, 
it is helpful to reflect: \emph{Does what I'm doing accord with Dhamma? 
What I'm saying, thinking, and feeling---does it accord with Dhamma?} 
If we are willing to investigate these questions closely, their answers 
will help us clarify whether our particular practices are leading us in 
the right direction.

It's important to also understand that what works for us will change, 
depending on conditions. For instance, just because something worked 
today doesn't mean it's going to work tomorrow, and what didn't work in 
the past may work for us now. This makes it necessary for us to adapt 
and experiment. Often Ajahn Chah would repeat a quote from one of his 
teachers, Ajahn Tong Rat, who taught that the practice is very 
straightforward and easy: ``If the defilements come high, then duck; if 
they come low, jump.'' In other words, we're to do whatever the 
situation demands---whatever works as long as it is not causing 
ourselves or others more suffering. This entails first asking 
ourselves, \emph{How might I work with this particular situation?} Once 
we have a sense of what might be a good approach, we put it into 
practice, try it out, and then evaluate the results.

This all points to an ongoing, evolving relationship between the 
workings of our practice and the workings of our minds. It takes time 
to discover skillful ways of engaging with that relationship; it's a 
learning process. But by sticking with this process, by taking a 
genuine interest in it, we can develop a good sense of what 
practices---whether from teachers or our own innovations---are truly 
beneficial, what practices accord with Dhamma, what practices genuinely 
work for us.

