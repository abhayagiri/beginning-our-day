\mychapter{This Buddha Isn't the Same as the Old Buddha}{Luang Por 
Pasanno}{April 2013}

Ajahn Chah used to emphasize the need for steadiness in our practice, 
especially with our application of effort---finding a level that pushes 
the edge a little bit, but is also sustainable and doesn't lead to our 
burning out. Sometimes we want to focus on tangible results, and it's 
certainly gratifying to feel we're really getting somewhere, that 
something's really happening. But I think it's much more important to 
focus on this kind of steadiness.

Steadiness requires patience. We need that patient quality of mind 
which is resilient and able to endure the different moods that come 
up---the highs of being exhilarated about the practice, and also the 
lows we inevitably experience. Moods oscillate like this; it's natural. 
So to achieve steadiness, patience must be applied along with effort.

The Buddha uses the image of water dripping very slowly into a huge 
earthen jug, drop by drop. If we focus on the individual drops, it may 
seem like nothing much is happening, as if the jug will never get full. 
But if the water continues to drip, slowly but steadily, the jug 
\emph{will} fill up over time. It's unavoidable. In the same way, 
steadiness and continuity of effort paired together with patience will 
bring results in our practice, even if it seems like nothing much is 
happening.

There's no need to aim for a sudden, spectacular impact or achievement. 
It's tempting to hold onto the image of the Bodhisatta sitting under 
the Bodhi Tree, resolving not to move until he had become enlightened, 
even if his blood dried up and his bones crumbled. We say to ourselves, 
\emph{That's inspiring. I'll try to do that.} Once when a monk said 
that to Ajahn Chah, Ajahn pointed back at him and said, ``Maybe 
\emph{this} Buddha trying to do that isn't same as the \emph{old} 
Buddha.'' What Ajahn meant was that the Bodhisatta's resolve was the 
culmination of hundreds of thousands of lifetimes spent building the 
\emph{pāramī}, with steadiness and patience.

So that's what we focus on---this steadiness of effort and patient 
endurance. Put them together and let them work for you.

