\mychapter{Leave No Trace}{Luang Por Pasanno}{June 2012}

There's an idiom in Zen: ``Leave no trace.'' It's an attitude ascribed 
to persons who do everything with clarity, efficiency and mindfulness. 
It's helpful to cultivate this attitude, both as an ideal within the 
mind and also in terms of the little things we do: paying attention so 
we do not leave a trace behind us when we're engaged in our daily 
activities. This can be helpful even with very simple things like 
preparing for \emph{pūjā} or getting ready for work. Wherever we go, 
we don't create a mess, we leave things neat and tidy. After washing 
our robes in the washing room, we wipe up the water around the sink; we 
make sure things are clean and tidy when we leave. Leaving our 
\emph{kuṭi}, we ask ourselves if everything is neat and tidy. We see 
that everything is done in a circumspect, crisp and clear manner. In 
this way, we can cultivate an attitude of leaving no trace.

There's an image for that in the Zen tradition: Apparently, when 
turtles walk their tails swing back and forth, sweeping away their 
footprints. That image suggests not leaving a trail of debris behind us 
when we walk from one place to another. Sometimes when the monks are 
getting ready for \emph{pūjā} or having tea in the monks' room, they 
open and close the cupboards without thinking of others who can hear 
this. For those of us sitting in the Dhamma Hall next door, it's 
bang-crash-bang-crash-bang-thump-thump. So we need to be sensitive to 
the impact we have on those around us and try not to leave a trace of 
noise. In terms of our relations with people, we can reflect on the 
things we say by asking ourselves, What kind of trace am I leaving in 
this conversations by the comments I've made? With our internal 
processes we can ask ourselves, \emph{What kind of trace am I leaving 
in my mind due to the moods I've picked up or the attitudes, views and 
perspectives I'm holding?}

Instead of leaving behind the debris of our mental, spoken, and bodily 
actions, we can cultivate an attitude of leaving no trace. One image 
used for an enlightened being is a bird that leaves no tracks in the 
air. That's a great reminder for us as we go about our daily tasks and 
activities.

