\mychapter{Joy: Rising Up and Going Forth}{Ajahn Yatiko}{May 2012}

I was speaking with somebody recently who shared that he was finding it 
difficult to settle into the joy of life, to sit back and enjoy being 
alive. He thought that something was wrong with that. I explained to 
him that, from a Buddhist perspective, it's not about settling into a 
joy that's supposedly inherent in life; rather, joy is something that 
comes from past action, from \emph{kamma}. As my father used to say, 
``There's no such thing as a free lunch.'' That's so true and a good 
reflection for the practice. If we want to experience joy, it takes 
effort. How could it be any other way? And while it may seem somewhat 
paradoxical, the effort needed for joy to arise must be directed toward 
letting go.

Now I have fairly passive tendencies, and so for me, letting go 
requires making an effort to counteract those tendencies. So letting go 
is not a passive experience, but an experience of going forth, as if I 
have to rise up and go forth into the present moment. It's a wonderful 
relief when I do this---it's joyful. On the other hand, for people who 
tend to be driven and goal-oriented, the process of letting go is very 
different. For them, letting go comes with the realization that they 
don't need to put forth a self-motivated, Herculean, obsessive effort. 
It still takes effort, but for them the effort mostly goes toward 
relaxing and calming the driven quality of their energy.

Once a sense of joy arises, it takes more effort to keep it going. We 
get onto our walking paths, walk, put forth effort, and, when the mind 
wanders and moves away from its center of awareness, we bring it back. 
At some point joy may arise. It's wonderful when that happens, but how 
long does that joy last? The image that comes to mind is one of those 
carnival wheels that spins around for about 30 seconds, then slows 
down, and finally stops. That's akin to the arising of joy and the way 
joy can be sustained. We put in effort---spinning the wheel---and from 
that effort joy arises. Then the wheel stops, and we exert our effort 
again. Over time, with our continued effort, we can keep the wheel 
spinning longer, and joy sustains itself longer.

When it comes to making our walking and sitting practice sustainable, 
we do have to enjoy it. We can turn our attention to that joy. Right 
now the weather is beautiful; the sun is out, and it's filtering 
through the trees as we walk or sit in the forest. It's quiet, 
peaceful, and easy to enjoy this opportunity to put forth effort toward 
something that's absolutely blameless and wholesome. It's a remarkable 
opportunity. If we can enjoy it, cultivate a sensitivity for the 
beauty, and develop the wonderful experience of being free and 
unburdened, then that's going to be a great help to our practice, our 
monastic lives, and our spiritual journey.

