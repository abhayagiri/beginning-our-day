\mychapter{How Do I Accord With Truth?}{Luang Por Pasanno}{May 2012}

There's a tendency to look for some technique or method we can follow 
and perfect, whether it's for cultivating mindfulness, concentration, 
or whatever. This is helpful at times, but no matter what method or 
technique we're trying to develop, we always need to consider whether 
we're doing it within the framework of Right View---whether we're doing 
it with a view that accords with Dhamma. If we're missing Right View in 
the way we're applying and cultivating our practice, no matter what 
method or technique we use, we're going to complicate things and create 
difficulties for ourselves. So we need to keep turning attention inward 
and asking ourselves, \emph{How do I accord with Dhamma? How do I 
accord with Truth? How do I align myself with a view that's in accord 
with reality and in accord with the Buddha's perspectives?}

Generally there are three ways we view experience: Being drawn toward 
the world, being drawn toward the sense of self, or viewing experience 
in accord with Truth. Often we're entangled with, caught up in, or 
propping up the feeling of self or the views and opinions that arise 
from the sense of self. Or it's the pull toward the world, relying on 
the world as a source of gratification, energy, or interest. When we're 
aware of that happening, we can draw our attention back and ask 
ourselves, \emph{How do I accord with Dhamma, with Truth?} This is 
where that sense of Right View is so important.

I remember a group of Buddhist teachers saying to Ajahn Chah, ``We 
teach about Right View. We've been studying about Right View. But what 
the heck \emph{is} Right View, anyway?'' Ajahn Chah smiled and held up 
the cup he was drinking from and said, ``Right View is knowing that 
this is a broken cup. It's just that much.'' So we need to recognize 
the impermanence, uncertainty and unsure nature of experience, rather 
than making assumptions about things from the bias of ``my way'' or 
from perceptions of the world as good or bad. We might think, \emph{Oh 
God. Everything is falling apart!} Or \emph{Everything is great and I'm 
going to get what I want. The world is going to gratify me!} Well, not 
sure. We need to stop and reflect to ourselves, \emph{What is sure?} We 
may see that what is sure is what we can know at this present moment. 
What's sure is we can make choices and try to guide ourselves in ways 
that are skillful.

When we make those choices, it is helpful to recollect that we are the 
owners of our actions, heirs to our actions, and that those actions 
have results. As soon as we formulate an idea, often something will 
come out of the mouth, or we do something with the body. Those actions 
have consequences, good or bad. We need to take responsibility for 
that, recognizing the causal relationship between particular actions 
and their results. How do we attune ourselves to this cause-and-effect 
conditionality?

We can begin by asking ourselves, \emph{What's going to lead to my 
well-being, happiness and peace? How do I avoid this confusion, 
difficulty, and suffering?} And we can keep taking responsibility, 
reflecting over and over again, \emph{I'm the owner of my actions, heir 
to my actions, born of my actions. Whatever actions I shall do, for 
good or for ill, of those I will be the heir.} But in order to take 
responsibility like this, we need to step back and clearly see: 
\emph{It's not sure. I don't have to believe my moods. I don't have to 
believe the world around me. I don't have to get caught in conditions. 
None of it is sure.} Once this sinks in, we can allow Right View and 
the Dhamma to guide us, rather than allowing ourselves to be pulled by 
the conditions of the world or the conditions of identity and self view.

