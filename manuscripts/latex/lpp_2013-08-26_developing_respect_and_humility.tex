\mychapter{Developing Respect and Humility}{Luang Por Pasanno}{August 
2013}

Before the monastery was established, I can remember Ajahn Amaro 
telling me that there was a Thai monk in Fremont, Ajahn Mahaprasert, 
who was keen to see a forest monastery succeed in the Bay Area. Since 
the founding of the monastery, Ajahn Mahaprasert's support has been 
unfailing.

Generally, in Thailand, at the beginning of the Rains Retreat or a bit 
earlier than that, most of the monks in monasteries will go together to 
pay respects to senior monks in the same province. This is important 
for a monastery and for one's individual practice. These trips are 
opportunities for showing respect and giving respect outside of one's 
own little circles as well as within one's circle. In America, we don't 
have many options, but we do have Ajahn Mahaprasert. We're fortunate, 
because he's a worthy monk to pay respects to.

When we reflect back on the discourse that the Buddha gave on the 
highest blessings, one blessing he points to is \emph{gāravo ca 
nivāto ca,} the qualities of respectfulness and humility. These are 
qualities that bring forth the highest blessings. They engender the 
sense of not having to carry the burden of a sense of self all the 
time---which \emph{is} an incredible burden. Sometimes we can lack 
humility and respect to the degree that we outwardly appear full of 
ourselves and carry our sense of self around showing it off to other 
people---which is not very inspiring.

We have the opportunity to establish a sense of respectfulness and 
humility not just once a year, but in our day-to-day lives: How we live 
with each other, how we treat each other, and how we engage with the 
world around us. So often we're thinking about meditation techniques 
and methodologies. But when we reflect on the qualities that help the 
mind become peaceful, we can see how often we overlook these 
fundamentals that are at the roots of our being---these qualities of 
respectfulness and humility that allow the mind to become quiet, at 
ease, and easily peaceful. If we've engendered these qualities within, 
then when we sit down to meditate, we don't have to wonder what 
techniques or methods will make us peaceful, because the peace is 
already there waiting for us.

