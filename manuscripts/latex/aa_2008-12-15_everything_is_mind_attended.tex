\mychapter{Everything Is Mind Attended}{Ajahn Amaro}{December 2008}

Even though the snow falls nearly every year, I'm always struck by the 
same impression: How much it changes the perception of world I live in. 
Water arriving in a particular form, turning all the horizontal and 
semi-horizontal surfaces white, the sound and the shape, everything 
highlighted in a strange, unusual way, all the color washed out. It's a 
very good teaching, isn't it?

It's like the mood of the mind. It can be so persistent when it's stuck 
on some track, excited, irritated, or worried about something. There 
seems to be a coloring of the whole world. This is my problem. I'm 
supposed to worry about this. I have to worry about this. This is my 
project. The only thing that really matters in the world are the 
railings on the Bhikkhu Commons kuṭi or clearing the trap under the 
sink at Casa Serena. The center of the world is this particular job we 
have, these letters that have to be written or this meal that has to be 
cooked. These things that the mind latches onto seem so real and 
permanent---our particular projects, responsibilities, fears, hopes, 
and desires. Well, of course it's that way. Can't you see? Then 
suddenly---snap! Overnight the world changes color and there's a whole 
different mood. Suddenly, Oh look, it's all different. How utterly 
transformed our perceptions can be merely by a different shape, 
coloring of the landscape, or the sound of the valley.

This is an excellent reflection on our moods. When we think we're in 
the middle of some difficult tangle, some big issue or conflict, an 
important project, a difficult relationship---suddenly it all shifts. 
Poof! It's not a problem anymore because it just changed. The thing 
that we thought we had to worry about wasn't anything we needed to be 
concerned about in the first place. It wasn't the way we thought it 
was. It was a mistaken impression. When we see and reflect in this 
way---how much the world can change when we're in the midst of some 
particular anxiety, worry, project, or activity---it helps to provide a 
perspective. Oh, this is only my impression. Now I see.

We get caught up in thinking, I have my name written on this particular 
issue. It's mine. I'm responsible for this project. I have to dig this 
hole. I have to fix this railing. I'm the one who has to pay the debt. 
It seems so real and important, and so much like it's the center of the 
world. We see what the mind has put onto it, what Luang Por Chah would 
call ``a mind-attended thing.'' He would say, ``Everything is mind 
attended.'' Even though in the Abhidhamma it says there are some things 
which are mind attended and there are some things which are not mind 
attended, Luang Por Chah would say, ``Well, actually, everything is 
mind attended.'' As soon as we know about something, we form an 
opinion, make a judgment, and create things with our thoughts.

A simple event like a snowfall, a change of the landscape, and we're 
reminded that this particular thing that's so significant to us is only 
significant because of our particular conditioning, our particular 
expectations, fears, hopes, or abilities. The thing's significance is 
not inherent, it's merely something the mind has added onto the thing. 
With thst realization we can carry out the work we need to do. We can 
fix those railings, dig this hole, cook that food, answer those letters 
and move that table around those difficult corners in a much more 
easeful and peaceful way. It's not the center of the world. We can see 
that it isn't so personal, so burdensome, or so much about ``me'' and 
``mine.'' When we realize this, then everything we need to attend 
to---the events of the day and the responsibilities that we have---is 
much more natural and easy to carry out. It becomes more like 
breathing. The body does it on its own. It doesn't have to be ``me'' or 
the ego that does it. Gravity works on its own---I don't have to 
\emph{do} gravity. When we start to function with that same kind of 
naturalness and easefulness, working and living in community becomes 
just like breathing.

