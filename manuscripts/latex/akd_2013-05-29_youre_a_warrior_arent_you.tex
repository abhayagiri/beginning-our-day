\mychapter{You're a Warrior Aren't You?}{Ajahn Karuṇadhammo}{May 2013}

Last night we had a discussion about the power of practicing in 
community. One of the things I remember fondly about an experience in 
community was during my third vassa away at Chithurst. When I first 
arrived, I was trying to get my bearings and slowly familiarizing 
myself with some of the community members. One of the monks, who was 
just junior to me, sat on my left and I noticed after a number of weeks 
he wasn't engaging or making any contact. He was very quiet and 
reserved. I didn't know what to make of it. It was several weeks later 
that I began to think, \emph{Well, perhaps he just doesn't like me or 
maybe I said something in the beginning that disturbed him.} At one 
point only he and I were in the day room resting after the meal. He had 
yet to say anything to me the entire time since I had arrived. I was 
sitting on one side of the room and he was sitting on the other and he 
broke into conversation saying: ``You're a warrior.''

I thought, \emph{What? What does he mean?} so I said to him, ``What?''

He replied, ``You're a warrior, aren't you?''

I thought to myself, \emph{Well, I've never thought of myself as a 
warrior.} So I asked him: ``What do you mean? Why do you say that I'm a 
warrior?''

He replied back to me, ``Well I know, because I'm a warrior too.''

I thought, \emph{What's going on in this guy's mind?} Again I said, 
``What do you mean?''

He responded, ``Well, I can see that you warry a lot. And I know that 
because I warry a lot too.''

I thought, \emph{Okay, now this makes sense}---I'd forgotten for a 
moment that I was in England and things like ``worry'' are pronounced a 
bit differently. This monk went on to explain that he had been 
suspicious of me for the first few weeks because I was an American and 
he had many different preconceptions of what Americans were like and 
did not want to get too close or engage. As it turned out, we became 
good friends over the year and appreciated each others company.

Last night here at the monastery, we were talking about living in 
community---the effect people have on us and how, particularly as 
Westerners, we have a strong individual self-identity and therefore 
construct so much \emph{dukkha} around that identity. With this 
individualism in our culture, we don't have a lot of experience living 
in community nor do we understand the benefits of living with and 
learning from other people in this way. Ajahn Chah's training heavily 
emphasized the power of living in community. It can be difficult to 
live this way because we all come from diverse backgrounds and have our 
different habits and ways of being in the world. But this kind of 
interaction, living and working with other people twenty-four hours a 
day, is a powerful practice.

I also briefly stayed at Amaravati during that same year. In the 
Amaravati kitchen, there was a vegetable-peeling machine that consisted 
of a big drum with a rough surface on the inside where vegetables 
whirled around. Carrots or potatoes were thrown into this machine and 
they banged up against each other and against the rough surface of the 
drum until all the skins were peeled off. When I first saw this I 
thought, \emph{Wow, this is just what living in community is like.} 
Part of it is the process of constantly being in a close living 
environment with other people until over time all the rough edges are 
worn down through interaction and give and take. There aren't a lot of 
people who would willingly throw themselves into a situation like a 
vegetable peeler, but that's basically what we have done here in the 
monastery.

There's a strength and power that we build up from engaging in formal 
meditation practice, being on our own, and establishing a sense of 
solitude. And this is, of course, important in the Forest tradition. 
But it's good to ask ourselves, \emph{How much self-concern and 
self-identity can be let go of, worked with, and honed down through the 
practice of being in community?} As individuals, we really aren't as 
important as we think we are. All the mistakes we make, the 
expectations we have of ourselves and other people, the difficulties, 
criticisms of others, criticisms of ourselves---all of these rough 
edges need to be acknowledged and seen for what they are. This 
acknowledgment and understanding can't take place if we don't have that 
kind of bump and grind with other people to help expose those rough 
edges. It's not easy---none of the practice is that easy---but we learn 
to whittle away the rough edges through the use of community and 
through self-reflection about how we are engaging in community. We do 
this by letting go of the expectations for ourselves or other people to 
be a particular way. It's a powerful practice to establish a state of 
mind that is not so self-involved. When we talk about the practice of 
letting go of the ego, the self-identity, I think we need to reflect on 
the importance of communal life and how lucky we are to have it. While 
we are living with others we can depend on that structure of support 
and make full use of it.

