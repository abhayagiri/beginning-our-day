\chapter{Preface}

``Beginning Our Day \ldots{}''

These quintessential words are spoken by Luang Por Pasanno before he
begins each of his morning reflections.

Five days a week, at Abhayagiri's morning meeting, work tasks are
assigned to the residents and guests living in the monastery. Shortly
thereafter, one of the senior monks offers a brief Dhamma reflection so
that the residents and guests have something meaningful to recollect
throughout their day.

These talks are given spontaneously and often address an event that is
about to occur, a condition that is already present in the monastery, or
a general teaching on Dhamma. The most common thread through all the
reflections is that of practicality: distilling the most important
teachings of the Buddha into pertinent and applicable practices. Though
many different teachings are touched upon, the fundamental aim is to
encourage the abandonment of the unwholesome, the cultivation of the
wholesome, and the purification of the mind.

While these teachings may be read together several at a time, readers
might find it more useful to focus on a single reflection so they can
easily recollect, contemplate, and make use of it throughout their day.

This book was made possible through the contributions of many people.
Over ten years ago, Pamela Kirby initiated the project when she placed a
recorder in front of one of the senior monks during a morning reflection
and proposed that a book be written. Matthew Grad, Jeff Miller, Ila
Lewis, Pamela Kirby, Ray Peterson, and Laurent Palmatier were the main
substantive editors of the material, enduring the long and difficult
process of editing the transcripts into compact and well-written
teachings. Ruby Grad and Pamela Kirby completed the copy editing. Sumi
Shin designed the cover. David Burrowes, Dee Cope, Josh Himmelfarb, Evan
Hirsch and Jeanie Daskais helped with further refining of the text. For
several years, Khemako Bhikkhu recorded the senior monks' reflections.
Kovilo Bhikkhu and Pesalo Bhikkhu provided corrections on an early draft
of the book. Suhajjo Bhikkhu generously dedicated a significant amount
of time on the overall book design and typesetting of the text.

The Kataññuta group of Malaysia, Singapore and Australia generously
brought this book into full production.

May these teachings bring insight into the nature of Dhamma and provide
a pathway toward the development of true peace and contentment.

\vspace{15pt}\noindent Cunda Bhikkhu\\
Abhayagiri Monastery,\\
Redwood Valley, California\\
May 15\textsuperscript{th}, 2014
