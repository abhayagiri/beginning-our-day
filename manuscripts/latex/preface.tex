\chapter{Preface}

``Beginning Our Day \ldots{}''

These quintessential words are spoken by Luang Por Pasanno before he
begins each of his morning reflections.

Five days a week, at Abhayagiri's morning meeting, work tasks are
assigned to the residents and guests living in the monastery. Shortly
thereafter, one of the senior monks offers a brief Dhamma reflection so
that they have something meaningful to recollect throughout their day. 

These talks are given spontaneously and often address an event that is
about to occur, a condition that is already present in the monastery, or
a general teaching on Dhamma. The most common thread through all the
reflections is that of practicality: distilling the most important
teachings of the Buddha into pertinent and applicable practices. Though
many different teachings are touched upon, the fundamental aim is to
encourage the abandonment of the unwholesome, the cultivation of the
wholesome, and the purification of the mind.
While these teachings may be read together several at a time, the reader
might find it more useful to focus on a single reflection so they can
easily recollect, contemplate, and make use of it throughout their day. 

May these teachings bring insight into the nature of Dhamma and provide
a pathway toward the development of true peace and contentment.

