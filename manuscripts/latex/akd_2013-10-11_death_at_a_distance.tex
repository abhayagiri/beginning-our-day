\mychapter{Death at a Distance}{Ajahn Karuṇadhammo}{October 2013}

The contemplations of old age, sickness, and death are themes that seem 
to be repeating themselves these days. Iris's recent diagnosis of lung 
cancer is just one example. Based on my own experience of having been a 
nurse, it is interesting to see how the mind immediately moves toward 
the technical aspects of the illness: \emph{What kind of tumor is it? 
What is the staging? What are the treatment options? What are the 
services that are available?} This is all based on my conditioning and 
background. Then sometimes, I move into: \emph{How is this for Iris? 
What is she going through? What kind of resources does it take to deal 
with this kind of information? How is she coping with this? What can I 
do to help her? How can I be most attentive without being overly 
solicitous?}

I recognize, both in myself and others, that this kind of speculation 
and questioning are expressions of a truly noble concern for what 
Iris's experience may be like and how we want to help. In some ways 
though, as we try to help we can take the focus off our own experiences 
and put off the need to reflect on ourselves. Many of the Buddha's 
teachings encourage us to pay attention to and contemplate the body in 
terms of aging, sickness, and death, especially as these subjects 
relate to ourselves. Nonetheless, it is difficult to acknowledge this 
until we find ourselves experiencing this on a personal level. So, we 
continue to think about that other person and her problem, hoping that 
by externalizing it in those noble ways, we might somehow be able keep 
old age, sickness, and death at a distance.

How can we internalize these thoughts? How can we remind ourselves that 
old age, sickness, and death is about me? We can say things to 
ourselves like: \emph{This will happen to me sometime in the future.} 
But that is another great way of putting a bit of a distance between 
ourselves and reality. What we need to do is adjust our views so we can 
see that not only is this happening out there to other people, or have 
some vague notion of it occurring in the future for ourselves, but 
right now, \emph{This is my body that is aging and there might actually 
be something happening to me now, in the present.} Then what does that 
mean? How should we contemplate that? This body is incredibly sensitive 
and fragile. We can be sensitive to the feelings of the body, the 
sensations, and come to understand the body as a sensitive and fragile 
condition of elements brought together by natural factors. It's an 
exceptionally delicate balance which can be lost at any moment. Just as 
Iris has a tumor in her lungs and tumors throughout her body, we can 
imagine that happening in our own bodies and take it to heart.

Try to do that as a daily reflection. If we are frequently able to 
bring this perception to mind, then the more real it will become for 
us. Paradoxically, this reflection can bring up a sense of ease and 
calm. This settled quality arises from insight into the characteristics 
of \emph{anicca} and \emph{anattā}, impermanence and not-self. Both 
characteristics become more apparent and real so that when serious 
illness or imminent death approaches there will be a familiarity with 
the nature of one's finite existence. The more those familiar feelings 
can be brought in, the less fear there is around letting go of that 
which really isn't ours to hold onto in the first place.

