\mychapter{When Help Is Needed}{Luang Por Pasanno}{August 2012}

One focus of our practice is to look after each other and help each 
other out. In one discourse where the Buddha finds Venerable Anuruddha 
and his friends living in the forest, we see that they are intent on 
formal practice; but whenever something needs to be done, they come out 
of meditation and help each other. That is a very beautiful story from 
the suttas. Our tendency, however, is to try doing everything by 
ourselves, which is usually not so comfortable. But when we learn to 
help each other, it's much more convenient for everyone. It's like the 
convenience of using two hands to wash. When one hand is trying to wash 
itself, the job takes quite a while and is less than thorough. But with 
two hands washing each other, it's quicker and everything gets clean.

Similarly, as human beings living together with duties and chores, we 
can learn to look out for the many ways we can help one another. A lot 
of this practice comes from the cultivation of mindfulness, of paying 
attention. It's so easy to have our blinders on and think, \emph{I'm 
only doing this task, this is what I'm doing.} We don't have enough 
attention and clear comprehension to reflect and ask ourselves, 
\emph{What are other people doing? Do they need assistance? Is there 
some way I can help out?}

When we reflect like this for the sake of others, it helps us cultivate 
mindfulness. And it is also a way we can step outside of ourselves. So 
often we set boundaries---me and my autonomous self. While this can 
have a useful function, it can also be quite limiting and isolating, 
and sometimes leads to selfish behavior. What we are doing instead is 
learning how to let go of the fixation on I, me, and mine---to 
relinquish that self-centric modality of living in the world. This 
fixation, the Buddha said, is one of the main sources of \emph{dukkha}. 
By looking out to see how we can be of assistance to others, we 
undermine that modality and give ourselves the opportunity to 
experience well-being.

