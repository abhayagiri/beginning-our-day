\mychapter{Inner Stillness}{Luang Por Pasanno}{July 2005}

As I was reading some teachings by Ajahn Munindo, a phrase he used 
resonated with me, and I find it useful to bring up and contemplate. He 
spoke of inner stillness; specifically, trying to attend to things, 
such as performing one's duties, or engaging with what needs to be 
done, with a quality of inner stillness.

Often the mind is impinged upon by the notion that it should be 
reacting, moving, or shifting. It's like the mind is waving its arms 
around trying to get attention: ``Don't forget about me!'' But if we 
return to inner stillness rather than going to the excitement, the 
movement, the reactions, and the ups and downs of the mind, then we can 
start to see arising and ceasing more clearly. We can recognize the 
pull of a particular impingement as simply something else that arises 
and ceases. This allows the mind to settle, which in turn fosters the 
quality of inner stillness even more.

Though our minds may come up with their justified indignations or 
justified attractions, if we hold these things within a framework of 
inner stillness, then we can make a choice to stop ourselves from being 
pulled in different directions. Instead, we can surround ourselves with 
this quality of inner stillness, and allow that to be what informs our 
practice and our engagement with the outer world.

Practicing in this way with our many duties will affect the experiences 
we have back in our dwelling places---when sitting or walking in 
solitude---so that this alone time, too, can be imbued with the quality 
of inner stillness.

