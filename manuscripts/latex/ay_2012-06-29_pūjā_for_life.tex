\mychapter{Pūjā for Life}{Ajahn Yatiko}{June 2012}

There's a passage in the suttas in which the Buddha talks about how 
long people live. He says a person who lives a long life might live 100 
years or a bit more. It's interesting that people lived that long in 
the Buddha's day. According to the suttas, there were many monks who 
lived to be 80, and a few as old as 120.

Let's assume, for the sake of contemplation, that the longest lifespan 
since the Buddha's time was 100 years. And let's say that my previous 
lives were all human. If I lived to be 100 years in each life, it would 
have been twenty-five lives ago that the Buddha was alive. I think it 
is helpful to recollect this regularly and make the connection between 
our own lives and the Buddha's life. We can connect with this 
remarkable human being who walked the Earth and set rolling the 
unstoppable wheel of Dhamma. By regularly making this connection to the 
Buddha, we are consciously bringing to mind this remarkable human being 
who lived, walked, breathed, and had sense experiences exactly like we 
do. Whatever we're doing, we can connect it to the Buddha.

Suppose we were to forget about the past, as if the past didn't exist, 
and imagine that today is the day of our birth, a fresh start. This can 
inspire us to make a determination to live the rest of our lives as a 
\emph{pūjā}---as an expression of gratitude to the Buddha: \emph{I 
want my life to be as well lived, meaningful and beautiful as possible, 
as a way of honoring the Buddha}.

This morning when we were chanting, I was looking at these beautiful 
flowers offered to the Buddha and feeling a lot of gratitude arise for 
the Buddha and what he's done. For many Westerners, the energy of 
devotion is not easily accessible, but one way to get in touch with it 
is through this sense of gratitude. If we reflect on what the Buddha 
did---his awakening to this remarkable and uncompromising Dhamma, his 
decades of teaching and exemplifying the Dhamma to innumerable 
beings---we can get a sense of the incredible, meaningful life he led. 
It can bring up a lot of gratitude, which comes very close to the 
quality of devotion. Whether we're feeling gratitude or devotion, we 
can connect it to a sense of doing pūjā for the rest of our 
lives---for this one lifespan. We could give this life over to the 
Triple Gem and make the whole practice a pūjā to the Buddha. This is 
a way of getting outside of ourselves. Sometimes we get caught up with 
ourselves, thinking, \emph{Is my practice going well? I don't think my 
meditation is very deep.} We can let all that go when we're connected 
to the quality of pūjā.

We can take the next twenty, thirty or forty years---whatever we 
possibly have left---and make our lives an offering to the Buddha. We 
could do everything as an expression of our gratitude. It's not an 
empty gesture where we think, \emph{I haven't realized anything 
significant from my practice so I might as well give it to the Buddha.} 
It's not like that at all. It's a refined and beautiful practice in and 
of itself. We can turn it into a whole meditation; we can use our whole 
day simply reflecting on this. Suddenly, we might find ourselves 
happily cleaning our shrine, no longer holding the attitude that, 
\emph{I'll clean my shrine, but only because it's monastery etiquette.} 
It's more like it is an expression of our practice, because the highest 
pūjā we can offer is to be mindful of the present moment and 
contemplate our experience. That's really what pūjā is about. If we 
think in this way, pūjā becomes a very valuable practice and is quite 
nourishing to our hearts.

