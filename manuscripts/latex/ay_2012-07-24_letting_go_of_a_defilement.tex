\mychapter{Letting Go of a Defilement}{Ajahn Yatiko}{July 2012}

Ajahn Liem is known for working all day long and encouraging the 
monastic community to contribute a great deal of work and service. 
That's a big part of his teachings. But how well does it fit with our 
practice here? When we are away from our \emph{kutis}, many of us carry 
with us the desire to return there right away and continue our formal 
practice of sitting and walking meditation. Suppose a senior monk tells 
us that the monastery just received a very large shipment of books, and 
that we'll need to help out with that as soon as the meal ends until 
eight o'clock at night. What would be our internal response?

As monastics, we're to set our intention toward staying aware of the 
present moment and letting go of our attachments to the future and 
past. That's what it means to be mindful and use wisdom in daily life. 
We open to the present moment and abandon any notion of the future. In 
a monastic environment like this, we're encouraged to let the future 
take care of itself. We let go and do whatever is needed, whatever 
comes up. Wherever there's an opportunity to serve or work, we do that. 
We simply let go of the future. In a monastic environment, we can do 
that because we trust that there will be plenty of time for formal 
practice when the time is right.

When we are able let go of the past and future, we are letting go of 
our fears, fixed views, attachments, and desires---we are letting go of 
our defilements. This is a profound experience. We can study all we 
want about Buddhism and have incredible knowledge of the scriptures in 
Pāḷi. We can even write a book on mindfulness, but still not know 
what mindfulness is or how to let go of a basic defilement. These are 
some of the most important aspects of the path---to practice 
mindfulness and understand how to let go of defilements. That's much 
more valuable than knowing all the suttas.

So we use every moment throughout the day to practice letting go---when 
we get back to our kutis and are doing walking and sitting meditation, 
when we are eating, when we are working and doing service---whether 
things are going well or not. We do this with a sense that the future 
doesn't exist, by opening to the present moment and doing whatever 
comes to hand. If nothing comes to hand, then we do walking or sitting 
meditation, remaining in the present, moment by moment, letting go of 
everything else. That's how we can build a strong and stable foundation 
for the arising of insight and the development of the path.

