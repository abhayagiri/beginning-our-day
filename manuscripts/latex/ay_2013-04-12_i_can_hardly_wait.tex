\mychapter{I Can Hardly Wait}{Ajahn Yatiko}{April 2013}

I was at the Island Hermitage in Sri Lanka this past winter, where it 
was very beautiful. While there, I did a fair amount of contemplation 
around the topic of illness, which I found very useful to my practice. 
The Buddha recommends it as one of the subjects for recollection: 
\emph{I am of the nature to sicken; I have not gone beyond sickness.} 
While walking on my walking path I would bring it to mind: \emph{This 
is going to happen to me, there's no question about it, sickness is 
going to come upon me and, most importantly, pain is going to come upon 
me.} This sort of reflection isn't meant to be depressing or increase 
anxiety. It simply functions as a means of facing an existential truth.

Here at Abhayagiri, we might be walking up and down on our walking path 
on a beautiful day. It could be seventy-five degrees, the flowers are 
out, and we feel strong and bright. Times like these are especially 
good for reflecting that illness is coming, and that, at some point, we 
will experience great pain. And rather than treating the matter in an 
abstract way, it's best to be realistic, asking ourselves, \emph{What 
does it mean that this body is subject to pain? How am I going to 
prepare for that? What does it mean that pain is going to come upon 
this body?}

This kind of reflection is valuable for several reasons. When we 
reflect like this, it is possible for the superficial things that 
preoccupy the mind to fade away into the background. It feels quite 
liberating to be mindful that pain or death will be experienced. And 
that awareness itself fosters a sense of readiness for the time when 
pain eventually does come to the body. We can also use this practice 
when illness or pain is actually present within us.

If we use our intelligence and think too much, it can lead to 
intellectual endeavors that are not helpful. Truly, what we want to 
cultivate is mindfulness, that awareness that can help us recognize 
that, \emph{This body is not who I am. This body is subject to many 
experiences---feelings, pain of all sorts, and death.} If we are 
mindful, this reflection can be uplifting---we recognize that we are 
facing something that is the truth of the condition in which we are 
living. It helps us feel much more familiar and intimate with these 
realities and, therefore, less afraid of them.

When Luang Por Sumedho was once asked how he feels about death, he 
grinned and responded, ``I can hardly wait.'' I can relate to that. 
Once we truly accept death we can simply see it as a fascinating 
transformation, as a great change that will come upon us. That is all 
that is happening, nothing more. We are moving into the space of a 
total unknown, and it is going to be a radical transformation. That's 
exciting and interesting, to say the least. Admittedly, death is 
uncertain, so we don't want to slip into superstition or ungrounded 
confidence with regard to it. We simply want to be willing to open 
ourselves to it, to trust in the \emph{kamma} we have made as human 
beings, and, especially, as Buddhist practitioners. By trusting in that 
kamma, we can have the courage to open up to the uncertainty of what 
will come. That is the real adventure of the death process.

We are all involved in this project together. There is no project more 
important than deeply coming to terms with and understanding our 
condition as beings who are subject to birth, aging, illness and, 
ultimately, death. Everything else is far down the list.

