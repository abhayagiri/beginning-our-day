\mychapter{Enjoying That Enough-ness}{Luang Por Pasanno}{June 2012}

It's beneficial for our practice to pay attention to how we use the 
four requisites---robe cloth, food, shelter and medicine---and reflect 
on how we rely on people's generosity and kindness when we receive 
these offerings. Inner qualities that arise from reflecting in this way 
are contentment and gratitude, which are said to be a source of the 
highest blessings---\emph{maṅgala}. These blessings are not only 
beneficial for us, but beneficial for others as well. People become 
inspired and confident when they see we are using the requisites in a 
wholesome way.

While reading Paul Breiter's book, \emph{One Monk, Many Masters,} I was 
reminded of Ajahn Chah saying, ``I have never seen any rich people in 
the world. I see a lot of people, a lot of visitors, but I've never 
seen any rich people in the world. All I've seen is people who don't 
have enough.'' That's a powerful reflection for us. No matter what the 
external conditions of abundance are, it doesn't actually mean one is 
rich or wealthy, because true wealth is not measured by the material 
goods one accumulates. The richness of one's life---having true wealth 
within the limits of the human condition---comes from having the 
ability to be happy and peaceful within those conditions.

We are learning to find satisfaction with the qualities of contentment 
and gratitude, rather than constantly seeking something more, something 
different, or something other than what we have. That's the way the 
mind usually works. The untrained mind is constantly seeking something 
else, whether it's in the material realm, in the realm of views and 
opinions or even in the realm of meditative states. It's constantly 
looking for something else, not content with what it has or what it's 
experiencing. The problem with the human condition is this constant 
seeking and, of course, not really finding. So we need to learn how to 
be someone who has enough, and to be someone who enjoys that 
enough-ness.

