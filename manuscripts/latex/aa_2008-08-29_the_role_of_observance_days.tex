\mychapter{The Role of Observance Days}{Ajahn Amaro}{August 2008}

The purpose of our weekly Observance Day is to put our usual daily 
tasks down and focus on the precepts and the formal spiritual qualities 
of our life. It's a day of recollecting and observing, of remembering 
the Dhamma and our original motivation for being here at the monastery. 
It's a time to remember the possibility we have as human beings to let 
go of all confusion, delusion, aversion, greed, and self-centeredness. 
It's a time for renewing our motivation and to begin again fresh.

As monastics, we shave our heads, taking the hair back to the root, 
back to the source, to begin again. There's that quality of tidying 
up---cleaning the shrine room, cleaning the kitchen, squaring things 
away, renewing our precepts. On a practical and symbolic level we're 
keeping the monastery tidy and looking after the things that have been 
offered for our use. On the internal level, we're remembering our 
priorities and helping to clarify the central principles of our lives.

The activities of the Observance Day---taking the precepts, shaving the 
head, having the all-night vigil, and putting aside the work 
routine---are all geared toward reinforcing the reason we're here at 
the monastery. We're not here to construct \emph{kuṭis}, post 
pictures on the website, cook meals, or any of the 10,000 tasks that 
occupy our attention. The whole point of this place existing, this 
gathering of human beings on this particular patch of hillside, is to 
realize the Dhamma---to let go of greed, hatred, and delusion. That's 
the reason we're here. Our preparations for the Observance Day and the 
day itself are all meant to help us to remember, \emph{Oh, right, 
that's what it's all for, of course. Why was I forgetting that?} That's 
the purpose of all the construction projects and committees. That's why 
we take care of these duties and keep the place clean and tidy. That's 
what it's for---the realization of Dhamma.

