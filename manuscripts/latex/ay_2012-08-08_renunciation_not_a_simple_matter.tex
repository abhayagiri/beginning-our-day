\mychapter{Renunciation: Not a Simple Matter}{Ajahn Yatiko}{August 2012}

Renunciation is a lovely reflection to bring up from time to time. 
Sometimes people talk about how they've renounced something they were 
strongly attached to. They say that, having renounced it, they're now 
done with it, once and for all. However, it's rarely as simple as that. 
Being attached to something means we don't want to let it go. Even so, 
if we recognize that it is harmful, a desire may arise to let it go 
after all, but we still have a difficult time doing so. Attachment and 
renunciation are a pair, and their relationship can be complicated, so 
it's good to reflect on them both equally. When we're attached to 
something, simply making a decision to renounce it doesn't mean we're 
done. It's a process.

Letting go from the heart takes passion, questing, searching, 
determination, time, and cultivating the right \emph{kamma}. 
Renunciation is something that requires commitment, time, and patience. 
It takes every opportunity we have to incline the mind toward letting 
things go. But we can start with small steps, letting go of little 
problems and complications. In doing so we find that, in time, we're 
making headway on the bigger issues we struggle with; letting go of the 
smaller problems and attachments eventually leads to letting go of the 
bigger ones. So we learn to let go where we can, renounce where we can, 
and, as a result, find peace where we can.

