\mychapter{Caring for Everything We Use}{Luang Por Pasanno}{June 2005}

Throughout the day it is helpful for us to recollect that we are a 
community of alms mendicants. We rely on what is offered to us as gifts 
of goodwill: robes, alms food, shelter and medicine---the four 
requisites. Traditionally, in Thai monasteries the monastic community 
brings the requisites to mind as part of its formal morning and evening 
chanting. The chants encourage the monastics to ask themselves, 
\emph{Did I use the requisites skillfully? Was I heedful when I used 
them? Do I understand their true purpose?} We too should reflect in 
this way. It's easy to expect that everything will be there for us and 
that everything will be of good quality. But as alms mendicants, the 
emphasis is on contentment with what we have and being circumspect with 
what we are using.

There's a story about a monk who was cleaning Ajahn Maha Boowa's kuti 
and threw away two used matches that were on the altar. When the Ajahn 
returned to his room, he asked his attendant, ``What happened to those 
matches? They weren't used up yet!'' It was Ajahn Maha Boowa's habit to 
use partially burnt matches for transferring the flame from one candle 
to another candle or to other objects; he wouldn't dispose of a match 
until it was completely burnt out. This example can inspire us to 
develop a sense of using things fully. The focus is not on our 
convenience, but on recollecting that these things are offerings; they 
have value, and we shouldn't waste them.

Another aspect of our relationship with material things has to do with 
respect and compassion for others in the community. It's really basic: 
make sure that people don't have to pick up or clean up after you. 
Return things to their proper place. After using a tool, put it back 
where it belongs, rather than leaving it out for someone else to put 
away. In the kitchen, putting a used dish in the sink doesn't magically 
make it clean and placed back in the cupboard. A real human being has 
to do that. We show respect and compassion for others by being 
considerate.

Beyond that, take responsibility for setting things right, even if they 
are not your assigned responsibility. If you see that something is out 
of place or hasn't been done, don't just walk by and leave it for 
someone else. Take the responsibility and initiative to be helpful. If 
everybody learns to take responsibility in this way, then it's not a 
burden for anyone.

Whether we are in the monastery or elsewhere, we rely on the requisites 
and other material items for our daily existence. Looking after these 
things is just an aspect of mindfulness---attending to what we're doing 
and what needs to be done in the present moment. This is not so mundane 
that we don't need to think about it. We learn to incorporate the way 
we care for material requisites into our day-to-day practice of 
mindfulness and cultivation of skillful qualities. The material realm 
will then become a more harmonious and pleasant place in which to live.

