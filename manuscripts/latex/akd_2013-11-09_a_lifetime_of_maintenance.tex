\mychapter{A Lifetime of Maintenance}{Ajahn Karuṇadhammo}{November 
2013}

I was looking around the monastery thinking about the tasks that we 
have today. I'm going to be working with a couple of other people 
checking out our septic system, seeing what's going in and whether it 
is working properly. We're also cleaning up different areas around the 
cloister. As I was looking down at the cloister construction site, I 
saw dug-out earth, cement, and rebar forming the foundation of what in 
the future will be our reception hall building.

I was reflecting on the fact that people don't often think about, see, 
or experience the underside of the monastery---the buildings, 
operations, functions, and infrastructure. When we look at a building 
that's completely built, like the monks utility building, we don't 
automatically visualize or see the foundation, the earthwork, or the 
dug-out area. We see the building itself that's structured on top of 
it. When we're eating, washing dishes, and putting our bowls away, we 
don't usually visualize the food that we eat eventually coming out of 
our bodies and going into the septic system. There's the starting of 
different projects and repairs, but we don't usually pay attention to 
them once they've been completed. In thinking about the beginnings and 
endings of these kinds of processes, what is it that we choose to 
experience? We view the world from what's on the surface and what is 
most pleasing to the eye. We selectively attune to that aspect of an 
experience because that's what we want to see. It's naturally how we 
live our lives.

For the most part, we do what we need to do to get through the day, but 
we don't pay attention to the aspects of existence that are unpleasant, 
unexciting, unattractive, tedious, routine, or mundane. We're usually 
working and living our lives in a way that's trying to get through the 
unpleasant and boring events, while at the same time looking forward to 
the pleasant bits---``the good stuff.''

As we are practicing, it can be helpful to pay attention to all of the 
routine maintenance and tedious tasks that we engage in, from the 
beginning of the day to the end of the day. Whether it's the areas 
where we live, the physical surroundings, or whether it's our bodies, 
or our minds, or the relationships we have with people---we are 
consistently maintaining these structures. We tend to miss or skip over 
that whole aspect of our daily lives---the routine maintenance. If we 
miss it, then it's as if we are walking through a cloud. There is the 
intention of experiencing the one or two moments that are exceptionally 
interesting, exciting or pleasurable, but the rest of it is just a 
cloud. With exception to the rare high and low spots, 90 percent of our 
lives are the routine ``in between'' experiences. But if we can bring 
mindfulness to the routine and mundane aspects of the day and not be so 
caught up in doing something to get somewhere exciting in the future, 
then there's a lot more peace and ease of mind and less desire momentum 
propelling us. We can settle back into a recognition of simply being 
here right now with whatever it is. And this is the most pleasant and 
peaceful abiding we can have.

