\mychapter{Skills for Letting Go}{Ajahn Ñāṇiko}{December 2013}

In our practice we are normally working with the core defilements of 
greed, anger, and delusion. Often when these defilements arise, the way 
we deal with them is through restraint. When we restrain the 
defilements it feels different than actually letting them go. With 
restraint, we continue to experience the defilement, it's an 
undercurrent in the mind. By contrast, when we genuinely let something 
go, there is a feeling of being completely refreshed and replenished. 
There can be an experience in both the body and the mind of being 
filled with a clear, cool, pure substance. Restraint is important as 
well, however. When there is irritation or anger in the mind, or a 
nagging desire that won't go away, we can first use restraint as an 
antidote by applying, for example, \emph{mettā} or \emph{asubha} 
practices. The mind will then loosen its grip on these defilements and 
let go.

Sometimes we may think it's impossible for the mind to let go, but it 
is important that we not think this way. The mind will let go in its 
own time according to its conditioning and its \emph{kamma}. In 
particular, letting go depends on the kamma of our skillful effort in 
the practice. We don't need to figure out when the mind will let go, we 
just keep practicing. Walking, standing, sitting, or lying down, we 
keep the Dhamma teachings in mind, moving forward with faith that true 
letting go is possible. It is not helpful to tell ourselves, \emph{I 
don't have enough pāramī to let go}. Letting go is a skill. Everyone 
has to put forth effort, especially when we are just beginning.

Learning how to fail is another important part of the letting-go 
practice. About a month ago I was trying to fix an old planer. It was 
difficult to get the blade removed. I was in the shop with Doug and I 
told him, ``I'm prepared to fail at this, but I'll give it a try.'' 
There was one screw with a stripped head and after awhile I finally 
managed to remove it. I was so excited when I took it out that I forgot 
to take out the other screw that didn't have any problems. I started 
prying the blade out and ended up breaking the planer entirely. 
Afterward I reviewed my mind and I thought, \emph{I told Doug that I 
was prepared to fail at this, but actually I wasn't prepared to fail at 
all.} I spent the next three days mired in \emph{dukkha}, full of 
reproach about my foolish mistake. It took quite a while to let that 
one go. So it's important to learn how to fail without taking it 
personally. After all, taking things personally is the very thing we're 
learning to let go of.

