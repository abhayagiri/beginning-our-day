\mychapter{Luang Por Chah's Approach}{Luang Por Pasanno}{July 2011}

Student: I've heard that in the beginning, Luang Por Chah used to lock 
the doors of the Dhamma hall during the all-night sits.

Luang Por Pasanno: I wasn't there back then. But he did have us sit in 
meditation right after the meal in all three of our robes---in the hot 
season! Over time, however, he came to rely more on wisdom than brute 
force.

S: What caused him to make this change?

LPP: Well, he learned that it was better to create the right 
environment for practice than to try to turn people who didn't want to 
practice into practitioners. He had a simile. He said: ``If you created 
a nice pasture and cows came in, they would eat the grass. If animals 
went into the pasture and didn't eat the grass, then you knew they 
weren't cows.'' That was his way of saying that if you create a good 
place for practice, real practitioners will practice. Other people 
won't, and there's no point in trying to change them.

S: Did he ever provide similes indicating that people can improve?

LPP: Every simile has a specific point, and it doesn't work outside of 
that. Ajahn Chah definitely encouraged people and told them they could 
do it if they tried. They had the teachings, and they were in a good 
environment. So if they tried, they could succeed.

S: I've heard that he often encouraged people to stay in robes even 
when they didn't want to.

LPP: That's right. For instance, there was Ajahn Toon. Every year after 
the Rains Retreat he would ask Ajahn Chah if he could leave the 
training, but Ajahn Chah would refuse. This went on for five years. 
After every Rains Retreat, just like clockwork, there would be Ajahn 
Toon with his offering of flowers and incense respectfully asking to 
leave the Saṅgha. Ajahn Chah would always talk him out of it or 
sometimes just get up and walk away. Ajahn Toon ended up staying a 
monk, and now he's a really good teacher in our lineage.

