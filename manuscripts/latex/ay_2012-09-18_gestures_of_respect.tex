\mychapter{Gestures of Respect}{Ajahn Yatiko}{September 2012}

Yesterday I clearly saw a defilement in myself that I'd like to speak 
about and share with you, so you can follow along with my process and 
possibly make use of it yourselves.

In the morning, I was putting out seats for \emph{pūjā} on the 
ordination platform. For a long time I'd been resistant to the idea of 
putting out a seat for Luang Por Pasanno when he's away, as he is now. 
There seems to be an ambiguity about whether we put out a seat for him 
when he's gone. Sometimes we do, sometimes we don't. There's no 
established etiquette for this.

I had the thought, \emph{He's not here, what's the point?} In the past, 
whenever someone put a seat out for him, something in my heart would 
roll its figurative eyes. But yesterday morning, as this reaction was 
coming up, I recognized what an unpleasant mind state it was. I said to 
myself, \emph{Wait a minute. Putting out a seat is an act of respect 
for Luang Por. Luang Por is our teacher, who has given us so much and 
helped us in so many ways that we know and don't know. This act reminds 
us of his presence in the monastery as the leader of the community.} So 
then, when I did put out a seat for him, it came with a nice feeling of 
relief to know that I could overcome this defilement of which I was 
previously unaware.

It reminded me of Ajahn Sumedho's story about washing Ajahn Chah's 
feet. When Ajahn Chah would come back from almsround, twenty monks 
would come running out to wash his feet. Ajahn Sumedho would roll his 
eyes and think how stupid it looked for twenty grown men to be washing 
one man's feet. He thought this was ridiculous and said to himself, 
\emph{I'm never going to do that.} This happened day after day, until 
he eventually realized that having this thought was causing him 
suffering. The next day there were twenty-one monks washing Ajahn 
Chah's feet; Ajahn Sumedho enjoyed doing it and felt really good about 
it.

By bringing this up, I'm not saying that putting out a seat should now 
be established as monastery etiquette; I don't feel so strongly about 
it, and if someone does not set out a seat, I think that's okay. 
Rather, I'm trying to encourage us all to reflect on our attitudes 
about respect and to question why we feel the way we do. We can simply 
ask ourselves whether these attitudes are suffering or not suffering.

From that we might sense the relief that can come from choosing an 
attitude that leads to a bright mind state instead of an unpleasant 
one. Showing respect is a nice thing; it brings up a nice feeling. It 
doesn't matter if people criticize us or think showing respect is 
stupid. If we do something that feels right, that feels kind, then it 
is a good thing to do. At the same time, it is not as if we are 
following a rule. It comes from free choice, and that's what makes it 
beautiful. If this were a rule and we put out a seat with a sense that 
this is what we were supposed to do, that wouldn't feel very special. 
It needs to come from the heart. And when it does come from the heart 
there is an attitude we can have of wanting to use whatever 
opportunities are available to show respect and be reminded of 
something good, something uplifting.

We can use that attitude in our practice no matter what situation we 
find ourselves in. We can do something good, something kind, be 
forgiving, patient, thoughtful, and helpful to each other. This is the 
foundation of group harmony. We have a harmonious community of both 
laypeople and monastics, and that harmony has its foundation in mutual 
respect.

