\mychapter{Paying Attention to Details}{Luang Por Pasanno}{May 2012}

When bringing Dhamma practice into our daily lives, it's important that 
we pay attention to the ordinary little details around us. If we hold 
to a broad, nebulous ``just be mindful'' attitude, we're unlikely to be 
clear about what we're doing. So it's important to keep ourselves from 
overlooking things, especially while living in community as we do here 
at the monastery.

Through our attention to ordinary details, we cultivate an attitude of 
mindfulness that's sharp and connected with what's happening around 
us---an attitude that is central to our training. Over time, this 
attitude permeates the way we relate to our minds and our experiences. 
And that's critical, because our defilements don't advertise themselves 
by holding up big signs that say, ``Greed! Hatred! Delusion!'' So, we 
need to look closely if we're to discern and understand the nuances of 
the attachment and delusion we experience. If we haven't cultivated the 
habit of paying attention to details---both external and internal---if 
we don't put forth the extra effort required for that, then we'll miss 
many significant aspects of our practice.

The Buddha pointed to \emph{yoniso manasikāra} as a vital part of the 
path. It refers to skillful and wise reflection and close attention to 
the root of things. This quality doesn't pop up by itself like a 
mushroom in the fall; it has to be cultivated. In our ordinary 
day-to-day activities, we can support this cultivation by, again, 
paying attention to details. We do this, for instance, by making a 
continuous effort to notice when any little thing needs doing, such as 
returning a tool to its proper place, and then doing what's called for, 
even if it's not ``our job.'' When everyone pays attention and takes 
responsibility like that, the monastery functions beautifully.

The Buddha based this entire path of liberation upon the experience of 
dissatisfaction, discomfort, stress, suffering---\emph{dukkha}. Paying 
attention to details is a doorway though which we can learn what leads 
to dukkha and what doesn't. While it may seem like a small thing, this 
attention is essential if we're to live our lives skillfully and in a 
way that opens us to the possibility of true peace and freedom.

