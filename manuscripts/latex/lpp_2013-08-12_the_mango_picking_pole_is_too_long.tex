\mychapter{The Mango Picking Pole Is Too Long}{Luang Por 
Pasanno}{August 2013}

Sometimes we feel compelled to be think, figure things out, analyze 
what we're doing, marshal all the logical reasons for directing our 
practice in a particular way. But in the end, it's simply busyness. And 
an important part of our practice is developing skills that prevent the 
mind from being trapped in busyness like that.

Ajahn Chah used the example of picking mangoes. In Thailand, mangoes 
are picked by using a long bamboo pole with a little basket at the end. 
Usually the mangos are on branches fifteen or twenty feet from the 
ground. Ajahn Chah would say, ``If our mango pole is twenty-five or 
thirty feet, it is much too long for the task. When considering what 
things could advance our practice, we might think having a Ph.D. or 
being really smart would help. But that may be like having too long a 
pole: too much thinking, too much intelligence, too much of figuring 
things out. It may not be what is needed.''

We do need to use reflection and investigation to understand our 
experiences. But that reflection and investigation needs to be 
appropriate and balanced. In the same way that picking mangoes requires 
a pole that's been cut to the proper length, we need to tailor our 
thought processes to the task at hand. And the task at hand is 
understanding: \emph{How do I not suffer? How do I not create problems? 
How do I not increase my suffering and confusion?} Those questions are 
more important than, \emph{Do I have all the information I need? Should 
I think more about my problem? What logic can I use to figure it out?} 
This can be mental overkill---more that what the Buddha would have us 
do.

Instead, we can learn to maintain a balance in our use of thinking and 
logical analysis. We do this by paying close attention to what it is 
that is useful in helping us prevent and ease suffering and confusion. 
We can learn to apply thought at the right time and in the right 
amount, and how to let go of thought when it's not necessary.

