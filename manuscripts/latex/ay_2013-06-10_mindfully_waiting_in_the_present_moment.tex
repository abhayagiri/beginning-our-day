\mychaptertoc{Mindfully Waiting in the Present Moment}
{Mindfully Waiting in\\the Present Moment}
{Ajahn Yatiko}{June 2013}

Sitting here in silence, some might say, feels like a waste of time. 
Sitting here waiting … waiting for something to be said. It could be 
a waste of time if we are sitting here waiting mindlessly. But it is 
not a waste of time if there is mindfulness present and an awareness of 
the present moment.

Usually, at this time of day, there is a sense of anticipation as the 
work period draws near. There are thoughts about what we are going to 
do and what we don't want to do, all of which are influenced by the 
different attitudes that we have toward work. So this silence 
beforehand can sometimes bring up feelings of anxiety, and we might 
think, \emph{Say something, it doesn't matter what it is, just say 
anything to fill the space.} Or the mind can slip into an animal state 
during which we zone out with eyes open, not really looking at 
anything. We are here, but not present. Our behavior is almost cow-like.

During this silence many different feelings and thoughts can arise, and 
we usually identify with them. We are like a fish caught with a hook 
and line that is simply pulling us along. But if mindfulness is 
present, we can see the hook and say to ourselves, \emph{I don't trust 
that.} We have the mindfulness to recognize the presence of experience, 
the presence of feeling and thought, and we can think, \emph{Wait a 
minute. This compulsion to attach and identify with the content of my 
consciousness is just like the bait on the end of that hook; it's 
trying to get me to bite, and once I bite I can be caught up in it for 
days, months, or even a lifetime.} With mindfulness, even if we get 
caught, there's no need to despair. When we recognize that we're 
caught, it is much easier to remove ourselves from that experience than 
it is to remove a fish from a hook. A large part of the battle is 
already won because that recognition brings us closer to the present 
moment.

The present moment is the place where we can recognize: there is the 
content of experience; there is something in the content that we find 
appealing---something that tempts us to make it our own; and there is a 
desire compelling us to grab onto that content. When we're connected to 
our present-moment experience in this way, there is the wisdom that 
tells us, \emph{I know this process of content-appeal-desire-compulsion 
is not to be trusted. I am going to step back from that and let this 
more spacious place of awareness and recognition establish itself. Then 
I can proceed from that place, rather than from the place of 
compulsion.}

So whenever we experience these empty times of waiting for something to 
happen, we can use them as opportunities to investigate and reflect on 
the content of consciousness. This is hard work. It is the work of 
spiritual life and one of the main activities we do with spiritual 
practice. It is not like having a livelihood in which we are given a 
clear task, day after day---a livelihood where we might have the 
attitude that, \emph{As long as I do this I don't have to worry about 
anything. I just do my job, go home, go to bed, and everything is okay. 
I don't have to give it much thought.} This is not how it is with 
spiritual life, nor with spiritual practice. We are not here merely to 
have a place to stay and food to eat---that would be a terrible 
motivation. The motivation to be here has to be for something noble, 
something that involves the dignity of work and the dignity of silence. 
And so whenever we sit in silence, whether in this room or someplace 
else, it's not a time for mindless waiting. It's a time for work.

