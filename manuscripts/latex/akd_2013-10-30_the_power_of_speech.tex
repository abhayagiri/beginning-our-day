\mychapter{The Power of Speech}{Ajahn Karuṇadhammo}{October 2013}

I was thinking about an aspect of our practice, right speech, and 
reflecting on the power of speech to affect us all. In some ways it 
affects us even more than bodily actions, which we think of as the 
coarsest way of acting out a mental condition. Speech can have an even 
broader effect because the number of people who can hear spoken words 
is greater than the number of people who can observe somebody's bodily 
action. Knowing how powerful it is, we need to put a great amount of 
care and attention into our speech because it can impact each one of us 
for better or for worse, depending on what is said. I believe that many 
of us here have a keen intellect with well-developed mental faculties, 
as well as the ability to rationalize and this often results in having 
some fairly strong views and opinions. I see that in myself. If we 
combine that with the ability to articulate, to put those ideas, 
thoughts, and intellectual capacities into speech, the outcome is that 
one has an extremely powerful tool. It is helpful for us to recognize 
that power and the possibility of its ramifications.

Thinking historically, we see how one person in this past century with 
a strong intellect and a powerful ability to articulate influenced an 
entire culture to the point that even though he may not have personally 
taken many lives himself, he was responsible for the destruction of six 
million Jews and millions of other people. He also justified horrible 
kinds of experimentation on human beings. All of this took place simply 
because of the power of his intellect and speech.

Conversely, speech can be used on a mass scale for good. We realize 
this when we look at our own teacher, the Buddha, who was able to use 
speech for the benefit of many, many people and 2600 years later, this 
ability to communicate and use speech effectively continues to have a 
strong influence worldwide. For better or for worse, speech can be an 
exceptionally powerful tool and here in our community, we need to be 
aware that it can have a potent and long-lasting effect.

If we have a keen ability to think, rationalize, and analyze, as well 
as the ability to articulate, it does not mean that our speech is 
always free from defilement. With all of those qualities the underlying 
negative tendencies can still be expressed. Sometimes we may feel a 
need to prove something or to be correct, or feel that something must 
be done in a certain way or try to position ourselves within the 
community in a way that is promoting our views. Those kinds of 
tendencies are an indication that we are not completely free of basic 
animal instincts. Our brains may have well-developed cortices and 
speech centers but we also have the lower functioning parts that are 
influenced by basic unwholesome tendencies toward territory, ownership, 
and a sense of ``me and mine.'' These tendencies are still part of our 
cultural inheritance, our kamma, and our biology. No matter how skilled 
we are with an ability to speak, no matter how clever we are in the way 
we think and analyze, these tendencies still come bubbling up.

We need to be aware of that and use our speech in a way that is 
conducive to harmony, because we also have the ability to harmonize. 
Think about how powerful it can be when there is a discussion here 
centered on Dhamma or when somebody tries to articulate some harmonious 
viewpoint, something that contributes to the welfare and well-being of 
the community, or speech that is backed up with good intentions toward 
kindness, support, clarification, and truth. We can put our focus there 
and use the skills we have learned from practicing the Dhamma to check 
our speech when kamma starts to bubble up. If a conversation is heading 
in the wrong direction, then one needs to be able to stop it, change 
direction, or withdraw if necessary. We can then move toward 
establishing speech that is not only well expressed but speech that is 
also grounded in kindness and an interest in supporting the welfare of 
the whole community.

