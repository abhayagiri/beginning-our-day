\mychapter{The Right Balance of Effort}{Luang Por Pasanno}{June 2005}

Since today is a Community Work Day, it's good to reflect a bit on 
effort. When we're working, how do we sustain our effort? How do we 
keep the kind of steadiness and pace that allows us to put forth effort 
without wearing ourselves out?

In meditation, we have a chance to notice how difficulties can arise 
when we're focused on trying to get or achieve something in our 
practice, when there's an agitated energy of doing. Or we can notice a 
holding back of effort and how the mind can sink with that. With these 
observations we better understand how effort works, and we can apply 
this understanding outside of meditation as well. When we're working or 
doing any kind of chore, it's helpful to consider how to apply effort 
so there's a steadiness, not a holding back or sinking, and not an 
agitated energy of doing.

We can also examine the way we limit ourselves through perceptions of 
what we think we can do. When we limit ourselves with perceptions, 
ideas, or fears, we're not able to put forth much effort. And contrary 
to these perceptions and fears, if our effort is balanced, then the 
more effort we put into something, the more energy we get back. There's 
a nice feedback loop of supporting energy that comes from putting forth 
the right kind of effort. When we learn to use effort in an appropriate 
way, we find ourselves buoyed up by the energy coming from that.

Being able to apply effort with steadiness and balance is an important 
skill. To develop this skill, we need to experiment with and look 
closely at how we apply effort and the results of applying effort. When 
we're able to apply ourselves in a steady, balanced way, we begin to 
get a feel for what the Buddha meant by ``unremitting energy, 
unwavering effort.'' This is a factor the Buddha said is crucial for 
liberation.

