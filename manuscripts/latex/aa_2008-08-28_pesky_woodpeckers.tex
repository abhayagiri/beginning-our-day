\mychapter{Pesky Woodpeckers}{Ajahn Amaro}{August 2008}

When we put up a building, we think it's our building, our \emph{sala} 
or our \emph{kuṭi}, but no one informs the geckos, the lizards, and 
the other creatures in the forest. They think it's their place too. The 
woodpeckers were around long before the humans showed up in this 
valley. This is the season when woodpeckers start to gather acorns and 
drill holes in the buildings, making little cupboards to stash acorns 
in. When people come along and put up a building, to a woodpecker, it's 
just a very odd-shaped oak tree. Before our beautiful new office 
building was even completed, the woodpeckers drilled a nice sequence of 
holes in this wonderful cedar trim all the way around the building. To 
a woodpecker, the trim is just a very nice, flat surface for stashing 
acorns with no bark or branches to work around. And we think, 
\emph{Pesky woodpeckers; they shouldn't be ruining our building}. But 
in reality, we've put up this attractive opportunity in their zone, and 
getting upset and irritated about it is simply proliferation---a story 
we've added on. As Luang Por Sumedho would often ask, ``Who is being 
foolish?'' Is it us or the woodpecker? In Thailand the rafters in the 
Dhamma Hall are great territory for wandering geckos. If a monk gets 
annoyed because he chooses to sit under the rafters and the geckos 
defecate on him, whose problem is that?

Our habit is to look at the world from a self-centered perspective: 
\emph{My preferences, My priorities.} We look at our body in that same 
way: \emph{My body has a right to be totally mobile, completely 
comfortable, pain free, and in an environment with the most desirable 
temperature at all times.} When that view gets intruded upon---when 
there's an infection, if the body is afflicted with a poison oak rash, 
or if it is too hot or too cold---then we become reactive and that same 
self-righteousness arises: \emph{This is an intrusion upon my space, my 
time, my convenience. I haven't got time for this; it shouldn't be like 
this.} If we believe those perceptions, then we endlessly create more 
\emph{dukkha} for ourselves---this sense of dis-ease and 
dissatisfaction---because it's just the voice of self view, isn't it?

We don't tend to look at our body as a food source, but it is. It's a 
collection of organic matter. I never really understood this until I 
was living in Thailand. At first I was resentful of mosquitoes coming 
in and feeding on me, but at some point I realized that the body is a 
large, pungent magnet, like a big supermarket sign advertising a food 
source where the doors are always open. We don't think of ourselves 
like that. We think, \emph{These darn mosquitoes are annoying }me!* 
These flies are landing on* me! But if we see that this body is simply 
a pile of heat-producing organic matter with interesting fumes, it 
makes sense that it would draw insects to it. If we shift the 
perspective a little, and drop that sense of self view, then we can 
similarly drop the burden of resentment, the feeling that life is being 
unfair.

That's the essence of the Middle Way---shifting from a self-centered 
view to the view that our bodies, the material world around us, and our 
minds are all natural systems---not ``me,'' not ``mine.'' And when we 
make that shift, when we adopt a view centered on Dhamma, rather than 
on the illusion of an independent self, then the world changes quite 
dramatically.

So we can take care of buildings at the monastery, protect the body 
from insects, while at the same time not create stress or irritation or 
the sense of burden that comes with self view. If we understand that 
everything is part of the natural order, then ``invasion'' from the 
outside is impossible, because everything occupies the same territory, 
it's all of the same order. With this understanding, the heart relates 
to it all in a radically different way.

