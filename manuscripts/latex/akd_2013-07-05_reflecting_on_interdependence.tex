\mychapter{Reflecting on Interdependence}{Ajahn Karuṇadhammo}{July 
2013}

In recent years there's been a modern Western interpretation of 
dependent co-arising that's derived from an explanation of the 
interdependence in the world, with the people in it being 
interconnected in a vast web of cause-and-effect relationships and 
experience---``It's all connected'' as people like to say. There's a 
belief that there's no type of action or activity in the world that 
doesn't have some sort of effect on the whole of existence, all action 
and interaction. This isn't a very accurate representation of what the 
Buddha's teaching on dependent origination is all about. It's actually 
pretty far from what the Buddha taught. Dependent origination is a 
teaching about how ignorance conditions the arising of suffering and 
all the various factors involved as well as the cessation of that 
entire process.

Nevertheless, the idea of interdependence is something that can be 
helpful to contemplate because even though it may not be valid in terms 
of dependent origination, it has a truth to it. We're very much 
affected by each other, dependent on each other, and influence each 
other, and we're inexplicably woven together through the various forces 
of kamma in our cyclical existence in \emph{saṃsāra}. With this, 
there are all the various ways that kamma works itself out that we 
don't really understand and can't possibly comprehend because it's so 
complicated. We find ourselves weaving through many, many lifetimes, 
receiving the results of our past actions, being involved with each 
other over and over again, and bound together in our commonality as 
beings coursing through saṃsāra.

We don't know exactly how that process works, but it is possible to 
understand that each of our rebirths in saṃsāra depends on our 
relationships with other people and, for better or for worse, how we 
respond to situations and the qualities we develop in relationship to 
each other. Due to causes and conditions, skillful actions and 
intentions that have been put in place by us in our past lives, by some 
fortunate set of incredible circumstances---some might call it a 
miracle---we find ourselves existing in the same space and time, right 
here and now, practicing with each other in this vast web of existence. 
So we need to ask ourselves, \emph{How well are we spending our time?}

I think it's important to reflect on that because it points out how 
much we need to rise up to this circumstance we find ourselves in and 
take responsibility for what we are doing right now, acknowledging what 
it's taken for us to get here and not wasting this precious 
opportunity. It's so unusual for us to all be here together with a 
strong interest in practicing the Dhamma---it's not a widespread 
inclination that's happening in the world. Perhaps there are small 
pockets of it here or there but by and large it's an incredibly rare 
opportunity. We can do our best to take full advantage of this 
situation because this life is short and we don't know exactly where 
we're going to end up the next time around. We keep the momentum going 
by cultivating the wholesome and skillful qualities we want to bring 
with us so these qualities carry on into the future in case we don't 
finish our work in this lifetime.

All of the wholesome intentions we cultivate now will condition what 
happens for us the next time around and, most importantly, will 
condition the quality of our lives right here and right now. We are 
working on letting go of unskillful tendencies, the aversion, greed, 
self-interest, and selfishness and cultivating qualities of virtue and 
generosity, to make a commitment to ending the cycle of suffering. 
That's how we can increase the potential of being in association with 
like-minded people in lifetimes to come. This is the meaning of true 
interdependence.

