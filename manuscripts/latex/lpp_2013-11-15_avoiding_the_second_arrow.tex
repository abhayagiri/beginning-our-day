\mychapter{Avoiding the Second Arrow}{Luang Por Pasanno}{November 2013}

The obvious thing for me right now is carrying this cold around: a 
thick head and not much energy. It's helpful to reflect on the reality 
of illness or discomfort that is natural and happens all the time. What 
comes to mind is the Buddha's discourse on the two arrows. Being struck 
by one arrow is painful and being struck by a second is painful as 
well. In terms of feeling, because each of us has a body, it is quite 
natural that we experience unpleasant sensations. And due to having a 
mind, it is also natural that we are sensitive to the sense contacts we 
experience. These are qualities of the first arrow. Becoming averse, 
worried or anxious about unpleasant feeling---or planning and 
proliferating about how to escape from sense contact---is the second 
arrow.

We can't avoid the first arrow, but we can avoid the second arrow. 
Because we have physical bodies, because we exist, our bodies naturally 
get ill and our minds change. There are always going to be certain 
elements of unpleasant feelings. Even the Buddha himself experienced 
unpleasant feelings and had physical ailments, like a bad back. There 
are places in the suttas where the Buddha turns over his teaching role 
to either Ānanda or Sāriputta because his back is too sore to 
continue sitting and teaching.

We can recognize and be attentive to the first arrow and not turn that 
arrow into something that then creates more complication, difficulty, 
and pain. We can reflect on that as we go about the day, asking 
ourselves, \emph{What are my reactions to unpleasant feelings? What are 
my habits and tendencies?} We can develop mindfulness and discernment 
to receive the first arrow skillfully and not look for ways to be 
struck by the second, third, or fourth arrow.

