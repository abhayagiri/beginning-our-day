\mychapter{Embodying Experience}{Luang Por Pasanno}{April 2011}

This morning I was recalling how James Joyce described one of his 
characters: ``Mr. Duffy lived a short distance from his body.'' In 
Dhamma practice, we're learning how to inhabit our bodies and our 
experiences of having a body. It's easy for us to become dissociated or 
distant. It's as if bodily experience becomes a projection of the mind, 
or we find ourselves treating the experience of having a body as if it 
were an abstract concept. This doesn't end up being very fruitful for 
us.

In his seminal discourse on mindfulness of breathing, the Buddha uses 
the verb \emph{paṭisaṃvedeti}, which means to \emph{fully 
experience} the body, as we breathe in, as we breathe out; to 
\emph{fully experience} feelings, as we breathe in, as we breathe out. 
It's a quality of inhabiting what is truly going on for us. If we stay 
even a slight distance from what we experience, then we don't see it 
clearly. And when we don't experience things clearly, we tend to create 
problems for ourselves.

That is why, when the Buddha taught the Four Noble Truths, he also 
taught a specific duty or response for each truth. For the Noble Truth 
of \emph{dukkha}---the experience of suffering, discontent, 
dissatisfaction, stress–he instructed that this has to be fully 
known. It's not until we fully know the experience of dukkha that we 
can begin to understand how we're contributing to it, getting entangled 
in it and a part of its cause. The Second Truth---the cause of 
dukkha---is to be relinquished; the Third Truth---the cessation of 
dukkha---is to be realized; and the Fourth Truth---the path or practice 
leading to the cessation of dukkha---is to be developed. But it all 
begins with the First Truth: fully knowing.

Over and over again, the Buddha encourages us to develop our practice 
by fully engaging with and fully knowing what we encounter and 
experience. This is the very foundation on which the practice rests.

