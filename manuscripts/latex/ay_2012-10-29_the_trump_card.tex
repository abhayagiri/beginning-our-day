\mychapter{The Trump Card}{Ajahn Yatiko}{October 2012}

During the recent Western Monastic Conference, I raised a question with 
the Christian monks who were there, regarding an orthodox belief. I 
asked, ``What happens to an unbaptized baby who dies in childbirth or 
an aborted fetus? Is it going to heaven or not?'' One of the monks 
answered, ``Well, technically it's not going to heaven, because it 
wasn't baptized.'' It's easy to think that's an outrageous belief. 
However, I like the way Brother Gregory explained the monk's answer. 
``Look at it this way: What trumps everything is that \emph{God is 
just}. You can reduce everything to this one concept. Everything else 
fits into this understanding. If some issue doesn't fit, then either 
you're misunderstanding the issue, or it's been improperly communicated 
over time. \emph{God is just}, so God is not going to send somebody to 
hell who doesn't deserve it.'' Fair enough. Maybe the ways of God are 
mysterious and beyond our understanding, and the bottom line is 
\emph{God is just}. That trumps everything.

Sometimes, over the many months and years of our Buddhist practice, 
doubts can creep in. In monastic life we have this form---the bowing, 
the routine, and the other structures we live within---and the doubt 
may arise, \emph{Did the Buddha really teach all the rules we find in 
the Vinaya?} Or we may doubt whether certain parts of the early 
discourses are legitimate. \emph{Did the Buddha really teach those 
suttas in the Dīgha Nikāya that seem so mythical?} How do we deal 
with that? In Buddhism we don't have this concept of a permanent deity 
or a just God---we can't rely on that. But there's another trump card 
we can play: \emph{The Buddha existed, and he was fully enlightened.} 
So I can reflect, \emph{Perhaps whatever doubts I have arise because 
the form has been distorted over time, or because I'm not understanding 
how to see through the particular issue at hand. But the bottom line 
is, The Buddha existed, and he was fully enlightened.} That's a 
powerful perception, and it's important that we do not let anything get 
in the way of that.

We can dwell on the things we don't like, the things we find 
frustrating---those things that create doubt---and we can even come up 
with persuasive reasons why something we doubt is, in fact, wrong. But 
the effect of dwelling and thinking like this is that the heart can 
become discontented. We needn't invite that sort of discontent when we 
can just as easily dwell on our trump card. \emph{The Buddha existed, 
and he was fully enlightened.} When I bring up this perception, there's 
immediate joy and love and a recognition that this form, with all its 
imperfections, has a lineage that connects the Buddha with ourselves, 
right now. Through hundreds of generations, we're connected in a direct 
and tangible way to the living, breathing, walking Buddha---who existed 
and was fully enlightened. We can easily forget that, because it may 
seem so far removed from our ordinary, day-to-day experience. So we 
need to make an effort to bring this trump card into consciousness.

This can be especially useful any time we're having issues with 
monastic life or for lay people who might be struggling with doubt or 
with their faith. It can also serve as a foundation for our meditation. 
If we feel dejected because our \emph{samadhi} meditation isn't 
working, or whatever, we can return to this one idea: \emph{The Buddha 
existed and was fully enlightened}. It is easily accessible to us and 
uplifting. As we return to this one idea, over and over again, we may 
well find that it trumps all our doubts and difficulties.

