\mychapter{Respecting Others' Boundaries}{Ajahn Yatiko}{May 2012}

I had a significant learning experience about two or three years after 
I had ordained. A good friend of mine whom I ordained with was the 
monastery's stores monk. One time he went away for a week to Pu Jom 
Gorm, a branch monastery. During that time, I became the stores monk in 
his absence. I was keen to be helpful and do something supportive or 
generous as a show of kindness to him. So I cleaned up and reorganized 
the stores room, and I thought I did quite a nice job. When this monk 
returned a week later, he was visibly upset and took it as a personal 
comment or statement that he wasn't doing a good job on his task as the 
stores monk. This wasn't my intention at all. I saw some things that 
would be good to do, so I went ahead and did them, though I was not 
appreciated for what I had done.

In monastic life, it's important to recognize and respect other 
people's boundaries. In this case, if I had been more sensitive, I 
would have considered that my friend might feel that what I was doing 
impinged on his role and duties as the stores monk; and if I felt 
something should be done differently, it might have been better to have 
kept that to myself. It can be difficult for the stores monk if all of 
us expresses our opinions about the way he should manage his job. Or I 
might have waited until he returned, approached him, and asked if there 
was anything I could do to help. Then if he'd said something like, 
``Yes, what did you have in mind?'' it might have been a good time to 
express an opinion while still respecting his boundaries.

To take another example, we hear the guest monk in the office giving 
some advice to someone over the phone, and we feel \emph{we} know what 
should be said, and so we tell him what we think. This can be very 
burdensome for the guest monk. We have to think about boundaries, 
because there are many different tasks in the monastery: abbot, work 
monk, monastery secretary, guest monk, stores monk, computer monk, 
kitchen manager, chores monk, librarian, accounts manager, and so on. 
These tasks can take a fair amount of effort and patience to deal with. 
So it may be helpful for us---before we decide to ``assist'' someone or 
express our opinion about how a task should be done---to ask ourselves 
whether we are creating more of a burden for the person who has taken 
on the responsibility of doing that particular task.

The subject of boundaries goes beyond respecting each other's duties. 
There are also boundaries around physical space. How do we enter a 
room? When we walk into a room with people inside, how do we enter 
their field of awareness? Do we simply walk right in and announce our 
presence, or do we respect and appreciate the space in the room, 
entering with care?

There are also property boundaries. Let's say I was missing something I 
owned. I'm a senior monk and have an attendant. If I thought that what 
I was looking for might be in my attendant's personal cupboard, I might 
ask him to look in his cupboard when I next see him. But I would never 
go through his stuff looking for something simply because I thought it 
might be there. It's a different story if it's an emergency, but in 
other contexts it's not something that I feel is the right thing to do.

Living in community as we do, we can think about boundaries and 
remember that we want to focus on our own practice rather than what 
other people are doing. As the Buddha explained, we shouldn't go 
outside our own domain into the domain of others, because if we do, 
Māra will get a hold of us (AN V. 6). We can think of this as 
understanding and respecting people's boundaries. So we do what we can 
to respect and honor physical boundaries, like property, as well as the 
boundaries delineated by roles. In this way, we support harmony and 
well-being within the community.

