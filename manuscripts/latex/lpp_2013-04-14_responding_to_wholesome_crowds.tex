\mychapter{Responding to Wholesome Crowds}{Luang Por Pasanno}{April 
2013}

We are preparing for the annual \emph{kaṭhina} ceremony today and 
visitors have already started arriving. In terms of our practice, 
everything that happens on occasions like this is merely sights, 
sounds, smells, tastes, touch, and mental objects. We tend to make a 
problem out of sense contact, but it's just that much, nothing more. 
Objectively speaking, what's taking place here? The day begins, there 
is a wave of people and activity that comes in---it arises. The day 
ends, the wave of people and activity goes out---it passes away. It's 
another arising and ceasing---an opportunity to pay attention to the 
way that things come and go.

Of course, there's more to deal with and manage today, because the 
number of visitors will be high. That's the reality of what's 
happening. But we don't need to get excited about it, create a story 
around it, worry about it, or try to get away from it. Rather, we can 
ask ourselves, \emph{How am I holding all of this?} Are we able to 
return to mindfulness, over and over again? Do we have the ability to 
sustain a sense of clarity, discernment, kindness, and well-being? We 
can chose to let go of excitement, confusion, irritation, and aversion. 
Reactions like those are extra and completely unnecessary.

As a monastic community, we're solely dependent on the generosity of 
the lay community, and this is a time when that generosity is being 
vividly displayed. What can we offer as a gesture of appreciation for 
this generosity that gives us the ability to live our lives? It isn't 
through our getting swept up in excitement. It isn't through our 
shrinking away because of irritation. It's through our ability to hold 
a steadiness and clarity in our minds. Our ability to be clear and 
steady has a lot more value than the short-term satisfaction of 
allowing our minds to get swept up in our reactions. This ability is 
the basis for our practice, but also serves as a model for others. And 
that's what the world needs. It's a wonderful gift when people come to 
the monastery and see a model of wisdom operating in a group of people.

There are people coming---that's not an illusion. So help each other, 
pay attention to the situation and look after what needs to be done. Be 
ready to respond with kindness and attentiveness to the people who 
arrive. Express your heartfelt appreciation. These people are here to 
offer support for the monastery. That kind of generosity is something 
to delight in, with a sense of gratitude.

