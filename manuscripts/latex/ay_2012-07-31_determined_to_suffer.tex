\mychapter{Determined to Suffer}{Ajahn Yatiko}{July 2012}

We suffer because we are not seeing things with the right perspective.

We're here in this monastery by choice, and all the work is done on a 
voluntary basis. It's a good place. Within this sort of environment, 
there is no need to suffer. Things are the way they are. But through it 
all, we still seem to have an insatiable need to suffer. We are 
absolutely determined to not let go or soften our perspective. It's as 
if we'd rather suffer than grow. The problem is, we don't realize that 
suffering is going to grow and grow and grow. Eventually we are going 
to have to let go of all the situations that cause suffering, such as 
loss, physical pain, uncertainty, or whatever it is we don't like. It's 
just the way it is.

The fact that we experience a particular object as something we don't 
like is because of the way we structure our values and priorities 
\emph{within}. None of what we experience from the outside---be it 
community, people, activities, or responsibilities---has an objective, 
factual existence. It's something that we experience through our 
internal senses. We give meaning to things, and, based on the meaning 
we give, we attach to this and hate that. We create an entire world out 
of what we experience, and we end up \emph{living} in that world. The 
world, as we know it, isn't actually out there; it's in our heads. 
Understanding this helps us to see that our suffering is coming from 
our own actions, views, and decisions, and not from the outside.

This sort of reflection is empowering, because it puts us in control. 
There's really nothing and no one else we can blame. This truth is 
especially apparent when we live in a monastery like this. It's good 
enough to carry out our practice. Perhaps it's not so apparent for 
people immersed in a cultural setting---local or global---where ethics 
are not a priority, and introspection is not encouraged or valued. In 
that situation, it's easier to convince ourselves that our choices and 
actions are taking place in an external world.

But that perspective doesn't hold sway in the environment we have here. 
So we can surrender to the situation, the routine, the principles, and 
the practices of the monastery and learn to let go. Through that 
quality of surrender, the heart starts to feel a sense of release. We 
start to gain a sense of confidence as to what the spiritual life is 
really about, and what's important in our life.

