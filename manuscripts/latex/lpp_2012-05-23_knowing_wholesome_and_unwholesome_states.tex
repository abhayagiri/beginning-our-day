\mychapter{Knowing Wholesome and Unwholesome States}{Luang Por 
Pasanno}{May 2012}

The work period and the regular chores we have in the monastery are an 
extension of our practice; it's important to consciously bring that 
point to mind. Otherwise, it's easy to fall into the habit of being 
wrapped up in the excitement and enthusiasm we feel about the work 
project we are doing or being indifferent and waiting for the work 
period to be over. Whatever perceptions and attitudes we may have 
regarding the chore we're doing, it's essential to take that chore as 
an opportunity for establishing and sustaining our mindfulness. And 
this is done with the help of \emph{yoniso manasikāra}, skillful or 
wise attention.

As we were having tea, I spoke about the importance the Buddha placed 
on yoniso manasikāra. Its function is to direct attention in an alert 
and discerning way, no matter what we may be doing. Without yoniso 
manasikāra, we tend to get caught up in proliferation, in a mood, or 
in reactivity, because we haven't been attending closely to the 
situation.

The Buddha often described wise attention as knowing when there are 
wholesome states present in the mind, and knowing when there are 
unwholesome states present. It's also a quality that supports right 
effort---generating the wholesome and discouraging the unwholesome. To 
help with that, wise attention serves as an anchor for our mindfulness. 
I tend to encourage mindfulness of the body---using the body as a 
foundation for mindfulness and awareness, and for directing attention 
skillfully.

As we go through the work period, and throughout the day, we can try to 
carry our attention in a clear and discerning way. Once we develop a 
continuity of wise attention, it will become a firm foundation for all 
of our growth in the practice.

