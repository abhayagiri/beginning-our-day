\mychapter{Mindful of Right Effort}{Ajahn Karuṇadhammo}{November 2012}

Both on and off the cushion, we can examine how the activity of daily 
life is brought into the practice of Dhamma. In terms of the Noble 
Eightfold Path, many path factors are concerned with activities off the 
cushion. Developing \emph{samādhi} with sitting is just one part of 
the path. There is so much more that one needs to do to practice well 
and correctly. If we think of practice as that which is only on the 
cushion, then we are going to miss almost all the opportunities in our 
lives for deepening the practice.

We should keep in mind throughout the day not only right mindfulness 
but right effort. We do this by tapping into the awareness of wholesome 
or unwholesome states presently occurring in the mind.

It is important to check in with our current mind states, periodically 
investigating the mood of the mind so that appropriate attention is 
paid to what is happening. We can ask ourselves if we are dwelling in 
an unhelpful hindrance of aversion, craving, or some sense of 
impatience. Do we try to rush and finish an activity so that we can be 
by ourselves or do something that's more interesting than what we are 
involved in right now? If there is a hindrance present, notice what 
that state of mind feels like.

If they are unwholesome mind states or states that take us to a place 
that is going to lead to more stress and suffering for us or other 
people, then we can switch our attention and move into a state that 
will have a more positive effect on the mind. We don't need to 
reinforce negative states and allow them to drag us down.

Or conversely, is what we are experiencing wholesome or something that 
is helpful to support in the mind? We might, for example, enjoy being 
part of a work team and have positive experiences with other people. Or 
we might work alone and enjoy a wholesome activity that's good for the 
monastery. We can notice and reflect on these wholesome states of mind, 
encouraging, supporting, developing, and maintaining them.

Our goal throughout the day then, both on and off the cushion, is to 
check in every now and again as to the mood and quality of the mind. 
From there, we adjust as needed to bring about the wholesome and 
decrease the unwholesome.

