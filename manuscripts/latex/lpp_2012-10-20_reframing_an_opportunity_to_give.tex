\mychapter{Reframing an Opportunity to Give}{Luang Por Pasanno}{October 
2012}

In one discourse, the Buddha speaks about the ``great gifts'' we give 
to countless beings by keeping the precepts. Through diligence in 
virtue, using the precepts to guide our conduct, we offer security to 
other beings---freedom from danger, fear, and animosity. These are 
great gifts, indeed.

Similarly, in terms of our daily lives in the monastery, it's helpful 
to reframe what we do and how we approach life so that we look at our 
activities as opportunities to give gifts of service. We can give a 
gift of service by doing a task that needs to be done instead of 
waiting for somebody else to do it. Rather than attending the morning 
\emph{pūjās} and group practices because we feel they are being 
imposed from the outside, we can take our attendance as an opportunity 
to give a gift to the community by supporting and encouraging each 
other in practice. There can be a completely different relationship 
with how we deal with our schedules if we are willing to view them in 
this way. Rather than considering our duties and chores as onerous 
tasks we have to put up with, we can relate to them as an opportunity 
to give a gift that sustains the monastery and the supportive community 
we live in together.

We mustn't relate to the practice like wage laborers. If we think of 
ourselves as working stiffs trudging through the practice to get our 
paycheck at the end of the week, then there's not much joy or wisdom 
arising from what we are doing. But when we relate to the practice as a 
series of opportunities to give what's beneficial to others, then it is 
in all respects a different matter, it has a completely different feel. 
And of course, as with any gift, the first beneficiary is ourselves. 
When we relate to the training and the life we are leading in terms of 
giving, it uplifts our hearts.

