\mychapter{Desire Creeps In}{Luang Por Pasanno}{August 2012}

It's worthwhile to watch the habits of desire and craving that keep 
creeping into the mind. Really notice and pay attention to desire, 
because it's insidious.

This isn't meant as a commentary on anyone's inability to recognize or 
understand desire or to work with it, but simply to say that it takes 
our concerted attention and a willingness to investigate to see how 
desire keeps creeping in. Most importantly, we need to be patient and 
willing to work with the different ways that desire comes up. Don't 
assume that because you've made a resolution, \emph{Oh, I really want 
to free myself from desire,} that it's not going to manifest in various 
ways. Try to be very practical and attentive. Desire always seeks an 
object---that's the way the mind works. It goes to an object and 
becomes interested.

A prime example is the desire for food. In one psychology study I read, 
researchers tested two groups: one with fewer varieties of candy put 
out in front of them and one with more varieties of candy. Participants 
in the group with the increased varieties of candy ate sixty-nine 
percent more. One of the issues this may illustrate is how desire can 
increase our habit of craving based simply on seeing a greater variety.

I read about another researcher who'd written a paper on desire and 
advertising. One day he was in a supermarket checkout line ready to buy 
a sealed box containing ten packs of chewing gum. The box was labeled 
something like, \emph{Ten Packs for Two Dollars}. A colleague who was 
with him at the time said, ``Hey, you've just written a paper on 
this!'' People get pulled into such things. They see \emph{Ten for Two 
Dollars,} and think, \emph{Wow, just twenty cents each---what a great 
deal!} They might not want to buy one pack for twenty cents, yet they 
get hooked into buying ten! It's merely marketing stimulating desire. 
So even a researcher who'd studied and written about this very 
thing–even he was hooked. Desire comes up.

That's just the way the mind works. Recognize that it's not personal; 
it's simply that desire seeks an object. Our job is to be attentive, 
reflective, and willing to investigate; to watch how desire keeps 
hopping around looking for an object, looking for 
something---\emph{anything}---it can be attracted to.

We investigate, but not in a harsh way. We do this by taking an 
attitude of curiosity about desire, rather than feeling we have to run 
around with a sledgehammer and annihilate it. By paying attention to 
desire and recognizing what desire is doing, we can see how silly it is 
when we get hooked. You can ask yourself, \emph{How dumb can I get?} 
That's the way to step back from it---by seeing desire clearly and not 
making a problem of it.

