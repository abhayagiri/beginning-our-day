\mychapter{Taints to Be Abandoned by Using}{Ajahn Yatiko}{July 2012}

In the Sabbāsava Sutta, All the Taints, the Buddha talks about. He 
says we can abandon taints by using the four requisites---robes, food, 
shelter, and medicine---in the appropriate way. It's an interesting 
reflection that taints are to be abandoned by consciously using the 
requisites. Even though we're renunciants and sometimes we have ideals 
about getting by with less robe cloth, sleeping outside, eating less, 
or some other idealistic standard, we always have to remember that the 
foundation of our lives is based on a very down to earth, grounded, and 
simple practice around robes, food, shelter, and medicine. Attending to 
these requisites with mindfulness and care helps us in our spiritual 
lives---that's the purpose.

A teacher once told me that when he eats, he does it as an act of 
loving-kindness to his body. He feeds the body because he's caring for 
it and expressing his desire to care for it. The body is quite fragile, 
and it can be helpful to reflect on whether we are relating to the body 
with loving-kindness or with a demanding attitude. Based on some 
ascetic ideal, we might live in a way that is not kind to the body, 
causing harm to ourselves. For the sake of our practice, we need to 
care for the body in a mature and balanced way, with kindness and 
discernment

Food is a valuable tool that allows for comfortable, easy living. We 
might identify with being really austere monks and regularly fast for 
lengthy periods of time. Or we might move toward the opposite extreme 
and take great delight in delicious food. In either case, we need to 
ask ourselves, \emph{Am I caring for the foundation of my practice? Am 
I using the requisites wisely?}

Whether we are relating to the requisites from a sensual perspective or 
an extremely austere perspective, this is not the path the Buddha 
suggests. Rather, we can reflect on whether we are using the requisites 
in a way that supports our long-term practice, and adjust our behavior 
accordingly. That is a valuable reflection to cultivate, apply, and 
strengthen over the years.

