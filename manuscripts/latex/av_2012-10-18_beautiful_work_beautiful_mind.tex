\mychapter{Beautiful Work, Beautiful Mind}{Ajahn Vīradhammo}{October 
2012}

Whether it's sewing robes or making a footpath, the Forest Tradition 
has a high standard of workmanship. But quite often we're asked to do 
things we're not competent in or used to doing. There's a learning 
curve we all go through in the Saṅgha. If we've never had to do 
welding and we end up assigned a welding job, or if we've never been an 
abbot and we end up being an abbot, it becomes a real training in how 
we learn new skills.

One of the monks at Chithurst was a very good carpenter and 
cabinetmaker, and in general, an excellent worker. He once suggested 
that one of the best ways to learn a manual craft or skill is to undo 
any mistakes we've made, rather than covering them up. I saw him do 
that several times. He would be building something quite complicated 
like a staircase, and, if he saw a mistake, he would reverse his steps 
until he'd gone back to the place where the mistake was made, and then 
he'd correct it. My tendency had been to keep going after making a 
mistake, and hope that no one would notice it. Watching this monk work 
I also noticed that once he had retraced his steps backward to correct 
some mistake he'd made, he tended not to make that same mistake in the 
future. Not surprisingly, his work was very beautiful.

Once I watched Luang Por Liem make a beautiful broom. It was a fabulous 
example of very mindful craftsmanship, his hands were so attentive and 
efficient. Luang Por obviously has the gift for it, but that gift has 
been greatly enhanced by his many years of training in mindfulness.

When our task involves working with others, it's a perfect opportunity 
to develop the beautiful skill of patience. For example, we may have a 
skill or aptitude for the task and already know how to do it quickly, 
but we're working with someone who is a bit clumsy and slow because he 
or she doesn't have that same training or aptitude. In that situation, 
we need to practice patience with each other.

Whether or not we already have the skills required to tackle a certain 
project we've been assigned, it's always rewarding to do things 
carefully and beautifully, as best we can. Abhayagiri's lodgings and 
buildings are quite beautiful. Whether it's the robe rail or the 
sleeping platform, it's all nicely done. I can appreciate everything it 
must have taken to complete these lovely works---the mindfulness and 
workmanship, the patience, and the willingness to learn.

