\mychapter{Putting Aside What I Want to Do}{Ajahn Yatiko}{December 2012}

In last night's Dhamma talk, Ajahn Karuṇādhammo said that one of the 
most valuable things we can do as residents of the monastery is develop 
generosity of heart by helping to create harmony in our community. 
Instead of looking at things solely from the perspective of what we 
like and what we want, we also consider the wants and needs of the 
group and what is best for everyone in the community. It is good to 
think of the Saṅgha as an organism of which we are a part.

There is a lovely passage that is often quoted from the suttas in which 
the Buddha is talking to a group of three monks living 
together---Venerables Anuruddha, Nandiya, and Kimbila (MN 31). When the 
Buddha asks how they cultivate harmony, Ven. Anuruddha says, ``Well, 
Lord, I ask myself, \emph{Why should I not put aside what I want to do 
and, instead, do what these venerable ones want to do?} Then I put 
aside what I want to do and do what they want to do.''

It is true that while we are here we cultivate solitude. Solitude is an 
important part of our practice, but equally important is the need to 
ask ourselves, \emph{What am I doing in solitude? Is it helping my 
practice or not? Am I cultivating wholesome, bright mind states in 
solitude, or might doing acts of service and helping the community be 
more beneficial for my mind?} We don't have to be out there all the 
time doing things. But whenever there is a possibility of helping the 
community or an individual, that is an excellent opportunity, in the 
moment, to say to ourselves, \emph{Okay, here's my chance to do 
something, to break out of my mold of solitude and contribute to the 
life of the monastery.} For example, let's say I needed a volunteer to 
give me a hand with something outside of the work period this 
afternoon, and I asked, ``Is anyone available for an hour?'' In a 
situation like that, while we may have it in mind to get back to our 
cabin and do some sitting, have a rest, read some suttas, or whatever, 
instead we could spontaneously drop that thought or desire and offer 
assistance. For this to be a special gift of generosity, it takes a 
willingness to be spontaneous---this is how it comes from the heart. To 
say to ourselves, in the moment, \emph{Sure, I'd be willing, I'll 
volunteer for that, I'd love to.} That's a beautiful way to create 
harmony and appreciation, and it's a fine example of the generosity 
that comes from the heart.

My intention is not to be one sided. I'm not saying that solitude is 
not important and that as monastics and people living in the monastery, 
we don't have the duty to cultivate a love for solitude. But we should 
also remember that a love of generosity is an important part of the 
path.

I remember a time when a person who'd lived as a monk for ten or 
fifteen years was clearly going through difficult times and was 
suffering from serious mental problems. One of the senior Thai Ajahns 
was talking about the reasons for this person's breakdown and said, 
``It's because he hasn't developed enough generosity in his practice.'' 
You could see that was true from this monk's behavior. He spent most of 
his time in solitude, but didn't do it in a skillful way. He tended to 
run off to be by himself at every opportunity. He rarely served or gave 
of himself. It is not as if other monks didn't want him in the 
monastery or didn't want to be around him. It was simply that they 
could not appreciate his presence, because he was never really there 
for anyone.

A community is like an organism that requires care, attention, and 
participation in order to function and remain beautiful, comfortable, 
pleasant, and healthy. As members of the community, each of us has to 
do our part in order to make that become a reality, because it does not 
come about accidentally or on its own. In terms of cultivating the 
path, I think this is worthy of reflection for our daily practice.

