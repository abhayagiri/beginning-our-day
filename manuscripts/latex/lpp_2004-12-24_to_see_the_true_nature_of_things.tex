\mychapter{To See the True Nature of Things}{Luang Por 
Pasanno}{December 2004}

Without clearly understanding the processes of our minds, we create all 
kinds of problems. We are dragged about by emotional states. For there 
to be personal and global peace, these states need to be understood. 
The ways of the mind need to be seen clearly. This is the function and 
value of Dhamma.

When we are feeling enthusiastic, we can easily give ourselves to the 
practice. But it can also happen that, at times, we feel completely 
disillusioned, even to the extent that we forget the original 
confidence and faith we had. But that is natural. It is like swimming a 
long way. We become tired. There's no need to panic though; we can 
simply be still for a while. Then, when we have regained strength, we 
continue. When we understand in accordance with nature, we understand 
these states will change. Despair, if that is what has arisen, will 
pass. We keep practicing. By observing our minds and seeing how our 
attitudes are continually changing we more clearly understand that 
impermanence is natural.

Isn't cultivating Dhamma as important as breathing? If we stop 
breathing, then we die. If we are not established in a right 
understanding of the truth---of the way things are---then also we die. 
We lose touch with what is truly good. If we are lacking the richness 
of truth in our hearts, then when we die and they cremate us, our lives 
will be worth no more than the handful of ashes we produce---and that's 
not much! We must investigate how to live in a way that truly accords 
with what the Buddha taught. Then we can live in harmony without 
conflicts, difficulties, and problems. \emph{Sīla}, morality, is that 
which shows us the Buddha's middle way. It points to the avoidance of 
the extremes of pleasure and pain; it means knowing the right amount. 
When we live in the middle way regarding actions of body and speech, 
then we don't cause offense to others; we do what is appropriate for 
human beings. The practice of formal meditation is to train our minds 
and hearts to stay in the middle way.

Many people who meditate try to force their minds to be as they want 
them to be. They sit there arguing with their thoughts. If their 
attention wanders, they forcibly bring it back to the breath. Too much 
forcing is not the middle way. The middle way is the ease that arises 
naturally in the mind when there is right effort, right intention, and 
right awareness. When the practice is right and there is ease of mind, 
we can simply watch the different states that arise and consider their 
nature. We don't need to argue with anything. Arguing only causes 
restlessness. Whatever emotion arises is within the domain of our 
awareness, and we simply watch. Whether it's joyful or the absolute 
opposite, each experience is within the boundaries of our awareness. We 
just sit, watch, recognize, and contemplate them all. They will 
naturally cease. Why do they cease? Because that is their nature. It is 
this realization of the true nature of change that strengthens and 
stills the mind. With such insight there is tranquility and peace.

What we call ``me'' is merely a convention. We are born without names. 
Then somebody gives us a name and after being called this name for a 
while, we start to think that a thing called ``me and mine'' actually 
exists. Then we feel we have to spend our lives looking after it. The 
wisdom of the Buddha knows how to let go of this ``me,'' this ``self,'' 
and all that pertains to it---possessions, attitudes, views, and 
opinions. This means letting go of the conditions that make suffering 
arise, and that requires taking the opportunity to see the true nature 
of things.

