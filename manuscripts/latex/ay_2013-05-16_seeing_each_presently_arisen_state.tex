\mychapter{Seeing Each Presently Arisen State}{Ajahn Yatiko}{May 2013}

There's this lovely passage from the Bhaddekaratta Sutta (MN 131): 
``Let not a person revive the past, or on the future build his hopes, 
for the past has been left behind, and the future has not been reached. 
Instead with insight let him see each presently arisen state; let him 
know that and be sure of it, invincibly, unshakably.''

Our lives are uncertain in so many different ways. All sorts of 
curveballs could be thrown to us at any time, disrupting our flow. If 
we understand this, knowing curveballs happen, we don't have to see 
them as problems. Any anxiety we have around the future is not to be 
denied as part of our experience. We can hold it, be with it, and try 
to act responsibly toward it. At the same time, we can have mindfulness 
and see anxiety as simply a phenomenon, a perception arising in the 
present moment. If we closely examine whatever it is we're worried 
about, thinking about or attached to, we see that very often it's 
nothing but an idea in our heads, and from that idea we can create a 
whole world of suffering. To penetrate and see this is an essential 
part of our practice.

We do need to take care of practical things, such as our tasks during 
the work period that starts a short while from now. But when we go out 
to do some work it is crucial that we do not let those insights 
slide---this reflection about the insubstantial nature of ideas and 
concepts, the past and the future, and their potential for causing us 
suffering. These insights are accessible and not that difficult to 
penetrate, but it takes determination.

We have to go into the stream of the compulsive mind that's going in a 
particular direction, very often the direction of \emph{bhavataṇhā}. 
Bhavataṇhā is the ongoing creation of and belief in a solid and 
substantial self or identity that we are constantly propping up with 
roles, habits, ideas, views, and opinions---\emph{I'm like this, I'm 
not like that}. We need to go against that bhavataṇhā, putting an 
anchor down in the stream of becoming. Penetrating through the illusion 
of bhavataṇhā is one of the most important aspects of our practice. 
It's more important than any of the little hobbies we have, the habits 
we've developed, the various ways we spend our time. We have to make 
sure that in our monastic lives we don't let the hobbies become 
foundational. If they do become foundational, then what \emph{should} 
be foundational---meditation---simply becomes a pastime. When that 
happens it's tragic, and we need to do our utmost to cut it off.

And to cut it off requires being aware of it. In the sutta ``Without 
Blemishes,'' the Buddha points to the case where a person with a 
blemish ``understands as it actually is, `I have a blemish in 
myself.''' He has a blemish and \emph{knows} he has a blemish, and 
therefore can be expected to exert himself to abandon that blemish. But 
a person who doesn't realize he has a blemish is not likely to do 
anything about it (MN 5). So this is the starting point---being able to 
recognize a defilement as a defilement, being able to call it that. 
That's step one. Step two is to work with it.

