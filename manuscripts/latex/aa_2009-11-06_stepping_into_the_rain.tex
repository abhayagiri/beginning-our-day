\mychapter{Stepping Into the Rain}{Ajahn Amaro}{November 2009}

When the weather turns wet and gray like this, the world is a bit less 
inviting outside, and it's easy to follow the natural instinct to seek 
shelter and find a cozy spot. We might find ourselves hanging around in 
the kitchen, the library, or the monks' office. It starts off with 
waiting for the rain to ease off, then an hour goes by, then two hours. 
We can end up spending hours doing a bit of unnecessary emails or 
thinking, \emph{Well, maybe I'll make that phone call, maybe I'll take 
a look at this or that website.} The whole afternoon goes by with us 
chatting away with each other, waiting for the rain or the mood to 
change. This is a good way to waste time.

For the monastics here who have dwellings off in the forest, as well as 
for the people who are staying here as guests, it's good to recall the 
basic principle that we have of not gravitating towards the communal 
cozy spot or a place where there are good chatting opportunities. Once 
the meal is finished and the washing-up is done, unless we have some 
urgent or significant business that requires our attention in the 
afternoon, we should pack up our things and go back to our dwellings.

It takes a certain resolution to walk out into a wet, gray afternoon, 
but once we are back in our dwellings, we find that solitude is the 
most delightful and helpful of companions. It takes an effort to turn 
and walk towards that. As the winter season is setting in and it's 
gray, misty, wet, cool weather, I strongly encourage people to refrain 
from huddling in that cozy spot looking for human company and cups of 
tea, and to instead step out into it. Go back to your dwellings. Spend 
time alone. Develop the path and turn your efforts toward the 
realization of Nibbāna. This is what we're here for.

The alternative to doing that is always available to us---the 
particular conversation we're interested in, this nice, cozy spot to 
settle in---the comfortable alternative. If we simply default to what's 
comfortable, what's interesting, the flow of a casual contact, hearing 
the news, looking for something to do, being engaged in 
something---anything---then we're really wasting our time. It's not 
helpful. It doesn't conduce to insight, concentration, or 
liberation---which is the purpose of this place and why this community 
exists.

We're not here to exchange information, contact others, or plan menus. 
Of course, these things have their place. They're all part of the 
everyday process of helping with the construction and maintenance of 
the buildings and with the cooking of food. But all these necessary 
tasks and all the extraneous little bits and pieces that demand our 
attention are not the purpose of our lives here. Those tasks and duties 
are merely the means by which we're fed and sheltered, supplying us 
with the requisites. Our lives here are for the purpose of developing 
the path and realizing Dhamma.

As the season changes and it becomes cool and damp, don't be blind to 
the influence the weather exerts over our minds. Take the opportunity 
to seek solitude, non-engagement, seclusion. These are the elements of 
the path that conduce to realization. These are some of the qualities 
that benefit and support our decision to live in this place, shaving 
our heads, putting on robes, keeping the precepts, following the 
routines and disciplines. And if we want to fulfill the true purpose 
for which we have come here---to realize the Dhamma---then we need to 
take the initiative, we need to back up our commitment with actions 
that match this greatest of all aspirations.

