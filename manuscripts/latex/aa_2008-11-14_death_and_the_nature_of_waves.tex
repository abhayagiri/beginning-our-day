\mychapter{Death and the Nature of Waves}{Ajahn Amaro}{November 2008}

We received news yesterday from Kondañña's partner that he's fading 
rapidly. He's in a hospice in San Francisco. Jay passed away a few days 
ago. Ajahn Karuṇadhammo is having surgery today. While we were 
sitting this morning after doing the \emph{paritta} chanting for Ajahn 
Karuṇadhammo and Kondañña, an image came to mind of different waves 
coming in from the ocean and moving toward the shore. There's the wave 
named Jay, having reached the shore and broken up, and the Kondañña 
wave, a wave moving steadily toward the shore. Similarly, with Ajahn 
Karuṇadhammo and all the rest of us, there are these little waves on 
the great ocean of nature, steadily, inexorably moving toward the 
shore. They sustain a form for a particular time---what we call Jay or 
whatever---they move along picking up material, and finally reach the 
shore and disperse. These energy waves flow through the oceans of the 
world. When they reach the beach, there's the sound of the waves 
crashing on the shore. And then the energy disperses.

When we reflect on the imminence of death, it seems like such a 
dramatic thing---a life coming to an end. It seems so personal. The 
Amaro wave seems so different from the Pasanno wave, the Cunda wave, 
the Ṭhitapañño wave. They all seem separate and different. But when 
we look at them from a broader perspective, it's merely the same ocean 
of stuff, with different patterns of energy moving through it. The 
waves have a certain coherence, a certain individuality that's 
apparent. But there's no need to get excited about this wave versus 
that wave, to worry about which one is closer to the shore, or to think 
that anything desperately bad has happened when a wave reaches the 
shore. It's simply what waves do. That's their ultimate destination, 
their nature. Their energy disperses.

Sickness and death are very much in our consciousness these days. Ajahn 
Karuṇadhammo going under the knife, Ajahn Toon passing away, Jay 
passing away, and Kondañña in his last days. So this is a good time 
to cultivate a sense of knowing the nature of waves, and to look upon 
our own lives and the lives of others with the same kind of equanimity 
and evenness. We're simply watching the ocean doing its thing. Waves 
forming, flowing in, breaking on the shore, washing out. That's it, no 
big deal.

We can also reflect on the distinctions between our lives and the lives 
of others, those we know and those we don't know. We can see how 
different textures of feeling can gather around particular waves. They 
seem so important because there's a name for each wave, the name of 
someone close to us. But the knowing quality within us reflects that 
it's just another wave. How could it be so different or special---even 
when it has our own name on it? This body, these aches and pains, these 
ailments, this lifespan. There's that knowing within us which can 
realize, \emph{Well, it's not such a big deal. Waves rise up, they 
move, they break. That's it, no big thing.} We have the ability to 
discover that quality of evenness, steadiness, and balance. We can have 
an unshakable and profoundly stable equanimity which is undisturbed in 
the midst of all that arises, moves toward the shore, and disperses.

