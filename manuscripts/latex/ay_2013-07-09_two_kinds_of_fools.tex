\mychapter{Two Kinds of Fools}{Ajahn Yatiko}{July 2013}

One of the things Ajahn Chah taught was what he called ``earthworm 
wisdom.'' For many people, earthworms aren't worth appreciating. But 
it's earthworms that till the soil, and if they weren't continually 
working away, the soil would be infertile and incapable of supporting 
growth. That's a nice reflection, something to chew on.

The higher aspects of the teachings are certainly worth reflecting on, 
and there is a lot of profundity to penetrate. But it's also true that 
we live in a community and, very often, community living is where the 
rubber hits the road. This is where we have contact, where we come 
together and rub up against our rough edges. Everyone in the community 
has the potential to make mistakes and cause offense or harm. When a 
mistake happens it can trigger different responses, both from the one 
who made the mistake, and from the one receiving the consequences of 
that mistake. In particular, the Buddha says, there are two kinds of 
fools in the world. One is a fool who, when having made a 
mistake---either an offense or a hurtful action towards another 
person---doesn't ask for forgiveness. The other kind of fool is one 
who, when asked for forgiveness, refuses to give it. Do \emph{we} 
sometimes act like one of those fools? That's something else to reflect 
upon.

We live in this community and hope that it is a wise community. We want 
to establish, maintain, and care for harmony here, because it is 
valuable and also because it is vulnerable. Like all relationships, 
those we have with our companions here require our generosity, and we 
have to put our hearts into that. We do this by opening our eyes, 
taking a look around, seeing how people are doing, and responding with 
our hearts. We reach out to people who look like they may be 
struggling, not doing so well, or who need a little bit of a lift. And 
in our hearts we forgive those who've made mistakes, whether they've 
asked for forgiveness or not. That's just ordinary kindness, but it's 
earthworm kindness---it's what creates an environment in the community 
that is very beautiful and uplifting. It provides the tilled soil from 
which fertility, growth, and development of individuals can take place.

Solitude in our practice is important in helping us to relinquish the 
unskillful views we've picked up in the past from misguided teachers, 
friends, books, and other unhelpful sources---views that have 
influenced us in ways not easily recognized. We need solitary practice 
to clearly see those views and the effects of our prior 
conditioning---to discern the Buddha-Dhamma for ourselves. But we have 
to be careful that our solitary practice doesn't create a type of 
selfishness or self-centered point of view. Living in community helps 
us remember to open our eyes and see that there are people here who are 
going through the same difficulties we are, and who may need the same 
sort of support we have received. It is helpful to be conscious of 
that, expand our vision, and care for the people around us. We are like 
earthworms tilling the spiritual soil of the community so it is a 
fertile place for growth in the Dhamma. When we come from a place of 
generosity and care, there can be a strong feeling that we are living 
under very special circumstances. This in turn uplifts our solitary 
practice and encourages a more encompassing perspective.

