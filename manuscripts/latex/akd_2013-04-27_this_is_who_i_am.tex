\mychapter{This Is Who I Am}{Ajahn Karuṇadhammo}{April 2013}

Before the work meeting, a few of us were talking about different kinds 
of characteristics, traits, and qualities we each have and how easy it 
is to indulge in our personal attributes. Some people are good 
visualizers and can easily create certain kinds of images in their 
heads. Others can remember music. I was reflecting how easy it is to 
take those characteristics that are an integral part of each of us and 
pick them up in a way that makes us think that's who we are. Whether 
it's positive qualities or whether it's negative qualities, we come 
into the world with these attributes based on past actions and past 
habitual conditioning. Whatever they are, we easily identify with them 
and believe that's who we are: \emph{I'm a person who has abilities in 
construction, carpentry, computers, or whatever our skills are} or, 
\emph{I'm a person who has a lot of anger, sensual desire, fear, 
anxiety.} We take these different kinds of qualities we experience 
throughout our lives and personalize them, creating an image of 
ourselves in our own minds.

We do the same with other people. We see certain qualities, 
characteristics, or habits so we identify a person as a particular type 
or someone who always has a certain characteristic. He's an angry 
person or she's a person who has a lot of sensual desire. Even if we're 
smart enough and know about Buddhist practice to the extent we don't 
really believe that's who we or they are, for the most part that's 
still how we operate. We continue to go through our daily activities, 
seeing the world of ourselves and the world outside through that 
perceptual lens of good and bad qualities. To counter this tendency, we 
persist in chipping away at that sense of solidity, that sense of a 
permanent self.

Every time we find ourselves lamenting over an unskillful quality or 
habit we have or puffing ourselves up thinking we're exceptionally good 
in some area, it's important to keep reflecting, \emph{Well, that's not 
really who I am.} These are just qualities that come from 
conditioning---sometimes through skillful attention, sometimes through 
unskillful attention, and they're essentially a conglomeration of 
images and ideas that have formed into the perception: \emph{This is 
who I am.} Try to see them objectively as conditioned qualities and 
don't take ownership. By recognizing this constant change and flow of 
characteristics moving through consciousness we can see clearly that we 
are merely holding onto shadows. We can learn not to hold onto these 
characteristics so tightly, and watch them arise and cease, realizing 
their conditioned nature. If we can see little bits and pieces of how 
these different characteristics are dependently arising based on causes 
and conditions, then that strong and limiting sense of self starts to 
slowly disintegrate. Through this process we can understand that we 
aren't bound by the sense of me, mine, myself---the sense of entrapped 
solidity. It's good to bring this up as a recollection moving 
throughout the day, letting go little by little, and not feeling so 
entangled by our images and perceptions.

