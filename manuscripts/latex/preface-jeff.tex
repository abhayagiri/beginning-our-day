\chapter*{Preface}
\addcontentsline{toc}{chapter}{Preface\protect\chapternumberline{}}

``Beginning our day \ldots{}''

These quintessential words are spoken by Luang Por Pasanno before he
begins each of his morning reflections.

Five days a week, at Abhayagiri's morning meeting, work tasks are
assigned to the residents and guests living in the monastery. Shortly
thereafter, one of the senior monks offers a brief Dhamma reflection so
that the residents and guests have something meaningful to recollect
throughout their day.

These talks are given spontaneously and often address an event that is
about to occur, a condition that is already present in the monastery, or
a general teaching on Dhamma. The most common thread through all the
reflections is that of practicality: distilling the most important
teachings of the Buddha into pertinent and applicable practices. Though
many different teachings are touched upon, the fundamental aim is to
encourage the abandonment of the unwholesome, the cultivation of the
wholesome, and the purification of the mind.

While several of these teachings may be read together at one time,
readers might find it more useful to focus on a single reflection so
they can easily recollect, contemplate, and make use of it throughout
their day.

This book was made possible through the contributions of many people.
More than ten years ago, Pamela Kirby initiated the project when she
placed a recorder in front of one of the senior monks during a morning
reflection and proposed that a book be written. Matthew Grad, Jeff
Miller, Ila Lewis, Ray Peterson, and Laurent Palmatier were the main
substantive editors of the material, enduring the long and difficult
process of editing the transcripts into compact and well-written
teachings. Pamela Kirby generously offered assistance at various stages
of the editing process. Ruby Grad helped with the copy editing. Shirley
Johannesen helped with the glossary. David Burrowes, Dee Cope, Josh
Himmelfarb, Evan Hirsch, Jeanie Daskais, Anagārika John Nishinaga, and
members of the \mbox{Lotus} \mbox{Volunteer} \mbox{Group}: Wendy Parker
and Viveka all helped with further refining of the text. Sumi Shin
designed the cover.  Jonathan Payne took the cover photos. Michael Smith
tumbled the stones for the back cover.

For several years, Khemako \mbox{Bhikkhu} recorded the senior monks'
reflections. Kovilo \mbox{Bhikkhu} and Pesalo \mbox{Bhikkhu} provided
corrections on an early draft of the book. Suhajjo \mbox{Bhikkhu}
generously dedicated a significant amount of time on the overall book
design and typesetting of the text.


\textit{The Collected Teachings of Ajahn Chah} published by Aruna
Publications in 2011 provided many of the terms in the glossary. 

Anonymous supporters of Abhayagiri generously funded the printing of
this book.

Any errors that remain in these reflections are my own responsibility.

May these teachings bring insight into the nature of Dhamma and provide
a pathway toward the development of true peace and contentment.

\vspace{\stretch{1}}
{\raggedleft
Cunda Bhikkhu\\Abhayagiri Monastery\\May 2014

}
