\mychapter{The Kamma of Listening}{Ajahn Yatiko}{July 2013}

Offering a morning reflection like this is a bit like planting a seed. 
What is said needs to be listened to with attention, care, and an open 
mind. When that happens, the seed has been planted in good soil. If the 
mind is not attentive or mindful, if it's off dreaming or thinking, 
then it gets no benefit from the reflection. But if the mind is 
attentive, focused, and keen to extract value, then even if what is 
said is very mundane or something we have heard a thousand times 
before, the mind can still benefit greatly. If we hear a reflection 
with the right attitude and a desire for what is wholesome, then value 
can be acquired even if we're not aware of it.

One of Ajahn Geoff's books is called \emph{The Karma of Questions}. 
That's a fascinating theme to reflect on: What kind of \emph{kamma} is 
involved in asking questions? It's not so much in getting an answer, 
but in asking the question that we create our kamma. Similarly, we can 
reflect on the kamma that is generated by listening to Dhamma. What 
does it mean to listen to Dhamma? Why is listening to Dhamma such 
profoundly good kamma? It's helpful to contemplate the value of 
listening to reflections and to appreciate receiving reflections even 
though it may sometimes feel as if we just want to get the experience 
over with as quickly as possible. If we listen attentively to Dhamma, 
we can realize insight right at that moment of listening, but also, 
days or even years later this Dhamma can give rise to wholesome 
thoughts, perceptions, and attitudes, planted like seeds in our minds. 
We never know when our listening to Dhamma---that initial seed---will 
bear fruit, but the quality of our listening will ensure the seed is in 
good soil.

