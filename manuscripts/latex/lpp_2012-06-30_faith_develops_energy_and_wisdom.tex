\mychapter{Faith Develops Energy and Wisdom}{Luang Por Pasanno}{June 
2012}

Faith is an essential part of our practice, and it's not something that 
magically appears on its own. Rather, the arising of faith takes 
effort. We need to direct our attention toward it, to frequently 
reflect on the arising of faith as a real possibility for us.

As Westerners, most of us are not on familiar ground when we reflect on 
faith. But it is an important quality for balancing the different 
aspects of our practice. In particular, faith comes first among the 
Five Spiritual Powers. It is used to balance wisdom---the fifth 
Power---and to support energy and effort---the second Power. How does 
that work? Wisdom arises from investigative analysis, and, without 
faith, investigative analysis becomes dry and we tend to lose energy. 
The qualities of faith include confidence and devotion, both of which 
naturally result in increased energy and effort. Energy and effort need 
to be nourished, and faith is an important part of that nourishment.

The opposite is true as well: When the wisdom faculty is lacking in 
faith, it often leads to cleverness and a superficially critical way of 
evaluating things. And without faith, wisdom can increase our 
negativity, which in turn can reduce our own energy and the energy of 
the people with whom we engage.

For many reasons, it is important to turn our attention toward faith, 
to reflect on it, so we can recognize faith as an essential part of our 
practice.

