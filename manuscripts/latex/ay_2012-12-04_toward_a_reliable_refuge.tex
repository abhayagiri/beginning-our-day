\mychapter{Toward a Reliable Refuge}{Ajahn Yatiko}{December 2012}

This morning was one of those times for me when meditation went quite 
well and was peaceful and bright. At the same time, I kept reminding 
myself that there's no condition, even in meditation, that can be a 
refuge. The same goes for life in general. It's good to contemplate 
that fact during our meditation. While attending to our bodies, 
feelings, and everything that's arising, we can say to ourselves, 
\emph{Nope, this is not a refuge}. When we do this, we find that our 
fear softens. There's nothing to fear, because there is no condition 
that is a problem. Neither the condition nor the fear is us. We don't 
get stuck on anything because we've separated ourselves from the 
problem, relaxing and stepping back.

The mind can obsess about different subjects---for example, the 
electrical system in the monastery. When we reflect that the electrical 
system or thoughts about it are not a refuge, then the mind can let go 
of obsessing about that and step back from it. The electrical system 
doesn't really matter. We are going to be dead soon, and in the bigger 
picture it doesn't matter how the electrical system is functioning.

This is the same with whatever mood we are experiencing. Moods are an 
obstacle to our practice, because we're so tempted to believe in them. 
A dark, despairing, frustrated mood, or a bright, blissful, joyful 
mood---it doesn't matter which---consciousness can get stuck on the 
perception or belief either way. More to the point, moods can't provide 
a reliable refuge---they come and go like the wind.

We should contemplate our experiences and remind ourselves: when 
they're good, that's not a refuge; when they're bad, that's not 
something to get stuck on. What we find with this practice is that, 
since everything we experience, feel, and imagine is moving and 
transient, none of that can be a true refuge. The refuge lies 
elsewhere. So we go through this process of deconstructing our 
experiences, stepping back from these experiences, and that's the way 
we move toward a reliable refuge.

