\mychapter{The Gift of Space}{Ajahn Yatiko}{September 2012}

It's a challenge to encourage people to grow in Dhamma. It's not a 
matter of simply telling people what to do or asking people to conform 
to a set of rules. We have to encourage and inspire people to dig deep 
within themselves to respond to situations in ways that are skillful. 
That often involves going against the grain. It's tempting to tell 
people what to do and to lay down a rule, but that's not the point.

If we see anger arise in other people, one way to quell the anger is to 
get really angry back and basically squash them with our words. That 
may shut them up, but there isn't much---if any---Dhamma present in 
that response. Another method is to manipulate them into to stopping 
their anger by using subtle, temporary tactics, but that also doesn't 
help them grow in Dhamma. What helps is meeting their negative energy 
with space and loving acceptance. If we give them the space to reflect 
on their own anger, they might realize that anger is a form of 
suffering.

Sometimes we give space but sense that no investigation has come from 
it. If we give people space and they don't investigate their anger and 
irritation, there's nothing we can do about that. There's no guarantee 
that giving space to negative energy will resolve the situation, but it 
does give the other person the chance to work with it. And if we 
\emph{don't} give space, if we respond to negativity with negativity, 
it's almost a guarantee that nothing profound will take place. Giving 
space is one of the best things we can do for people; it's noble. We 
don't condone what they're doing at the moment, but by giving them 
space, we encourage them to grow and gain insight into their own mental 
processes.

So this is a gift that we can give to others, a gift that others can 
give to us, and a gift we can give to ourselves as well. Whenever we 
receive this noble gift, we should use it as much as possible to 
lovingly look into the mind and see what actions cause suffering for 
ourselves and suffering for others.

