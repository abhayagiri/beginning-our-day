\mychapter{Reversing the Tendency to Decline}{Ajahn Yatiko}{May 2013}

Last night a number of the monks had an opportunity to go up to Ajahn 
Dtun's \emph{kuṭī} for a discussion on Dhamma. One of the themes 
brought up was the tendency for personal standards to decline in one's 
practice. This is an important trend to examine. We can look at the 
direction our practice has taken over time---how we started, how we've 
been practicing, and where our practice seems to be headed. That's 
something for us monastics to explore, because we can sometimes think 
of our practice as merely the form of being in robes. But actually, 
it's the heart that's practicing; the practice is not about taking on a 
role or simply putting on a robe. So we should ask ourselves, 
\emph{Where is this leading, in the long run? What is the destination 
of the practice we're doing now?}

There are two extremes for monastics that come to mind. One is moving 
toward the sense world, which strikes me as a nihilistic place with an 
emptiness to it. The other is moving toward some state of 
being---becoming a certain type of person, identifying with a role or 
believing that our goal is to become something different, improved, 
efficient, or whatever. These are the two extremes. So we can reflect 
on this and ask ourselves, \emph{Is our practice moving in the 
direction of cessation and peace? Or is it moving toward the world of 
the senses or becoming?}

We can bring mindfulness and attention to this inquiry so as to see the 
direction of our practice. Once we see this direction clearly, we can 
reflect on what we need to adjust or correct. We do this so that our 
practice is moving in a more direct, straight, and less wavering 
direction toward a state of peace, understanding, and calm.

