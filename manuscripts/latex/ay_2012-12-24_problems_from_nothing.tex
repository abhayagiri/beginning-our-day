\mychapter{Problems From Nothing}{Ajahn Yatiko}{December 2012}

If we were to look calmly and clearly at our experience living here in 
the monastery, we would see that our lives are extraordinarily simple. 
However, as human beings, we have this great penchant for creating 
problems out of nothing, a tendency that is quite self-harming. We 
create these problems with our perceptions and our desires. We pay 
attention to them, dwell on them, think about them, obsess over 
them---and so they stick with us. By dwelling on a particular issue, it 
grows and becomes more entrenched in our being and more entrenched in 
our perceptions. Before we know it, we're in a world that seems 
tremendously complicated, and we don't recognize that this is something 
we've created out of nothing. Behind these problems are nothing but our 
conflicted and contradictory desires.

With the path laid out for us by the Buddha, we can aim to see this 
process and explore how it is that we create these problems for 
ourselves. This is what will alleviate suffering and stress and 
ultimately prevent us from creating these complications in the first 
place. There's nothing here in the monastery that is a real concern. We 
have food to eat, water to drink---all our basic necessities are well 
taken care of. There's nothing that has to be done. Furthermore, we 
live in a harmonious community. Yet somehow we create a situation that 
seems so complicated. We all create these issues, from the most junior 
to the most senior of us. We can reflect on this and stop attending to 
things that we think are our personal dilemmas. We can observe this 
complicating and unnecessary process, put it down, and realize, 
\emph{This is ridiculous, I don't have to suffer about this at all.} 
When we gain insight in this way, we come back to this peaceful, 
simple, spacious lifestyle that we're offered as renunciants.

