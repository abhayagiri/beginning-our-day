\mychapter{Whose Is This?}{Ajahn Karuṇadhammo}{August 2013}

A few days ago I noticed that just above the door of my kuṭi, there 
is a large colony of yellow jackets that has decided to make its home 
under my roof. Yellow jackets are notoriously territorial. As the 
summer continues on and becomes hotter and drier, there is less food 
and as the colony size grows the wasps tend to get more and more 
aggressive. I have been stung by yellow jackets in the past and their 
stings can be pretty painful. Given these circumstances I knew that I 
needed to take care of the situation. The first thoughts that went 
through my mind when I initially saw them were: \emph{They are in my 
kuṭi. This is my kuṭi and they shouldn't be there. I need to do 
something to get them out.}

I tried to figure out a way to do that so that they would be deterred, 
but not harmed. I thought I could accomplish this by spraying them with 
water, albeit from a good distance. It seemed to me that would 
aggravate them enough so they would leave their nest long enough for me 
to take it down and move it further away. I've sprayed them a number of 
times over the past few days and it does, indeed, stir them up. In 
fact, a couple of them have found me more than twenty feet away and 
I've been stung twice. After the first couple of days observing them in 
their territorial and protective nature, I was reminded of how really 
foolish it is to have any sense of ownership around ``my kuṭi.'' It's 
likely that their sense of possession is even stronger than mine to the 
point of aggressively defending their position. For them it is ``their 
kuṭi,'' not mine, and they are simply defending their territory 
against me. This is another reminder of the sense of ownership and 
control we have, the sense we have of being the center of experience 
and generally the center of the universe. We tend to carry this sense 
around with us most of the time.

So it's beneficial to pose these questions to ourselves: \emph{To whom 
does this belong? To whom does this kuṭi belong? To whom does this 
monastery belong?} The other night in Vinaya class we were talking 
about the ownership of property, especially as it pertains to monks. We 
say the Abhayagiri Monastic Corporation owns the property of the 
monastery. Well, does it really? What about all the other beings that 
exist in the buildings? They probably have a sense that they own the 
whole monastery.

This feeling of ownership also extends to all of the things that give 
us a sense of who we are, particularly the human body. We tend to think 
of this body as ``my body, my foot, my knee, my back,'' and do what is 
necessary to defend ourselves against all of the situations that might 
come and threaten the body. We treat illnesses and we take care of 
whatever has befallen us with a sense of possession, a sense that this 
has happened to ``my body.'' We take antibiotics or put on different 
ointments and salves to defend the body against all of the organisms 
that are taking over. But the millions of organisms that live in our 
bodies may think this body is theirs (if they do think). It's helpful 
to ask ourselves: \emph{Whose body is this? Whose feelings are these? 
Whose thoughts are these? Whose opinions are these? Whose neuroses are 
these? Whose problems are these?} We have such a strong sense of 
ownership around all of this. We hope to do whatever is possible to 
reasonably maximize comfort and skillfulness, yet, if we have a sense 
of personal attachment, personal identity to any of this, then at some 
point when things change, we are bound to suffer.

This is a reflection we can keep in mind as we practice throughout the 
day. Even though we spend a lot of time developing skillful means and 
use a fair amount of intention and will, we need to ask ourselves, 
\emph{Who is in charge? Whose is this? To whom does this all belong?} 
We don't have to come to any specific logical conclusions. These 
questions are asked to open up a space for uncertainty and they help us 
see our bodies as changing phenomena over which we truly do not have 
any ultimate control.

