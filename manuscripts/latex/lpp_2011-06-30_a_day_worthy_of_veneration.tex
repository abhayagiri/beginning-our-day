\mychapter{A Day Worthy of Veneration}{Luang Por Pasanno}{June 2011}

This evening is the night of our Lunar Observance. It's an opportunity 
to recollect the refuges and precepts, and to reflect on the direction 
we want to guide our spiritual practice. In the Thai language, the 
Observance Day is called the \emph{Wan Phra}, which means ``Holy Day'' 
or ``Monk Day,'' or maybe ``Day Worthy of Veneration.''

As Ajahn Chah used to say, the Buddha made these special days part of 
our training, and it's not too much for the Buddha to have asked that 
we observe them---to have regular days worthy of veneration. We do it 
once a week. Over a month, we get four days worthy of veneration and 
twenty-six ordinary days, which is a good balance. The problem is we 
sometimes treat the days of veneration much like ordinary days, so we 
end up with lots of ordinary days and not so many days worthy of 
veneration.

The Wan Phra offers us a quiet time, a day for stepping back from the 
busyness of our ordinary activities. Here at Abhayagiri, we 
consistently encourage everyone to do that. For instance, we remind the 
community to refrain from turning on the computer or continuing with 
work projects. The Observance Day is a time to step back and keep the 
mind from being cluttered with these kinds of activities.

By truly observing the Wan Phra, we have a regular day set aside for 
reflection, a day to ask ourselves, \emph{What do I take as a refuge? 
What is worthy of veneration? How well do I interact with the world 
around me? How might I cultivate virtue and integrity? How do I want to 
live my life?} Reflecting like this is not aimed at setting the highest 
possible standards for ourselves; it's not for trying to achieve 
theoretical ideals. It's simply to recognize what's useful---what works 
to decrease the discontent and suffering in our lives. Discernment of 
that nature, cultivated on Observance Days, motivates us to turn 
towards relinquishment, giving up, and letting go of things. This is 
central to the Buddha's teaching and to the very ethos of our practice.

Observance Days are an opportunity to step back from the busyness of 
ordinary activities, to reflect on our lives and practice and to 
cultivate letting go. When used in these ways, they truly become days 
worthy of veneration.

