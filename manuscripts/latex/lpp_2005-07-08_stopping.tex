\mychapter{Stopping}{Luang Por Pasanno}{July 2005}

The other night Ajahn Sucitto gave a Dhamma talk about the type of 
\emph{kamma} that leads to the end of kamma. And one type of kamma he 
spoke about was \emph{stopping}. The importance of stopping is often 
overlooked. We get so caught up in doing, becoming, activity, and 
engagement that we don't attend to stopping---it's a neglected aspect 
of our practice. This isn't about sitting around doing nothing, because 
that's a form of doing as well. It's about stopping habits of greed, 
irritation and confusion. It's about stepping back and ceasing.

Throughout the day, during work periods and in meditation, we can 
reflect on how much the mind gets swept up in activity and in 
identifying with that activity. We can also reflect on how this 
tendency moves the mind toward restlessness and agitation from which we 
lose our center: One little push, the ball starts rolling, and pretty 
soon we've worked ourselves into a state of anger and conflict. With 
desire, it often starts with the merest sort of interest, which then 
becomes liking, fascination … and then we begin to lose control. We 
don't know how to stop this momentum of desire because we're not used 
to stopping. It's also helpful to reflect on the occasions when we do 
happen to stop. We might ask ourselves, \emph{Does stopping mean I 
can't function? Or is it more so that when I stop and stand in 
awareness, I accomplish things more skillfully than when the mind is 
swept up by moods and thoughts?}

These reflections help motivate us to learn how to stop the mind, how 
to stop the flow of proliferating thoughts and moods that draw us into 
attraction or aversion. If we learn how to stop the mind and exercise 
that skill frequently, then even in challenging circumstances, we will 
be able to bring things to a point of stillness inside and return to a 
clear center of awareness.

So how do we learn to stop? When we feel the mind moving and 
proliferating, how do we stop and come back to a place of awareness 
where we can attend to what the mind is doing and is intending to do? 
To begin, we can recognize when the mind has stopped on its own. Then 
we can observe what stopping itself feels like---the actual 
\emph{experience} of ceasing to engage with mental activity and mental 
impulses. Once we become familiar with how stopping feels, we'll know 
when our conscious efforts to cultivate stopping are on the right 
track. It's helpful to make this cultivation into a regular 
exercise---something to frequently work, play, and experiment with.

Certainly we need to engage in activities and attend to our duties and 
responsibilities, but we can bring an attitude of stopping into the 
midst of this activity, whether it's the stopping of obsession, worry, 
fear, competition, aversion, or whatever. We may try to replace such 
unwelcome mind states with something we feel is more appropriate for 
``good Buddhists,'' but trying to replace one thing with another is yet 
another form of doing. Instead, if we simply attend to stopping, we 
learn to trust in fundamental clarity and wisdom.

