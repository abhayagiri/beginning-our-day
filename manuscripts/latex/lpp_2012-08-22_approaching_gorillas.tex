\mychapter{Approaching Gorillas}{Luang Por Pasanno}{August 2012}

Bring mindfulness to your day-to-day activities. We need to keep 
reminding ourselves of that. It's easy to say to ourselves, \emph{Just 
be mindful.} But we hear it so often that it can seem trite and can 
lose its impact. The practice of mindfulness, the application of 
mindfulness, learning how to direct mindfulness, are key foundations of 
our practice and training. Once we learn how to establish and balance 
mindfulness, we can more clearly understand the mind and its reactions.

The biologist George Schaller was the first person to study silverback 
mountain gorillas. People who read his studies were amazed that he was 
able to get close enough to the gorillas to observe their habits and 
social structure and even the primitive language they used. People 
would ask how he was able to achieve this. His answer was that he 
approached the gorillas in a manner that was very different from the 
way others had tried and failed: he never carried a gun. This allowed 
him to establish a trusting relationship with the gorillas and approach 
them with a sense of ease.

Now, think of what our minds are doing all the time. We carry weapons 
around–our ``guns'' of views and opinions, our moods and biases, our 
reactivity, and all our extra baggage. They keep us from getting close 
enough to clearly see what the mind is doing, its underlying 
tendencies, or what its potential capability might be. We're not able 
to approach our experience mindfully.

We need to put down the weapons of our likes and dislikes, our views 
and opinions, and the constant chatter and commentary that we're always 
overlaying onto the world around us. These weapons constantly separate 
us from the world and from our experience. Once we've put our weapons 
down, we can take up the fundamental quality of mindfulness: we can pay 
attention and be aware without moving in with reactivity. That way, we 
can begin to establish a quality of real peace, while also learning to 
coexist with our own minds.

