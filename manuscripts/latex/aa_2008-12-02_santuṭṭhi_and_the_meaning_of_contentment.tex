\mychaptertoc{Santuṭṭhi and the Meaning of Contentment}{Santuṭṭhi and
the Meaning\\of Contentment}{Ajahn Amaro}{December 2008}

Contentment, or \emph{santuṭṭhi}, is often talked about in the 
context of material possessions, particularly in our reflections on the 
four requisites of robes, alms-food, lodging, and medicine. It's the 
quality of being content with whatever is offered---the food that's 
presented to us each day, whatever shelter is available to us for one 
night, and whatever robes and medicine are accessible to us. That's an 
important aspect of contentment, and a very grounding one---to have few 
material needs and few material possessions. There's a story about a 
man who found everyone of his possessions torched in a large bonfire. 
All that remained were some ashes and the buckles from his boots. His 
response was, ``Well, good, now I don't have to bother with all that 
stuff anymore.'' This is what it means to be content with what we 
have---doing without if need be or living in our dwelling places with 
the thought that it is only a roof over our head for one night. This is 
our training.

It's also important to consider how santuṭṭhi permeates all aspects 
of the path. It's not solely a matter of renunciation, or not being 
moved by desire, agitation, or fear in terms of material things or how 
we relate to other people. Contentment is also the basis for 
concentration, \emph{samādhi}---that quality of being content with 
this moment, this breath, this footstep, this feeling in the knee, this 
sound in the room, this quality of mood. To be content with just this 
moment is of enormous importance in terms of samādhi, concentration. 
And if there's discontent---I have to become more concentrated, gain 
more insight, be more comfortable, I have to, I have to---then even 
though we may be well intentioned, discontent continually creates a 
cause for agitation and a lack of focus. As a result, samādhi is far 
away.

The quality of contentment can drift off in certain ways. It can turn 
into dullness or laziness, or an urge to switch off, while at the same 
time we're thinking of ourselves as being content. Actually, we're 
simply steering the mind toward numbness, a non-feeling state, or a 
feeling of wanting to getting rid of, not bothering with. That's not 
contentment. And contentment isn't a quality of begrudging resignation. 
Oh well, I'm stuck with this mood, this particular problem or feeling. 
I'll just grit my teeth and bear it. I'll just wait for this to be 
over. This is merely dullness, a nihilistic attitude. By contrast, when 
we are content, there is a bright, radiant quality present. Contentment 
has a great lightness and clarity to it.

Contentment can also drift off in the opposite direction, into 
complacency, self-satisfaction, being pleased with ourselves. I'm fine. 
I don't need to do anything with my mind. Everything is perfect. That's 
taking it too far in the opposite direction. Contentment is a bright 
and energetic state, but it's also free from self-view and 
self-centeredness. It's not colored by an I-me-mine attitude.

Santuṭṭhi is not only a basis for samādhi, but also for 
\emph{vipassanā}, insight. It's the ability to be content with seeing 
this feeling, this thought, this mood, or this memory as a pattern of 
nature. We're not buying into it, trying to read a story into it, or 
claiming it as self or other. Contentment allows us to leave things 
alone. A painful memory arises, does its thing, and ceases. An exciting 
fantasy arises, does its thing, and ceases. An important responsibility 
arises, does its thing, and ceases. That's it. This is a characteristic 
of contentment, we're able to leave things alone. The 
\emph{saṅkhāras}, the patterns of the world---we can let them be. 
It's not because we're switching off, we don't care, or we're 
resentfully resigned to some situation. Rather, it's a gentleness and 
presence of mind, a sense of the fullness of being. We're not needing 
to extract something from this thought, this feeling, this moment, or 
this experience.

Vipassanā is based on being able to attend simply to the process of an 
experience, rather than buying into its content. This requires 
restraint and, in particular, sense restraint: not maneuvering to get 
more in the way of our requisites, not getting fussy and picky in terms 
of robes, food, shelter, or medicine. It's very basic. But the things 
we learn on such a basic, material level reach right to the core of our 
training, our spiritual practices, and the development of insight. It's 
the same with the quality of contentment---being at ease, tapping into 
the fullness of Dhamma, the completeness of Dhamma. It's always here 
because contentment is related to the quality of wholeness. It reflects 
the wholeness and the completeness of Dhamma. Nothing is missing, 
nothing needs adding or taking away. Knowing this, we can be content 
with the way things are.

