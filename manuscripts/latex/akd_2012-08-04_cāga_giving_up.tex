\mychapter{Cāga: Giving Up}{Ajahn Karuṇadhammo}{August 2012}

I was listening to Ajahn Yatiko give out the work assignments and 
observing how he adjusted midcourse as he received information about 
all the people who were or weren't here and who did or didn't do a 
particular task. It was easy to marvel at the flexibility and ease he 
conveyed as he juggled around in his mind what was going to take place 
with the complicated work assignments and multiple tasks.

I remember when I was the work monk for a few months. It wouldn't be 
impolite to say that it was a disaster. At least that's what it felt 
like to me. Hearing Ajahn Yatiko just now, I was reflecting on the 
different ways that people offer themselves and the different skills 
they have. There is an impulse to give, to serve, to offer oneself to 
the community in whatever way one can.

During my retreat, I spent some time reflecting on the theme of giving. 
In the Pāḷi language there are a couple of different words used for 
giving. There is \emph{dāna}, generosity or giving, and \emph{cāga}, 
which has a broader scope than generosity. Cāga refers to the aspect 
of giving as well, but it can be used more with the tone of giving 
over, giving to something, or giving up in terms of relinquishment. 
There is a sense of handing over or giving toward some higher ideal 
that can manifest in many ways. We can give up a material object as an 
offering with a sense of selflessness. This might be an object we like 
or something we would like to keep, but for the sake or benefit of 
somebody else we give it over to someone. With cāga, when we give, it 
is not with a feeling of loss but rather with a feeling of fulfillment. 
There is a sense that we are getting more from giving up, from 
relinquishing than from holding on.

In our daily lives and in the monastery we give up time to people, give 
up self-concern and ask ourselves, \emph{What can I do to make life a 
little bit easier or a little more pleasant for somebody else?} We may 
see that someone is overworked or overstretched in what they are doing 
and we keep our eyes open for the opportunity to make a gesture of 
giving. For the residents here, it might mean that we don't go up to 
our \emph{kuṭis} for another fifteen minutes while we help somebody 
do something, giving up what we would like to do for the benefit of 
somebody else.

There is a story from the suttas about three monks who are all 
practicing well and living harmoniously. When questioned by the Buddha 
about this, one of the monks explains that his success is due to asking 
himself, \emph{Why should I not set aside what I wish to do and do what 
these venerable ones wish to do?}

We can also think about cāga in regard to our views and opinions. This 
is a monastery with a lot of independent-minded folks, some of whom 
have strong personalities (I put myself in that category). There can 
sometimes be a sense that I know the best way to do something, 
therefore that is the way we should do it. It can be a real workout to 
notice that inclination, suspend it, and acknowledge that even if we 
are right, maybe it's not the best way to proceed. There are many 
collective ways that we do things in the monastery, and even if we 
don't agree with them, we can acknowledge we have a common agreement to 
do as the group does and let go of our views and opinions about the 
matter. This is what we all signed up for. We operate in a container, 
the monastery, that has a fair number of protocols. In this way we give 
ourselves over to the community in the form of cāga, maintaining 
harmony even if it means letting go of some of our views and opinions.

On a more transcendent level, we are giving up and giving over to the 
Dhamma, relinquishing to the practice and the training for the 
realization the Buddha talks about as the endpoint of our practice. 
Whatever it is that we find we are holding to, clinging to, adhering 
to, we meet with an attitude of cāga, of giving up, giving over, and 
relinquishing. It can be a formidable task to let go because we often 
hold onto our old habits of finding temporary happiness in temporary 
things. We give up those tendencies for the benefit of having a 
long-term sense of satisfaction and completion. Cāga is one of those 
qualities that leads to satisfaction, and we can develop it in ways 
toward ultimate realization.

