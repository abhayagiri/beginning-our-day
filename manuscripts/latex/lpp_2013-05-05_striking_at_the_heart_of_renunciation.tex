\mychapter{Striking at the Heart of Renunciation}{Luang Por 
Pasanno}{May 2013}

One of the teachings Ajahn Chah emphasized most consistently was on the 
theme of uncertainty---that everything is not for sure. In a monastery, 
for instance, it's common for the number of visitors to increase, like 
today, and then decrease; they're here for a while, then they 
disappear. This creates a constant sense of circumstances being 
uncertain, always changing.

We tend to conceive of our practice and training as being under 
conditions over which we have some control. We can take this 
opportunity today to investigate and reflect upon how we deal with our 
practice when it becomes apparent that conditions are really not under 
our control, when the monastery becomes an open house for more activity 
and people.

When circumstances change in a way we like, we tend to get excited and 
happy; when they change in a way that we don't like, we may get upset 
and irritable. In both cases, we're being swayed by the circumstances 
around us, and that's a shaky foundation on which to build our 
practice. So how can we stay present with what's happening without 
getting lost in the changing circumstances? The way to do that is 
through the quality of our attention---how we direct our attention, how 
clear we are, and how mindful. But mindfulness isn't always going to be 
present. For this reason, establishing an internal quality of 
renunciation is quintessential for us as practitioners.

The rules and conditions that are part of living in a monastery create 
a framework of external renunciation, such as the giving up of material 
things. That framework is in place to support an \emph{internal} 
quality of renunciation, which needs to be cultivated at all times. 
Internal renunciation means we're not desperately holding on to 
circumstances, moods and feelings. Giving up material things is not 
that difficult. When we give up our moods, views, opinions and 
preferences we are striking more at the heart of renunciation.

We learn to bring this quality of renunciation into our daily lives and 
interactions with others, so that, when circumstances change, we can 
let go and adapt to them. Today, for example, there's a large group of 
people visiting and things are happening all around us. If we try to 
control the situation---try to make things happen the way we 
want---we're likely to create problems. But everything will run 
smoothly if we turn to this internal quality of renunciation, doing our 
best by letting go of our moods, views, and preferences.

