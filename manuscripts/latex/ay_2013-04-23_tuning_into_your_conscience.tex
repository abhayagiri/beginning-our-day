\mychapter{Tuning Into Your Conscience}{Ajahn Yatiko}{April 2013}

There's a certain danger in these everyday reflections. For those of us 
who are community residents---who are here all the time and have 
listened to these reflections every day for years---the danger is that 
we might develop an attitude that has us thinking, \emph{The senior 
monk is just doing his thing, giving a reflection, and it doesn't 
really affect me. I don't need to pay attention or internalize any of 
it.} \emph{Here's my chance to zone out and think about something else.}

If that is our experience then it's a real loss, because the purpose of 
these reflections is to help create the conditions for all of us to 
practice and positively influence the atmosphere of the monastery in 
some way. Some reflections may include information or a practical 
reminder, but the essential purpose, motivation, and intention is to 
provide a form of communication and energy that's going to uplift, 
encourage, and support the community.

As an encouragement today, I'd like each of us to tune into our own 
situations. On some level we're the ones who know what needs to be said 
and what we need to hear. So spend a couple of minutes in silence 
tuning into your conscience and asking yourself, \emph{What is it that 
I need to hear? What is a good thing for me to hear? What kind of 
reflection is really appropriate to my situation now?} The purpose is 
not to make us comfortable and happy. Rather, it's to foster an 
understanding from which we can say to ourselves, \emph{Truly, this is 
what I need. This is what's going to be good for me and my practice.} 
So spend a few minutes imagining what that reflection might be. This 
requires sensitivity---a tuning into the conscience. Please, take a few 
minutes for that.

