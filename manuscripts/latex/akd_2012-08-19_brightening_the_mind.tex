\mychapter{Brightening the Mind}{Ajahn Karuṇadhammo}{August 2012}

Recent discussions and Dhamma talk reflections have focused on the 
theme of right effort. Many of us can be so caught up in what we think 
of as Dhamma practice or meditation practice that we create a narrow 
focus for ourselves. Several of us here came to Buddhism with a focus 
on the practice of meditation in the context of silent retreat, 
oftentimes with a very specific technique related to quieting the mind. 
Sometimes it's easy to get the idea that Buddhist practice boils down 
to right concentration or a specific meditation method. But the 
Buddha's Eight Fold Path is much broader than that. Right effort, one 
of these path factors, is a significant skill to be developed. With 
right effort, we are learning to be sensitive to the effect our minds 
have on what it is we're doing. We are examining and asking ourselves, 
\emph{Are wholesome states increasing or unwholesome states 
decreasing?} That's our benchmark. That's our measure of whatever it is 
we're doing in what we call our Buddhist practice.

For instance, the theme today for Upāsikā Day, ``Brightening the 
Mind,'' focuses on the formal side of meditation, which is often---but 
not explicitly---talked about in terms of contemplation. There are many 
kinds of active reflections the Buddha gave to help prepare, brighten, 
and soften the mind. And by building on those qualities we can develop 
more refined states of concentration. So that is our orientation---to 
learn how to develop and prepare the mind, not only with formal 
meditation practice, but also in our daily activities.

The rest of the path is oriented toward the other supports and 
developments of right view and the attitudes and actions that are part 
and parcel of our practice to liberate the mind. We follow this path by 
paying attention to whatever it is we are doing, in the livelihood 
we've chosen and in the activities and various ways we express 
ourselves in body, speech, and mind. We can ask ourselves if what we 
are doing or thinking is leading toward wholesome states of 
mind---toward more peace, contentment, and satisfaction, or toward less 
stress, suffering, clinging, and holding on to unskillful habits of 
responding in the world. We can use that as the benchmark around all of 
our activities, everything we do. This in turn allows us to let go of 
the belief that meditation or practice is something we do only after 
our activities are done, when we get up to our kuṭis or find a place 
to sit quietly. We are gaining insight into the perspective that it's 
all practice right here and right now, toward the increasing of 
wholesome mind states and the decreasing of the unwholesome ones.

