\mychapter{Developing Samaṇa Saññā}{Luang Por Pasanno}{July 2005}

Yesterday, at the City of the Dharma Realm in Sacramento, I was giving 
instructions to twenty-eight \emph{shramaneris} who were preparing to 
ordain as \emph{bhikshunis} next month. It was quite a delightful time. 
Their sincerity was tangible.

One of the ideas I brought up with them is developing \emph{samaṇa 
sañña,} the perception or recollection of being a religious seeker. 
We function out of our perceptions. We perceive something to be 
interesting or desirable, and we get excited. We perceive something to 
be worrisome or troublesome, and we start to have aversion or 
negativity. Perceptions are always informing how we relate to things. 
The Buddha encouraged us to develop the perception or recollection of 
being religious seekers. In that way, we can relate to the 
circumstances in which we find ourselves, and to the people with whom 
we live from a very different perspective. By perceiving ourselves to 
be religious seekers---those who are seeking peace---we encourage 
ourselves to always relate and act in the best possible manner.

So how do we conduct ourselves? What do we do as seekers? How do we 
engage in our responsibilities and duties in order to maintain a 
quality of peace? How do we fulfill that aspiration? We do this by 
reminding ourselves and recollecting, \emph{Yes, this is what I am, 
this is what I'm doing and most valuable for me to be doing---seeking 
peace, seeking truth}.

As we take on duties or have contact and engagement with each other, we 
can relate to each other as fellow seekers of truth and peace, rather 
than as objects of aversion, attention, or interest, or just somebody 
else who can fulfill a function. There are times when we may think, 
\emph{There's the person who does the computer work. There's the person 
who is the kitchen manager. There's the person who does this or that}. 
We might see that person only in a particular role or having a certain 
type of personality. This really limits us and limits everybody else as 
well.

Instead, we can recollect ourselves as samaṇas and develop a 
perception of each other as fellow seekers of peace, fellow seekers of 
truth. This helps us support our own practice, our own daily living in 
a way that is peaceful and encourages us to live skillful lives in the 
monastery.

