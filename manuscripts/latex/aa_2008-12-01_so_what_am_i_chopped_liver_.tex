\mychapter{So What am I, Chopped Liver!?}{Ajahn Amaro}{December 2008}

A theme we explored on the Thanksgiving Retreat was ``complaining and 
blaming.'' I thought it would be a useful theme because our culture 
tends toward complaint. \emph{If I'm suffering, then the way to the end 
of suffering is to complain or blame. I'm suffering, therefore it's 
somebody else's fault. I've been treated unfairly. This isn't right. It 
shouldn't be this way.} This is powerful conditioning in our lives. I 
remember a New Yorker cartoon with a student asking a monk, ``You say 
life is suffering, but isn't it also complaining?'' It's useful to take 
a period of time to reflect on the unconscious or semiconscious way we 
react to the experience of suffering---to reflect on the urge to be 
critical, to be negative, to complain, or to find fault in ourselves or 
in the things around us. While reflecting like that, we can broaden our 
view by inquiring into the matter. \emph{Why do I think I shouldn't 
have to experience this illness, this pain, this weather, this food, 
this person sitting next to me?}

Then we can broaden our view further by consciously evoking a sense of 
appreciation and gratitude for the gifts and opportunities we have in 
our lives. This is a way to catch the mind's habitual movement toward 
criticism or complaint, its movement toward the classic 
glass-is-half-empty attitude. Evoking gratitude goes directly against 
that complaining, criticizing, blaming mind. But we need to make sure 
that this gratitude isn't based on a ``think pink'' attitude---trying 
to sugarcoat things and pretend that we're not really feeling critical 
or negative. It doesn't help much to paste an artificial expression of 
gratitude on top of a negative mood or a feeling.

We begin with listening to the critical, blaming, or complaining mind, 
and hearing what that mind is saying. What's it coming up with? Is it 
the feeling of being unfairly treated, slighted, left out, or ignored? 
Can we can hear the mind's cry of righteous indignation, \emph{So what 
am I, chopped liver?!} We receptively listen to the affronted, hurt, 
wounded, abandoned, irritated feelings, and hear the mind coming up 
with the reactions and thought processes that follow those feelings. We 
are simply allowing this experience to be known---this narrow, painful, 
reactionary state of complaining or feeling slighted. By bringing 
awareness to that, fully knowing its reactive quality, we can recognize 
and inquire, \emph{This is a really painful state. Why would I choose 
to react like this? Why would I want to carry this around and burden my 
heart with this?} We're not saying to ourselves, \emph{Oh, I'm supposed 
to be grateful now, I should plant some gratitude in here.} Instead we 
are simply seeing the painfulness of our narrow, self-centered 
reactions. Once we see this, then the very acknowledgment of that 
painfulness can enable us to let go and relax. In the broadening of our 
views and attitudes, what arises is gratitude. We are able to 
appreciate the bigger picture, the gifts and the lessons we have 
received, and the potential opportunities we have in the world.

