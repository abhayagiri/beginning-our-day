\mychaptertoc{Sweep What's in Front of Your Broom}
{Sweep What's in Front\\of Your Broom}
{Luang Por Pasanno}{August 2012}

We have a very full schedule here at the monastery over the coming 
days. That's how it is, sometimes there's a lot happening. When this 
occurs, it's helpful to have a perspective that doesn't make things 
complicated or difficult.

I remember at Wat Nanachat, one of the things visitors were asked to do 
in the morning, while the monks were out on alms round, was to sweep 
the monastery grounds---and it's a fairly large monastery. One morning, 
while an \emph{anagārika} was sweeping, a new guest came out, looked 
at the grounds, and said, ``Are we supposed to sweep all of this?'' The 
anagārika replied, ``No, just what's in front of your broom.'' It's 
helpful to keep this sort of perspective.

But we can easily stray from that and tell ourselves, \emph{Oh, there's 
this to do, and that to do; there's this person, and then that person.} 
It turns into something that sounds complicated and overwhelming. In 
the end, though, it's just the person in front of us that we're dealing 
with, the particular chore or task that needs to be done now, breathing 
in and breathing out, moment by moment.

Here in the monastery, once the morning work period is over and the 
meal is finished, then there's the afternoon. Take the time in the 
afternoon for meditation, for reading Dhamma, for some quiet time. No 
need to think about the things that may need doing in the future.

Remember that it's only what's in front of us that needs to be done. As 
we maintain that perspective, we realize that things do get done. They 
may not get done as quickly as we wish, or in the way we think they 
ought to be done, but we can only do what we're doing. It's helpful if 
we don't lose ourselves in a lot of thinking and complication.

That's a big part of personal practice. Without that we might start 
thinking of all of the things we need to do to become a proficient 
meditator or practitioner: \emph{I've got to get my precepts down, 
learn the Vinaya, learn the chanting; get this meditation technique, 
and that meditation technique, and this other one I haven't even tried 
yet; and there's this reflection and that understanding of this aspect 
of Buddhist philosophy.} And then we might think, \emph{Oh, this is 
hopeless. I'm just giving myself more suffering and more difficulty 
than I ever had before!}

But if we drop all of that and attend to just one breath at a time, to 
one mental state at a time, and that's all---if we can attend to things 
from that perspective---then everything is doable.

