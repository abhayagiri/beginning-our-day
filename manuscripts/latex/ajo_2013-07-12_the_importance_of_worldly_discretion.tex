\mychaptertoc{The Importance of Worldly Discretion}
{The Importance of\\Worldly Discretion}
{Ajahn Jotipālo}{July 2013}

One word mentioned in the Mettā Sutta is \emph{nipako}. It is often 
translated as ``wise,'' ``chief,'' or ``highest.'' Bhikkhu Boddhi 
translates it as ``discretion'' and talks about it in terms of worldly 
skills or practical wisdom. One way to reflect on this translation of 
nipako is in the realm of human relationships. By looking back on a 
conversation, for instance, we can ask ourselves whether we had been 
gentle and appropriate in that situation, and whether we had been 
deliberately paying attention to the quality of our speech at that 
time. In this way we can hone our communication skills and learn how to 
live more harmoniously in community. When we get upset or are involved 
in a misunderstanding or miscommunication, this word nipako points to 
how we can use discretion and practical wisdom with this type of 
experience. We can learn how to skillfully approach a person and work 
out a problem even in a situation that is difficult or uncomfortable.

Nipako comes from the root word \emph{nipa,} which means ``to lay 
low.'' So I think of it as not jumping into a conversation with the 
first thing that comes to mind, and not trying to be the first person 
to come up with a quick solution. Instead, it is more akin to paying 
attention and watching.

We can also think of nipako in terms of learning and seeking guidance. 
For example, it's important to take the time to learn how things work 
instead of jumping in and trying to fix some particular thing when we 
really don't know how. Otherwise, we might damage what ever it is, and 
create a bigger problem in the long run. We need to learn to ask. This 
applies to meditation as well. It's good to try new approaches and 
explore different techniques on our own, but it's also good to ask and 
seek advice from those with wisdom.

So with this quality of nipako we can apply our discretion and worldly 
skills in all situations. We can learn to ``lay low'' with 
circumspection, while learning from those who know what we don't. These 
are all qualities highly commended by the Buddha.

