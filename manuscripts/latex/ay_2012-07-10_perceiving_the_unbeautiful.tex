\mychapter{Perceiving the Unbeautiful}{Ajahn Yatiko}{July 2012}

Yesterday Luang Por Pasanno spoke on the theme of \emph{asubha 
kammaṭṭhāna}---contemplating the unbeautiful. This topic can be 
embarrassing for Westeners to talk about because it seems like such a 
foreign and strange concept---deeply personal, almost taboo. Why do we 
contemplate the unbeautiful? The answer isn't immediately obvious. But 
if we look at this question in terms of our own experience, we begin to 
see its benefit in our practice.

There is a strong tendency to judge and slap a perception on everything 
we experience, and it could be said that whatever we experience is 
filtered through these perceptions. In particular, we habitually 
distinguish objects as either beautiful or unbeautiful. There is 
something within us that creates this perception. But in fact, it's 
merely a perception, not the ultimate truth of things; we can perceive 
anything in the world as being beautiful or unbeautiful depending on 
the way we've been conditioned to look at it. But when we look at the 
body, we most always see it as beautiful and ignore its unbeautiful 
features. That's our default bias. There is a deeply ingrained tendency 
towards creating and perceiving the beautiful in the body, to become 
attached to it, worry and fight about it, and become distressed when we 
lose it.

We all have \emph{rāga-taṇhā}---sensual desire---as part of our 
being. If our parents didn't have rāga-taṇhā, we wouldn't be here. 
When it's present, we want to perceive the beautiful in the body. When 
it's not present, the body doesn't seem so beautiful anymore---it 
doesn't appeal to us in the same way. We're no longer in the grip of a 
desire that can lead to painful mind states and regrettable behavior. 
This is where asubha practice comes in: When we start thinking about 
what comprises the human body---bones, sinews, blood, intestines, 
undigested food, the heart, the whole lot of it---we may come to 
realize that these objects have an unattractive quality to them.

But when left to our own devices, we tend to only focus on the thin 
layer of skin that wraps around these unattractive objects, ignoring 
everything else. Or we convince ourselves that the skin is of a 
different nature than the rest of our body---but how could that be so? 
How could the skin be of a different nature? The unbeautiful quality of 
skin and of what it contains are exactlty the same, but because we've 
created a fixed perception that the skin is beautiful, we see it in 
that way.

The fact is, nothing in the world is inherently beautiful or ugly. 
These are merely qualities we create and project onto the world. We can 
create an entire universe of perceptions like those, which then dictate 
to us what we want or don't want. It becomes a heavy experience---a 
heavy reality---which is completely unnecessarily. Asuba practice can 
help liberate us from this tyranny of entrenched perceptions.

Initially, there can be a strong resistance to asubha kammaṭṭhāna. 
This resistance is rāga itself---the desire that does not want to give 
up the perception of beauty. Recognizing and investigating this 
resistance can be very interesting and a profound experience in its own 
right as well as a direct way of working with fear. Even so, we might 
say to ourselves, \emph{I didn't take up the Buddha's teachings to 
focus on the unbeautiful!} But actually, this is a significant part of 
the Buddha's teachings, and it doesn't in any way overshadow the beauty 
of those teachings. In fact, the feelings that arise from asubha 
practice are themselves quite beautiful---feelings of letting go, 
release, and freedom. It also helps, in overcoming our resistance, to 
recall that asuba practice is to be done within the supporting 
framework of loving-kindness towards ourselves and others.

However, we \emph{do} need to apply this practice with discernment, to 
watch out for any difficulties that arise. If we find ourselves getting 
negative results from the practice, it's perfectly fine to put it 
aside, for years if necessary. But at some point it's worth working 
with the difficulties, because asubha kammaṭṭhāna is such a 
potentially rewarding practice. Eventually, if we use this practice 
repeatedly, we will likely discover that it's an extremely useful tool 
for the realization of Dhamma.

