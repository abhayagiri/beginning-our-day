\mychapter{We Don't Have to Struggle}{Ajahn Jotipālo}{November 2013}

Before coming to this practice, most of us had positive ideas about 
meditation---that it brings peace and happiness, even bliss---and those 
ideas are what motivated us to become practitioners. For many of us, 
however, the reality is that meditation can often be a struggle, rather 
than an experience of bliss and happiness. When we close our eyes, 
there's can be a lot of physical or mental pain. Or there can be a 
sense of pushing away from our experiences, or a general feeling of 
discomfort and tension.

When I think about my own practice, what often creates the struggle I 
experience in meditation is my attitude. I can have an attitude that 
I'm only meditating to experience peace, not pain. That's a problem. So 
I can ask myself, \emph{Is it possible to drop that attitude and 
relax?} If I can drop the attitude, I have the capacity to bring that 
sense of peace and tranquility into one single moment---the present 
moment.

Alternatively, when we find we are unable to bear with a particular 
form of pain, we can also turn our attention away from that pain. If 
there is pain in the knee, for example, why put our attention on that 
if our minds cannot hold a sense of steadiness around it? Why bring 
that into our attention? Instead, we can find some other part of the 
body that is not in pain so we can experience that feeling of 
present-moment peace right there.

Ajahn Buddhadāssa used to talk about enlightenment in the sense of 
experiencing this moment of peace with clarity and then becoming 
familiar with that experience. Of course it is impermanent, it comes 
and it goes, but it is a taste. And we can train the mind to incline 
toward that state more and more often, to make it stretch out a little 
longer, and return to it throughout the day. With training, this 
experience of peace and clarity can become a touchstone for the mind, 
and we can turn toward that in meditation when we find ourselves 
struggling with pain or discomfort.

But the first thing to do when we find ourselves struggling, is to 
remember that we do not have to struggle. Then we can focus on the body 
or the breath or whatever it is for us that brings that sense of peace 
and tranquility. Or we can reflect on any on of the Seven Factors of 
Enlightenment---mindfulness, discernment, energy, rapture, tranquility, 
concentration and equanimity---in order to establish a more grounded 
state of mind.

There are many avenues we can take to cultivate the peace and happiness 
in meditation which brought many of us to the practice in the first 
place. And now that we have become committed to the practice, our job 
is to explore those avenues for ourselves, to discover which ways lead 
to the most fruitful results.

