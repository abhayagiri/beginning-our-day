\mychapter{Clean Kuṭi, Clear Mind}{Luang Por Pasanno}{May 2013}

Keeping our kuṭis, our dwelling places, in order helps to keep our 
minds in order as well. So anytime we leave our kuṭis, we should make 
sure everything is put away, neat and tidy.

It's easy to let things slide, to tidy up only once a week, or 
whatever. But if we have a habit of letting things get a bit messy 
before tidying up, we're apt to develop a somewhat lax attitude about 
everything, which would make it very difficult to lift up and sustain 
clarity in meditation. By contrast, when our attitude is to keep things 
tidy, moment to moment, we're developing the same quality of mind 
needed to stay with our meditation object, moment to moment, which 
allows the mind to settle and clarity to arise.

Just as maintaining an orderly kuṭi helps to keeps the dust and dirt 
from finding places to hide in our living environment, it also promotes 
the mental qualities needed to expose those places in the mind where 
the defilements hide away, unobserved. When we're trying to understand 
the subtleties of the mind, we don't want to have dark corners where 
the defilements can hide, because they'll tend to hang out there 
forever. So we need to develop habits like this, which assist us in 
keeping the mind spacious and perceptive. That way, we can see our 
conditioning, our mental patterns---everything in the mind that creates 
suffering and discontent.

This mundane task of keeping our kuṭis in order can be extended to 
our general environment as well. This is a part of our training to make 
the mind clear, steady, and discerning. When we are consistent with 
this, these qualities will become part of the mind's normal way of 
being---its default setting. We won't have to make an effort to lift 
them up in the mind, because they'll already be there for us, primed 
and accessible. This in turn will make the mind bright and ready for 
work.

