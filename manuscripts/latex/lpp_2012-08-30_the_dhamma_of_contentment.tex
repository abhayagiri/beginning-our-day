\mychapter{The Dhamma of Contentment}{Luang Por Pasanno}{August 2012}

Contentment is a good theme for all of us to consider. In doing so we 
want to learn how to be content with the circumstances around us, as 
well as with our own minds, internally.

Most of the agitation, negativity, and faultfinding that the mind 
cranks out is not so much about any big event that's happening outside. 
Almost invariably, it is a lack of internal contentment. When the mind 
is internally unable to find contentment, externally it finds something 
to be excited about, upset about, agitated about, or have an opinion 
about. It's usually really believable! We come up with the logic and 
all of the good reasons to justify our states of mind. There are plenty 
of good reasons if we look for them. But oftentimes, what's overlooked 
is the question, \emph{Why can't I be content with this present moment, 
with this circumstance, with my mind and feelings as they are?}

This is a very important investigation. It's a fundamental basis for 
progress in practice. Until we learn how to direct our attention in 
that way, we're almost always driven by discontent and end up being 
caught up in some sort of sensual fantasy or internal rant or something 
that, at the very least, takes us out of the present moment. The 
challenge is to be able to draw attention to what's arising and 
investigate: \emph{How can I be content with this present moment? How 
can I be content with myself?}

When the Buddha talked about being a refuge unto ourselves and taking 
Dhamma as a refuge, he didn't mean that we take refuge in the Dhamma of 
discontent. Our refuge is in the Dhamma of contentment, the ability to 
not be pulled away from the present moment. This is absolutely 
essential when we're talking about meditation---for the mind to become 
settled, peaceful, and still we need to have the ability to be content 
with the breath or some other meditation object. In the \emph{suttas}, 
the Buddha describes contentment as one of the characteristics of a 
great being or a noble one, an \emph{Arya}. We learn to be content with 
our robes, almsfood, lodging, and with our cultivation, our development 
of meditation.

This aspect of contentment is a fruitful area for investigation. We can 
experiment with it and find ways to draw our hearts closer to that 
quality.

