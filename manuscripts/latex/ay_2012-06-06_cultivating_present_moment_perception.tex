\mychaptertoc{Cultivating Present Moment Perception}
{Cultivating Present\\Moment Perception}
{Ajahn Yatiko}{June 2012}

When we think about something repeatedly, it tends to become a fixed 
perception for us, and we think about it even more. For example, if we 
have a work project and think about it throughout the day, we build up 
that perception and our tendency to think and proliferate around 
it---it becomes a world unto itself, a perceptual world we inhabit. 
This world-creating tendency builds momentum throughout the day, 
throughout the week; when we get up to our \emph{kuṭīs} in the 
afternoon and do sitting and walking meditation, these worlds we've 
created manifest themselves, and we think about them yet again.

But the corollary is also true. If we cultivate the perception of the 
present moment, that too will develop momentum. This does not mean, 
however, that cultivating this perception comes easily. When we get 
onto the walking path, for instance, there may be a tendency to think 
about computers, to think about a work project, or whatever. The 
untrained mind naturally wants to wander or obsess and proliferate---to 
disconnect from what is happening now. So we need to exercise 
discipline, saying to ourselves, \emph{No, now is the time to be in the 
present moment.}

Sometimes getting the mind back on track simply requires us to relax 
our mental efforts. This can calm the mind so it's able to come back, 
seemingly of its own accord. Other times we need to exert a conscious 
effort to reconnect with the present. At first when we do that, we can 
only sustain the connection for a little while before losing it again. 
That's because it's a perception; left to its own devices, it comes and 
goes, like all other perceptions. If we want to sustain this 
present-moment perception, to develop a mastery around it, we have to 
cultivate the perception repeatedly, over and over again. When some 
little obstacle pushes us off in this direction or that direction, we 
do our best to find our way---to find a direction in which we can drop 
all obstacles and come back to the present.

Experiencing the present moment is always a relief and always 
refreshing. If we're not experiencing it that way, it's probably not 
the present moment! It's possible there's some residual attachment 
we're not seeing or acknowledging. So while it is true that cultivating 
a perception of the present moment is our duty, it is also true that 
fulfilling this duty is a most gratifying and uplifting experience. 
Remembering that point, we can undertake our duty with enthusiasm and a 
light heart.

As monastics living here at Abhayagiri, we have the opportunity and 
support to cultivate this present-moment perception. And such 
cultivation is vital---it's an invaluable tool for developing the mind 
and creating a foundation for deep insights into who we are, into the 
nature of this mind and body. So let us use this day for cultivating 
our perception of---and connection with---the present moment.

