\mychapter{Examining Uncomfortable Experiences}{Luang Por Pasanno}{June 
2013}

Two days ago it was the anniversary of Ajahn Chah's birth. Many aspects 
of his life are well worth recalling and reflecting upon. Certainly one 
of them is the practical approach he used to teach and encourage us. He 
always emphasized the importance of reflecting on the Four Noble 
Truths, and the experience of \emph{dukkha}---suffering, dis-ease, 
discontent---and the different ways we create dukkha within the heart. 
In this, Ajahn Chah embodied the quality of fearlessness---he had no 
fear when looking at the uncomfortable aspects of his own experience. 
By contrast, there is a tendency many of us have to try and run away 
from the uncomfortable experiences in our lives, to gloss them over or 
put some kind of spin on them. Instead, we need to look at them 
closely, without feeling intimidated.

I remember an example Ajahn Chah gave about this. He said, ``Sometimes 
you get a splinter in your foot. It's kind of small and you don't feel 
it all the time, but every once in a while you step in a particular way 
that causes you to be irritated by it. So the thought arises, \emph{I 
really have to do something about this splinter.} Then you carry on and 
forget about it. But the irritation keeps returning until at one point 
you say to yourself, \emph{I really need to do something about this 
splinter now. It's not something I can put up with anymore.} So finally 
you dig it out. The experience of dukkha, is the same.''

It's not as if we're experiencing dukkha all the time; for the most 
part, we live incredibly comfortable lives here. But there's this 
recurring sense of dis-ease, discontent, dissatisfaction. We need to 
make a strong determination to investigate it, understand it, resolve 
it, relinquish it, not shrink back from it and really try to dig it 
out. We are trying to understand the nature of the human 
condition---the condition that keeps us cycling back to that feeling of 
dis-ease. It's a willingness to patiently put forth effort and to work 
with our experiences. This is an opportunity for us to learn how to do 
that in our daily lives and to learn how to do that in our formal 
meditation. It's possible to be carrying that investigation with us at 
all times.

