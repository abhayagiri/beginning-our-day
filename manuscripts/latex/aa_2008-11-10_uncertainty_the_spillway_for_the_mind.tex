\mychaptertoc{Uncertainty: The Spillway for the Mind}
{Uncertainty: The Spillway\\for the Mind}
{Ajahn Amaro}{November 2008}

With our friend Jay's health in such a precarious condition, this is a 
good time to reflect on uncertainty. Ajahn Karuṇadhammo was saying, 
``The doctors and nurses keep telling me that Jay can't last much 
longer. But I've seen so many times where, against all reason and 
medical possibilities, people continue to survive for a long time.'' 
This simple recollection can be helpful. We don't know how much longer 
Jay will be with us. Similarly, with the work scene, we don't know 
where the water tank leak is or even if there is a leak. We don't know 
if all the different work crews will get where they're supposed to this 
morning. We don't know if Jim will be attacked by fronds of poison oak. 
\emph{We don't know}.

Throughout the course of any day, there are thousands of different 
situations either on the grand scale, like someone's life ending, or on 
the minuscule level---\emph{Where have I left that hammer? What am I 
going to make to go with the broccoli? Whose driving the truck up the 
mountain?} We don't know. Instead of feeling frustrated because we're 
anxious and without a plan, we simply can recognize, \emph{I don't know 
where that tools is, what to make with the broccoli, or whose driving. 
I don't know what's going to happen next or if this is going to work. I 
don't know.} We are bringing that quality of not knowing into our 
attention, rather than trying to get some information so we \emph{can} 
know, or feeling frustrated or anxious because we don't have a plan or 
haven't figured out a particular problem. When we reflect on not 
knowing, we are letting go of our incessant need to ameliorate 
uncertainty---that refuge we usually take in making sure that 
everything has a plan, an answer, or some worldly solution.

Ajahn Chah would often say that this reflection on uncertainty is the 
flag or emblem of the Noble Ones. He'd also say it's like a spillway 
for a dam. When we build a dam, we need to have a spillway to relieve 
the pressure and divert the excess water. Ajahn Chah would say, 
``Uncertainty is the spillway for the mind.'' That's what relieves the 
pressure in our lives and in our experience of the world.

So remember: It's uncertain, we don't know. When the mind makes a 
judgment, calling this good or bad, right or wrong, we can reflect, 
\emph{Is it really a good thing? Don't know. Is it a bad thing? Don't 
know.} When we cultivate that reflection on uncertainty from a place of 
wisdom rather than from a place of self-view and anxiety, it can serve 
as a spillway that relieves the pressure. Right there we can feel the 
relief in the heart. Often we think, \emph{Oh, this is going to be 
great, now I'll be happy, everything is going to work out.} When 
thoughts like that arise, we need to follow them up with wisdom, 
\emph{How presumptuous to believe I could know that. Of course, I don't 
know.} By frequently reflecting in this way, we can learn to stop 
looking for certainty in that which is intrinsically uncertain. What a 
relief.

As the day proceeds and we go about our work tasks, bear in mind that 
\emph{we don't know}. Has Jay passed away? Is he still alive? Is he 
still breathing? \emph{Don't know.} Whether it's concerning Jay or 
concerning the tasks at hand, notice how the quality of the mind 
depends on whether or not we're seeing uncertainty in relation to 
whatever judgment or activity is taking place. When we see uncertainty 
clearly---the uncertain nature of our lives---we directly experience 
the qualities of relief and ease.

