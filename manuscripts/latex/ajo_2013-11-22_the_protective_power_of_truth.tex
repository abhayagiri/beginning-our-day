\mychapter{The Protective Power of Truth}{Ajahn Jotipālo}{November 
2013}

In quite a few of our \emph{paritta} chants there is the line 
\emph{etena sacca vajjena sotthi te hotu sabbadā}, which is roughly 
translated as, ``by the utterance of this truth may there be safety, 
protection.'' About a month ago I was on my solo retreat reading a book 
about the stories behind the parittas, the protective chants. And there 
was this really touching tale from the \emph{Quails Protection Chant} 
for warding off fires. As the story goes, there is a baby quail that is 
alone in its nest out in a field and a fire begins to approach the 
nest. The quail is too young to take flight and not strong enough to 
get itself to the ground and run off. Its parents are out foraging for 
food. The baby quail is really in a desperate situation so it makes a 
statement of truth, which is, ``I'm alone, without my parents, I'm 
weak, I can't fly or run away, and there is a huge fire approaching. By 
the utterance of this truth may I be protected.'' In the story, the 
wind then shifts and the baby quail is saved.

I was reflecting on this in terms of truth and honesty---really being 
present for what is happening as well as taking stock of the situation 
we're in. If there is some difficult interaction we have with someone, 
we can try to catch our reactions before getting upset or offended and 
try to relate to the truth of our experience. Sometimes when we state 
the simple truth that, \emph{This hurts}, we can cut it off right 
there. We don't have to go any further into the difficult situation. 
And that's a good thing, because if we did go into it further we'd 
likely complicate the matter---\emph{Why did this have to happen to me? 
It's not fair. I didn't do anything to deserve this!}---and then get 
lost in negativity. So instead of doing that, we can emulate the baby 
quail. Even if the wind had not shifted and the quail had died, well at 
least he wasn't blaming anyone, wallowing in self-pity, or being 
negative.

When Ajahn Sucitto was on \emph{tudong} pilgrimage in India, a group of 
robbers attacked him and the layperson he was traveling with. In a book 
they later wrote together, Ajahn's traveling companion said of himself 
that during the robbery he reacted like an animal, fighting and fleeing 
out of fear. Ajahn Sucitto just stayed there with the robbers, quietly 
and calmly dealing with them. When the leader of the robbers held an ax 
over Ajahn's head and was about to murder him, Ajahn Sucitto calmly 
recited, \emph{Namo tassa bhagavato arahato sammāsambuddhassa}, 
``Homage to the Blessed, Noble, and Perfectly Enlightened One.'' He 
came into the present moment, made an honest assessment of the 
situation, and found himself truly ready to die. Needless to say, Ajahn 
lived to tell the tale, and it might well be that his honest presence 
and devotional recitation on that day was what preserved his life.

There is a real protection and power which we receive when, instead of 
focusing on what might happen in a negative situation, we bring 
ourselves into the present moment, honestly and truthfully being 
available to the circumstances we find ourselves in. There is no need 
to wait for fires or murderous robbers to confront us---we can take 
advantage of this protective power in any moment of our day-to-day lives

