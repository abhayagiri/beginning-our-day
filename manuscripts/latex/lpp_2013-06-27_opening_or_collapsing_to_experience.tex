\mychapter{Opening or Collapsing to Experience}{Luang Por Pasanno}{June 
2013}

As we apply our practice to the circumstances in which we find 
ourselves---whether we are working in the kitchen, out on the trails or 
in the office---we apply mindfulness to the present moment while also 
connecting and establishing our continuity of mindfulness to the 
posture of the body. There's a natural tendency, as we are sitting at 
the computer, working in the kitchen or out on the trails, to move 
forward and hunch. We can make this habit conscious, re-grounding and 
reconnecting with the body by opening up the chest and breathing 
comfortably. We are not trying to strut around with a military 
posture–we can open the body, the chest and the heart base by 
expanding our posture. Another way to see it is opening up our presence 
to how we engage with the world around us.

With the habit to collapse in and slump forward we can try to 
continually bring in that sense of an expanding and spacious presence. 
These are little reminders with which we can ask ourselves, \emph{Am I 
present? Am I here?} If we do this, we end up acquiring a continuity of 
awareness that is then brought into our sitting meditation. When we are 
sitting in meditation, if we are slumping and collapsing our posture 
and not really putting effort into the present moment, we tend to 
collapse in the mind as well. Ajahn Chah put a lot of emphasis on 
continuity and steadiness in practice and training. This is shown in 
the little things that we do and not in some sort of major heroic 
activity. We continue to place our attention on the body and the 
present moment with a sense of opening. It's an important attitude to 
carry into our daily practice.

