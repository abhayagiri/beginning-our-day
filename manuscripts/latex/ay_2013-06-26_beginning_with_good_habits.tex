\mychapter{Beginning With Good Habits}{Ajahn Yatiko}{June 2013}

It is possible to have freedom from a mind that seems compulsively 
locked into habits and mind movements. We have the ability to 
completely put aside those habits and rest in a silent, quiet, 
spacious, aware, calm, and devotional place. The path that leads to 
this place is the path laid out by the Buddha---a transformative path 
that releases us from the habits of mind that cause us suffering.

As far as habits go, it's much easier to develop good habits right from 
the beginning rather than trying to correct our bad habits once they've 
set in. When people first come to live at a monastery--- especially 
junior monks, novices, \emph{anagarikas,} or long-term lay 
residents---that is the time for them to develop good habits. It's much 
easier to accomplish in the early years, because the energy to do so is 
most available and present. So that's the time to reflect on and 
establish the practices recommended by the Buddha.

Whenever we perform an action there's a tendency for that action to be 
repeated---that's an aspect of \emph{kamma}. If we chose to act in a 
certain way under certain conditions, then whenever those conditions 
recur, we're likely to act in the same way. So when new monastics right 
away get caught up into busyness, distraction, and work projects, then 
it's likely they will continue getting caught up like that, over and 
over; it will become their habit. Fortunately, this applies to our 
wholesome actions as well.

It is a great refuge for us to recollect that we've exerted a 
significant amount of effort in various aspects of the 
practice---study, meditation, different devotional practices, service, 
and other wholesome actions. This is a meaningful source of comfort, 
because we know that what we've done in the past is repeatable---we 
have established an ability to practice well. This can give us a sense 
of satisfaction, which is especially important when we're going through 
a difficult and challenging stretch. If I had never put forth that 
effort in my early years, I think it would be very difficult for me to 
establish those wholesome habits for the first time now, because I 
wouldn't know I was capable of practicing in that way.

