\mychapter{A Bowl Full of Light}{Luang Por Pasanno}{August 2013}

I have been reading a book on Hawaiian spirituality, and there's a 
beautiful image that Hawaiians use: Each one of us is born into the 
world with a bowl full of light; and for each unskillful choice we make 
throughout our lives---getting caught up in anger, conflict, or 
selfishness---it's as if we put a rock in the bowl. The more rocks that 
are placed in the bowl, the less room there is for light.

In our daily-life practice, by examining what it is we are doing, we 
can reflect on whether we are a being of light or a being of rocks and 
pebbles. When we recognize that we have accumulated any kind of rocks 
or pebbles, we learn how to tip over the bowl and dump the rocks out. 
This helps us look after that bowl of light and return to making 
choices that are more skillful.

Light is a universal image that is used across all religious and 
spiritual traditions. In the first discourse of the Buddha, the 
Dhammacakkappavattana Sutta, there is the contemplation of the Four 
Noble Truths and the implementing of the Eightfold Path. The stock 
phrases that follow this are: \emph{cakkhuṃ udapādi, ñāṇaṃ 
udapādi, paññā udapādi, vijjā udapādi, āloko udapādi:} Vision 
arises, knowledge arises, wisdom arises, clear seeing arises, and light 
arises. There is a sense of light coming into being. From the Buddhist 
perspective, when we establish and develop a continuity of training 
with the Four Noble Truths and the Eightfold Path, it brings light into 
the mind and into our being.

