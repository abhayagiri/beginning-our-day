\mychapter{Making the Embers Bright Red}{Ajahn Yatiko}{June 2013}

Due to our past kamma, we have found ourselves in a Buddhist community 
where there is harmony and good feelings, and, most importantly, we 
have teachings that are encouraging us in the practice and leading us 
to truth and peace. We have the space and time to practice and 
relatively good health, and our requisites are well provided. We can 
recognize this and appreciate the opportunity that we have. This is a 
helpful reflection, but it is also important to have a clear insight 
into the extreme rarity of this kind of situation.

When we develop this insight, it's like waving a fan in front of a fire 
to make the embers bright red. We need to go over this insight and work 
at it until we really see it. We can do this by setting a goal, not so 
much to attain \emph{samadhi} or to have perfect mindfulness, but, 
rather, to arouse a clear, vivid sense of the urgency of our situation. 
If we are successful, and this insight arises, it can be very 
exhilarating.

When the embers are bright red from this insight, then it's like 
starting an engine; the engine takes care of itself once it gets going. 
We start this engine by making an effort to see the specialness of our 
situation, our environment, the teachings, and the comparative rareness 
of this experience. We can ask ourselves, \emph{How many beings are in 
a situation where they are so tight up against the conditions of life 
that they can't separate themselves from those conditions and it just 
seems impossible to get a perspective on the Dhamma?} That's the case 
for the vast majority of beings in the universe. Go over this insight 
into the rareness of our opportunity, and work at it over and over 
again until the embers are bright red and the engine is humming along 
without effort. We can let the practice take care of itself from there.

