\mychapter{This Is the Dissolving}{Ajahn Sucitto}{May 2007}

\emph{What if I get it wrong? What if it doesn't work? What if I'm left 
here alone?}

Just look over the edge of that ``what if.'' Let your mind open up and 
realize that you've been running away from phantoms. Examine the 
attitudes you might have like, \emph{What if I get it wrong?} We've 
been getting it wrong all our life---it's no big deal! Everybody's 
making mistakes and losing it. We've all been blundering, not noticing, 
disrespectful, impolite, unkind. We've been talking too long or not 
saying enough. Everybody's getting it wrong. Why don't we just practice 
forgiveness instead and get out of this trap? You forgive me, and I'll 
forgive you, okay?

And yes everything is breaking down. But we can get through that, we 
can be bigger than that. We can be bigger than getting it right and 
wrong. We can be bigger than success and failure, praise and blame, 
gain and loss. What a relief to get out of the game. There will be some 
pain though, and we might even cause pain. Maybe the way out of it is 
being a bit more relaxed, at ease, broad-minded, and less concerned 
about being right, perfect, on time, prepared, well-defended, and 
approved.

We can go down into our bellies, relax, breathe out, open up, and trust 
that as human beings we have what it takes to be human beings. We can 
learn from that process, and we can learn to not make a big deal out of 
it. We can learn to not get confused by it or expect miracles out of 
it. And then we can learn to let go of it, to come out of it---and to 
come out of the agitation and self-consciousness that arises up in 
these forms.

This is important to learn, isn't it? It is nothing esoteric or 
high-minded, just basic sanity. But the wonderful thing is that, 
although it is very ordinary, it is also extremely profound because it 
starts to dismantle all those reflexes that seem so ingrained and out 
of control---reflexes that grip us, push us around, make us flustered, 
say things we wish we hadn't said, or make us feel bad.

We can simply start to dismantle all that stuff, to dismantle our hold 
on all that need---the need to be something, prove something, get 
somewhere---until we can be, miraculously, right here, in a place that 
doesn't have a location. Because it doesn't have a location, we never 
leave it. Because we never leave it, we \emph{can't} leave it. So there 
isn't any kind of grief or sadness, no ups or downs, no holding on or 
worrying.

In the ongoing truthfulness of our practice it's important to sift 
through all the static and white noise that the emotions and the mind's 
programs set up. It's important to see and get a handle on what 
intention feels like in our nervous system. And the same with the 
quality of attention---to see how big, narrow, tight, or bound we feel 
when we're occupied with a series of thoughts---how our attention 
bunches up with that sort of proliferation. We start to get a real 
sense of how this feels in the body. And then we start to get a sense 
of what it's like to release it all.

This is what we practice. This is the dissolving. Dissolving involves 
the letting go of control, self-image, and self-territories. For that 
to happen, it has to be a comfortable ride. And we can feel that 
comfort in the breathing in and the breathing out, in good friendship, 
and in moral living. We get the sense that, \emph{It's okay. It's okay 
to be here}. That gives us the ability to trust the process and to 
trust the practice.

