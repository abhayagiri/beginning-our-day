\mychapter{A Mirror on Desire}{Luang Por Pasanno}{June 2005}

Reflecting on the four requisites---clothing, food, shelter and 
medicine---we can try to be more clear about what we need and what we 
merely want or desire. This can often become a bit cloudy in the mind. 
If we try to rely on what is truly necessary for a simple life, and 
question the desire or feeling of need that comes up, then things can 
become a bit clearer, and we are better able to understand how it is 
that desire keeps pushing us around.

Our lifestyle is so simple that the mind focuses on things very 
strongly and builds a case for what it thinks it really needs. But when 
we reflect, investigate, and question, we realize that it isn't 
actually a need at all---it's just another desire. So use the 
requisites as a means to understand the mind; use them as a mirror to 
see clearly what appears in the mind and what the mind is scrambling to 
get.

Having this prescribed lifestyle is very fruitful and encourages us to 
reflect more clearly about the nature of our desire and attachments. 
Being an alms mendicant with requisites and living in the monastery 
with this fundamental simplicity gives us a clear opportunity to 
unravel the way the mind complicates things, makes things problematic, 
and takes us away from fundamental contentment. The tools of our 
lifestyle are the Vinaya, the teachings, the requisites, our day-to-day 
activities, and the practice itself. We don't necessarily live by an 
extraordinary standard, but we use these tools that give us the 
opportunity to reflect, investigate, and learn to understand the mind 
in and of itself.

