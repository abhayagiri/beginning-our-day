\mychapter{Breaking the Momentum}{Ajahn Yatiko}{April 2013}

We can take these next few minutes as a time to establish mindfulness 
and provide ourselves with a break. We can break the momentum of the 
mind, which so easily gets caught up in becoming, especially when we 
have ongoing projects and duties to attend to. It can be so easy for 
the mind to obsess about unfinished tasks, keeping itself in a chaotic 
world. If we find ourselves stuck thinking about a project 
incessantly---trying to get it organized, straight, and complete---we 
can take a few minutes like this to stop.

If we can't stop that momentum, at least we can look to see if we have 
the mindfulness to recognize this flow of thoughts. We can pay careful 
attention to the mind that is creating time, a future, plans, and 
everything else. We may find that to be a very valuable reflection when 
this momentum of becoming is such that we can't seem to stop it. We try 
to recognize and contemplate that this momentum and becoming is 
happening right now, just like everything else in the universe. That 
recognition alone is a very useful form of mindfulness.

When we're not able to step outside of our usual patterns of thinking, 
and instead find ourselves caught up in the flow of becoming, or the 
momentum of any kind of pattern---and we're identified with it---then 
it is similar to a form of craziness. Luang Por Chah famously once said 
that three seconds without mindfulness is like three seconds of 
madness. So even though we are wearing robes, or are dedicated lay 
practitioners, we can still share the exact same chaotic mental state 
as ninety-nine percent of the rest of the planet. In those instances we 
are caught up in the same momentum, identified with it, not questioning 
it, and flowing on and on in the stream of becoming. So we need to use 
these opportunities and formal structures like meetings, 
\emph{pūjās}, Dhamma talks, and reflections to break that pattern. 
For you laypeople, living outside of a monastery, it is important to 
encourage yourselves to find environments where you can create a break 
in that flow of momentum. This can provide a space to establish 
mindfulness and allow you to reflect on your life skillfully. Otherwise 
life ends up being determined by patterns that have already been set in 
motion in the past---you're simply riding a wave. It may be difficult 
to find such places for yourselves, but it is important that you try.

These morning reflections are not just something to bear with and get 
over so we can get on to our jobs; they're here for us to use, to take 
up as subjects for contemplation. If we don't see very clearly how to 
use them in a skillful way, then we need to take that on as a 
reflection in and of itself. We can spend some time on the walking path 
or while we are sitting and ask ourselves, \emph{What is the morning 
reflection for? What's the right attitude to have when it is given?} 
Based on what arises out of this questioning, we can challenge 
ourselves further and ask, \emph{If this is the right attitude, is it 
something that I manage to have and cultivate? Or have I allowed myself 
to get into the habit of resisting formal structures?}

We are giving ourselves the space and time to pay attention to this 
flow of becoming and to use these reflections and wholesome structures 
to break the mind's unwholesome habits. By doing that we are slowing 
down the momentum of being relentlessly caught in cycles of rebirth and 
suffering. And when we have learned how to uproot this cycle, we come 
to a natural place of peace and freedom.

