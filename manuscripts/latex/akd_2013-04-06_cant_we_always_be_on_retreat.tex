\mychapter{Can't We Always Be on Retreat?}{Ajahn Karuṇadhammo}{April 
2013}

We're still in the process of coming off our three-month retreat and 
adjusting to the different pace, level of energy, and engagement. I've 
been reflecting on some of the words I've heard from others and what I 
sometimes hear in my own head: \emph{Why can't we always be on retreat? 
Why can't we always live this particular way? Can't we slow down and 
have more space and time for formal practice in the day?} The logical 
answer is there's a monastery that needs maintaining. All the buildings 
we live in and the different things we need as requisites have to be 
looked after. That's the logical reason. If we didn't have Abhayagiri, 
we wouldn't have any place for retreat. So we all need to pitch in and 
help.

We're working on a greater perspective, a long-term goal in this 
practice, and our abilities to realize peace of mind need to be ones 
that aren't dependent on specific circumstances. We need to be able to 
develop mindfulness, clarity, and ease of living in whatever situation 
we find ourselves, whether it's retreat, engagement, work, or 
community. We're developing an all-encompassing freedom that can be 
realized in any circumstance, whether in quietude or engagement. One of 
the qualities that's so important in developing this is patience---not 
to expect big realizations or quick understanding. This is a gradual 
path and requires a tremendous amount of patience. It's not the kind of 
patience where we grit our teeth and simply bear with it, but rather, 
it is a spacious, wide-open acceptance of the way things are.

I remember as a very junior monk, a time when I had some difficult 
interactions with another community member. I was trying to strategize 
how to better cope with it and more skillfully handle the situation 
because there was a lot of frustration and anger arising in me. I tried 
the usual antidote, developing loving-kindness, but it wasn't working 
because there was a sense of resistance, strain, and aversion with 
trying to get rid of the particular circumstance. I realized that more 
important than directly responding with loving-kindness was cultivating 
a sense of patience---both for this other person and for myself---while 
I tried to develop the skills needed to handle and cope with the 
situation. Practicing with patience helped me see more clearly what my 
expectations were. It allowed me to be more open and I sensed, 
\emph{Okay, this is deeply ingrained in me and it's going to take some 
time. It's not always going to be pleasant, but I need to open myself 
up to the circumstances and be a bit more spacious with it.}

That's what the quality of patience is all about. It's not a quality of 
gritting one's teeth and bearing with the circumstance until things get 
better, or bearing through the next nine months until we get three more 
months of retreat. It's about opening to one's experience and 
developing a sense of clarity and understanding that this is a 
long-term process. The more we can be patient with ourselves and with 
the practice in our living situations, the more we can open to whatever 
it is that comes our way and use it as a tool for learning about 
ourselves.

