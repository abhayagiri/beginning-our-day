\mychapter{Earthworm Practice}{Luang Por Pasanno}{November 2008}

With the drawbacks of physical existence, the Buddha instructed that we 
reflect on the theme of death as an antidote to our interest in being 
reborn into the conditioned realm. These reflections are meant to bring 
up a sense of urgency: there is no time to waste. There aren't 
unlimited opportunities for spiritual practice. We have excellent 
conditions right now, and we should make use of them. The image that 
the Buddha uses for this sense of urgency is a person whose hair is on 
fire. There's a real motivation to put that out, to deal with the 
situation.

It's easy to put things off, to find various good and logical excuses 
to be pulling back a bit on our effort. The motivation for making the 
effort, for putting attention back onto the practice, can lose its 
quality of urgency. We may turn to some social engagement, some 
conversation, some distraction, and it might be interesting, it might 
have some tangential benefit, but we must realize the need to set that 
aside and return to being mindfully attentive to what we're doing.

One of the problems with having a sense of urgency is the feeling of 
flailing around---putting out effort in a sporadic way and not being 
able to sustain it. We start off enthusiastically, really buckling 
down: \emph{Okay, back to the practice, I'm really going to stick with 
it this time.} Yet we're not able to sustain it. We swing back and 
forth. Sometimes the results of our efforts might not seem so dazzling. 
They might not make us think, \emph{Now I'm really getting somewhere!} 
They may not even seem interesting. So we can get frustrated and pull 
back and then, sometime later, it's: \emph{Okay, time to buckle down 
again.} But consistency is what's important and paying attention to 
being consistent.

I remember Ajahn Chah's advice on how to maintain constancy in the 
practice, addressing the issue of first wanting to really push and then 
feeling frustrated, ``Can you learn to practice like an earthworm? They 
can't know where they're going, but they keep moving along. So get your 
head down, simply go forward, just have earthworm practice. Keep 
moving, constant and consistent.''

