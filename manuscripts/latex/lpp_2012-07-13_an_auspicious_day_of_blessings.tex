\mychapter{An Auspicious Day of Blessings}{Luang Por Pasanno}{July 2012}

Today is Friday the 13th, and by tradition some people believe that 
it's an unlucky day. And many people have different ideas of why Friday 
the 13th became known as an unlucky day. In the Thai tradition and the 
Asian tradition in general, there are lucky days, unlucky days, 
auspicious and inauspicious times. Ajahn Chah used to say that whatever 
day we are doing something wholesome, that is an auspicious day. He 
also said there is no such thing as a day in and of itself that is 
inauspicious or unlucky. We are completely dependent on gathering our 
own resources into doing something that is skillful, beneficial and 
wholesome. So the opportunity for doing that which is wholesome is in 
itself what makes it a blessing in the world. The Pāḷi word 
\emph{mahāmaṅgala} means the highest blessings. The Mahāmaṅgala 
Sutta encourages the cultivation of that which is skillful: association 
with good people and developing inner virtues that are beneficial to 
both ourselves and others. When we cultivate these virtues, we create 
the blessings of a skillful life, sharing those blessings with the 
people we're living with and the people we come into contact with. And 
that is how we bring great blessings into the world.

