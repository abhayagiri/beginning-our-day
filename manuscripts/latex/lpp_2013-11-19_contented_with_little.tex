\mychapter{Contented With Little}{Luang Por Pasanno}{November 2013}

I recently read that in 1985---which is not so long ago---the average 
American purchased about 32 new articles of clothing per year. By 2010 
it was up to almost sixty---and that was when the economy was in a 
tailspin. It seems many people are becoming more and more focused on 
consuming, without much consideration for what they already have.

By contrast, part of the ethos of living in a monastery as a 
\emph{samaṇa}, a religious seeker, is developing the qualities of 
simplicity, frugality and the care we take in using and reusing things. 
For instance, I have a cold now and have been blowing my nose quite 
often---I don't know how many times a day. So I try to emulate Master 
Hua, who would use the same tissue over and over again, until it became 
apparent that it could no longer be used again. I must admit that 
despite trying, I cannot match his austerity in this regard. However, 
it's a good illustration of frugality and taking care of what we 
use---we don't need to throw a tissue away after only using it once, 
but instead, we can keep reusing it until it's completely worn out. 
Simplicity and frugality are about doing little things like that: 
paying attention to \emph{all} the things we use, taking care of them 
and avoiding waste. Living like this helps protect the culture of 
modesty and contentment we've established here.

There's an idiom in Thai that refers to ``one who is contented with 
little.'' Reflecting on this can inspire a life without complication. 
We can live very simply, being conscious of how we use things, and, 
contrary to what one might think, this doesn't make us miserable. 
Contentment is quite the opposite of misery. Living in a conscious way 
like this can make us happier by fostering internal qualities that lead 
to ease and well-being. We realize that we don't require so many things 
to keep us happy and comfortable. It's an internal experience based on 
internal qualities.

