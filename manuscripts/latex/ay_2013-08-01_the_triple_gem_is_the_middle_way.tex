\mychapter{The Triple Gem Is the Middle Way}{Ajahn Yatiko}{August 2013}

It's interesting how things change when Luang Por Pasanno leaves. 
Suddenly the energy in the community shifts. One of the roles a senior 
monk takes on when he's the head of the community is to figuratively 
place himself into the center of the community, and from that central 
point, provide a focus for the community, be inspiring and encouraging 
of others. His role is part of the external world we live in here. But 
we each occupy an internal world as well, and each of us is at the 
center of that world. And from that center, when push comes to shove, 
we're each responsible for our own inspiration, encouragement, and 
well-being.

So we have to find a way of centering ourselves that doesn't depend on 
some person of authority or other external form of support---to find a 
center that can serve as a point of focus and a stable refuge. If we 
take our own desires and selfish cravings as a refuge, that's a recipe 
for suffering. When we do that, we're headed toward an empty, 
dead-ended, self-centered solitude that's very painful to experience. 
On the other extreme, we could place the external world at our center, 
seeking refuge in trying to serve and help, focusing all of our 
energies on those in need. But one of the cold, hard facts of this 
world is that there are many more beings in need than we could possibly 
help in a meaningful way. So trying to center ourselves in unreserved, 
unremitting service often leads to suffering as well, to frustration 
and burnout.

The middle way for all of this---for finding a center---is based on the 
Triple Gem: The Buddha, the Dhamma, and the Sangha. We can rely on the 
Triple Gem for our center. We already know how much time we waste 
focused on our own little thoughts, ideas, perceptions, aspirations, 
hopes, and anxieties. We can get out of that mode by calling to mind 
the Triple Gem---something extra-mundane and transcendent. We can let 
our internal worlds revolve around and focus on that. When we focus on 
the Triple Gem as our center, the focus is neither on ourselves nor on 
other people; yet it is something that radiates everywhere, toward all 
beings, including ourselves. And although maintaining this center 
requires effort, we need not depend on external people or conditions to 
make it happen.

