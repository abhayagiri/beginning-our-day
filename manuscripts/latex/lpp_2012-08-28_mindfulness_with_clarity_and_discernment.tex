\mychaptertoc{Mindfulness With Clarity and Discernment}
{Mindfulness With Clarity\\and Discernment}
{Luang Por Pasanno}{August 2012}

Yesterday at tea time, we were talking about \emph{sampajañña}, clear 
comprehension, which is a quality we can reflect on in our daily 
practice. When the Buddha speaks about mindfulness---\emph{sati}---he 
rarely treats mindfulness as an isolated quality. It's usually in 
conjunction with some other quality---particularly, clear 
comprehension, \emph{sati-sampajañña}. Without clear comprehension, 
our mindfulness tends to be rather narrowly focused. We can forget that 
the Buddha is encouraging us to have a broadness of attention and 
awareness and a reflective quality while we are cultivating 
mindfulness. So we need to make clear comprehension an integral part of 
our cultivation and development of mindfulness.

Clear comprehension has different functions, duties, and purposes. 
There is clear comprehension of the object of our attention, but there 
are many other aspects as well. For instance, the sense of being free 
of delusion is one example of sampajañña operating within the mind. 
With non-delusion we have a sense of clearly comprehending our biases 
or lack of biases and refraining from bringing a self-position into our 
experiences.

We can also clearly comprehend how to practice in a way that's in 
keeping with whatever time and place we happen to be in. Ajahn Chah 
used to give a classic example of this having to do with a senior monk 
who would visit him at his monastery. In the morning they would go on 
alms round together into one of the villages around Wat Pah Pong. Ajahn 
Chah said that this was often a source of slight irritation, because, 
although this monk was very mindful, he didn't clearly comprehend the 
situation. As the visiting monk would lead a group on alms round, he 
would mindfully walk towards a buffalo pen, because he didn't notice 
that he wasn't walking on the road anymore. Or, while he was being very 
focused, he would fail to recognize that he was walking on the left 
side of the road and all of the villagers were waiting off to the right 
side of the road. He would mindfully walk past them and start heading 
off into the paddy fields. Ajahn Chah would say, ``Go right, go 
right,'' or ``Go left, go left.'' Mindfulness requires clear 
comprehension so that we are aware of the appropriate time and place in 
each circumstance.

Another aspect of clear comprehension pertains to the different 
personalities and temperaments people have. Different people need 
attending to in different ways, and we need to adjust our actions and 
attitudes accordingly. Otherwise we may offend, upset, or 
miscommunicate with the people around us. Clear comprehension allows us 
to see how to choose the best response for each person we interact with.

In the \emph{Visuddhimagga}, Ācariya Buddhaghosa, suggests that 
sampajañña is a synonym for discernment; specifically, it's the 
discernment we use to determine what Dhamma to apply, depending on the 
situation we're in and the experience that's arising for us.

As we go about our day, we use reflective awareness to get the best 
sense of how to apply sati-sampajañña. We do not need to apply 
mindfulness and clear comprehension in a mechanical way. When we do our 
daily chanting, we recollect the qualities of the Dhamma, and one of 
these qualities is \emph{opanayiko}: leading inwards. The function of 
our practice is to draw our attention inwards; to draw the world and 
interaction with the world around us, inwards. That is how we are able 
to hold things with clarity and discernment, applying sati-sampajañña 
to see the Dhamma clearly.

