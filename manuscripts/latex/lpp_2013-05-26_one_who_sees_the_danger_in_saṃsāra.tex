\mychaptertoc{One Who Sees the Danger in Saṃsāra}
{One Who Sees the Danger\\in Saṃsāra}
{Luang Por Pasanno}{May 2013}

Contemplations and reflections on death and dying are a means of 
focusing attention, and prioritizing where we choose to put our 
attention. We can ask ourselves, \emph{What is it we are becoming 
absorbed in? What are we letting the mind run away with?} These are 
essential contemplations because they help us make the 
practice---including relinquishment and generosity---our first priority.

The literal translation of the word \emph{bhikkhu} is ``one who begs 
for alms.'' But in keeping with the commentaries, Ajahn Chah used to 
emphasize bhikkhu as meaning ``one who sees the danger in 
\emph{saṃsāra},'' the danger in the endless round of birth and 
death, the endless wandering. That's a useful image for the unfocused, 
inattentive mind---the mind that does not prioritize, that drifts 
aimlessly about. It wanders to moods of like and dislike, moods of 
resentment and attachment, and continues to wander on. This reflection 
on one who sees the danger in saṃsāra applies to everybody, not only 
monks. It can encourage us all to make good use of our faculties, to 
take the opportunities afforded us through our bodies and our minds.

So we recollect and reflect on our impending death and the endless 
cycle of saṃsāra, in order to sustain our attention and to keep the 
practice as our first priority.

