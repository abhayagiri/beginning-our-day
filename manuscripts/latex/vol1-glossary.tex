\chapter*{Glossary}
\addcontentsline{toc}{chapter}{Glossary\protect\chapternumberline{}}

{\parindent 0pt \parskip .5em

\glossentrylang{Ajahn}{Thai}{From the Pāḷi ācariya, literally “teacher”;
often used in monasteries as a title of senior monks or nuns who have
been ordained for ten years or more.}

\glossentry{Anagārika}{Literally: “homeless one.” An eight-precept male
postulant who often lives with Bhikkhus and, in addition to their own
meditation practice, also helps with certain services that are forbidden
for Bhikkhus to do, such as, using money, cutting plants, or cooking
food.}

\glossentry{anattā}{Not-self, ownerless, impersonal. One of the three
characteristics of all conditioned phenomena.}

\glossentry{anicca}{Impermanent, inconstant, unsteady. Ajahn Chah often
translated it as “not sure.” One of the three characteristics of all
conditioned phenomena.}

\glossentry{asubha}{Unattractive, not-beautiful. The Buddha recommended
contemplation of this aspect of the body as an antidote to desire, lust,
and complacency.}

\glossentry{bhikkhu}{A Buddhist monk; a man who has given up the
householder’s life to join the monastic Saṅgha. He follows the
Dhamma-Vinaya—the doctrine and the discipline—the teachings of the
Buddha as well as the Buddha's established code of conduct.}

\glossentry{brahmavihāra}{The four “sublime” or “divine” abodes that are
attained through the development of mettā, karuṇā, muditā, and
uppekhā—boundless loving-kindness, compassion, appreciative joy, and
equanimity.}

\glossentry{Buddha}{The historical religious leader and teacher, who lived around 
2500 BCE in the Ganges Valley of India. He established a monks', nuns', 
and lay order under what he called the Dhamma-Vinaya—the doctrine and 
the discipline. The word Buddha literally means “awakened one” or 
“enlightened one.”}

\glossentry{Dhamma}{A broad, deep, and unique concept in Indian culture.
The Buddha adopted the term to address his understanding of the truth of
the way things are—how suffering is a natural phenomena in the world and
how human beings can learn to be permanently released from this
suffering, thus ending the cycle of rebirth.}

\glossentry{dhamma}{Used as a term to define natural phenomena of the
world, including phenomena of the mind.}

\glossentry{dukkha}{“Hard to bear,” unsatisfactoriness, suffering,
stress. One of the three characteristics of all conditioned phenomena.}

\glossentry{Eightfold Path}{See Noble Eightfold Path.}

\glossentry{Forest Tradition}{The tradition of Buddhist monks and nuns
who have primarily dwelled in forests emphasizing formal meditation
practice and following the Buddha’s monastic code of conduct (Vinaya).}

\glossentry{Four Noble Truths}{The first and central teaching of the
Buddha about dukkha, its origin, cessation, and the path leading towards
its cessation. Complete understanding of the Four Noble Truths is
equivalent to the attainment of Nibbāna.}

\glossentrylang{kamma}{Sanskrit: karma}{Volitional action by means of
body, speech, or mind.  Kamma always leads to an effect (kamma-vipāka).}

\glossentrylang{khandha}{Sanskrit: skandha}{Heap, group, aggregate.
Physical and mental components of the personality and of sensory
experience in general. The five bases of clinging: form, feeling,
perception, mental formations, and consciousness.}

\glossentry{kuṭi}{A small dwelling place for a Buddhist monastic; a
hut.}

\glossentrylang{Luang Por}{Thai}{Venerable Father, Respected Father; a
friendly and reverential term of address used for elderly monks.}

\glossentry{Māra}{Evil, craving, and death personified as a deity, but
also used as a representation of these elements within the mind.}

\glossentry{mettā}{Loving-kindness, goodwill, friendliness. One of the
four brahmavihāras or sublime abodes.}

\glossentry{Middle Way}{The path the Buddha taught between the extremes of 
asceticism and sensual pleasure.}

\glossentry{mindfulness}{See sati.}

\glossentrylang{Nibbāna}{Sanskrit: Nirvāna}{Final liberation from all
suffering, the goal of Buddhist practice. The liberation of the mind
from the mental effluents, defilements, the round of rebirth, and from
all that can be described or defined. As this term also denotes the
extinguishing of a fire, it carries the connotations of stilling,
cooling, and peace.}

\glossentry{Noble Eightfold Path}{Eight factors of spiritual practice
leading to the cessation of suffering: right view, right intention,
right speech, right action, right livelihood, right effort, right
mindfulness, and right concentration.}

\glossentry{noble silence}{Taking on the practice to only speak when
necessary.}

\glossentry{Pāḷi}{An Indo-Aryan language, speculated to have been spoken
around the time of the Buddha's life. The earliest Buddhist scriptures
were written down in Pāḷi some 500 years after the Buddha's passing and
the Theravada school of Buddhism takes these texts as its authority.}

\glossentry{paññā}{Wisdom, discernment, insight, intelligence, common
sense, ingenuity. One of the ten perfections.}

\glossentry{pāramī}{Sanskrit: pāramitā}{Perfection of the character. A
group of ten qualities developed over many lifetimes by a bodhisatta:
generosity, virtue, renunciation, discernment, energy/persistence,
patience or forbearance, truthfulness, determination, good will, and
equanimity.}

\glossentry{Paritta}{Literally: protection. Auspicious blessing and
protective chants typically recited by monastics and sometimes lay
followers as well.}

\glossentry{pūjā}{Literally: offering. Chanting in various languages typically 
recited in the morning and evening by monastic and lay followers of  a 
particular teacher, in this case the Buddha. Typically these 
recitations pay homage to the Buddha, Dhamma and Saṅgha.}

\glossentrylang{Rains Retreat}{vassa}{The traditional time of year that
monks and nuns would determine to stay in one location for three months.
Some monastics will take this time to intensify their formal or
allowable acetic practices. Monks and nuns will refer to themselves as
having a certain number of vassa which signifies how  many years they
have been in robes.}

\glossentry{right effort}{One factor of the Eightfold Path describing
four methods in which a practitioner endeavors to keep his or her mind
free from unskillful qualities. This is done by preventing the arising
of unwholesome qualities that have not yet arisen, abandoning
unwholesome qualities that have arisen, encouraging wholesome qualities
that have not yet arisen, and maintaining and developing wholesome
qualities that have arisen.}

\glossentry{right speech}{One factor of the Eightfold Path describing
the proper use of speech: refraining from lying, divisive speech,
abusive speech, and idle chatter.}

\glossentry{samādhi}{Concentration, one-pointedness of mind, mental
stability. A state of concentrated calm resulting from meditation
practice.}

\glossentry{samaṇa}{Literally: a person who abandons the conventional
obligations of social life in order to find a way of life more 'in tune'
(sama) with the ways of nature. A contemplative or wandering acetic.}

\glossentry{sampajañña}{Clear comprehension, self-awareness, self
recollection, alertness.}

\glossentry{saṃsāra}{The cyclical wheel of existence, literally:
‘perpetual wandering,’ the continuous process of being born, growing
old, suffering and dying again and again, the world of all conditioned
phenomena, mental and material.}

\glossentry{Saṅgha}{On the conventional level, this term denotes the
communities of Buddhist monks and nuns; on the ideal (ariya) level it
denotes those followers of the Buddha, lay or ordained, who have
attained at least stream-entry, the first of the transcendent paths
culminating in Nibbāna.}

\glossentry{sati}{Mindfulness, self-collectedness, recollection,
bringing to mind.  In some contexts, the word sati when used alone
refers to clear-comprehension (sampajañña) as well.}

\glossentry{sīla}{Virtue, morality. The quality of ethical and moral
purity that prevents one from unskillful actions. Also, the training
precepts that restrain one from performing unskillful actions.}

\glossentrylang{five spiritual faculties}{pañca bala}{A list of qualities
the Buddha gathered for the attainment of supernormal faculties or
powers. They are: faith, energy, mindfulness, concentration, and
wisdom.}

\glossentrylang{sutta}{Sanskrit: sūtra}{Literally: ‘thread.’ A discourse
or sermon by the Buddha or his contemporary disciples. After the
Buddha’s death the suttas were passed down in the Pāḷi language
according to a well established oral tradition and finally committed to
written form in Sri Lanka just around the turn of the common era. The
Pāḷi Suttas are widely regarded as the earliest record of the Buddha's
teachings.}

\glossentry{Triple Gem}{The ‘Threefold Refuge’—the Buddha, Dhamma, and
Saṅgha}

\glossentry{Upāsikā Day}{A day for Abhayagiri lay devotees to visit the
monastery and partake in an afternoon teaching.}

\glossentry{Vinaya}{The Buddhist monastic discipline, literally,
‘leading out’, because maintenance of these rules ‘leads out’ of
unskillful states of mind. The vinaya rules and traditions define every
aspect of the bhikkhus’ and bhikkhunīs’ way of life.}

\glossentrylang{wat}{Thai}{A monastery.}
