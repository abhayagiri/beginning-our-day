\mychapter{Beyond Determinations}{Ajahn Yatiko}{December 2012}

If we wish to overcome obstacles in the heart to experience the 
transformation of the defilements, we need to recognize that it cannot 
be done through force. This is an important principle in our training. 
If we come across states of anger, irritation, or disappointment with 
ourselves or other people, and we're getting stuck in these states and 
dwelling on them, it doesn't work to simply make a determination to 
force ourselves to change them. We may think, \emph{Okay, I'm going to 
try harder and make a new determination so I won't get stuck in these 
states.} That's not going to solve the problem.

Even so, when done properly, making a determination can be extremely 
valuable. And when a determination is valuable, if we slip up on it, 
then it's helpful for us to recommit to the determination straight 
away. But with deeply rooted internal states---such as irritation, ill 
will, being judgmental, and other defilements that linger within---we 
can't control our way out of them. We can't successfully determine, 
\emph{I lost it; therefore, I'm going to try harder and not let it 
happen again.} Instead, it needs to come from a state of investigation 
and study. It doesn't help us or other people if we criticize ourselves 
for slipping up---and it also doesn't help to tell ourselves not to be 
critical.

We can't force ourselves to be happy. But there can be an investigation 
and a realization that we don't have to suffer over something. It 
really is possible to take a negative state, loosen it, and say to 
ourselves, \emph{I don't have to suffer over this–it's really not 
necessary for me to suffer.} If that sort of loosening seems 
impossible, it merely means we haven't found the right entry point into 
the state we're working with. There \emph{is} some entry point, some 
place where we're clinging and hanging onto negativity. And there 
\emph{is} a way to let go of it. We have to know that, to have faith in 
that, and to trust in that. We can say to ourselves, \emph{This is not 
who I am, and there is some way to let go of this. There's somewhere 
inside where I can loosen my grip. There has to be. If there wasn't, 
that would be the end of this whole spiritual path.}

So when these states arise, to a certain extent, determinations can be 
helpful to deal with them. But in order to fully uproot these states, 
we have to use introspection, investigation, and self-study, reminding 
ourselves that we don't have to suffer over anything. If we can be 
clear on that, then we can even work on states that seem impossible to 
deal with, like physical agony---things that seem out of our control 
and things that really \emph{are} out of our control. We can find a way 
to loosen our grip so we don't suffer. And that, of course, is the core 
of our practice.

