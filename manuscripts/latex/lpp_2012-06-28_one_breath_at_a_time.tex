\mychapter{One Breath at a Time}{Luang Por Pasanno}{June 2012}

Distributing the daily work assignments seemed a bit complicated today. 
That's the nature of organizing many people living together. When there 
is one person living in one place, it's fairly simple. With two people 
it's a little harder, and it gets exponentially more complicated the 
more people there are who need organizing.

For this reason, we need to learn the skills of living together, so 
that our own interactions and how we relate to each other don't get 
overly complicated. And in general terms, as practitioners, it is 
essential that we cultivate the quality of simplicity in the ways we 
relate to things. In reality, there is one moment at a time, and we 
take one step at a time when we do things. It's not all that 
complicated.

But the mind tends to leap forward towards proliferation, so we need 
the patience to step back and ask ourselves, \emph{What do I have to 
deal with right now?} Mostly all we have to deal with right now is 
breathing in and breathing out. There's not much to do. If there is a 
task to be done, then we can learn how to apply patience and attention 
to it. We put attention on what we are doing, the way it impacts the 
people around us and the circumstances we are in.

As we do this, we realize that the morning has passed, another day has 
moved on. Very simple. We can recall this quality of simplicity as we 
pay attention to the reactions in the mind, the moods that keep popping 
up, and the proliferations that keep hounding us. We can remember: 
\emph{One breath at a time, one step at a time}. It's all very 
uncomplicated, and there's not much to be done.

