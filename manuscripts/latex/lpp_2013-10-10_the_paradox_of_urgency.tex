\mychapter{The Paradox of Urgency}{Luang Por Pasanno}{October 2013}

We may feel there exists a paradox or inconsistency having to do with 
some of the teachings that guide our spiritual practice and how we 
conduct our daily lives. First we're told it's important to have a 
sense of urgency, and then we're told to relax, let go, and cultivate a 
feeling of spaciousness.

On one side, the Buddha encourages us to cultivate the quality of 
\emph{saṃvega}, which is usually translated as ``a sense of spiritual 
urgency.'' It's the recognition that life is short and everything is 
uncertain, so we need to take this practice seriously. We urgently need 
to apply our attention and effort in order to fulfill the teachings and 
not miss out on the opportunity we have.

On the other side, we need to learn how to relax and not get caught up 
in the feeling of urgency. We tend to pile stuff onto ourselves, 
thinking, \emph{I have to get this done and that done and …} Or we 
might feel an urgency with regard to future planning: \emph{We have to 
have a budget meeting, and we have to have an insurance meeting, and we 
have to have a business meeting, and …} This agitating sort of 
urgency can arise not only with our chores and duties, but with our 
internal spiritual practice as well: \emph{I have to get my mindfulness 
of breathing straight, and keep my reflections going, and …} With 
this sense of urgency in trying to do everything all at once, we end up 
becoming frantic, and, ironically, very little gets done well or 
efficiently---it really doesn't work. So instead we can remind 
ourselves, \emph{Hey, it's just one thing at a time, one moment at a 
time, one breath at a time.} We can learn to create some space for 
ourselves and not carry the ``I have to'' tendency around with us; nor 
the impulse that tells us that everything is so urgent.

As we reflect on these seemingly paradoxical qualities of urgency and 
spaciousness, we may find that they are not inconsistent after all. We 
can apply both of them in our practice, with a sense of balance, 
emphasizing one or the other depending on the circumstances. In doing 
so, we need to come from a place of clarity and steadiness. Clarity is 
needed to anchor spaciousness and letting go, and to guide those 
actions that arise out of a sense of urgency. Steadiness is needed to 
bring good results from those actions, and to keep spaciousness from 
turning into spacing out. This place of clarity and steadiness is where 
paradoxes like these dissolve and fade away.

