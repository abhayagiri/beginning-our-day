\mychapter{Being With Resistance}{Ajahn Jotipālo}{November 2012}

In one of the Dhamma talks Ajahn Chandako gave recently, he said the 
best way that we, as monastics, can support people in the world, 
support ourselves, and support the people who support us, is to develop 
the monastery and develop our individual practices. We can do this by 
investigating our tendencies and mind states to see whether we are 
moving toward contentment and communal harmony or if we are moving 
toward a tendency to control---trying to push people around with our 
words or attitudes, or trying to manipulate conditions to meet our 
preferences. It's important to stop and ask ourselves, \emph{What's the 
tendency there? What's the mental habit?}

As we all know, during these fifteen-minute morning work meetings, the 
work monk assigns a job to each of us. When I was a junior monk, I 
would feel a tense sense of resistance during the entire meeting 
because there were certain things I didn't like being asked to do. 
Years later, when I became the work monk, these tendencies weren't 
there at all because I had control. But even later, when it was again 
someone else who had the work monk job, my tendency toward resistance 
came back. This is only an example---a reminder that our tendencies to 
feel resistance, or any unwholesome tendencies, simply arise out of 
causes and conditions. It's important not to see them as wrong, but 
simply as tendencies of the mind.

With that understanding, we can investigate what is happening for us in 
the moment when resistance arises, and give ourselves space around the 
feeling that is present. We don't need to push these experiences away 
by bossing people around, trying to change things, or becoming 
reclusive. Rather, we can look at the feeling of resistance and learn 
how to \emph{be} with it. We can observe how the feeling changes, or 
investigate the fear that underlies resistance, breaking it all down. 
When we do this, it's possible to see that there's nothing worth 
fighting against---there's really nothing to be gained by being caught 
up in a mind state of resistance.

