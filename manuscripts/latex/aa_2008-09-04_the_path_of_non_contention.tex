\mychapter{The Path of Non-Contention}{Ajahn Amaro}{September 2008}

Often when we practice loving-kindness, \emph{mettā,} it involves an 
active well-wishing to all beings, such as when we repeat the phrases, 
``May you be happy, may they be happy, may all beings be happy, 
healthy, safe, at ease,'' and so forth. Certainly that's an important 
part of loving-kindness meditation. But in a more essential, practical 
way, the quality of mettā is not only a well-wishing toward other 
beings; it also has to do with how we relate to our own mind states and 
the way we handle the different moods, feelings, and perceptions that 
arise within us. If we are repeating those mettā phrases and 
cultivating those sentiments toward external beings, but yet internally 
relating to our own mind states in a semi-conscious or unconscious, 
reactive way, then all of those noble sentiments and qualities we 
direct outwards don't have much fuel; they don't have much of a 
foundation. To make the practices of mettā meditation really fruitful 
and genuinely relevant and effective, there needs to be both the 
external element and the internal element---relating with 
loving-kindness toward our own mind states, moods, bodies, thoughts, 
and feelings.

When there's an irritation, an impatience arising, a sense of things 
having shortcomings, or when we feel ourselves to be imperfect or not 
beautiful in some way, then it's easy for us to criticize and blame 
ourselves. We can quickly get upset with a mind that won't stop 
thinking or has stray thoughts and unwanted memories, ideas, fears, 
feelings, desires, and dislikes. Ajahn Sumedho addresses all these 
tendencies by expressing the essential attitude of loving-kindness as, 
``Not dwelling in aversion.'' Rather than expecting to be affirmatively 
affectionate toward our bodies, thoughts, and feelings, it's merely a 
sense of not finding fault with them, not dwelling in aversion. That 
much is doable.

To establish the attitude of loving-kindness---the genuine heart of 
mettā---is to establish within ourselves a heart of non-contention, a 
heart which is accepting and accommodating of all mind states. This 
doesn't mean to say that we are approving of every thought we have, or 
that all our feelings of selfishness, violence, aggression, jealousy, 
and fear are beautiful, wonderful---adornments for the world. We're not 
trying to pretend or allow ourselves to get lost in delusion, confusing 
the skillful and unskillful. Instead we're simply trying to recognize 
that everything belongs, whether it's a noble and wholesome thought, or 
a selfish, fearful, jealous, or greedy thought. They all belong. 
They're all attributes of nature. When we cultivate this quality of 
non-contention and acceptance toward all our inner 
qualities---feelings, thoughts, perceptions, memories, ideas, and 
fantasies---then the heart is not divided, and there's unity, a 
unification of the heart.

And without this fundamental unity of mettā in our hearts---this 
fundamental welcoming inclination and recognition that our internal 
attributes, good or bad, are all aspects of the natural order---then 
it's impossible to have a genuine, substantial attitude of 
loving-kindness toward others, because the heart is divided.

So we can take some time to pursue and explore these themes and to 
reflect on them, seeing throughout the course of the day how often the 
mind wants to contend against our own bodies and their limitations, our 
feelings and thoughts, and the world around us. We can take it all so 
personally. While simply digging a hole in the ground we might think, 
\emph{That rock is determined to get in my way!} But mettā is the 
heart of non-contention. We can learn how to work with the world so 
that regardless of how obstructive things may seem to be, how unwanted 
and unbidden, we can recognize that there's no need to start a fight or 
contend against the world---it's up to us.

When this recognition takes hold, we can see that there is always a 
path to working with the way things are, a path that leads us toward 
greater clarity and peacefulness. And all of this is based fully upon 
our cultivating an attitude of non-contention, of basic loving-kindness.

