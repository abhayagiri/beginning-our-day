\mychapter{What the Body is Supposed to Do}{Ajahn 
Karuṇadhammo}{October 2013}

I remember the first experiences Luang Por Liem had with his heart 
difficulties in Thailand. Apparently the condition had been going on 
for quite some time. He had symptoms of fatigue and probably shortness 
of breath among other things. It wasn't until it was quite progressed 
that he did anything about it. When asked why it took him so long to 
look into it, his response was something like: ``I thought this was 
what the body was supposed to do.'' That's the nature of the body, 
isn't it? And even if we do fix something in the body, something else 
is bound to eventually break.

All of the things that happen to us that we think shouldn't happen are 
simply reflective of the way \emph{saṃsāra} is. It wouldn't be 
saṃsāra if things were always going perfectly. By the mere existence 
of saṃsāra, we know that everything in the material world, the realm 
of the human body or psyche is going to either break, change, or 
deteriorate according to certain laws of biology and kamma. These are 
natural laws that create the flow of arising and passing away. If we 
reflect, we can see how our expectations take us in a direction that 
sometimes leads to dukkha. For example, we get ``the'' diagnosis and we 
think, \emph{Why me? I've done everything right. It's too soon. Why 
should this happen? Going to doctors and having treatments isn't what I 
was planning to do for the next few months or years!} We can feel 
extremely agitated and afraid in situations like this. Or maybe 
something in the material world breaks and we really don't have time to 
deal with it---it feels frustrating. All of us experience these kinds 
of things, but we try to keep in mind and reflect that all of 
this---particularly the experience of old age, sickness, and death---is 
what we should expect to have happen. In fact, it's a miracle that it 
doesn't happen a lot sooner for most of us because this body is in a 
very fragile condition. Ajahn Chah is quoted as saying, ``Everything we 
need to know can be learned from nature, if we just watch and pay 
attention.'' Whether it's in the material, physical, or mental realm, 
we can watch the nature of everything. We can see it arise, have its 
own life, and then change and pass away. That's the fundamental insight 
we need to have to truly understand saṃsāra and eventually realize 
complete liberation.

