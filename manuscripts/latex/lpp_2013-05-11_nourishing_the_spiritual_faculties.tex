\mychapter{Nourishing the Spiritual Faculties}{Luang Por Pasanno}{May 
2013}

We just had Ajahn Dtun staying at the monastery, and he very generously 
shared teachings with us. For most people here, I think, his presence 
and way of being were uplifting and generated the arising of faith. 
Now, whenever faith or confidence arises in us---whether from the 
presence of a teacher or whatever the source---it's important to use 
that faith to benefit our practice. The wholesome result of faith is 
the inclination to apply effort or energy. In this way faith is 
nourishment for effort and energy in our practice.

These two qualities are among the Five Spiritual Faculties: 
\emph{saddhā}, \emph{viriya}, \emph{sati}, \emph{samādhi,} 
and\emph{pañña;} faith, energy, mindfulness, concentration and 
wisdom. None of these factors stands alone. They support, feed and 
nourish each other. We need to attend to saddhā---faith or 
confidence---in a skillful way, so that it helps nourish the qualities 
of effort, mindfulness, wisdom and samādhi. Together these qualities 
bring a steadiness of mind, a clarity of awareness. In the suttas, the 
Buddha teaches that the Spiritual Faculties are a direct source of 
awakening. So when we generate and nurture them, we are directly 
nourishing the conditions for awakening.

In his teachings, Tan Ajahn Dtun emphasized samādhi---concentration or 
collectedness of mind---the fourth Spiritual Power. In doing so, he 
also specified that the function of samādhi is to stabilize and steady 
mindfulness. Ajahn Chah taught this as well. This is important because 
it's easy for us to get caught in up our preconceptions about the 
various states of samādhi and its purposes. Having read or heard about 
them from various teachers, we may misunderstand the chief reason to 
practice samādhi---again, it's the stabilization of mindfulness. When 
mindfulness isn't steady, we have to return to a wholesome meditation 
object and use formal exercises for developing samādhi, such as the 
mindfulness exercises described in the Satipatthana Sutta.

Further, samādhi and mindfulness are mutually dependent---each 
nourishes the other: samādhi develops through the continuity of 
mindfulness practice, and the stabilization of mindfulness requires a 
continuity of samādhi. The continuity of samādhi and the stability of 
mindfulness are essential components of the path to awakening.

Each of these wholesome and skillful Spiritual Faculties positively 
influences and supports so many aspects of the Buddha's path. We can 
get a glimpse of their power when we see them exemplified in the elder 
teachers whom we come in contact with. So it's important that we take 
full advantage of these teachings and teachers. They are invaluable 
supports for our practice and spiritual cultivation.

