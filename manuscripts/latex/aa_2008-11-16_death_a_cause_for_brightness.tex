\mychapter{Death: A Cause for Brightness}{Ajahn Amaro}{November 2008}

There is a skillful and beautiful Buddhist tradition for families and 
friends of someone who has passed away. The family members and friends 
come to the monastery and make offerings that support the monastic 
community. They receive \emph{puñña}, merit from these offerings, and 
they dedicate that merit to the deceased in whatever state of being he 
or she may have moved on to. In countries like Thailand, Sri Lanka and 
for Buddhists who live in the West, there are certain occasions when 
family and friends come to make these offerings: Some short period of 
time after the death, then perhaps six months after the death, and then 
the year anniversary of the person's passing away.

In the chanting that is done on those occasions, we don't recite versus 
like, ``Don't worry, he's gone off to be with the \emph{devas} forever. 
The Lord is looking after him and he'll be happy up in heaven with the 
bunnies and the blackberries.'' Rather, we chant in a more reflective 
way, in terms of the wholesome, unwholesome, and neutral dhammas, the 
internal and external dhammas, which comprise the different mind states 
and qualities of experience. Also, for those left behind, there are 
recollections such as, ``All that arises passes away. Whatever comes 
into being disintegrates, and in it's passing there is peace.'' They're 
not deliberately consoling on an emotional level, but very realistic. 
``Yes, life came to be, and now it's ended. It's dissolved.'' This 
acknowledges the sadness when someone close to us dies, without 
wallowing in that sadness. At the same time, we don't suppress it, 
trying to sugar over everything by thinking, \emph{She's gone to a 
better place}. Well, maybe she has or maybe she hasn't.

In the Buddhist tradition there's a great deal of realism around the 
process of death, and it's important for us all to cultivate this 
quality of realism. What we know is that a life came into being and now 
it's ended. That much we can be absolutely sure of. There's a natural 
feeling of loss and sadness, a sense that the person was around, a 
friend, close to us, but now gone. Even the Buddha experienced the loss 
of those close to him. There's a famous passage that takes place after 
the Buddha's two chief disciples, Sāriputta and Mahamoggallāna, had 
passed away. The Buddha says to the gathered Saṅgha, ``The assembly 
seems empty now that Sāriputta and Mahamoggallāna are no longer 
here.'' So even a fully enlightened being like the Buddha can know and 
sense the loss of friends and companions. That's only natural and to be 
expected. It's realistic.

The customs of gathering together, taking precepts, making offerings, 
dedicating the merit to benefit an individual, are ways of taking that 
feeling of sadness and loss on the occasion of a person's death and 
uniting it with an act which is intrinsically wholesome, a brightening 
act of generosity and kindness. Acts of generosity and dedicating merit 
bring happiness, brightness, and invigoration to the mind and the 
heart; over time, a succession of gathering, making offerings, and 
creating wholesome \emph{kamma} in relationship to that person, slowly 
transforms the occasion of their death from being something associated 
with an experience of loss and absence into something that is much more 
a cause for brightness and happiness to arise.

