\mychapter{The Impact of Right Speech}{Ajahn Karuṇadhammo}{April 2012}

There are three ways we act on the inclinations, impulses, and 
intentions that come through the mind and this is through body, speech, 
or mind. In a monastery, where there are many restraints on our 
activities, we can particularly notice the action of speech. Because 
actions and ideas are often expressed through speech, it's good to 
focus attention on this habit so we can learn about ourselves through 
our speech patterns. In the monastery we attempt to speak not only the 
truth, but to also speak without anger, without talebearing, and 
without frivolous or unnecessary speech. Overall, people do a good job 
with the practice of right speech. However, speech is a difficult area 
of practice, and wrong speech can come out unexpectedly.

We come to the monastery with speech habits formed through family 
upbringing and the company we've kept. In Western cultures there can be 
an encouragement for people to express their thoughts openly, without 
considering how it might affect other people. This conditioning can 
manifest in speech indicative of trying to get ones way or get what one 
wants or being overly persistent. We can also express frustration, 
impress people, or present ourselves in certain ways that might be 
different from how we really are.

These habits in our minds can easily influence the speech we use in the 
monastery and it can take constant vigilance to restrain ourselves from 
speaking in unskillful ways. Sometimes it's appropriate to say nothing, 
as when practicing noble silence. But at other times in a monastery 
there is a need for communication. We need to talk with each other to 
engender a sense of communal living and support as well as maintain 
harmony and well-being. If something needs to be communicated or 
somebody needs support, then skillful speech is appropriately 
encouraged. Nevertheless, we must watch the underlying impulse or mood 
in the mind that serves as the basis for speech. It's important to be 
careful with our speech because people are sensitive and regrets can 
arise when others are hurt through the use of wrong speech. The Mettā 
Sutta suggests that we be not only straightforward in speech, but 
gentle as well. Even if our speech is true, we must be mindful of the 
impact our words have in a community as well as when we are engaging in 
the world outside of the monastery.

