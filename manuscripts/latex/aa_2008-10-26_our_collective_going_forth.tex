\mychapter{Our Collective Going Forth}{Ajahn Amaro}{October 2008}

This is a big day for Venerable Kaccāna. After arriving at Abhayagiri 
about two and a half years ago, today he is making the commitment to 
ordain into the \emph{Bhikkhu Saṅgha} and is taking on the precepts 
of training. It's helpful during occasions like this to reflect on the 
process of ``going forth'' into the 
Saṅgha---\emph{upasamapadā}---which means ``lifted'' or ``raised 
up.'' It not only reflects an outer process, but an inner process as 
well. It's the formal commitment of an individual to this particular 
training and his acceptance into the group of monks. It also reflects 
an inner commitment and an inner change that's useful for us to 
consider, whether we have hair and wear trousers or have a shaved head 
and wear a \emph{sabong.}

Essentially when we talk about ordaining, it's usually in terms of 
going forth from the household life into homelessness, from being a 
\emph{garika} to an \emph{anāgārika}, one who lets go of the 
household life. But in many ways, it's more about going forth from 
self-centered thinking to seeing in terms of Dhamma. It's going forth 
from confusion to clarity, from a life of being half awake or not awake 
at all to wakefulness. That's something that is useful for all of us to 
reflect on whether we're living as a lay practitioner with commitments 
and responsibilities in the world, as an anāgārika, as a 
\emph{sāmaṇera}, or as a \emph{bhikkhu} who has already ordained. If 
we've formally made the commitment, have been ``raised up'' into the 
Saṅgha and have already gone forth, still, the most important aspect 
of what we are doing is that going forth from confusion and 
self-centered thinking to being awake. Unfortunately, we sometimes hang 
on to the formal commitment of having gone forth and taking the 
precepts of a monk, so that after years of experience, we can forget 
the part about waking up, about going forth from confusion to clarity 
and seeing in terms of Dhamma.

Whether we're a lay person or a monastic, this auspicious day of 
Sāmaṇera Kaccāna's going forth can encourage all of us to go forth 
in terms of our attitudes, the way that we relate to the world, 
choosing to be mindful, choosing to be awake, and choosing not to be so 
self-concerned or self-obsessed.

On a practical level we have many tasks to pull together today in 
preparation for the ordination. But, it's good to bear in mind these 
reflections, not just as a philosophical aspiration, but also in terms 
of how we work with each other. We can bring that quality of 
mindfulness and self-relinquishment to the work we're doing---to our 
concerns, the tasks we have, and the way we relate to other people. 
Whether carrying a cumbersome bench through the forest, maneuvering a 
ladder, or setting up the ordination platform, we're bringing that 
quality of wakefulness and attentiveness to the time and place of the 
situation and the way we're functioning with each other. That quality 
of attentiveness will then inform all of our tasks so that the day 
itself becomes a resonance of the gesture of going forth. That's what 
the ceremony is all about. In this way it really makes going forth 
alive and meaningful, informing all of our lives. It's not only helping 
outside, it's also helping us to work on our inner lives, our inner 
worlds, together. It's our collective going forth.

