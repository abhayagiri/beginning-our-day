\mychapter{Truthfulness in Speech}{Ajahn Yatiko}{June 2013}

One way we are generous with each other here at the monastery is in the 
realm of truthful speech---an attribute that is essential in one who is 
walking the Buddha's path. The notion of truthful speech is deep and 
profound. It doesn't simply mean saying things that are truthful; It 
also implies expressing ourselves honestly---being able to clearly 
communicate with others, even when there's a thorny issue at hand.

Sometimes our attempts at communicating can be hopelessly indirect, 
which is not helpful. For example, if we feel somebody is being unkind 
to us, rather than letting them know we might instead ostentatiously 
increase our ``kindness'' around them, hoping they'll get the message 
and change. This rarely works and can sometimes be seen as passive 
aggressive. So it's important to establish genuinely truthful speech to 
directly express our needs to others, at the appropriate time, so they 
can hear and understand us. This takes practice, but it's very much a 
part of the path.

More generally, it is also important to act with a generous heart, no 
matter what we're doing. This requires an attitude of taking 
responsibility, recognizing that all institutions---families, 
monasteries, and communities of all kinds---require people to take 
responsibility, open their eyes, and see what needs to be done to help 
others.

Sometimes when we see somebody in need, physically or emotionally, we 
may think to ourselves, \emph{Well, this is a big place and someone 
else will take care of the situation}. When people think in this way, 
it is like placing an invisible force field around others so that they 
can't see them or their problems. Ajahn Amaro has colloquially referred 
to this phenomenon as if people are approaching the world with an ``SEP 
field''---a somebody else's problem field. In a large community like 
this, it can be easy to take the SEP field approach and ignore the 
problems of visitors or those we live with. But we mustn't let these 
people slip through the cracks or fall by the wayside. This requires a 
resolve to help those in need as best we can, and not just for their 
sake. Our holding to a such a resolve will help improve the overall 
sense of well-being, functionality, coherence, harmony, and brightness 
in the community we're living in.

