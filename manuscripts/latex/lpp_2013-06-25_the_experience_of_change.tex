\mychapter{The Experience of Change}{Luang Por Pasanno}{June 2013}

The weather these days is itself giving us something to reflect upon: 
Yesterday and last night it was raining. This weekend it's expected to 
be over 100 degrees. It should be obvious to us that this is definitely 
uncertain! Ajahn Chah kept encouraging us to investigate the truth of 
uncertainty---that the nature of things is ``not sure''---and to resist 
the inner voice that says, \emph{Without a doubt, things are going to 
be like this, or like that.}

Of course, the external environment is only one source of uncertainty; 
there's the internal environment, as well---our moods, perceptions, 
thoughts, and feelings---and it's this internal environment that's most 
important to examine. Sometimes we may think, \emph{My practice is 
going well. My practice is moving in the right direction. It's really a 
sure thing …} and then it changes. Other times we're feeling stuck in 
a pit of difficulties and may think, \emph{This is really what I am and 
everything is hopeless. This is a sure thing …} and that too changes. 
It's important to point the mind toward the experience of change, 
uncertainty, and impermanence, so it can see the true nature of things. 
Doing this creates a sense of stability and develops the mind's 
capability to know things---to know when we're upset; to know when 
we're peaceful; to know when we're experiencing something wholesome. 
This quality of knowing is our place of refuge.

The uncertainty of things is simply how they are. But it's the way we 
\emph{respond} to uncertainty that we need to focus on. That's what 
reveals the areas in which we have more work to do, and the areas in 
which we have a good handle on the practice. What happens when we 
experience some difficulty and anger arises? What happens when we get 
what we want and an experience of happiness and well-being arises? How 
do we respond to that? Do we take it for granted? Do we make 
assumptions? Do we create certainty around it? Do we get caught up in 
it? Do we create a sense of self around it?

Working with uncertainty in this way is our practice. It's what gives 
us the opportunity to realize true peace. True peace cannot be found in 
a passing mood or a state of mind. True peace comes through recognizing 
the fundamentally uncertain nature of things.

As we go through our day, there's a constant, ongoing flow of change. 
Pay attention to that. Whether doing chores, engaging with people, 
sitting in formal meditation---be attentive, without forcing the mind 
in any way. This is how we can strengthen our place of refuge, the 
quality of clear knowing.

