\mychapter{Where Did It All Go?}{Ajahn Karuṇadhammo}{August 2012}

Today is Beth's birthday. We were talking a few minutes ago about how 
quickly it all happens, how fast the time passes, particularly as one 
gets older. Time tends to telescope and move much faster with age. Beth 
mentioned her mother was recently looking in a shop and saw a 
reflection in the window. Her first thought was, \emph{Who is that old 
lady?} Then she realized that it was herself! I remember a time when I 
was about ten years old and thought to myself, \emph{Wow, in the year 
2000 I'll be forty-five years old.} All of these experiences of time 
are very malleable and fleeting.

We can use this perception of time as a reminder that the life that we 
have is a certain length and we don't know how long it's going to be. 
It may not last beyond the next few minutes or it may go on for many, 
many years from now. The point is that we don't know. For most of the 
time in our lives we are either thinking about the past or projecting 
into the future. We miss the point that it's all happening right here 
in this moment.

So we need to use a sense of urgency, knowing there is an aging 
process, illness, and death. And we don't know when illness and death 
will occur. We only know that it will occur at some point in time. With 
this type of reflection we can develop a sense of urgency to encourage 
our minds to dwell in the present because the present is where we have 
the ability to change and create favorable conditions for the 
development of our minds. It's not thinking about the past, having 
regrets about the past, or making projections into the future; it's 
about responding to what comes our way right here and right now in this 
very moment. Of course, it's easy to believe our thoughts, \emph{Well, 
the time isn't quite right. Maybe when the conditions are better I'll 
start practicing seriously. I have plenty of time left and there's just 
too many responsibilities right now and too many goals I have to 
accomplish. As soon as I get this part of my life together then I'll 
really be able to practice.} But that's a kind of false thinking that 
we get ourselves into. It's a misperception that we're going to have 
all that time when the conditions are perfect, because the conditions 
are never going to be perfect. We have to take the opportunity right 
here and right now to take what's coming our way and work very closely 
with greed, hatred, and delusion.

It's not as if we're trying to create a different person in the future 
who will then be more skilled and capable, \emph{If only I didn't have 
so much anger or craving … I'll work on that so in the future I will 
become somebody who can be much more free.} While it is true that 
things develop over time, it is the effort we put forth right here and 
right now, seeing our reactions and habits just as they arise that 
allows us to change our perceptions and understand our human 
experience. If we keep on putting it off or spend time regretting all 
the things we did in the past or project into the future without paying 
attention to right here and now, then we've missed the opportunity to 
make a genuine change in our present experience. Realization and 
genuine shifts of perspective don't occur somewhere off in the distant 
future. They occur right now in this moment.

