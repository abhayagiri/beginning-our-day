\mychapter{Non-contention Is Full Awareness}{Ajahn Amaro}{December 2002}

Under the Bodhi tree the Buddha's response to death---in the form of 
Māra's threats, cajolings, temptations, and attempts to cause 
doubts---was not life-affirmation, going into deep \emph{jhāna} to 
evade Māra, blasting him with a \emph{vajra} bolt, trying to be 
reasonable and negotiate on Māra's terms, or trying to justify 
himself. Instead, the Buddha's response was a fearless wakefulness. 
Almost invariably, throughout the accounts of the Buddha's meetings 
with Māra, as soon as he is aware of the malefactor's presence, he 
says: ``I know you, Māra.'' And the game is over.

Maybe this is a myth or maybe not, but such tales maintain their power 
through their congruity with truth as we experience it. When Māra 
knows the Buddha has seen the hook inside the bait, he knows his victim 
is not going to bite. Māra is defeated in that gesture of knowing. 
This suggests that the opposite of death is not birth, 
life-affirmation, or the destruction of death, but rather, the opposite 
of death is wakefulness.

Perhaps the most meaningful way of considering the encounters between 
the Buddha and Māra is to regard them as depicting the arising of 
unwholesome, ego-based states in the mind of the Buddha. They portray 
the instinctual fears, doubts, and desires that arise, but have no 
place to land. When using the myth as a map of our own psyches, Māra 
represents our ego-death experiences---loneliness, anger, 
obsessiveness, greed, doubt---and the Buddha's example points the way 
for our hearts to respond most skillfully, with a wise, wakeful, and 
radical non-contention. For as soon as we contend against death we've 
bought into Māra's value system and bitten his hook---when we hate and 
fear death, or want to swamp it with life, Māra has won. We have 
``gone over to Māra's side and the Evil One can do with [us] as he 
likes'' (SN 35.115). We can perhaps run with his line for a while, but 
sooner or later Māra's going to reel us in.

Non-contention is not a passivity, a denial, or a switching 
off---numbly suffering the slings and arrows as they thump into us. 
Rather, non-contention is full awareness. The Buddha doesn't say ``It's 
okay Māra, do your worst, I won't stand in your way.'' No, the point 
is to defeat Māra---but the way we defeat him is by not contending 
against him. In one of the most often quoted passages of the 
Dhammapada, the Buddha states, ``Hatred does not cease by hatred, but 
by love alone. This is an eternal truth.''

