\mychapter{Like a Snake Shedding Its Skin}{Ajahn Yatiko}{May 2013}

Many of us often live in a fog of time. We can become caught up in this 
fog---completely absorbed and obsessed with experiences that happened 
in the past or we hope will happen in the future. It's as if this 
perception of time has a reality to it that's independent of our own 
machinations and creations of the mind. But it's not separate, it is 
created and a part of our perception like everything else we 
experience. Sometimes we get entrapped and absorbed in it through the 
different physical and emotional difficulties we have, and we can't 
seem to rise above it.

But sometimes we \emph{can} rise above it. Sometimes we can experience 
a clear perception of the present moment---the whole aspect of the mind 
that creates time through projection, memory, and hope. It is a vivid, 
here-and-now perception of what we're observing---like a portal, where 
we slip through the delusion of the created world we live in and, 
suddenly, we're in the present moment, which is grounded in reality and 
has a sense of authentic truth.

A metaphor comes to mind of a snake shedding its skin. The Buddha uses 
this image in the most profound sense, but I think it also applies to 
the present moment, where the snake's skin stands for time: this scaly, 
falsely-created outer skin of delusion that's part of our own making. 
We move through it and leave it behind, which makes us naked and 
vulnerable. It is not vulnerable in the sense of being fragile or 
easily damaged. It is vulnerable in the sense of being able to receive 
the reality of the present moment---including everything we create in 
our minds---from a bare, new, and open field of awareness. It's 
freeing, liberating, lucid, and it has the quality of truth to it.

When we're caught up and locked into difficulties and can't access that 
truthfulness of the present moment, we have only one option: patience. 
Sometimes we can't connect with the present moment because we're 
captivated by our actions and the results we're receiving from our 
actions. When that's the case, we have to be patient. There's nothing 
else we can do. Just because we cannot connect with the present moment 
doesn't mean we should give up on the Buddha's path and pack it in. We 
need to be patient and keep with the practice, reestablishing our 
wholesome intentions and our capacity for experiencing a present-moment 
awareness.

