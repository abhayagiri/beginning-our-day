\chapter*{Glossary}
\addcontentsline{toc}{chapter}{Glossary\protect\chapternumberline{}}

{\parindent 0pt \parskip .5em

\glossentrylang{ajahn}{Thai}{Literally, “teacher.” From the Pāli word
ācariya; often used in monasteries as a title for senior monks or nuns
who have been ordained for ten years or more.}

\glossentry{anagārika}{Literally, “homeless one.” An eight-precept male
postulant who often lives with bhikkhus and, in addition to his own
meditation practice, also helps with certain services that are forbidden
for bhikkhus to do, such as, using money, cutting plants, or cooking
food.}

\glossentry{anattā}{Not-self, ownerless, impersonal.}

\glossentry{anicca}{Impermanent, inconstant, unsteady. Ajahn Chah often
translated it as “not sure.”}

\glossentry{asubha}{Unattractive, not-beautiful. The Buddha recommended
contemplation of this aspect of the body as an antidote to desire, lust,
and complacency.}

\glossentry{bhikkhu}{A Buddhist monk; a man who has given up the
householder’s life to join the monastic Saṅgha. He follows the
Dhamma-Vinaya (the doctrine and discipline), the teachings of the Buddha
as well as the Buddha’s established code of conduct.}

\glossentry{brahmavihāra}{The four sublime or divine abodes that are
attained through the development of mettā, karuṇā, muditā, and upekkhā
(boundless loving-kindness, compassion, sympathetic joy, and
equanimity).}

\glossentry{Buddha}{The historical religious leader and teacher who
lived around 2500 BCE in the Ganges Valley of India. After his
enlightenment, he established a monks’, nuns’, and lay order under the 
instruction of what he called the Dhamma-Vinaya—the doctrine and
discipline. The word Buddha literally means “awakened one” or
“enlightened one.”}

\glossentrylang{Dhamma}{Sanskrit: Dharma}{In general, a spiritual or
philosophical teaching describing the natural state of reality. When
used in this book, Dhamma specifically refers to the teachings of the
Buddha: a systematic understanding of suffering, its cause and how one
applies oneself to eliminate this suffering, thus ending the cycle of
rebirth.}

\glossentry{dhamma}{Used as a term to define natural phenomena of the
world, including phenomena of the mind.}

\glossentry{Dhamma-Vinaya}{The Doctrine and Discipline. The name the
Buddha gave to the religion he founded. The conjunction of the Dhamma
with the Vinaya forms the core of the Buddhist religion.}

\glossentry{dukkha}{“Hard to bear,” unsatisfactoriness, suffering,
stress.}

\glossentry{Eightfold Path}{See Noble Eightfold Path.}

\glossentry{Forest Tradition}{The tradition of Buddhist monks and nuns
who have primarily dwelled in forests emphasizing formal meditation
practice and following the Buddha’s monastic code of conduct (Vinaya).}

\glossentry{Four Noble Truths}{The first and central teaching of the
Buddha about dukkha, its origin, cessation, and the path leading toward
its cessation. Complete understanding of the Four Noble Truths is
equivalent to the realization of Nibbāna.}

\glossentrylang{kamma}{Sanskrit karma}{Volitional action by means of
body, speech, or mind. Kamma always leads to an effect (kamma-vipāka).}

\glossentry{Kaṭhina}{A traditional cloth offering ceremony held at the
end of the annual Rains Retreat (Vassa) celebrating community harmony.}

\glossentrylang{khandhas}{Sanskrit skandha}{Heap, group, aggregate.
Physical and mental components of the personality and of sensory
experience in general. The five bases of clinging: form, feeling,
perception, mental formations, and consciousness.}

\glossentry{kuṭi}{A small dwelling place for a Buddhist monastic; a
hut.}

\glossentrylang{Luang Por}{Thai}{Venerable Father, Respected Father; a
friendly and reverential term of address used for elderly monks.}

\glossentry{Māra}{Evil, craving, and death personified as a deity, but
also used as a representation of these elements within the mind.}

\glossentry{mettā}{Loving-kindness, goodwill, friendliness. One of the
four brahmavihāras or sublime abodes.}

\glossentry{Middle Way}{The path the Buddha taught between the extremes
of asceticism and sensual pleasure.}

\glossentry{mindfulness}{See sati.}

\glossentrylang{Nibbāna}{Sanskrit Nirvāṇa}{Final liberation from all
suffering, the goal of Buddhist practice. The liberation of the mind
from the mental effluents, defilements, the round of rebirth, and from
all that can be described or defined. As this term also denotes the
extinguishing of a fire, it carries the connotations of stilling,
cooling, and peace.}

\glossentry{Noble Eightfold Path}{Eight factors of spiritual practice
leading to the cessation of suffering: right view, right intention,
right speech, right action, right livelihood, right effort, right
mindfulness, and right concentration.}

\glossentrylang{Observance Day}{Thai: Wan Phra}{Once a week in Thai
monasteries, monks and lay people set aside work duties and devote their
time for a day (and sometimes all night) to formal practice. If the
practice is continued until dawn the next day, the monks and laity will
often refrain from lying down until dawn.}

\glossentry{Pāli}{An ancient Indian language related to Sanskrit. The
teachings of the Theravada school of Buddhism were transmitted orally in
Pāli for hundreds of years before being written down at the beginning of
the Common Era in Sri Lanka.}

\glossentry{Pāli Canon}{The standardized collection of Theravada
Buddhist suttas written in the Pāli language.}

\glossentry{paññā}{Wisdom, discernment, insight, intelligence, common
sense, ingenuity. One of the ten perfections.}

\glossentrylang{pāramī}{Sanskrit: pāramitā}{Perfection of the character.
A group of ten qualities developed over many lifetimes by a bodhisatta:
generosity, virtue, renunciation, discernment, energy/persistence,
patience or forbearance, truthfulness, determination, good will, and
equanimity.}

\glossentry{paritta}{Literally, “protection.” Auspicious blessing and
protective chants typically recited by monastics and sometimes lay
followers as well.}

\glossentry{pūjā}{Literally, “offering.” Chanting in various languages
typically recited in the morning and evening by monastic and lay
followers of a particular teacher, in this case the Buddha. Typically
these recitations pay homage to the Buddha, Dhamma, and Saṅgha.}

\glossentrylang{Rains Retreat}{Vassa}{The traditional time of year that
monks and nuns determine to stay in one location for three months. Some
monastics will take this time to intensify their formal or allowable
acetic practices.  Monks and nuns will refer to themselves as having a
certain number of Vassa which signifies how many years they have been in
robes.}

\glossentry{right effort}{One factor of the Eightfold Path which
describes how a practitioner endeavors to prevent or abandon unwholesome
qualities as well as maintain and develop wholesome qualities within the
mind.}

\glossentry{right speech}{One factor of the Eightfold Path describing
the proper use of speech: refraining from lying, divisive speech,
abusive speech, and idle chatter.}

\glossentry{samādhi}{Concentration, one-pointedness of mind, mental
stability. A state of concentrated calm resulting from meditation
practice.}

\glossentry{sampajañña}{Clear comprehension, self-awareness,
self-recollection, alertness.}

\glossentry{saṃsāra}{Literally, “perpetual wandering.” The cyclical
wheel of existence. The continuous process of being born, growing old,
suffering and dying again and again, the world of all conditioned
phenomena, mental and material.}

\glossentry{Saṅgha}{This term is used to conventionally describe the 
community of ordained monks and nuns practising the teachings of the
Buddha. However, from a noble or ideal view, it specifically describes
the followers of the Buddha, lay or ordained, who have realized one of
the four levels of awakening: stream-entry, once-returning,
non-returning, or Nibbāna.}

\glossentry{saṅkhāra}{Formation, compound, formation; the forces and
factors that form things (physical or mental), the process of forming,
and the formed things that result. Saṅkhāra can refer to anything formed
by conditions, or, more specifically, thought-formations within the
mind.}

\glossentry{sati}{Mindfulness, self-collectedness, recollection,
bringing to mind.  In some contexts, the word sati when used alone
refers to clear-comprehension (sampajañña) as well.}

\glossentry{sīla}{Virtue, morality. The quality of ethical and moral
purity that prevents one from engaging in unskillful actions. Also, the
training precepts that restrain one from performing unskillful actions.}

\glossentrylang{five spiritual faculties}{pañca bala}{A list of
qualities the Buddha gathered for the attainment of supernormal powers
or faculties. They are: faith, energy, mindfulness, concentration, and
wisdom.}

\glossentrylang{sutta}{Sanskrit sūtra}{Literally, “thread.” A discourse
or sermon by the Buddha or his contemporary disciples. After the
Buddha’s death the suttas were passed down in the Pāli language
according to a well established oral tradition and finally committed to
written form in Sri Lanka just around the turn of the common era. The
Pāli suttas are widely regarded as the earliest record of the Buddha’s
teachings.}


\glossentrylang{taints}{āsava}{Mental effluents, fermentations, or
outflows. Four qualities that taint the mind are sensuality, views,
becoming, and ignorance.}

\glossentry{Triple Gem}{The Threefold Refuge: the Buddha, Dhamma, and
Saṅgha.}

\glossentrylang{tudong}{Thai}{The practice of wandering in the country
and living on alms food.}

\glossentry{Upāsikā Day}{A day for Abhayagiri lay devotees to visit the
monastery and partake in an afternoon teaching.}

\glossentry{Vinaya}{The Buddhist monastic discipline or code of conduct.
The literal meaning of Vinaya is “leading out,” because maintenance of
these rules leads out of unskillful states of mind. The Vinaya rules and
traditions define every aspect of the bhikkhus’ and bhikkhunīs’ way of
life.}

\glossentry{Visuddhimagga}{A post-canonical collection compiled by the
Bhikkhu Bhadantācariya Buddhaghoṣa in the fifth century. It is a
treatise explaining in detail the path to enlightenment.}

\glossentrylang{wat}{Thai}{A monastery.}
}
