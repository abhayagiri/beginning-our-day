\mychapter{Snow on a Forest Trail}{Ajahn Jotipālo}{December 2013}

This morning I was walking through the forest, on a path that I've 
walked thousands of times before, and everything looked completely 
different. Two conditions had changed: there was moisture in the air 
and the temperature had dropped below freezing. This completely 
transformed the path. Most of the trees were covered with heavy snow 
and ice. It was quite beautiful.

I was contemplating this in terms of the characteristic of not-self, as 
a reminder of how we can take our thoughts, moods, and opinions and 
believe them to be who and what we are. We may have a thought and then 
some reaction, like anger, but that anger is not who we are---it's 
merely a transient phenomenon, like the snow on the trail. It's merely 
the result of some condition that came into being. Whatever habitual 
reactions we may have, they have been conditioned into us. They're all 
due to past causes. And just like snow will melt given a certain 
temperature, when the conditions change for us, our reactions may can 
change.

I used to be bothered by a couple of people in the community, the way 
they habitually reacted in certain situations. Then one day it hit me 
that those people were merely reacting exactly as they'd been 
conditioned to react in that particular situation. This has given me a 
lot of space around difficult interactions that occur with other 
people, and I've been able to recondition my own habitual reactions. 
When I experience aversion to somebody, I can think to myself, \emph{If 
they could act differently they would act differently. }Or I might also 
think\emph{, When the conditions change, this may no longer be their 
habitual reaction. }It can be that impersonal.

This is something we all can do. And when we encounter a situation with 
that sort of wisdom, we can respond with compassion, equanimity, and 
understanding, and not experience the dukkha of aversion. We can 
transform our inner landscape, as with the snow fallen on a forest 
trail.

