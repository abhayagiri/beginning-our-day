\mychapter{Comfortable in Any Circumstance}{Luang Por Pasanno}{July 
2013}

It looks like it's going to be hot again today. Most of us are 
uncomfortable when it's hot like this. So what we need to do, as Dhamma 
practitioners, is learn how to adapt. We learn to dwell with 
mindfulness and equanimity whether things are to our liking or not. The 
tendency is to wait for conditions we like, and, when they arise, only 
then do we say to ourselves, \emph{Okay, now I can practice.} It's an 
attitude that can easily become habitual. But being tied to the human 
condition as we are, it's rare for things to be just right. So what we 
need to do is develop a willingness to work with conditions as they 
are. If it's really hot, then we adjust our pace accordingly; perhaps 
we slow down what we're doing. We may say to ourselves, \emph{Really, 
truly, I can't practice now. I have to wait for this to be over and 
then I can practice.} That's a common response, but we shouldn't waste 
our time with it.

Often we deal with imperfect conditions by getting in touch with our 
``inner complainer'' that's whining away, going on and on about how 
miserable we feel. Instead of mindlessly doing that, we can use 
challenging circumstances as a means of investigating the habit of 
complaining. We do this by watching the mind trying to convince itself 
that physical discomfort automatically means we have to experience 
mental discomfort. When the mind is adopting that sort of 
misunderstanding and complaining about the circumstances, we observe 
how this simply perpetuates suffering. It's not that we're trying to 
sugar coat the tendency to complain by saying to ourselves, \emph{Oh, 
isn't this wonderful? I just love it when it's 108 degrees outside.} 
Instead we're facing reality and being honest with ourselves. At the 
same time, we understand that simply because circumstances are less 
than ideal, they do not also have to be a source of complication or 
oppression.

The point is to distinguish between the direct, physical experience and 
the layers of mental complication we add to that experience. When we do 
that, it gives us an inner refuge, allowing us to be comfortable in any 
circumstance. That's one of the magical things about Dhamma practice. 
We can be at ease and clear in any circumstance if we're willing to 
direct our attention in a skillful way.

