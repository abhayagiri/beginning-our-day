\mychapter{Small Choices Lead to Big Decisions}{Ajahn Yatiko}{November 
2012}

A lot of big decisions we make in life are dictated by the many small 
decisions we make on a daily basis. That can be a very powerful 
reflection, and one to keep in mind.

In the book \emph{Crime and Punishment} by Dostoyevsky, the main 
character offhandedly fantasizes about killing a certain woman and 
stealing her money. He's not really serious about it, but he asks 
himself, ``What if I \emph{were} serious?'' He plays around with the 
idea, but realizes that he's certainly not committed to the idea, and 
decides not to proceed any further. But then, without even being aware 
of it, through a host of minor decisions, he manages to put into place 
various minor conditions that make the murder a real possibility. He 
goes from thinking the murder ridiculous, to feeling like he has no 
choice in the matter---he has to commit the act---and he does. It's 
like setting in motion an avalanche: a little movement here, a little 
there, doing this, doing that---then all of a sudden those seemingly 
minor conditions come together and produce a catastrophic result.

So it behooves us to reflect on the little choices we make, the small, 
moment-by-moment decisions triggered throughout the day by the various 
situations we find ourselves in. This reflection is intended to 
encourage us to be more circumspect about all those small choices. We 
need to remember the causative potential of a choice we make in a 
moment of heedlessness. It might seem like a small matter---and it 
might indeed \emph{be} a small matter---but when coupled with other 
small matters, the cumulative effect can have a great impact. We can 
feel that external circumstances have trapped us into making a big 
decision in which we have no choice. But external circumstances aren't 
the culprits. We are. It was all those little, heedless choices we made 
along the way.

