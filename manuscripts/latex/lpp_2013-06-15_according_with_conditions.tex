\mychapter{According With Conditions}{Luang Por Pasanno}{June 2013}

In terms of living as monastics and lay practitioners, there are two 
helpful principles we can return to again and again in our daily life. 
The first of these is learning how to accept and adapt to whatever 
conditions we find ourselves in. This doesn't mean being indifferent or 
not dealing with things, but really engaging with conditions in a 
skillful, attentive way. With so many visitors coming to the monastery 
today we may think, \emph{I don't want this; it's not how I like it.} 
Or we may welcome the interaction and find ourselves drawn into that 
contact. Neither reaction is very skillful. The wholesome alternative 
is to be mindful of the ways we react to the various conditions we 
encounter. What are our habits? How can we develop habits that better 
accord with Dhamma, that accord with changing conditions, so that a 
sense of equanimity and balance is more readily available to us? That's 
very much a part of monastic training in this lineage---and how Ajahn 
Chah trained the monks who came to live with him.

The second principle is renunciation---\emph{nekkhamma}---which is an 
integral part of adapting to conditions. The English word 
``renunciation'' suggests that we're pushing away or running away from 
something. But that doesn't reflect the real meaning of nekkhamma, 
which is a sense of rising up to conditions with a noble attitude. It's 
a quality that brightens the mind and allows us to engage with the 
Dhamma. If we neglect opportunities to practice nekkhamma, we miss much 
of what monastery training is for. Ajahn Chah used to speak about 
people who became discontented while practicing in a monastery. They 
might leave and go out into the forest, which was fine for a while, but 
then they'd get fed up with the forest and go off to the seashore to 
practice, and after a while they'd get fed up with that, so they would 
go off to the mountains, and after a while they'd get fed up with that 
too. They neglected to practice renunciation in the circumstances they 
were in---they didn't engage or rise up to that opportunity.

So in our daily lives, it's important that we apply these two training 
principles: accepting and adapting to conditions, while sustaining a 
noble attitude of renunciation. These principles can serve as 
aspirations for everybody, lay and monastic, because we're all apt to 
spend so much time and effort trying to manipulate circumstances to get 
what we want, grumbling and complaining about how things are. Instead, 
we can learn to accommodate, rise up and meet conditions with a sense 
of relinquishment, letting go of discontent. When we practice like this 
we are likely to find that there is no need to make conditions into a 
problem for ourselves.

