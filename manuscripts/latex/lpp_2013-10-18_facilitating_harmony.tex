\mychapter{Facilitating Harmony}{Luang Por Pasanno}{October 2013}

In several suttas, the Buddha points to \emph{cāga} as a quality that 
facilitates harmony. \emph{Cāga} is an interesting word. It means 
``giving'' or ``sharing'' and also ``giving up.'' So it's not only the 
quality of generosity, but also the ability to give up our fears, 
views, and opinions---things that end up creating moods and feelings of 
disharmony.

Another quality the Buddha points to that facilitates harmony is 
\emph{piyavācā}---endearing, timely, and kindly speech. We use 
piyavācā in all our interactions, such as when we express our wishes, 
needs, and requests. With piyavācā, it is said, our speech will be 
``loved by many.''

It can be a challenge to apply piyavācā when we're tired or when 
we're dealing with a lot of people at the same time. Cāga, too, can be 
difficult, especially when we're feeling stressed or fearful. But 
they're both part of the training. We train to recognize when our 
speech isn't endearing or kind and then we reestablish our intention 
and start again. Likewise, we train to recognize our resistance to 
sharing or giving up and then make an effort to let go. By starting 
over with a new intention of cāga or piyavācā, we again place 
ourselves in a position of creating harmony.

