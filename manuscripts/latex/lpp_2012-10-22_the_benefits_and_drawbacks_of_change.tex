\mychaptertoc{The Benefits and Drawbacks of Change}
{The Benefits and\\Drawbacks of Change}
{Luang Por Pasanno}{October 2012}

There is a feeling of change today since the rain has come. The rain on 
the roof is an unfamiliar sound after months and months of dry summer. 
As I came down this morning for \emph{pūjā}, there was the smell of 
moisture in the air. We can pay attention to that feeling of change.

The mind tends to look at things from a one-sided perspective. We might 
be people who love the warm summer and hate the cold rainy time, or we 
might like the colder wet weather and hate the heat. But nothing is 
ever one-sided like that. Everything has its benefits and drawbacks, 
and the Buddha instructs us to investigate that.

This includes investigating the drawbacks of things we like and are 
excited about. It's starting to get cool, the dust has settled, the 
plants are coming out, and the forest is regenerating---these are the 
benefits of this lovely weather. At the same time, a gutter is leaking 
and dripping down the side of the wall onto the deck and needs to be 
repaired.

There was a group of Pau Auk Sayadaw's lay disciples who were planning 
on coming up today to have a look at Abhayagiri. They were thinking 
about how to develop their new property for Pau Auk and were excited to 
see what we've done here. However, we received an email from them this 
morning saying that, with the new rain, they found leaks in several of 
their buildings and need to fix them right away. So they have to 
postpone their trip for some time.

There are benefits and drawbacks to everything. If we think in terms 
of, \emph{I like this}, or \emph{I don't like that}, we end up trapping 
ourselves. By looking at experience from a broader perspective and 
applying Dhamma to it, we can more easily recognize that everything has 
two or more sides and see more clearly how problematic it is to take a 
personal position on what we are experiencing. That way we're not 
investing in the I-me-mine of how we think things should be. Rather, 
we're examining the way things are, reflecting on both their benefits 
and their drawbacks.

