\mychapter{This Pūjā May Be My Last}{Ajahn Jotipālo}{September 2013}

There is a fairly well-known sutta where the Buddha indicates that one 
who contemplates death about every few seconds develops mindfulness of 
death heedfully, with diligence; while one who contemplates death every 
few minutes or more develops mindfulness of death heedlessly, with 
sluggishness (AN 8:73). It only takes two or three seconds for someone 
to die from an aneurysm. It's much the same with a massive heart attack 
or any number of other medical conditions and accidental calamities. We 
can be fully functional one moment, and dead two or three seconds 
later. So we have good reason to contemplate death quite often, as the 
Buddha suggested.

Sitting on the Ordination Platform this morning for pūjā, I had an 
unusually strong sense of how beautiful it was, just sitting there. I 
asked myself why I might be feeling this way. We are getting ready to 
go on a trip to Yosemite and there is a national disaster occurring 
just 20 miles from where we will be camping. When I looked online 
yesterday to check out the conditions, there were three major weather 
alerts for Yosemite valley. One was a warning about severe 
thunderstorms, another was a flash flood warning, and the third warned 
about unhealthy levels of smoke in the air due to the large fires 
burning in the area. And of course, there was the risk of fire itself. 
I said to myself sarcastically, Oh, this looks like a wonderful place 
to go camping! Luang Por Pasanno is urging me to go and says it's a 
great opportunity. I keep thinking, Does he just want to send me off to 
my death? Sitting there during pūjā, I had the sense that it really 
could be my last pūjā on that platform.

This is true for every single one of us. Maybe you're not heading off 
to Yosemite, but death could occur at any time---when you're driving 
into Ukiah, or taking the small diesel vehicle up the monastery road, 
or perhaps working in the kitchen where the gas oven might explode. We 
don't need to be obsessive or neurotic about death, but it is good to 
have it constantly in the back of our minds.

Just before coming to Abhayagiri, I used to be on the maintenance staff 
at the Insight Meditation Society, IMS, in Barre, Massachusetts. I 
remember being strongly attached to certain opinions about the way 
things should be done with work projects there, such as using 
particular types of paint and cleaning products. These sorts of 
opinions were a major part of my consciousness in whatever project I 
was involved. But after leaving IMS and coming here, I realized that I 
didn't have a care in the world about what kind of paint they were 
using in Barre. My consciousness of being at IMS had, in essence, 
died---not in a negative way, but because I no longer had any input 
there. When Ajahn Yatiko was the work monk here, he had to put a lot of 
effort into making sure that everything was done correctly on the work 
scene. But I have a suspicion that Ajahn Yatiko, now living in Sri 
Lanka, isn't worried at all about how we are handling the work scene at 
the moment.

These are some of the reasons why I think it's good to contemplate 
death. These contemplations are especially helpful when we find 
ourselves hooked into some minor thing we feel is so important. If we 
were diagnosed with terminal cancer, then some trivial issue or a 
difficulty with a particular person would probably seem insignificant. 
So I encourage everyone to frequently bring up these contemplations on 
death. And the next time you're sitting on the Ordination Platform for 
pūjā, enjoy the beauty there, and consider, This pūjā may be my 
last.

