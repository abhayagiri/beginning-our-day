\mychapter{Loving-Kindness for Ourselves}{Ajahn Yatiko}{December 2012}

When we do \emph{mettā} meditation---loving-kindness meditation---it's 
often good to start with ourselves. But when doing that, it is 
important that we not put ourselves under our thumbs---making demands 
about who we are and what we should be. When this happens, it's as if 
we're looking at the mind the way judgmental parents look at their 
child, power tripping and demanding that the child behave in a 
particular way. \emph{I'm going to tell this kid what to do, and he's 
going to do what I say!} There can be no joy in approaching the mind 
with the assumption that it will tow the line if we force it to do so. 
Joy has to be unfettered, with a sense of real freedom.

When we practice loving-kindness meditation, we turn our awareness and 
consciousness toward our sense of being, our sense of existence, and we 
give ourselves the space and freedom to be exactly as we are. There are 
no demands placed on us---not on our physical bodies or on our 
personalities. All of our expectations are irrelevant; they are a 
fiction. We give ourselves freedom to be what we are, and from that 
freedom we wish for ourselves, \emph{May I be happy.}

For many of us, there can sometimes be a sense, or even an insight, 
that our \emph{modus} \emph{operandi} is not to be happy. For myself, 
when I look at my own behavior, it doesn't always feel like it's coming 
from a wish for happiness. It can be like that for all of us, so we 
cultivate mettā as an antidote. We turn our awareness toward mettā, 
and eventually we may have a deeper insight: \emph{Actually, I really 
do want to be happy, and my heart genuinely wants that freedom.} The 
greatest freedom we can experience is not the freedom of letting 
ourselves do what we want out in the world. The Buddha says that 
freedom in its highest form is freedom from affliction; it's not from 
the gratification of the senses. It's simply the experience of sitting 
here with very little suffering occurring for us; we're not making 
problems out of our experience.

When we have loving-kindness for ourselves, we want to imbibe the sense 
that we are what we are and we don't need to make a problem out of any 
of it. We want to tune into the sense that generating loving-kindness 
for ourselves \emph{is} what our hearts want---we want to be happy and 
free from suffering. We genuinely want to care for ourselves. It's 
almost insane how we don't act in our own best interests, how we forget 
that this is actually what we need to be doing. So we need to make a 
conscious effort to value our own well-being and to recognize the 
heartfelt wish for our own happiness.

When doing loving-kindness meditation, we give ourselves space, 
freedom, and acceptance. We express ourselves from a nonjudgmental 
attitude, \emph{I am what I am, and however I am, that's okay. }We 
bring up the wish, \emph{May I be happy.} Then we relax into that space 
and see what happens.

