\mychapter{Listening to Reflections}{Ajahn Yatiko}{May 2013}

I'd like to share an experience I've had, with the intention to say 
something useful---to plant some seeds: Sometimes when monastery 
residents or visitors attend these morning reflections, they can place 
the teacher on a pedestal, as if he is about to say something that will 
magically bring them insight; but, after listening for a minute or so, 
they make up their minds that nothing worthwhile is being said and then 
tune out. It's as if they feel it's the teacher's responsibility to 
infuse them with something profound. But it's not. As listeners, it is 
\emph{our} responsibility to pay attention and extract meaning from 
whatever is said. It's up to us. That's a change in attitude from 
simply expecting the teacher to have all the information and deliver 
something of value while we remain seated passively. With a different 
attitude we could say to ourselves, \emph{I'm the one with the power 
here. It's my life, and I have to do what is going to be beneficial and 
meaningful for myself.}

We should remember that. At the end of a reflections talk---whatever 
has been said---we're the ones who have to take responsibility for its 
impact. What are we going to do with what's been said? The teacher's 
reflection can be something extremely simple. But if we are listening 
to it with the right attitude and in the right way, it can trigger 
something very valuable in our minds and hearts. Conversely, we might 
hear a talk that is extremely profound, rich, deep, subtle, and 
meaningful. However, if we are only sitting here because we always sit 
here at this time of day or we do not make an effort to extract meaning 
from what is said, then we are not going to get anything out of it. It 
all comes down to the quality of attention we bring to the present 
moment and how we listen to the Dhamma.

