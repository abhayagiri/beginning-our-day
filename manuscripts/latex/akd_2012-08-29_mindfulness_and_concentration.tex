\mychapter{Mindfulness and Concentration}{Ajahn Karuṇadhammo}{August 
2012}

Yesterday, at teatime we were talking about right concentration, 
\emph{sammā} \emph{samādhi}. One of the guests staying here had a 
question about mindfulness versus concentration. It's a good subject to 
reflect on because we can sometimes be hard on ourselves when we are 
trying to concentrate our minds and the practice and doesn't feel like 
it's going so well. This can happen when we have some fixed ideas about 
the nature of concentration. As Luang Por Pasanno says, even the word 
``concentration,'' taken by itself has a connotation of a narrow focus 
that's exclusive of other experience. He used an analogy between this 
tray here and samādhi. The tray is a good example of samādhi in that 
we need a firm foundation and the glass on the tray doesn't form a 
strong base like the tray does. In samādhi we are looking for a tray 
rather than a glass. After he explained this---a few minutes 
later---the glass spontaneously burst in front of us. I don't know if 
it was merely by chance, but it was a well-timed example of the 
instability of that glass samādhi.

It's good to keep in mind that sammā samādhi is dependent on and 
supported by the right application of the other path factors, 
specifically right mindfulness and right effort, but also all of the 
other factors: right view, right intention, right action, right speech, 
and right livelihood. All of those factors have to be in play for 
samādhi to be right concentration. It is not independent of those 
other factors and when we don't have significant amounts of time for 
intense, long periods of formal sitting meditation, the work we do with 
the other factors of the Eightfold Path forms that firm foundation, 
making it a broad base of practice. To begin with, we have to have 
right view, starting with the knowledge that the actions we take in 
body, speech, and mind all have an effect in firming up that foundation 
and establishing peace of mind. How we go through the day---the 
mindfulness we have when walking, doing dishes, working outside, 
working in the office and the attention we bring to what we're doing, 
even though it is quite active and engaged---is helping to form that 
firm foundation.

When we skillfully practice the Dhamma all of those factors work 
closely together and act as a base for that type of collectedness of 
mind. It's not a forced activity. We engage throughout the day with all 
of the other aspects of the Noble Eightfold Path to have more ability 
to enter into a quiet, collected, enjoyable, peaceful state of mind. 
This concentration is not a result of having to go into sensory 
deprivation or exclusion of all experiences so we have a few moments of 
peace. That's not the kind of samādhi that is going to be stable, 
long-lasting or even enjoyable. When we go through periods of practice 
that don't seem very fruitful or when we feel our minds will never 
settle down, we can bring to mind that the sitting practice and 
development of concentration is an important part of practice, but it's 
only one part of the practice. It needs to be supported by and firmly 
grounded in the other factors of the path as well.

