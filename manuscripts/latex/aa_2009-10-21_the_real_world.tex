\mychapter{The Real World}{Ajahn Amaro}{October 2009}

What does a Buddhist monk know about the real world, anyway? It's a 
common question because there's a sense of the monastery being an 
isolated sanctuary where we say, ``Good-bye cruel world,'' and then 
come into our beautiful, sacred space, and suddenly we're spiritual. 
That's a bit of a sweeping generalization, but it's often the way 
people think. What is a monastery? What is the purpose of a sanctuary 
like Abhayagiri? And what is the ``real world?'' Anyone who has stayed 
here for more than a few hours realizes that far from getting away from 
the real world, monastery life is designed to be a place where we meet 
that world---the real world of our own perceptions, preferences, fears, 
desires, and opinions.

Even those of us here might think, \emph{Oh, there are so many 
difficulties in the world---social stresses, problems of climate 
change, the collapsing economy, so many suffering beings in the world. 
What am I doing in a monastery? How am I helping? Am I just trying to 
hide away from the real world?} Those are reasonable questions. But 
what we find at the monastery is that because there is a meeting with 
the real world of our own minds and bodies and the physical reality of 
our existence, we are, in an important way, more genuinely engaged with 
the real world than when we're running around outside. The blur of 
activity in an ordinary, everyday life, even when it is involved with 
compassionate and beneficial activity, can also entail being 
disconnected both from oneself and those around oneself.

Living in the monastery and undertaking monastic training is about 
bringing our attention to ordinary everyday activities. The way we 
carefully put two pieces of PVC pipe together, chop an onion, clear a 
trail, or edit a Dhamma talk, and the way we carefully bring our 
attention to each of those activities---attuning the mind to the 
present moment---is what creates a sacred space. That is what makes 
this place a monastery, rather than an aggregation of individuals 
following their own wishes, opinions, and habits. Sanctuaries such as 
this are a tremendous benefit and blessing to the world, because they 
let people know there are places where others will not lie to them, try 
to cheat them, try to flirt with them, get money from them, or wish 
them harm. This is a tremendous gift.

We may ask ourselves, \emph{How am I helping the world by chopping 
onions or pruning branches on the trail?} When a reasonable doubt like 
that arises, it's important to realize that the intention to bring 
mindfulness, care, and the cultivation of unselfish conduct into these 
simple and apparently insignificant acts is a way in which we 
\emph{are} helping the world. The very fact that this monastery exists 
at all, with a couple dozen people choosing to live and train 
themselves in this way---it does have an effect. It is a beneficial and 
guiding presence in people's lives all over the planet and is truly 
something in which to rejoice.

