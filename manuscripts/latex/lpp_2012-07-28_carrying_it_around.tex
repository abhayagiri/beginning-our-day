\mychapter{Carrying It Around}{Luang Por Pasanno}{July 2012}

We've had visitors from Thailand in the past four weeks, and I think 
it's important to have this kind of contact and connection with elder 
monks in our tradition. During this time, there's been a fair amount of 
coming and going, which has been conducive to a lot of busyness. So now 
we can allow the monastery to settle into a bit more simplicity---the 
simple rhythm of our daily duties, chores and practices. We'll take the 
next bit of time to practice on our own in the evenings, giving each 
other the opportunity for a bit more space and solitude.

Sometimes we have space, but we just fill it up, distracting ourselves 
with socializing and chatting. This is a good time to get back to our 
\emph{kuṭis} and keep that theme of simplicity going. We can sit in 
our kuṭis by ourselves, proliferating about all sorts of things, and 
miss that opportunity for inner simplicity. Alternatively, we can 
attend to the simplicity of watching the breath or doing walking 
meditation at our kuṭis and learn how to delight in that.

When the Buddha taught his Aunt Mahāpajāpatī about the Dhamma in 
brief, he emphasized non-proliferation or non-complexity as a 
characteristic that aligns itself with \emph{Dhamma-Vinaya}, the 
teachings and the discipline. We can add our complications and 
complexities to everything around us and forget that this training, 
this practice, in and of itself---if we attend to it and maintain our 
focus---is a natural shedding of complication and complexity.

We have a habit of carrying around all sorts of proliferations and 
complications. Because we believe in them, we invest in them. Just 
think of all that complexity, all the planning, worrying about, and 
going into trepidation about some external event that's going to 
happen, or not going to happen, and carrying around in our minds all 
the people in our social environment: \emph{Oh, this person said that 
and that person said this,} on and on. When we're caught up in that 
sort of thing, it's time to ask ourselves: \emph{Is this really 
necessary? What's the point?} \emph{If the Dhamma-Vinaya of the Buddha 
is for non-proliferation and non-complication, then why do I insist on 
carrying all that around? How do I put it down? How do I return to the 
principles of Dhamma-Vinaya and continue with the training?}

Part of the answer is to give ourselves the time and space to practice 
at our dwelling places in the forest. Obviously, we still need to 
exercise skill when coming into contact with other people. But 
\emph{kāyaviveka}---physical solitude---helps develop the heart and 
mind. That's what we're here for, so let's focus on that.

