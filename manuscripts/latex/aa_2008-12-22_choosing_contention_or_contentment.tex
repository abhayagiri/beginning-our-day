\mychaptertoc{Choosing Contention or Contentment}
{Choosing Contention\\or Contentment}
{Ajahn Amaro}{December 2008}

On the winter solstice I led a daylong retreat at Spirit Rock with the 
theme, ``Maximum Darkness.'' We investigated the experiences of death, 
loss, and sadness. It was a suitably dingy, gray, wet day, appropriate 
to that dolorous subject. For many years, one of the exercises I used 
in my meditation---and a theme I had others use on this daylong 
retreat---was to imagine the current sitting I was doing to be the last 
minutes of my life. When we did this together for the day, I didn't let 
the others know how long the sitting was going to be---ten minutes, 
twenty minutes, half an hour---so that we could really focus on each 
breath, each moment, as if it were the last moment of our lives.

I encouraged the participants to reflect: \emph{If this truly was the 
last few minutes of my life, then what happens to my priority list?} If 
we take seriously that the bell is the last moment, then consider 
what's important and see what comes up in the mind with all of those 
things to do, the anxieties about going to the dentist, and so on. Just 
reflect on that, \emph{How do I relate to the things I've left half 
done, the things I'm so proud of, the things I regret?}

I've found this theme very useful, and for years when I was at 
Chithurst Monastery I would make it a daily practice. Luang Por Sumedho 
would always be the one ringing the bell at the end of the sittings, 
and at the beginning I would imagine, \emph{Okay, this is the last 
forty-five minutes or last hour of my life. What will my mind be 
dwelling on when the bell is rung? Can I drop it and be ready to go by 
the time the sound of the ringing fades?} It's often quite shocking, 
the kind of things that the mind is obsessed about or focused on. So we 
can ask ourselves, \emph{Would I really want to be thinking of this or 
dwelling on that at the time of death?} That reflection then allows us 
to develop the capacity to drop things, to let go.

One woman at the retreat saw a dichotomy in her mind, she saw a certain 
choice. The choice was between contention and contentment. What took 
shape in her mind was a clear distinction between being ``content 
with'' the way things are, or ``contending against'' the way things 
are. I thought it was quite insightful and interesting how her mind had 
produced that way of formulating the dichotomy. She authentically 
summarized the choice we all have at any moment during the day. Working 
out in the cold with the rain dripping down our necks, carrying large 
uncomfortable objects across slippery mud not wanting to drop them, 
finding places to store 300 apples and oranges in a confined space, 
whatever it might be, we have the capacity to \emph{contend against}, 
and we have the capacity to be \emph{content with}---to attune 
ourselves with the way things are as we go about our tasks. It's a 
choice.

Often we relate to obstacles in our lives as being unavoidable. It's as 
if they've been put in our way deliberately, and we feel burdened and 
frustrated by them. There's a quality of resentment, negativity, a 
desire to avoid or to switch off. But it's important to recognize that 
even if something is particularly obstructive---carrying a large clumsy 
object across slippery wet clay in the rain---it's up to us whether we 
tense up about it and contend against it. It's our choice to buy into 
our story and contend against the way things are, becoming anxious or 
irritated. Or not. This is a theme I often use for my own practice, to 
recognize, \emph{Its my choice whether I make a problem out of this, or 
not. It's my choice to be content with this, or not.}

Our application of mindfulness and clear comprehension, \emph{sati} and 
\emph{sampajañña,} has a lot to do with recognizing that we have 
choices in how we react to the world around us and to our internal 
states of mind. If we also apply the quality of wisdom or discernment, 
\emph{pañña}, then we will choose the path of non-contention, of 
cultivating the capacity to attune. Even in the midst of the most 
difficult or challenging circumstances, we don't have to dwell in 
aversion, we don't have to contend against these circumstances. We can 
find a quality of peace and clarity in relationship to every situation, 
even those we would not choose. We begin to see that there's no thing 
in life that is inherently obstructive or unsatisfactory. It all hinges 
on the way we choose to handle it.

