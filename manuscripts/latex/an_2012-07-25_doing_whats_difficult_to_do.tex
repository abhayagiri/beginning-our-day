\mychapter{Doing What's Difficult to Do}{Ajahn Ñāṇiko}{July 2012}

Living in a monastery can be very difficult to do---eating one meal a 
day, keeping precepts, trying to live and work together as a harmonious 
community. But as Master Hua has said, if we want to practice Dhamma, 
we have to do what's difficult, what others would not choose to do. 
Even though most people wouldn't choose to live in this way, there's an 
enormous benefit to what we're doing here. Living in community, we 
learn how to hold things lightly and take responsibility for what's 
happening in our minds.

During the work period, it's easy for one little event to trigger 
anger, irritation, or some sort of desire. If this happens when we're 
interacting with someone, the mind convinces us that it's the other 
person's fault, as if these defilements came from that other person. So 
it's good to remember that irritation, anger, and desires come from 
within our own minds. They arise there and then seek external objects 
to latch on to. That's why sometimes, no matter how easygoing we may 
be, when we come in contact with some little thing or other we can 
become irritated: the conditions for that irritation were already 
there. And when we get stuck in irritation or anger with another 
community member, we can forget that it's \emph{dukkha.} We're 
suffering and the other person is suffering too. Sometimes we can get 
angry at someone else because they're angry. This is ridiculous, but 
it's the way it works. The Buddha said, ``If you don't react with anger 
to an angry person, then you win a battle hard to win.'' So when 
interacting with others, we may need to swallow our pride from time to 
time, as difficult as that may be.

I remember during my third \emph{vassa} in Thailand, I was doing 
walking meditation every day for several hours after the meal. There 
was a lot of doubt coming up. I was literally driving myself to tears 
with constant doubts in the mind. There I was on the walking meditation 
path, no one else is around, and I'm mired in suffering because of this 
unstoppable mental proliferation. That's how it is sometimes. But these 
experiences are good to have---by going through them, we become 
stronger. Even so, it's difficult to do.

Master Hua has said that we're aiming to develop a ``long-enduring 
mind''---a mind that can sustain itself through the months and years 
without sinking far down, becoming dark, falling out of the robes, or 
not wanting to be in a monastery anymore. To develop this long-enduring 
mind, we need to come back to peace, to the skillful actions that help 
sustain us, and to the factors of letting go. We need to come back 
again and again, moment by moment. It may be difficult to do, but the 
reward is true freedom.

