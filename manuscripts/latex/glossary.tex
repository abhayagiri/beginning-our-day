\chapter{Glossary}

\textbf{Ajahn} (Thai) From the Pāḷi \emph{ācariya}, literally 
``teacher''; often used in western monasteries as a title of senior 
monks or nuns who have been ordained for ten years or more.

\textbf{Anagārika} Literally: ``homeless one.'' An eight-precept 
postulant who often lives with monks or nuns and, in addition to their 
own meditation practice, also helps monks and nuns with certain 
services that are forbidden for monastics to do, such as cutting plants 
or storing and cooking food.

\textbf{anattā} not-self, ownerless, impersonal.

\textbf{anicca} impermanent, inconstant, unsteady. Ajahn Chah 
translated it as: not sure.

\textbf{arahant} Literally: a `Worthy One'. A person whose mind is free 
of defilement, who has abandoned all ten of the fetters that bind the 
mind to the cycle of rebirth, whose heart is free of mental effluents 
or outflows, and who is thus not destined for further rebirth. A title 
for the Buddha and the highest level of his noble disciples.

\textbf{asubha} Unattractiveness, not-beautiful. The Buddha recommended 
contemplation of this aspect of the body as an antidote to lust and 
complacency.

\textbf{bhikkhu} A Buddhist monk; a man who has given up the 
householder's life to join the monastic Saṅgha. He follows the 
Dhamma-Vinaya---the doctrine and the discipline---the teachings of the 
Buddha as well as the Buddha's established code of conduct or 
heightened virtue.

\textbf{brahmavihāra} The four ``sublime'' or ``divine'' abodes that 
are attained through the development of boundless loving-kindness, 
compassion, appreciative joy, and equanimity.

\textbf{Buddha} The word Buddha literally means ``awakened one'' or 
``enlightened one.'' The historical religious leader and teacher whose 
given name was Siddhārtha Gautama of the Sakyan clan. He is said to 
have been born around 2500 BCE in Lumbini, modern day Nepal. After 
leaving his family at the age of 29 and practicing various acetic 
practices for seven years, he took a new direction in practice toward a 
middle path between asceticism and sensual indulgence. Shortly 
thereafter, at the age of 35, the Buddha realized enlightenment. He 
went on to establish a monks', nuns', and lay order, giving teachings 
for the next 45 years. The Buddha died at the age of 80.

\textbf{buddho} Used in the literal sense its meaning is ``awake'' or 
``enlightened.'' It is also used as a meditation mantra, internally 
reciting \emph{bud-} on the inhalation and \emph{-dho} on the 
exhalation.

\textbf{Dhamma} Literally: the truth. The teachings of the Buddha 
explaining the Four Noble Truths---how all beings create suffering for 
themselves and how human beings can learn to be permanently released 
from this suffering, ending the cycle of rebirth.

\textbf{dhamma} used as a term to define natural phenomena of the world.

\textbf{Dhamma-Vinaya} The Doctrine and Discipline. The name the Buddha 
gave to the religion he founded. The conjunction of the Dhamma with the 
Vinaya forms the core of the Buddhist religion.

\textbf{dukkha} ``Hard to bear,'' unsatisfactoriness, suffering, 
stress; one of the three characteristics of all conditioned phenomena.

\textbf{Eightfold Path} See noble Eightfold Path.

\textbf{Forest Tradition} The tradition of Buddhist monks and nuns who 
have primarily dwelled in the forest emphasizing formal meditation 
practice and following the monastic code of conduct (Vinaya) as laid 
down by the Buddha.

\textbf{Four Noble Truths} The first and central teaching of the Buddha 
about dukkha, its origin, cessation, and the path leading towards its 
cessation. Complete understanding of the Four Noble Truths is 
equivalent to the attainment of Nibbāna.

\textbf{hindrances} Impediments to progress in the practice of 
meditation. They are: sensual desire, ill will, sloth and drowsiness, 
restlessness and anxiety, and skeptical doubt.

\textbf{jhāna} Mental absorption. A state of strong concentration 
focused on a single physical sensation or mental notion. Development of 
\emph{jhāna} arises from the temporary suspension of the hindrances.

\textbf{kamma} (Skt. karma) Volitional action by means of body, speech, 
or mind, always leading to an effect (kamma-vipāka).

\textbf{karuṇā} Compassion. One of the four brahmavihāras or 
sublime abodes.

\textbf{kaṭhina} A traditional cloth offering ceremony held at the 
end of the annual Rains Retreat celebrating community harmony.

\textbf{khanti} Patience or forbearance.

\textbf{khandas} (Skt. skandha) Heap; group; aggregate. Physical and 
mental components of the personality and of sensory experience in 
general. The five bases of clinging: form, feeling, perception, mental 
formations, and consciousness.

\textbf{kuṭi} A small dwelling place for a Buddhist monastic; a hut.

\textbf{Luang Por} (Thai) Venerable Father, Respected Father; a 
friendly and reverential term of address used for elderly monks.

\textbf{Māra} Evil and temptation personified as a deity ruling over 
the highest heaven of the sensual sphere; personification of the 
defilements, the totality of worldly existence, and death.

\textbf{mettā} Loving-kindness, goodwill, friendliness. One of the 
four brahmavihāras or sublime abodes.

\textbf{Middle Way} The path the Buddha taught between the extremes of 
asceticism and sensual pleasure.

\textbf{mindfulness} See sati.

\textbf{Nibbāna} (Skt. Nirvāna) Final liberation from all suffering, 
the goal of Buddhist practice. The liberation of the mind from the 
mental effluents, defilements, the round of rebirth, and from all that 
can be described or defined. As this term also denotes the 
extinguishing of a fire, it carries the connotations of stilling, 
cooling, and peace.

\textbf{Noble Eightfold Path} Eight factors of spiritual practice 
leading to the cessation of suffering: right view, right intention, 
right speech, right action, right livelihood, right effort, right 
mindfulness, and right concentration.

\textbf{noble silence} Taking on the practice to only speak when is 
necessary.

\textbf{Pāḷi Canon} The standardized collection of scriptures 
(suttas) in the Theravada Buddhist tradition that have been recorded in 
the Pāḷi language.

\textbf{paññā} Wisdom, discernment, insight, intelligence, common 
sense, ingenuity. One of the ten perfections.

\textbf{papañca} mental proliferation, unskillful thought.

\textbf{pāramī} (Skt. pāramitā) Perfection of the character. A 
group of ten qualities developed over many lifetimes by a bodhisatta: 
generosity, virtue, renunciation, discernment, energy/persistence, 
patience or forbearance, truthfulness, determination, good will, and 
equanimity.

\textbf{paritta} Auspicious blessing chants.

\textbf{pūjā} Chanting in various languages typically recited in the 
morning and evening by monastic and lay followers of the Buddha. 
Typically the recitations pay homage to the Buddha, Dhamma and Saṅgha.

\textbf{Rains Retreat} (Pāḷi: vassa) the traditional time of year 
that monks and nuns would determine to stay in one location for three 
months. Some monastics will take this time to intensify their formal or 
allowable acetic practices.

\textbf{refuges and precepts} Traditionally the laity will ask to take 
the ``refuges and precepts'' by repeating after a monk recites them. 
They take refuge in the Triple Gem (Buddha, Dhamma, and Saṅgha) and 
make the intention to follow the moral training precepts for lay 
people: not killing, not stealing, not engaging in harmful sexual 
behavior, not lying and not taking intoxicants. There are four 
additional renunciant practices for laypeople which include not 
engaging in any sexual behavior, not eating after midday until dawn the 
next day, refraining from entertainment, beautification, and adornment, 
and not sleeping on a high or luxurious bed.

\textbf{right effort} One of the eight path factors of the Eightfold 
Path describing four methods in which a practitioner endeavors to keep 
his or her mind free from defilement. This is done by preventing the 
arising of unwholesome qualities that have not yet arisen, abandoning 
unwholesome qualities that have arisen, encouraging wholesome qualities 
that have not yet arisen, and maintaining and developing wholesome 
qualities that have arisen.

\textbf{right speech} One of the eight path factors of the Eightfold 
Path describing the proper use of speech: refraining from lying, 
refraining from divisive speech, refraining from abusive speech, and 
refraining from idle chatter.

\textbf{sālā} Hall. Another word for mediation hall or Dhamma Hall.

\textbf{samādhi} Concentration, one-pointedness of mind, mental 
stability; state of concentrated calm resulting from meditation 
practice.

\textbf{samaṇa} Contemplative. Literally, a person who abandons the 
conventional obligations of social life in order to find a way of life 
more 'in tune' (sama) with the ways of nature. Also, a wandering acetic.

\textbf{sammā} Right or correct.

\textbf{sampajañña} clear comprehension, self-awareness, self 
recollection, alertness.

\textbf{saṃsāra} The cyclical wheel of existence, literally: 
`perpetual wandering,' the continuous process of being born, growing 
old, suffering and dying again and again, the world of all conditioned 
phenomena, mental and material.

\textbf{Saṅgha} On the conventional level, this term denotes the 
communities of Buddhist monks and nuns; on the ideal (ariya) level it 
denotes those followers of the Buddha, lay or ordained, who have 
attained at least stream-entry, the first of the transcendent paths 
culminating in Nibbāna.

\textbf{saṅkhāra} Formation, compound, formation – the forces and 
factors that form things (physical or mental), the process of forming, 
and the formed things that result. Saṅkhāra can refer to anything 
formed by conditions, or, more specifically, thought-formations within 
the mind.

\textbf{sati} mindfulness, self-collectedness, recollection, bringing 
to mind. In some contexts, the word sati when used alone refers to 
clear-comprehension (sampajañña) as well.

\textbf{sīla} Virtue, morality. The quality of ethical and moral 
purity that prevents one from unskillful actions. Also, the training 
precepts that restrain one from performing unskillful actions.

\textbf{five spiritual powers} (Pāḷi: pañca bala) A list of 
qualities the Buddha gathered for the attainment of spiritual or 
supernormal powers. They are: faith, energy, mindfulness, 
concentration, and wisdom.

\textbf{sutta} (Skt. sūtra) Literally, `thread'. A discourse or sermon 
by the Buddha or his contemporary disciples. After the Buddha's death 
the suttas were passed down in the Pāḷi language according to a well 
established oral tradition, and were finally committed to written form 
in Sri Lanka just around the turn of the common era. The Pāḷi suttas 
are widely regarded as the earliest record of the Buddha's teachings.

\textbf{taints} (Pāḷi: āsava) Mental effluent, fermentation or 
outflow. Four qualities that taint the mind: sensuality, views, 
becoming, and ignorance.

\textbf{Three characteristics:} suffering, impermanence, and not-self.

\textbf{Triple Gem} The `Threefold Refuge'–the Buddha, Dhamma, and 
Saṅgha

\textbf{tudong} (Thai) The practice of wandering in the country and 
living on alms food.

\textbf{upāsikā day} A day for Abhayagiri lay devotees to visit the 
monastery and partake in an afternoon teaching.

\textbf{Vinaya} The Buddhist monastic discipline, literally, `leading 
out', because maintenance of these rules `leads out' of unskillful 
states of mind. The Vinaya rules and traditions define every aspect of 
the bhikkhus' and bhikkhunīs' way of life.

\textbf{vipassanā} Clear intuitive insight into physical and mental 
phenomena as they arise and disappear, seeing them for what they 
actually are---in and of themselves---in terms of the three 
characteristics and in terms of suffering, its origin, its cessation, 
and the way leading to its cessation.

\textbf{Visuddhimagga} A post-canonical collection compiled by the 
Bhikkhu Bhadantācariya Buddhaghoṣa in the fifth century. It is a 
treatise explaining in detail the path to enlightenment.

\textbf{Wan Phra} Observance Day or Monk's Day. Once a week in Thai 
monasteries, monks and lay people set aside work and devote their time 
for a day (and sometimes all night) to formal practice. If the practice 
is continued until dawn the next day, the monks and laity will often 
refrain from lying down until dawn.

\textbf{wat} (Thai) A monastery.

