\mychapter{The Breath Through the Fog}{Ajahn Yatiko}{December 2012}

Ajahn Chah said that when watching the breath, it's important to 
understand that our thinking doesn't have to stop. This is a very 
useful point. Often while we're practicing to watch the breath, we get 
lost in a train of thought and eventually remind ourselves to come back 
to the breath. In many guided meditations, we hear the phrase fairly 
often, ``Come back to the breath.'' We can start to feel that thinking 
is a problem, as if there is a battle between thinking and the breath. 
It can become unpleasant, because we're fighting against our thinking. 
Rather than battling like that, we might instead imagine that our 
thoughts are like fog. We can be aware of the fog of thought, and when 
the breath becomes more prominent we can observe the breath.

Another way to look at it is that the breath is always there, waiting 
to be observed. We look through thoughts to the breath like we look 
through a fog to see a light post or beacon. Rather than feeling we 
have to stop the thinking, we try to see through it. It's okay for 
thinking to be there if we have this attitude of trying to look through 
it. Observing in this way, we're not compelled to give thoughts a lot 
of attention; we're not interested in them. If we are able to observe 
our meditation experience from this point of view, it's easy to 
remember that our efforts in meditating aren't for the purpose of 
thinking, but rather that they're aimed at connecting with the breath. 
By having this attitude of looking through our thoughts, it can help us 
feel more harmonious toward them. By contrast, when thoughts arise, if 
we respond with, \emph{Not again! I have to watch the breath}, then we 
have a sense of failing. This defeats our aim of cultivating ease and 
contentment.

In daily life as well, we often perceive thoughts as a problem we have 
to get rid of. And the same attitude can arise with the other 
experiences in our lives. We have certain emotions, moods, and 
interactions we want to get rid of because we perceive them as 
unhelpful, irritating, and annoying. We get into a battle between what 
we like and what we don't like. Instead, we can perceive experiences to 
be like fog. When we see them this way, there's nothing we have to 
suffer over. Ultimately, we will understand that there is no experience 
or feeling we have to hold onto or be afraid of in this world, in this 
life. If we have our hearts set on peace, truth, contentment, and 
virtue, then, with that as a refuge, we don't have to fear anything. 
When we commit ourselves to these principles and values, everything we 
experience throughout the day is easier to let go of. We can see 
through it all.

Everything becomes transparent in the light of this attitude, at least 
it does when we can \emph{access} that attitude. To do that, we have to 
put it into words we can repeat to ourselves, so that we can come back 
to this perception of seeing through the fog of thought and experience. 
This can be as simple as telling ourselves, \emph{What I am being 
distracted by is just thought or some other experience, nothing more, 
and it is okay for it to be here in my mind while I am following the 
breath.} This is the way the mind can settle and become clearer and 
more peaceful. We should try to see through everything in this way. 
When we do this with thoughts, they tend to subside. Thinking becomes 
less problematic; the breath becomes more evident. Clarity and peace 
arise.

