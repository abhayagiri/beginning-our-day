\mychapter{Abhayagiri is Complete}{Ajahn Amaro}{November 2012}

I'm experiencing a very worldly delight in non-involvement, 
non-responsibility. It's lovely to be here as a visitor, not having to 
feel responsible for making all the threads come together, as I did 
before leaving for Amamavati a couple years ago. But it's also 
important to note---especially during the busy festival season we're in 
right now---that the complications that are difficult are not the 
external ones, like the logistical nightmares the work monk has to deal 
with. It's the logistical nightmares that are \emph{inside} that are 
the real troublemakers. It's always good to bring that to mind. It's 
not the external complications that really make things difficult, it's 
the way we pick things up, create complications, and tangle things 
within ourselves. That's the real cause of tension, of \emph{dukkha}, 
of stress within.

Every monastery is the same during the festival season. It draws 
together a large complex array of different tasks that need to be taken 
care of with so many extra people lending a hand while also being part 
of the mix. As Luang Por Pasanno was saying yesterday, we need to lean 
into the wind in a very conscious way so as to counteract the tendency 
we have to create inner complication, because there's such a great 
potential during this time for getting caught up, being busy, 
resentful, or excited.

Yesterday when some of us were walking around with Tan Ṭhitābho in 
the morning, we saw so many new \emph{kuṭis} and the new workshop 
which were all were built over the last couple of years since I 
departed. These were things that Luang Por Pasanno, myself, and others 
would fantasize about. \emph{It'd be a nice place to have a kuṭi 
there, or we could do this here, or maybe we should put the workshop 
there}. Many of these things have suddenly become a reality. Of course, 
these physical changes took place slowly and steadily over time. So 
much has evolved since June 1st, 1996 when I, along with Anāgārika 
Tom, now Ajahn Karuṇadhammo, Debbie, and a gang of others rolled up 
to the newly purchased property. On that very first evening, once we'd 
cleaned up the house, set up the domed tents that we'd be living in, 
gotten ourselves sorted and settled, I remember thinking, \emph{Now the 
monastery is complete, now it is done}.

With those words, I was taking a leaf out of Luang Por Liem's book. I 
remember when preparations were underway for Luang Por Chah's funeral 
and a \emph{cetiya} for his cremation was under construction. A whole 
new eating hall, road system, water towers, and over 630 toilets were 
all under construction simultaneously. During that time, there was 
someone who was touring the monastery and saw all these different 
construction projects. He was quite amazed and bewildered. He then saw 
Luang Por Liem who was running the whole show and had just come down 
off of the roof of a \emph{sālā} with a welding torch in hand. The 
man said to Luang Por, ``This is incredible, this is amazing, there's 
so much happening here, I bet you'll be really glad when it's 
finished.'' And Luang Por Liem responded in his inimitable Luang Por 
Liem way, ``I finish it every day.'' That is a very simple observation, 
but coming from Luang Por Liem, it's not merely a nice thing to say. 
It's not just sophistry; it's the actuality. Yes, we have bare girders 
here and wet concrete over there and so many pits for the concrete 
rings to go in under all of these toilets and they are all sitting out 
there in heaps. But it's completely finished, just as it is. This is 
what it is, right now.

When there's a lot of activity going on---going from here to there, 
finding this and taking it over there, picking up these gas bottles and 
moving them over there, taking them to the wrong place and then taking 
them back---there can be a current of becoming, the flood, the 
\emph{oghā} of becoming, which can become very intense. So it is 
important during the flow of activity and doingness to be leaning into 
the wind, to be leaning against that.

When we do something as simple as fill up the gas tank, we can think, 
\emph{Okay, now Abhayagiri is complete. Everything is fully completed. 
Everything is done.} We reflect in that way, even though part of our 
worldly instinct might say, \emph{Yeah but, but, but, look at my list! 
I have so many things to do and they are important and they have my 
name on them and I can't just brush them away.} But with the 
reflection, \emph{Abhayagiri is complete}, we can keep that worldly 
perspective in its appropriate place and recognize that, within a 
larger context, it's just as Luang Por Liem was expressing: It's 
finished. Even when the gas tank is half filled, it's finished. As 
you're carrying along the carpets or untangling the flags, it's 
finished. Even though the knot is still there, it's finished.

That's because the Dhamma is here and now. The Dhamma is 
\emph{akāliko}, timeless, and it's \emph{sandiṭṭhiko}, apparent 
here and now. The Dhamma doesn't simply happen when the knot is 
untangled or when the carpet is laid out and all of the food is cooked. 
It's not, \emph{Okay, the Dhamma is here now, it wasn't here before}. 
The Dhamma is always here. During the morning reflection, it's here 
already, not only after the reflection is finished and we get on to the 
practice during the day. It's here now.

If we remember that---really let the mind awaken to that---then that 
presence to the Dhamma will inform our every action. We can then attune 
to the \emph{citta}, the heart---to that quality, that fundamental, 
timeless presence of Dhamma---in the midst of activity. Then any 
external complications won't contribute to any internal complications, 
to any internal \emph{papañca}.

So without further ado I offer these thoughts for your consideration 
today. Enjoy, as they say.

