\mychapter{Practicing With the Five Hindrances}{Ajahn 
Karuṇadhammo}{December 2012}

In our Dhamma practice, many of us work will with the five hindrances 
throughout the day---sensual desire, aversion, sloth and torpor, 
restlessness and worry, and doubt. It is helpful to pay attention to 
these hindrances and work with them because they are not only an 
obstruction when we sit in meditation, they're also concrete 
manifestations of the underlying tendencies: greed, hatred, and 
delusion. These are the tendencies that keep us moving through 
\emph{saṃsāra}, cyclical rebirth. By examining the hindrances, we 
can see how these underlying tendencies manifest themselves in our 
daily lives. They tend to be more obvious to us if we are accustomed to 
them and start to work with them in a constructive way.

As Dhamma practitioners, we sometimes may think that our sole purpose 
is to be mindful of a particular state of mind that is occurring for 
us. This paradigm is correct in some ways, but not complete. We may 
think that if we're aware of being with a mind state then that alone is 
sufficient to deal with it. In other words, if we know that it's there, 
then, presumably, we can watch its rising, its maintaining, and its 
cessation. This is a primary skill in our practice we need to develop: 
a straightforward, nonjudgmental awareness of a particular mind state.

However, there are also concrete and applicable antidotes we can use to 
work with the hindrances. In the Satipaṭṭhāna Sutta, the Buddha 
said that the five hindrances are to be known like all the other 
objects of mindfulness. In addition to knowing, we can reflect on how 
an unarisen hindrance arises, how we can deal with a hindrance that's 
already arisen, and how we can prevent its future arising. It's more 
than bare noting of a particular hindrance---we are learning more about 
it: how it comes to be, what encourages it, what nourishes it, and what 
denourishes it. We can understand how to work with it in a real, active 
sense when it's overwhelming and doesn't respond to bare attention.

It's a significant part of our practice to know how to recognize the 
hindrances. Although we probably have experienced most of them many 
times, we can have particular tendencies in regards to which ones we 
move to first. We might tend toward escaping from discomfort by using 
sensual gratification and indulgence. Or we may react to challenging 
situations with immediate irritation by using a verbal retort or with 
resistance by having an internal sense of heating up. Likewise, we may 
become confused and doubtful or shut everything out by annihilating 
ourselves with sleep or worry and become restless and anxious.

We take note of where our buttons get pushed and the responses that 
tend to be most habitual, and start working with these aspects 
throughout the day. We do this to help prevent the hindrances from 
arising. One of the ways this is done is by working with a hindrance 
before it has arisen in the mind. For example, with aversion or 
irritation, we learn how to develop its antidote: loving-kindness. By 
developing loving-kindness, we are increasing its strength and 
availability so we can access it any time we need to use it.

As we become more familiar with our tool bag of antidotes, we weaken 
and diminish the power the hindrances have over us. ``''We can 
therefore feel more confident and proficient in dealing with them so 
that we are not so easily blown off course.

