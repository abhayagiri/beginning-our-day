\mychapter{Work Is Giving}{Luang Por Pasanno}{July 2005}

Since today is a community work day, there is an opportunity to work 
together for the benefit of the monastery and to recognize how we look 
at opportunities for work. Sometimes we may feel that the word 
\emph{work} already has the connotation of drudgery, a burden in some 
way. But that's merely how we perceive it and therefore create a 
certain feeling around it. Instead, we can change our perceptions and 
see work as an opportunity for giving---giving that benefits others or, 
rather, giving that benefits somebody other than ourselves. These are 
opportunities to do something quite noble.

We can also look at work in terms of something that is enjoyable. In 
the English language, the word \emph{work} doesn't have the connotation 
of being enjoyable. However, in Thai, the same word that one uses for 
\emph{work} is also used to mean festival. If it's a festival, it's an 
opportunity to get together and have a good time. This creates a 
completely different perception than drudgery. It is interesting seeing 
how many connotations we bring to words. Certainly it's the perception 
we have of work and what we bring to it that gives us the opportunity 
to put our intention on generosity, sharing, participating, and 
helping. These intentions brighten the mind and help us enjoy ourselves 
at the same time.

