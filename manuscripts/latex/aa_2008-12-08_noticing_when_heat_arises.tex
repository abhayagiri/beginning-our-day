\mychapter{Noticing When Heat Arises}{Ajahn Amaro}{December 2008}

It often strikes me how the Fire Sermon is the shortest of the three 
cardinal discourses of the Buddha, taking less than fifteen minutes to 
chant. Yet during the course of the Buddha's teaching it, a thousand 
\emph{bhikkhus} became \emph{arahants}. In the Pāḷi Canon that's the 
largest number of people who were completely liberated during the 
course of listening to a single Dhamma talk. It's a powerful teaching, 
although it can seem unremarkable in certain ways.

In the sutta, the Buddha goes through a list explaining how all of the 
aspects of the senses are burning with greed, hatred, and delusion. The 
eye, eye consciousness, eye contact, the feeling that arises with eye 
contact, visual objects, everything to do with the process of 
vision---and the same with the processes of hearing, smelling, tasting, 
touching, and mental activities---they're all burning with passion, 
aversion, and delusion. He then says, ``\emph{Evaṃ passaṃ bhikkhave 
sutva ariyasāvako cakkusmim'pi nibbindati}---Seeing thus, the wise, 
noble disciple becomes dispassionate toward the eye, toward visual 
objects, toward eye consciousness, toward eye contact, and the feeling 
that arises with eye contact,'' and so on through the remaining five 
senses.

It's extraordinarily simple and direct. First, seeing that the senses 
are on fire; second, recognizing that they're on fire, agitated, 
heated; and then third, responding with coolness, \emph{nibbindati}, 
dispassion. \emph{Nibbidā} is related to \emph{Nibbāna}, 
``coolness''---there's a cooling in relationship to the senses. And 
with that cooling down, that dispassion, the hearts of those thousand 
bhikkhus were liberated. It's extraordinarily simple. In the sutta, it 
almost seems as if nothing has happened, as if very little instruction 
was given. But like many other Buddhist teachings, particularly in the 
Pāḷi tradition, if we blink, we miss it. The teachings can be quite 
subtle. They're not very demonstrative or highly elaborate. This 
particular teaching is like that. So we can easily miss the key piece: 
having the mindfulness to recognize---to recognize the quality of 
burning, to recognize that things are agitated and heated, to recognize 
the friction around what we see, smell, taste, hear, touch, and think.

During the course of the day, it's helpful to bring mindfulness to 
recognizing that the quality of friction, tension---what the Buddha 
calls heat, \emph{ādhitaṃ,} burning. The feeling can be one of 
ownership regarding some tool, the interest in a particular task, or 
the irritation with an exceptionally obstructive piece of rock that 
won't move out of the way. Whatever it might be---some computer program 
or printer that won't obey---whether aversion, delusion, or passion, it 
can be extremely subtle and thus hard to recognize. It can also be 
obvious, gross, and clearly visible---but even so, there's no guarantee 
we'll recognize it. Recognition requires mindfulness.

As we bring mindfulness to the course of our days, we can observe the 
sense world and how we respond to what we see, hear, smell, taste, 
touch, and think---the mental realms of moods, memories, ideas, and 
plans. We can notice when that heat arises. We notice the heat of 
\emph{rāga}, of passion, wanting, and desiring; the heat of 
\emph{dosa}, of aversion, being irritated, upset, obstructed; the heat 
of \emph{moha}, delusion, caught up in reactions, opinions, 
assumptions, projections, and other deluded states. These are all 
aspects of heat in the context of this teaching.

By bringing that quality of mindfulness---evaṃ passaṃ, seeing 
thus---we're able to recognize that feeling of ownership, for example. 
We can recognize the heat that the mind generates out of simple things. 
\emph{Oh look, I'm getting upset with this machine, I'm getting excited 
about this plan I have, I'm claiming this painting project as mine.} A 
task we didn't even know existed before it was assigned to us, suddenly 
becomes ``mine.'' \emph{I hope no one sees that awful wood cut I made. 
``Look, that cut's not square and we all know who did it!'' That would 
be so embarrassing.} By owning something, it can become our great 
achievement or our terrible crime---the heat of pride or the heat of 
shame. So again, ``Evaṃ passaṃ bhikkhave--- the wise, noble 
disciple, seeing thus, becomes dispassionate.''

We can see how the mind creates these stupid, absurd reactions and 
projections about the world, and at the same time, what an amazing, 
wonderful capacity we have to cool down, let go, and not create heat 
around these things. The wise, noble disciple becomes dispassionate 
toward the eye, ear, nose, tongue, body, and mind. Even though that's a 
subtle teaching in some ways, it's also incredibly essential and 
helpful. When we take the opportunity to cool down, to let the fire go 
out, then, lo and behold, life becomes much easier, more pleasant, and 
open. There's less heat, friction, and abrasion in our world.

