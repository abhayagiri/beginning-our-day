\mychaptertoc{The Importance of Informal Meditation}
{The Importance of\\Informal Meditation}
{Ajahn Amaro}{August 2008}

Our teacher Ajahn Chah strongly discouraged us from holding a 
perception of meditation practice as different and apart from the 
activities of our ordinary everyday lives. There can be a tendency to 
think that meditation is what happens when we're walking up and down 
the meditation path or in the Dhamma hall when we have our legs crossed 
and our eyes closed---that the rest is merely ``the other stuff'' we 
need to take care of so we can do ``the real thing'' of meditation. But 
Ajahn Chah was very keen to point out that we're capable of suffering 
in all modes of our lives, whether we're sitting on a cushion with our 
eyes closed, sitting in a classroom, working in the kitchen, driving 
down the freeway, or chatting with our friends. We can make 
difficulties for ourselves and experience happiness and unhappiness in 
every aspect of our lives. So it's important to bring all those 
different aspects of our lives into the scope of meditation.

Rather than thinking that meditation equals the activities in the hall 
or the walking path, it's more skillful to think in terms of ``formal'' 
and ``informal'' practice. When we sit with our eyes closed and legs 
crossed, that's formal practice. When working around the monastery and 
taking time with each other, this is the same practice, but in an 
informal mode. The informal practice is sometimes more challenging than 
that of trying to be mindful when everything is still, quiet, and 
highly controlled while we sit on our meditation cushions. So while we 
go about the various work tasks this morning, we can bring attention 
and mindfulness to them.

We can notice the different moods we're feeling. Are they exciting? 
\emph{This is great. This is so wonderful!} Or are we thinking, 
\emph{Oh no, another thing for me to fail at?} Maybe there's a feeling 
of pride. \emph{My lines are straighter than anybody else's. Just look 
at these lines I painted!} The mind that creates problems can be very 
active. But with practice we're able to see through whatever state of 
mind happens to be in play.

We can watch the flow of moods, feelings, and thoughts as they go 
through the mind. We're not suppressing them or feeling that we 
shouldn't be experiencing these kinds of thoughts or attitudes, but 
neither are we buying into or being caught up in them. Instead, we're 
simply seeing that this is how the mind works. When things go well, 
we're happy. When things go badly, we're unhappy. That's just the way 
it is. We are recognizing that this is simply a flow of feeling and 
perception we're engaged in, noticing for example the different changes 
of the day. When we start the work period, it's nice, cool and, bright. 
\emph{Ah, this is so lovely. What a beautiful place.} Then, when the 
temperature starts to pick up later in the morning, we might think, 
\emph{Oh, it's really hot. It's only 10 o'clock, and I'm dying on my 
feet here.} We notice how our attitudes changed with the changing 
conditions. \emph{Oh, how interesting. When it was cool, I felt happy. 
When it was hot, I felt oppressed. One is a feeling of comfort, the 
other is a feeling of discomfort.} We can see these as nothing more 
than different impressions and moods that come and go through the mind. 
We don't need to make a problem out of them.

By doing this, We're not negating anything, not thinking we shouldn't 
be feeling uncomfortable. We're merely seeing, \emph{When it's like 
this I'm happy; when it's like that then I'm unhappy.} We see how 
fickle the mind can be, how easily we can get caught up in perceptions 
and moods. And we simply bring mindfulness to that. The mind that's 
fully attentive to the present moment sees how easily we get caught up, 
and in that seeing there's a coolness of heart. There's an easefulness 
in the attitude. We find ourselves more easily able to go with the ups 
and downs of praise and criticism, success and failure, and with all 
the mundane, ordinary experiences that flow through the day.

Of course, applying this kind of attention and attitude isn't only for 
the morning work period, but for the entire course of the day: when the 
work period finishes, when we gather for the mealtime, after the meal 
is finished, when the dishes are done, and throughout our open time in 
the afternoon. We try to keep bringing our attention and the reflective 
attitude of mind to the flow of mood and perception. This is what's 
meant by ``mindfulness of every day life.'' It's as much a part of the 
meditation as is the formal, quiet sitting---perhaps even more so. The 
habit is to easily get caught up in the different conversations we're 
having or in the work we're engaged in so that we don't notice what's 
going on inside us. We're so busy and interested in active and external 
things. If we first apply a bit of attention to how we're feeling, and 
attend a bit more closely to our attitude, this will help us find and 
retain a quality of balance and easefulness. We have the potential to 
be interested in what we are attending to, but it takes some effort to 
draw the attention inward.

So these are some of the ways we learn to change the lens of our 
perceptions with regard to formal and informal meditation. Eventually, 
we may come to understand that meditation is an all-day and sometimes 
all-night activity. The more we are able to bring mindfulness and 
attention into every aspect of our lives, the more opportunities there 
are for wisdom and understanding to arise.

