\mychaptertoc{The Dedication of Merit and Blessings}
{The Dedication of Merit\\and Blessings}
{Luang Por 
Pasanno}{May 2013}

The dedication of merit and blessings is a practice that is very common 
and ordinary in Buddhist cultures like Thailand. This practice helps 
counteract our tendency to focus on the problems, the flaws and the 
obstacles we believe we have to overcome, whether real or perceived. We 
can set all that aside, and instead, bring our attention to the 
conditions in our lives that are blessings---those conditions that 
align themselves with what is meritorious and good.

It's not so difficult to do. If we think of the surroundings we're 
in---they're about as idyllic as we could hope for. We're not oppressed 
by war, famine, or pestilence. It's an incredibly fortunate time and 
place we live in. These are blessings. And while we are under 
government regulation with our building codes and other mundane 
details, the government doesn't prohibit monasteries like ours from 
existing. So we have this opportunity to live here, in an American 
Buddhist monastery, and to practice the Dhamma.

In addition, there are so many people who freely offer their support to 
us. We have more than adequate food, dwelling places, clothing and 
medicine. We're supported every day by people's generosity, and we live 
in dependence on them. It's essential for us to recollect that 
truth---that blessing---and by frequently practicing the dedication of 
blessings to others, we can keep that recollection fresh in our minds.

As monastics, we are not only on the receiving end. It's also our role 
to give and to share. In carrying out this role, we have the 
opportunity to reflect on what we are offering, and to reflect on our 
attitudes with regard to giving. This is one aspect of making sure we 
are worthy recipients. And it underscores the fact that our 
relationship with those who support us is one of mutual generosity, 
which is, in itself, a great blessing.

Now just consider the culture of virtue that's been established here. 
We've all committed ourselves to living in a virtuous way, with 
integrity, according to the moral precepts. That too is a blessing, to 
live in a situation where there is this quality of integrity and trust. 
It supports, uplifts and encourages us.

So these are some of the many blessings that we're virtually swimming 
in here. By bringing them to mind, we are able to share and dedicate 
them. They can become a field of blessing and merit, not just for 
ourselves, but for others as well. And so, attending to our blessings, 
bringing them to mind, reflecting on this field of merit---this is 
important to do as we go through our day.

The foundation for this practice is selflessness. That's what really 
brings about the sense of merit and blessing---we are willing to set 
aside our personal agendas and preferences, our views and opinions and 
everything that comes out of our obsession with ``me'' and ``mine.'' 
Certainly, these things come to the surface in all of us, and I'm not 
suggesting we deny or try to annihilate them. Rather, this is about 
redirecting our attention away from self-oriented concerns, so that we 
can step into this field of blessing and merit.

There is a verse from the Dhammapada where the Buddha instructs us to 
do things that are aligned with what is good and meritorious. He 
encourages us to do them over and over again because, as we develop an 
affinity for acting in that way, it will lead to our ease and 
well-being. This certainly applies to the practice of dedicating merit 
and blessings. It is something that can hold us in a place of ease; and 
if we are grounded in that place as we apply effort in our practice, 
our efforts will be held within that same quality of easefulness.

There is a lightness to experience when people live with a sense of 
generosity and selflessness. We can share in that, and dedicate the 
goodness of merit and blessings that come from it.

