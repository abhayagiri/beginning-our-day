\mychapter{Deconstructing Personas}{Ajahn Karuṇadhammo}{August 2013}

Recently I've been reflecting on the ways we project ourselves to the 
world, and the relief that comes when we set these projections down. 
Many of us feel the need to constantly create a persona based on either 
how we think we should be, or how we want others to view us. If we can 
get an idea of how that process occurs, then we have the potential to 
catch it in its tracks. By noticing this fabrication before it becomes 
too involved or too set in, we're able to use reflection to let it 
deconstruct. The less we try to create a sense of who we are, or who we 
want people to see us as, then the more natural and at ease we can be.

Earlier I suggested seeing if walking meditation could be used as a 
reflection---walking back and forth with the thought in mind:* I don't 
have to be anybody right now.* By doing this for a short period of 
time, we are planting the seed that may help us stop creating an 
identity to present to the world. Instead of forming that identity, we 
simply see, if even for a finger snap, what it's like to no longer 
create that image in our minds.

When there is a period where we don't need to think too hard---moving 
from one place to another, in a moment of transition or while getting 
up to go to the bathroom during the work period---whatever mundane 
activity it may be, we can notice this. We can ask ourselves, \emph{Can 
I drop this created image of being somebody who's operating in the 
world? Could I at least notice whether or not I'm holding this 
perception of a person who behaves in a certain way---with a role, a 
purpose, an image or projection of a solid independent individual?} 
Then try to let go of that created self-image for just a moment, see 
what it might feel like. Feel the sense of relief and ease that arises 
when it's simply a few body parts and a few stray thoughts moving 
through space and time.

