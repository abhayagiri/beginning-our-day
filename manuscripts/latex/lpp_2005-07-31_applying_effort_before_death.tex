\mychapter{Applying Effort Before Death}{Luang Por Pasanno}{July 2005}

One of the American monks in Thailand, Tan Paññāvuḍḍho, has just died. 
He had been spending the Rains Retreat at Ajahn Dun's monastery. 
Yesterday morning, he didn't show up for alms round. A monk was sent to 
check on him, and he was found lifeless in the bathroom. It looked like 
he fainted while standing up, fell, and hit his head on the way down.

We remind ourselves of these concepts: \emph{I am of the nature to age, 
I have not gone beyond aging; I am of the nature to sicken, I have not 
gone beyond sickness; I am of the nature to die, I have not gone beyond 
dying.} We may think these concepts apply to sometime in the 
future---somewhere else, not now---but that is not the case. We carry 
old age, sickness, and death with us all the time. Something that is 
very mundane and we do all the time, such as getting up at night to go 
to the toilet---well, tonight may be the last time for us to do that. 
It's always important to recollect and remember that, making sure we 
are using our time skillfully.Tan Pannyavudo was a very diligent and 
sincere monk. He had a career track waiting for him in the world but 
chose not to take it. Instead, he became a monk, used his time 
skillfully, and put effort into his spiritual life.

However long we have to live, it is so important to put effort into 
that which is skillful, directing our effort toward the deathless. This 
can have an extraordinary benefit, not only for us, but for everyone 
else. The effort we put into disentangling ourselves from 
\emph{saṃsāra}---cyclical rebirth---is of immense benefit. 
Saṃsāra keeps weaving its entangling web because of the 
inappropriate effort we make buying into it. However, by looking at the 
realities of the human condition, we can apply appropriate effort to 
free ourselves from saṃsāra, and do that which is of the most 
benefit for our lives.

