\mychapter{Using the Communal Life Skillfully}{Ajahn Yatiko}{August 
2013}

At one point while we were meditating together this morning, I became 
vividly aware of the silence. Suddenly, it became extremely still, and 
I had a very strong sense of collective practice, a sense of shared 
activity with fellow practitioners. It was a very beautiful moment. I 
think religious groups in general provide a sense of community or 
family, because people in those communities spend so much time with 
each other, in mutual care and support. That's very meaningful and 
valuable.

At the same time, living in community can mask the truth of our own 
aloneness. When it comes time to die, we die alone. The people around 
us when we die, with all their good wishes and best intentions, still 
continue to live their own lives and deal with their daily tasks. I 
imagine that in many cases this fact is not fully apparent to those who 
are dying. They might be lying there wondering, \emph{Doesn't anybody 
realize this is happening to me, that I'm leaving soon?} They're not 
fully in touch with the natural fact that life goes on, whether it 
includes them or not. We are born alone; we die alone. Moreover, even 
while we are alive and have many close relationships, it is inevitable 
that those relationships will end eventually, one way or another. 
Perhaps a long-term friend disrobes, or our teacher disrobes or dies, 
or a partner or family member leaves us or dies. This experience is 
very painful for many people, but we are subject to having that 
experience at any time. That's a truth that living in community can 
mask.

Despite this truth, living in community can serve as a useful crutch. 
Our existential situation can be compared to a person who breaks a leg; 
while the leg is healing, there's a need for physical support, like a 
pair of crutches. In this respect, religious institutions and 
relationships are like crutches. This is not a value judgment, it's 
simply a fact and a recognition that crutches are important for people 
in need of support. In a community, we live in the presence of other 
people. We have the responsibility to care for them. We provide them 
with crutches, and they do the same for us when the need arises. And 
when we help others, it lessens our self-centeredness and our unhealthy 
sense of self-importance. That's good for us and good for others. 
Everyone wins.

With the loss of friends, we can be reminded of our own existential 
aloneness. At the same time, we can remember the support we receive in 
community. We can remember that, when people leave us, the community 
will encourage us to take our loss in the spirit of letting go, rather 
than encouraging us to tightly grasp our crutches, hoping we'll never 
be without them. Crutches are not supposed to be around forever. Their 
purpose is to assist and lend strength until they are no longer needed. 
So we care for ourselves, care for each other, and care for the 
community in which we live, while recalling and respecting the fact of 
our aloneness.

