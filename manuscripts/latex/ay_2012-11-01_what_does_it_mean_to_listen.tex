\mychapter{What Does It Mean to Listen?}{Ajahn Yatiko}{November 2012}

What does it mean to listen? How many difficulties arise simply through 
poor listening skills and not allowing others the space to say what 
they want to say? It's not easy to find people who know how to simply 
listen. A good listener doesn't focus on the content of what is being 
said. Whether it's good or bad is not the point. The point is to simply 
listen, to let people be what they are. This holds true for listening 
to oneself, as well.

There's the mind state of listening and there's the mind state of 
judging, and they are completely different experiences. There is an 
open, spacious, attentive, and awake quality inherent in a mind that is 
truly listening. Suppose everybody in this room right now had what we 
might call a listening mind state, a state in which the mind is 
sensitive and open. Given this shared quality, if everyone were to move 
around in the room, coming in and out of everybody else's personal 
space, there would probably be a sense of harmony. Compare that to a 
room filled with twenty or thirty people holding tightly to some 
judgment or view. It's likely that if they were all to move about the 
room, it would be clank-and-clunk, everybody stumbling into each 
other's space in some sort of disharmonious way. So listening, and its 
quality of attentive spaciousness, is a beautiful skill to develop.

Truly listening allows things to be the way they are. It allows us to 
be what we are. I'm not saying we should ignore those aspects of our 
lives and practice that clearly need work and attending to. But I think 
most of us are already pretty good at thinking of the hundred-and-one 
ways we could be better. What we often neglect, however, is the wish to 
be heard, which everyone has. Responding to that wish with compassion 
requires that we develop the skill of receptive, non-judgmental 
listening---listening to others and listening to ourselves.

