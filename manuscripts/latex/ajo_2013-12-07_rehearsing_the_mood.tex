\mychapter{Rehearsing the Mood}{Ajahn Jotipālo}{December 2013}

In communicating with other people there is always room for mistakes to 
be made and issues to come up around misunderstandings. When these 
problems occur, we can sometimes find it helpful to seek advice from 
someone else to help clarify what we want to say to the other person 
the next time we speak with him or her. We do this so we are sure we 
will be speaking factually, and at an appropriate time.

I was listening to a talk by Ajahn Soṇa and he gave a beautiful piece 
of advice in regards to this type of situation. Once we are clear about 
wishing to speak with someone in regards to a previous interaction, 
Ajahn Sona suggested, ``Don't rehearse the story or what we want to 
say; instead, rehearse the mood, rehearse the state of mind we want to 
be in when we speak.''

Do we want to be experiencing anger when we speak to the other person? 
If we really want to alienate ourselves and encourage that person to 
get upset, then we can rehearse anger. But if we want to help the other 
person or decrease tensions, then we can rehearse a compassionate mood 
or one based on loving-kindness. We can encourage that in ourselves.

And this isn't only useful when we wish to speak with someone about an 
issue. We can do this all the time. We can contemplate and encourage 
the moods we wish to establish in our minds. While we are working with 
someone, for example, we can ask ourselves, \emph{What's the mood I 
want to be in right now?} If we can deliberately generate a wholesome 
mood within ourselves, we allow our minds to more easily open to the 
Dhamma.

