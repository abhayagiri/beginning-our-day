\mychapter{Putting Our Moods in Their Place}{Luang Por Pasanno}{July 
2013}

I've been thinking about Ajahn Ñanadhammo, recalling his most 
memorable interaction with Ajahn Chah while living at Wat Pah Pong.

One day while out on alms round, Ajahn Ñanadhammo had a slight 
argument with another monk and became stirred up and upset. When he 
returned to the monastery, Ajahn Chah smiled at him and said very 
sweetly in English, ``Good morning.'' Of course Ajahn Ñanadhammo was 
really tickled with that, and this feeling of happiness arose that was 
uplifting for him.

In the late afternoon, he went over to Ajahn Chah's dwelling place. 
Oftentimes monks would go over there in the afternoon when laypeople 
came by to pay respects and ask questions; it was a good opportunity to 
hear Ajahn Chah give advice. As evening approached, Ajahn Chah sent 
everybody off to \emph{pūjā}, except for Ajahn Ñanadhammo, who alone 
was asked to stay. He sat beneath Luang Por Chah, massaging his feet. 
In the distance he could hear the chanting of evening pūjā, with the 
stillness of the forest almost palpable. Ajahn Ñanadhammo described it 
as an ethereal, heavenly experience, and he felt quite blissful.

Suddenly, Ajahn Chah pulled his foot away and kicked Ajahn Ñanadhammo 
in the chest, sending him flat on his back. This was quite a shock, of 
course. Then Ajahn Chah said, ``You're not really paying attention to 
the practice or the training. You have an argument in the morning and 
get upset, carrying around a mood of ill will. Then all it takes is one 
person to say good morning to you, and you go off into a happy mood, 
spending the day proliferating about that. You come over here and even 
more happens that pleases you, so you get into an even happier mood. 
Next I put you flat on your back, and you're confused. That's not the 
mind of a practitioner, that's not the mind of somebody who is training 
in Dhamma. You have to be able to stop yourself from following your 
moods. You do your best not to be caught by them, believe in them, or 
buy into them. That's what defines a practitioner.''

For all of us, moods are woven into the fabric of our lives. We go up 
and down, get inspired and depressed, energetic or enervated by our 
moods. None of that is the essence of practice. To really practice is 
to see through these conditions, to see them clearly, as they really 
are: \emph{This is just a mood. This happens to be something I like. 
This happens to be something I dislike.} It doesn't mean we don't have 
feelings, but we resist picking them up and running with them all the 
time.

As we go about our day, we need to look at the nature of our 
impressions---to look at the contact, the feelings, and the way moods 
want us to go---and stop ourselves from giving in to them. We have to 
be willing to look closely at moods and challenge them. With peaceful 
moods, for instance, if we get what we want and things go the way we 
prefer, that's not very stable, that's no refuge at all. The refuge is 
in \emph{sati-paññā}---mindfulness and wisdom. We use sati-paññā 
to cultivate truth-discerning awareness at all times. We know our 
moods, we know their arising and their fading away. Whether they are 
pleasant or unpleasant, we don't get caught by them nor do we follow 
them, we learn to relinquish this tendency of mind.

