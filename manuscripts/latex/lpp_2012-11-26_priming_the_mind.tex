\mychapter{Priming the Mind}{Luang Por Pasanno}{November 2012}

As we get into the cool season, there are things around the monastery 
that need to be taken care of and chores that need to be finished 
before the winter retreat begins. It's a good time to be paying 
attention to all that, as well as trying to keep things fairly quiet.

Last week, during the Thanksgiving Retreat, I didn't really have a plan 
in mind for what I was going to teach. I ended up talking a lot about 
\emph{anicca}, \emph{dukkha} and \emph{anattā}, and how these three 
characteristics display themselves in our practice and how central they 
are to our practice. We constantly use reflections of impermanence, 
unsatisfactoriness, and not-self when investigating our experience. 
This is not a contemplation we only do on retreat or during a time of 
formal meditation; it's an investigative perspective that we bring to 
all of our experience, both in formal practice, as well as in the 
day-to-day application of our mindfulness and contemplative living.

It is quite essential in the practice to bring that investigation into 
the mind, so that the mind is seeded or primed. Last night at teatime 
we were talking about a video used to demonstrate a psychological 
phenomenon: It shows a circle of people people passing two basketballs 
around. Some people in the group are wearing black and the others are 
wearing white. One ball is being passed between the players wearing 
black and the other ball is being passed between the players wearing 
white. The people viewing the video are told to keep track of the 
number of times the ball gets passed between the players wearing white. 
As a result, viewers focus on the players in white shirts and the ball 
to the exclusion of most everything else. So when somebody in the video 
dressed in a gorilla suit walks through the group of people throwing 
the basketballs, beats his chest, and walks off, most people don't even 
notice!

That's because the mind was primed to focus in a particular way---a way 
that produced an overall state of heedlessness. But if we prime the 
mind to view experience in terms of anicca, dukkha, and anattā, in 
terms of these universal characteristics, then we start to see those 
characteristics more consistently and clearly. However, we tend to 
overlook those fundamental truths of existence, thereby missing the big 
picture. We get caught up in our personal stories, worries, fears, 
likes, and dislikes. So we need to prime the mind for viewing our 
experience through the lens of Dhamma, because otherwise we overlook 
it. That's very much a part of our training in mindfulness, reflection, 
and investigation: keeping a particular agenda in mind.

Of course, the agenda here is to see how the Buddha's fundamental 
truths actually apply to our experience, how those truths affect us. 
The beneficial effects of seeing these truths are clarity, 
relinquishment, and loosening the grip of clinging and delusion. By 
priming the mind to see though the lens of Dhamma we are supporting 
wholesome inclinations toward spiritual truth and spiritual peace, 
rather than unwholesome inclinations that spur the mind into keeping us 
clouded in delusion.

