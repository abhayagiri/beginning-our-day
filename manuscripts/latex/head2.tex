%!TEX TS-program = xetex
%!TEX encoding = UTF-8 Unicode
\documentclass[11pt,openany]{memoir}

\usepackage{fontspec}
\defaultfontfeatures{Mapping=tex-text}
\setromanfont{Gentium Basic}

\usepackage{geometry}
\geometry{%
    paperwidth=5.25in,
    paperheight=8in,
    top=54pt,
    bottom=68pt,
    inner=54pt,
    outer=45pt,
    footskip=30pt
}

\usepackage[gray]{xcolor}
\definecolor{authorgray}{gray}{0.2}

\makechapterstyle{mainmatter}{%
    \chapterstyle{default}
    \setlength{\beforechapskip}{9pt}
    \setlength{\midchapskip}{0pt}
    \setlength{\afterchapskip}{10pt}
    \renewcommand*{\printchaptername}{}
    \renewcommand*{\printchapternum}{}
    \renewcommand*{\printchaptertitle}[1]{\huge\itshape\centering ##1}
}
\chapterstyle{mainmatter}

\newcommand{\mychapter}[3]{%
\chapter*{#1\\[5pt]{\normalfont\large#2 • #3}}
\addcontentsline{toc}{chapter}{#1\protect\chapternumberline{}}
\addtocontents{toc}{\noindent\hspace{18pt}{\color{authorgray}\itshape\footnotesize #2}\par}
}
\newcommand{\mychaptertoc}[4]{%
\chapter*{#2\\[5pt]{\normalfont\large#3 • #4}}
\addcontentsline{toc}{chapter}{#1\protect\chapternumberline{}}
\addtocontents{toc}{\noindent\hspace{18pt}{\color{authorgray}\itshape\footnotesize #3}\par}
}

%\addtocontents{toc}{\vspace{1.2em}}
%\addtocontents{toc}{\vspace{14pt}}

\renewcommand{\cftbeforechapterskip}{.4em}
\renewcommand{\cftchapterfont}{\normalfont}
\renewcommand{\cftchapterformatpnum}[1]{{\normalfont #1}}
\makepagestyle{contents}{}{}{}

\sloppy
\raggedbottom
\pagestyle{plain}
\parindent 18pt

\usepackage[defaultlines=2,all]{nowidow}

\renewcommand{\normalsize}{\fontsize{11pt}{15.0pt}\selectfont}

\begin{document}
%\renewcommand{\baselinestretch}{1.1}

\frontmatter
\addtocontents{toc}{\protect\thispagestyle{plain}}
\addtocontents{toc}{\mbox{}\par}
\pagestyle{empty}

{\center
\vspace*{80pt}
\huge
Beginning Our Day

\par}
\clearpage

\thispagestyle{empty}
{\footnotesize\raggedright
\vspace*{\stretch{1}}

Beginning Our Day

\vspace{1em}
Abhayagiri Buddhist Monastery\\
16201 Tomki Road\\
Redwood Valley, CA 95470\\
www.abhayagiri.org\\
707-485-1630

\vspace{1em}
\copyright{} 2014 Abhayagiri Buddhist Monastery

\vspace{1em}
This work is licensed under the Creative Commons
Attribution-NonCommercial-NoDerivatives 4.0 International License.

To view a copy of this license, visit
http://creativecommons.org/licenses/by-nc-nd/4.0/

\vspace{1em}

\textit{sabbadānaṃ dhammadānaṃ jināti.}

The gift of dhamma excells all gifts.
}

\clearpage


\pagestyle{plain}
\tableofcontents*

\setlength{\afterchapskip}{15pt}

%\thispagestyle{plain}
\clearpage

\thispagestyle{empty}
\mbox{} \clearpage

\chapter{Acknowledgements}

This book would not have been possible without the contributions of many
people.

Pamela Kirby initiated the project when she asked to place a recorder in
front of one of the senior monks during a morning reflection.

Matthew Grad, Jeff Miller, Ila Lewis, Pamela Kirby, Ray Peterson, and
Laurent Palmatier were the main substantive editors of the material,
enduring the long and difficult process of editing the transcripts into
compact and well-written teachings.

Ruby Grad did most of the copy editing. Sumi Shin designed the cover.
And David Burrowes Dee Cope, Josh Himmelfarb, Evan Hirsch, and Jeanie
Daskais had a hand in editing, proofreading, or transcribing.

Over several years, Khemako Bhikkhu recorded the senior monks
reflections on a daily basis. Kovilo Bhikkhu and Pesalo Bhikkhu provided
corrections and suggestions on an early draft of the book. Suhajjo
Bhikkhu generously dedicated a significant amount of time on the overall
book design and typesetting of the text.

There are many others who have helped directly and indirectly to make
this book a reality. The extended community of Abhayagiri, both lay and
monastic, provided the essential support and resources for this project
from beginning to end.\\
\mbox{}\\
Cunda Bhikkhu\\
October 26th, 2014\nowidow[5]

\chapter{Preface}

The title of this book, ``Beginning Our Day,'' comes from the statement
made by Luang Por Pasanno each day he shares his morning reflection.

Five days a week, at Abhayagiri's morning meeting, work tasks are
assigns to the monastic residents and lay guests living in the
monastery. Shortly there after, one of the senior monks gives a brief
Dhamma reflection so that the residents and guests may have something
tangible to recollect throughout the work period and for the rest of the
day.

The talks are given spontaneously and often address an event that is
about to occur, a condition that is already present in the monastery, or
a general teaching on Dhamma. The most common thread throughout the
reflections is that of practicality: distilling the most important
teachings of the Buddha into pertinent and applicable practices. Though
many deep teachings are touched upon, from right view and not-self to
death and dying, the fundamental aim is to encourage the abandonment of
the unwholesome, the cultivation of the wholesome, and the purification
of the mind.
The Dhamma teachings given at Abhayagiri's morning meetings have been
recorded for the purpose of one day creating a book of short
reflections. Although the teachings gathered here can be read together
at one time, the intention of this book is to provide a single daily
reflection for readers who would like to have something they can easily
recollect, contemplate, and make use of throughout their day.

May these teachings bring insight into the nature of Dhamma and provide
a pathway toward the development of true peace and contentment.

\mainmatter


\mainmatter
