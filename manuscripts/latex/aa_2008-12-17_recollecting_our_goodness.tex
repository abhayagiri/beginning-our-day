\mychapter{Recollecting Our Goodness}{Ajahn Amaro}{December 2008}

When we're engaged in a lot of activity, we can become so focused on 
the details of what we're doing that we forget there's an element of 
generosity and goodness in our actions. When Ajahn Sumedho was a young 
monk, Ajahn Chah recommended that he recollect his good qualities. He 
couldn't understand what Ajahn Chah was talking about because his mind 
was quite busy and filled up with negative thoughts and emotional 
habits. He thought he was basically a selfish, nasty, horrible person 
and an embarrassment to the robes.

Luang Por told him, ``If you're such a bad person and really evil you 
wouldn't want to live with Buddhist monks. We would be the last people 
you would want to spend time with. If you're so off the mark and if 
your mind is so given to unwholesomeness, you wouldn't want to be 
around people who are honest, refrain from stealing, refrain from 
consuming intoxicants, people who don't behave in unruly ways. The last 
thing you'd want to do is be around virtuous people.'' When Luang Por 
Chah said this to him, he was quite startled by it. He thought to 
himself, You know that's true. I can easily forget that I'm living in a 
Buddhist monastery with Buddhist monks. There must be some reason why 
I'm doing this.

We can become quite focused on our faults, wrongdoings, and the things 
we said that were less than generous, friendly, helpful, or patient. 
Even after someone's asked us for advice, what we think about later is 
how much better it would have been to have said this or that. Oh gee, I 
really didn't get that right. The attention goes to all our 
shortcomings, failures, and weaknesses, and we obsess on those 
qualities.

That's how it was for Ajahn Sumedho. So when Luang Por Chah suggested, 
``Why don't you recollect all of your good qualities?'' there was no 
pigeonhole for him to put them in. At that time any good quality he 
had, he saw as inflated, egotistical, proud, or conceited. But Luang 
Por Chah said, ``No. This is cāgānussati, recollecting your goodness, 
recollecting your own generosity. It's a completely normal concept.'' 
Now, since we've had the good fortune to hear Laung Por Chah's advice 
to Ajahn Sumedho, it would behoove us to put that advice into practice 
for ourselves.

One way for us to do that is to recollect all the good effort we're 
making in looking after the kitchen, providing food, looking after the 
construction projects, the kuṭis, the buildings, or the micro-hydro 
project we're setting up. These are acts of generosity and kindness, of 
putting forth effort and putting forth our time to help other people. 
Many people have helped with the construction of the kuṭi down by the 
Bhikkhu Commons. Of those people, how many are actually going to live 
in it? Probably a very small proportion. The effort that we make is not 
simply in the single task we do, but also in the wholesomeness and 
goodness it supports---the kusala kamma it supports. By providing this 
dwelling, this food, these Dhamma talks, we bring enormous blessings 
into our lives.

Take Tan Thitapañño, for instance. He's wrestling with the 
intricacies of the Expression Engine software and the obstructive 
passwords that won't let him use the program. When doing such hard 
work, it can be easy to forget about the fact that there are people all 
over the planet who delight in what this work accomplishes, \emph{Oh 
look, a new Dhamma talk on the Abhayagiri website. How marvelous! This 
is fantastic!} This isn't simply about trying to cheer ourselves up or 
look on the bright side. This really is the bright side. Earlier in the 
year a fellow came to Abhayagiri from Liverpool, England. He was so 
happy to be here. He didn't even stay for a full day. He happened to be 
in the country on holiday and took a chance to visit us. He said, ``We 
have this little meditation group in Liverpool and we listen to 
Abhayagiri talks all the time. It's so great to be here.'' He was 
bubbling with happiness. We can forget that our lives are connected 
with little groups of people like that all over the planet. In our 
small efforts to keep the bodies fed, to keep the shelters workable, to 
provide Dhamma talks on the website, to offer publications, to pay the 
bills---every little piece is bringing goodness into the world. That's 
kusala kamma, wholesome action.

It's not indulgence, egotism, or pride to be reflecting on that 
goodness. The Buddha himself encourages cāgānussati, recollecting our 
generosity, because that brightens and brings joy to the mind. We can 
take some time to recollect all the efforts that we're making on the 
practical front as well as on the bhāvanā, the meditation. The word 
anumodanā is rejoicing in the goodness that has been done. The cynical 
mind says, Yeah, well, that's one thing. But I've really got some 
serious problems. Basically I'm just a defiled mess. I really am! We 
need to listen to that voice with compassion, but at the same time, we 
don't want to let it run our lives. So we listen, and then gently park 
it to one side. We can reflect, Even if I'm filled with utterly ghastly 
defilements, still there are things I've done that have helped people. 
That's an undeniable fact. There's some goodness or brightness that 
I've brought into the lives of others. My effort to be a little bit 
more patient has benefited other beings. How wonderful! By letting that 
brightness inform our lives, we can be encouraged and gladdened by it.

