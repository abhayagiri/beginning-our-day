\mychapter{An Internal Space of Mindfulness}{Luang Por Pasanno}{July 
2012}

Currently, there are many visiting monastics and laypeople here, and 
the numbers are at the high-end of what we are used to. In some areas 
of the monastery there is limited space and close quarters. So in the 
monk's room, the kitchen, and various other places where we are doing 
activities together we need to attend to the ways we interact with each 
other. It is easy to slip into frivolous, sarcastic, or negative speech 
that doesn't help us or anyone around us. So when in proximity with 
people, try to maintain a mindful and respectful space. If we're not 
attentive to that, and barge into crowded places being pushy or 
anxious, it tends to have an agitating influence and is not 
particularly pleasant for anyone. So let's try to return to an inner 
space of mindfulness.

How do we maintain mindfulness? All of us living at the monastery are 
here because we are interested and inspired by the concepts of 
mindfulness and wisdom. But sometimes when we find ourselves situations 
where we are more socially active and engaged than usual, mindfulness 
and wisdom fall by the wayside, and the result is not beautiful. So 
instead, we can pay special attention to maintaining mindfulness.

It's not that difficult to be mindful when we are on our own in our 
\emph{kuṭis}. But when we're engaging with others, when there are 
duties that need to be done, or when something needs doing right away, 
it takes effort to stop ourselves from being swept up in agitation and 
effort to maintain a space of mindfulness and letting go. What we are 
letting go of is suffering. When we use mindfulness and relinquishment, 
there's less suffering. With less mindfulness and less relinquishment, 
there is more suffering. It's a very simple equation, but one we can 
easily forget.

