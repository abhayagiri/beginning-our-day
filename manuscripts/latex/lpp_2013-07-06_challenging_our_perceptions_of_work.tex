\mychapter{Challenging Our Perceptions of Work}{Luang Por Pasanno}{July 
2013}

It's helpful to consider and reflect on how we perceive work. If we 
bring up the word ``work,'' what comes to mind? There may be a feeling 
of drudgery, that it's onerous, maybe even odious. We often view work 
as something we're beset with, something we have to get done and out of 
the way before we can be comfortable and at ease. Or we may think of 
work as something that keeps us from our meditation practice. These are 
merely perceptions that come up in the mind, and it's helpful to 
challenge them.

If we do challenge them, we can learn to bring mindfulness into the 
work period, which helps us develop a continuity of attention and 
reflective clarity. When we approach the work period like this, we 
realize it isn't something to get out of the way, but rather, it's an 
important aspect of the training. We don't want to be like so many 
people who live their whole lives waiting for the work to be over: 
\emph{When I get this done, I'll be happy and I can relax.} … And as 
death approaches, they're still waiting.

It's also helpful to remember that the work we do in the monastery 
serves to maintain the facilities here; we're looking after the places 
we live in, for our own benefit and for the benefit of others. When we 
perceive our work in this way, we can recognize that it's an 
opportunity for giving, for generosity, and for putting forth effort 
that benefits everyone. And we're reminded that generosity is not 
limited to material things, but includes acts of service, wholesome, 
skillful, and selfless actions. As a result, our work can become a 
source of well-being and happiness.

In Thai, the word for work also means festival. This blurs the 
distinction between work and enjoyment. During my time in Thailand, I 
was always struck by the northeastern villagers who came to the 
monastery to work. They really enjoyed themselves. They saw this work 
as an opportunity to do good things together. We can take a page from 
their book and use our work as an opportunity to develop a sense of 
lightness, to enjoy pitching in together and enjoy doing what is 
beneficial.

