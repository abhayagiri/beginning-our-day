\mychapter{Spacious Practice}{Ajahn Karuṇadhammo}{June 2013}

Having a sense of spaciousness is an important quality to look after in 
one's meditation practice. It's easy to become contracted and narrow in 
what we focus on, particularly during the work period when we can be 
quite involved with the task at hand. During this time we can easily 
remain involved, absorbed, enmeshed, and identified with the work we're 
doing. While intense focus like this can serve a purpose, the 
cultivation of an open and spacious mind can help keep this focus in 
the context of practice. This is talked about in certain meditation 
circles as keeping a ``broad choiceless awareness.'' It's a concept 
that is quite useful, but often misinterpreted. I've always appreciated 
Luang Por Pasanno's description of an unfocused spaciousness of mind as 
a nebulous nothingness that we are \emph{not} trying to cultivate. We 
are not striving for spaciness, because spaciousness is different from 
that. This quality of space and being open carries with it a sense of 
alertness, attentiveness, and clear comprehension. Spaciousness doesn't 
mean that we drift, totally object-less---it's important we still 
maintain a frame of reference while we open up to that space.

This is quite a useful quality to cultivate. It's easy to get involved 
in our mental worlds but we can take the opportunity, even while 
working or walking from one place to another, to broaden the focus a 
bit. This helps to create a little more space in the mind and body 
without drifting off into a nether world. It reminds me of something 
Ajahn Sumedho used to say, possibly quoting Ajahn Chah: ``Oftentimes we 
think of the mind as being within the body. We have this body and 
inside it is the mind. But it may be even more skillful to consider 
that the body is in the mind.'' So with that broad, spacious, relaxed, 
open awareness we can keep the object of attention on the body itself, 
and see that the body is contained within that awareness. That's a good 
focus to maintain in the forest---walking up and down the trails, 
walking from one work site to another, or when we are doing anything 
using our bodies---we can tune into a relaxed, open, state of 
awareness. In this way we're not closed in on our mental worlds, but 
rather holding very clearly in our minds the object of the body moving 
through space. It may be a simple object, but it's one that keeps us 
tuned into the present moment, and helps us avoid the experience of 
drifting into spaciness.

This broadness of mind is also useful in sitting meditation---keeping 
the breath within that spacious context of an expansive awareness. We 
can pick up and cultivate any object within that spaciousness. For 
example, when cultivating the brahmavihāras, the qualities of 
loving-kindness, compassion, altruistic joy, and equanimity, the states 
of mind that are talked about as boundless, we can establish that sense 
of an open and expansive mind, even before we start to pick up one of 
the brahmavihāras. That provides a broad, open container for the 
brahmavihāras where they can be developed. With any object we are 
developing, we can bring this spacious quality into our daily 
activities by establishing an open, easeful sense of the mind, and a 
broad awareness that is alert and attentive to what it is that is 
happening in the present moment.

