\mychaptertoc{Krueng Yoo: The Tool That Sustains Us}
{Krueng Yoo: The Tool That\\Sustains Us}
{Ajahn Yatiko}{July 2012}

It is worth reflecting on the Thai phrase \emph{krueng yoo}. 
\emph{Krueng} literally means ``tool.'' When it is combined with 
\emph{yoo}, the loose translation is ``a tool used to sustain.'' With 
Dhamma in mind, krueng yoo can mean a practice that is used to help 
sustain one's spiritual existence. So we might reflect and ask 
ourselves, \emph{In my daily life, what do I use to occupy my time? 
What is the practice that sustains me?}

At times we may say to ourselves, \emph{I've been scattered lately, and 
I really want to focus more}. So we make a determination, \emph{I'm 
going to be more focused in what I do}. That can be a pitfall if our 
resolve comes from the standpoint that our situation is not acceptable 
and from a desire to control or coerce, rather than the standpoint of 
taking an interest in examining our experience. It's like parents who 
constantly tell their child what to do, trying to force the child to 
act in certain ways; after a while, the child doesn't bother listening. 
The child and parents can end up with a split. It can be the same with 
our minds---we can have this split as well. We might try force 
ourselves to focus or practice harder, but there's a part of the mind 
that can rebel, that doesn't want to do be forced.

So rather than trying to force the mind, we can instead have an 
internal dialogue. \emph{Why is it that I don't want to practice 
sometimes? What is that about?} Without forcing and imposing our 
control, we can get to the root of the problem, through skillful 
investigation. Instead of \emph{telling}, we're \emph{asking}---we're 
probing, inquiring, and looking for the defilements that lurk within. 
Forcing ourselves to do something doesn't actually deal with any of the 
defilements. We need to investigate: What are the conditions that give 
rise to the defilements? What causes them to hang around? This gives 
rise to an understanding of ourselves, and an understanding of 
suffering. We might say that investigating in this way is our krueng 
yoo.

