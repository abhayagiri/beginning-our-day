\mychaptertoc{When Generosity Motivates Our Practice}
{When Generosity Motivates\\Our Practice}
{Ajahn Yatiko}{April 2013}

This morning in meditation I was reflecting on the various offerings 
that have been made to me in my life. It's a really wonderful exercise 
to do from time to time. This sitting cloth was made by Ajahn Ñaniko. 
I'm wearing clothes that Dennis gave me. David's mother, Ayya 
Santusika, gave the monastery this vibrating alarm clock. Ajahn 
Jotipālo made this wooden holder for the bell. Ajahn Saññamo gave me 
the socks I'm wearing, which were given to him by Tan Khemako, who 
received them from a layperson. The flowers on the shrine came from our 
friend Apple.

As monastics, it's nice that, for almost any material item around us, 
we can name the specific person or group of people who offered it. By 
doing this we realize that absolutely everything we have is a gift from 
faithful and generous people. As alms mendicants, we rely on the 
faithful generosity of the laity to provide us with food as well as 
other material supports---robe cloth, shelter, and medicine. Our 
survival is sustained through the gifts of others. When we reflect on 
this, quite naturally the result is a sense of gratitude and 
appreciation.

At the same time, such generosity and support can cause us to ask, 
\emph{Why am I allowing myself to receive all this goodness and 
kindness?} The reason is that, having faith in the Buddha's teachings, 
we became intent upon being free from greed, hatred, and delusion; and 
so we came here to devote ourselves to the Buddha's path, which takes a 
great deal of commitment and effort. This is why we receive so much 
goodness from others, which in turn inspires us to practice well.

However, this dynamic can create a problem. Many of us come from a 
strongly guilt-driven culture, and we can say things to ourselves like, 
\emph{Everyone is supporting me, so I should practice hard, I should do 
sitting meditation and walking meditation. I should learn the suttas.} 
For us, the word \emph{should} can mean that if we don't do it, we're 
really bad monks. That's the completely wrong approach. Instead, we can 
reflect that, \emph{Since everything is given to me, even this body 
doesn't belong to me. It belongs to the faithful---it's merely loaned 
to me by them.} With that in mind, there's still the sense that we 
\emph{should} practice, but it comes from a completely different 
\emph{should}. It's more like a voluntary \emph{should.} It comes from 
a wholesome desire to practice because we know that it's worthwhile, a 
decent thing to do. We come to realize, \emph{It's for my own benefit, 
and it's for other people's benefit}.

As we walk on the meditation path, we can contemplate the difference 
between these two kinds of \emph{shoulds}. There's a significant 
difference between the \emph{should} of guilt, and the \emph{should} of 
the natural and wonderful activity we're interested in doing. We can 
reflect on the blessings of our lives, the things that have come to us, 
and realize what a wonderful opportunity we have to practice and 
cultivate the path of the Buddha.

