\mychapter{Applying a Wholesome Attitude}{Luang Por Pasanno}{May 2012}

Yesterday Ajahn Saññamo gave us a reading about sweeping leaves---how 
sweeping can be an integral part of our practice. His reading came from 
a book detailing the practices at Ajahn Baen's monastery. It is a big 
monastery with large open areas, so every day, everybody goes out to 
sweep. In his monastery, sweeping is part of the practice and training. 
That's the way it should be for us as well---and not only with 
sweeping. All the various chores, duties, and responsibilities we have 
are part of the training. They're not simply things to fill the time, 
or excuses to take a break from practice to get something ``practical'' 
done. Rather, they allow us to examine the attitudes we bring to the 
activities we perform and to evaluate the way we spend our time.

And how \emph{do} we spend our time? Do we spend it thinking about 
ourselves, and resenting anything that impinges on our preferences, 
views, and opinions of how we imagine things should be? Or do we spend 
our time engaged in what we do with generosity and kindness, with a 
sense of relinquishment? Do we put energy and effort into our chores 
and duties, or do we try to slide by, thinking to ourselves, \emph{If 
people see me moving around the place, maybe they won't notice that I'm 
not really getting much done.} These are some of the attitudes we might 
bring to our practice. It's helpful to skillfully engage with these 
attitudes, because they can give rise to unwelcome, problematic moods. 
Once a mood like that does arise---if we're engaged with what's going 
on---we can respond with an energetic attitude, asking ourselves, 
\emph{How can I shift this mood in a positive way?}

There's a story Ajahn Sumedho tells about sweeping during his early 
days at Wat Pah Pong. It is a standard practice in forest monasteries 
to sweep the grounds, and Wat Pah Pong is no different. The day before 
each Wan Phra, the bell would ring to indicate that it was time for 
sweeping, and all the monks were supposed to go out and sweep the 
large, dusty central areas of the monastery. Ajahn Sumedho relates how 
he didn't like going out into the heat and the dust, nor did he like 
the activity of group sweeping, so he'd usually wait to join the group 
until he was about the last person to come out. Because he took so long 
to join the group, he'd often get the last broom, which was usually 
some scruffy old thing, made of a few little twigs that didn't really 
do much. Ajahn Sumedho said he would go along and scratch a bit at the 
ground, stand, wait, internally grumbling and complaining, saying to 
himself, \emph{This is really stupid. I don't like this. Why do we have 
to do this?} His mind would go on and on. Of course, Ajahn Chah noticed 
what was happening, and one day during the sweeping period he walked 
past Ajahn Sumedho and said, ``Wat Pah Pong---is it suffering?'' Ajahn 
Sumedho reflected on this, \emph{Wat Pah Pong---is it suffering? No, of 
course not! Wat Pah Pong isn't suffering! It's me! It's not Wat Pah 
Pong. I like Wat Pah Pong. I'm the one making suffering out of this. 
It's me.} That was a powerful and significant insight for him---to see 
that the external situation is one thing, and whether or not he adds 
suffering to it is another. After that, Ajahn Sumedho put more energy 
and enthusiasm into the sweeping. By doing so and reflecting, he 
decreased the suffering for himself and turned sweeping into an 
enjoyable experience.

We can use our chores in the same way---to bring up energy, a sense of 
relinquishment, generosity, service, and mindfulness. And doing that 
helps to provide continuity for our practice; it keeps us reflecting on 
what's happening in the mind, what's going on with our attitudes and 
perspectives. We can see more clearly the story the mind is telling us, 
so we learn how to work with it in a skillful way. Sometimes it's easy 
to focus on getting a particular chore over with, so we can go back to 
our dwellings and do ``our practice.'' In doing this, we forget that 
our mental state and what's going on in the mind \emph{is} our practice.

