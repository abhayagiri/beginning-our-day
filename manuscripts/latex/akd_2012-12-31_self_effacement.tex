\mychapter{Self-Effacement}{Ajahn Karuṇadhammo}{December 2012}

Developing insight around the aspect of \emph{anattā}, not-self, is 
often understood to arise from a sudden insight, awakening, or 
penetration that results from the culmination of earnestly practicing 
\emph{samatha} and \emph{vipassanā}, concentration and insight. In a 
sense there's a truth to that. A deep and penetrating insight does come 
through investigation. But there's a slow, gradual process that's part 
of coming to this insight of not-self as well. The words that describes 
this slower process best are ``self-effacement''---a gradual thinning 
or wearing away of the self making process, this sense of me and mine, 
and the ownership and identity around all the things we usually 
identity with. The process of self-effacement is a progressive and 
steady process, a whittling away over time through various practices, 
not only the practice of meditation and development of insight around 
the three characteristics, but in everything we do as part of community 
training. Many of the things that we incorporate into our daily lives 
and daily practices are ways of increasing this kind of 
self-effacement, this gradual diminishing of the sense of identity. 
It's a process that can take many, many years.

Dhamma practice and training, particularly in the monastic form, is a 
long-term endeavor. There are many ways that our structures and 
community life help support self-effacement. For example, the practice 
of generosity is an external way of reducing the sense of 
self-importance. We give of ourselves by offering our time and support 
to each other and by making an effort in how we communicate with one 
another. We are also generous with the work duties we take on and how 
we contribute around the monastery. Little by little, if done 
sincerely, it's all a part of the self-effacing process.

The hierarchy is another way that we can encourage this process. As 
much as we try to establish an environment where people feel 
comfortable to express themselves about the way things should happen 
there's also the sense of hierarchy that we give ourselves to. We have 
business meetings and regular circle meetings where people contribute 
in a free way. And when decisions need to be made we utilize this 
hierarchy based on seniority. Basically, it's boiled down to one very 
simple aspect of who's been around the longest in robes. Hopefully, 
there's also a bit of experience and maturity that comes along with 
years in the robes, so it's not completely an arbitrary structure. Most 
importantly, it's also a way of learning to let go of self views, self 
opinions, or the thought, \emph{My way is the right way}. If there's 
some sort of disagreement or something that needs to be worked out or 
decided on with different views and opinions, then seniority plays a 
part in that. We turn over a bit of selfishness to the monastery 
hierarchy. As Ajahn Yatiko was saying, if somebody misses a morning 
meeting, an evening meeting, work or whatever other communal activity 
we have, then the proper protocol is to mention one's absence to the 
senior monk as a way of being open and honest. This is not encouraged 
as a punitive guilt trip, but as a way of acknowledging our mistakes. 
All of these actions add up little by little---working together, 
bumping up against each other, accommodating each other, giving up our 
own views and opinions---all of these are subtle ways of working toward 
self-effacement.

A number of us came to the monastery with a sense of wanting to become 
something or somebody. It's sometimes nice to develop a set of skills 
or a set of training that we can call our own, as a way of feeling like 
we can contribute something unique. That's not a bad intention, but it 
can sometimes be held to as an identity. For example, wanting to become 
the best work monk, the best guest monk, the best attendant, the best 
meditator, the best teacher, the most knowledgeable in suttas, or 
whatever it is that we want to identify with. All of these skills can 
be useful tools for community life, but we can also develop a sense of 
identity around them---becoming this or becoming that---and doing what 
we might be doing if we were engaged in worldly careers. The idea is to 
pick up all of these things, use them, develop skills, but do this for 
the purpose of maintaining community life, offering service, and really 
looking at how to keep on giving up self identities and a sense of 
self-importance in the same process. Little by little, bit by bit, with 
this process we're realizing that there really isn't any ``me'' here 
who's doing all of this. We are using skills to develop the Eightfold 
Path, to encourage each other, and to contribute to Saṅgha harmony. 
Gradually this process supports us in developing the insight into who 
we are not.

