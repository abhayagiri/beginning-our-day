\mychaptertoc{Loving-Kindness and Living in Harmony}
{Loving-Kindness and\\Living in Harmony}
{Luang Por Pasanno}{May 2005}

The Buddha taught the \emph{Saranīya Dhammas}, which are the six 
principles of conciliation or harmonious living. The first three of 
these Dhammas are based on bodily, verbal and mental acts of 
loving-kindness. By generating loving-kindness, we create a strong 
condition for the arising of wholesome kamma. When we establish harmony 
with others, we build on that foundation of loving-kindness. The fourth 
Dhamma, which encourages sharing or giving, also has its basis in 
loving-kindness. When one is living together in a community, as we do, 
this quality of sharing is essential to creating a sense of mutual 
congeniality and care. The fifth principal is virtue, which establishes 
trust---and thereby harmony---among people living together. And when we 
trust each other, we have a sense of mutual loving-kindness as well. 
The last Saranīya Dhamma regards holding a Noble View of the potential 
for freedom, the ending of suffering. When we see that we and others 
can realize this Dhamma, and that our companions have the same View, it 
creates feelings of kinship and kindness based on this harmonious 
perspective.

How do we bring up loving-kindness? What do we do when we feel it's 
impossible? And when it does arise, what makes us lose it? For 
instance, sometimes when I bring up loving-kindness myself, fear 
arises. The Buddha taught that loving-kindness is appropriate in all 
situations, so it is helpful for us to reflect on these questions and 
try to answer them for ourselves. By investigating, we can learn to 
encourage and sustain loving-kindness, to work with it and challenge 
any aversion we may have to it. This doesn't mean we'll get it right 
all the time. But we should reflect on how to bring about 
loving-kindness in our thoughts, actions and speech, because it's 
through loving-kindness that we live in a harmonious way.

