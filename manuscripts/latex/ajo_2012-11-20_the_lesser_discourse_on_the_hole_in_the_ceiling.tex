\mychaptertoc{The Lesser Discourse on the Hole in the Ceiling}
{The Lesser Discourse on the\\Hole in the Ceiling}
{Ajahn Jotipālo}{November 2012}

In the earlier days, the Dhamma Hall was not quite as nice as it is 
now. There was still carpet on the floor. One day Ajahn Amaro was 
giving a talk when it was raining really hard, just as it was this 
morning. As he was speaking, water started to leak from the ceiling and 
drip down right in front of where I'm sitting now---in pretty much the 
same place as the current leak we are having. Ajahn simply kept on 
giving the talk. It wasn't as if nothing was happening, but the new 
leak became part of the talk's theme of unexpectedness and uncertainty. 
I think he titled the talk \emph{The Greater Discourse on the Hole in 
the Ceiling}. The title alone made it memorable.

A little drip like that may look like a minor issue, but people who 
understand construction even a bit know that if there's a little drip 
inside, there's a hole someplace on the outside. Maybe there's a hole 
underneath a shingle and the water is getting in and flowing down into 
the rafters. And the hole could be anywhere---it could be twenty feet 
away from where the water is dripping into the hall. That water is 
coming down and it's taking the path of least resistance. Maybe it hits 
a two-by-four and goes a different direction, then twenty feet later it 
hits another two-by-four. Meanwhile, inside the room here, the leak 
looks like it might come from a little hole in the ceiling that we 
could fix by putting a piece of tape over it. But a proper fix can be 
quite complicated, and if it's not taken care of quickly, then it's 
possible that all the water up there could saturate the drywall ceiling 
boards and the whole ceiling could collapse. So even what looks like a 
little drip can be quite serious.

This is often the case with a seemingly little issue like that. We can 
always put a towel on the floor and ignore the drip, or use a bucket 
temporarily as we've done right now. But the wise thing to do is to 
investigate what's wrong and come up with an appropriate plan to fix 
it. That could mean finding its root cause and fixing it ourselves, or 
calling in a professional to help with the job.

The same is true in living the holy life, in our meditation practice, 
and in developing Dhamma. There might be something on the surface, 
manifesting in our behavior or attitudes---something that we notice, 
but others don't. Or maybe other people notice something about us, but 
we don't notice it ourselves. It could be minor. But if there is some 
troublesome issue that is the underlying cause, it can fester and grow. 
It can create many difficulties and problems for us and for other 
people. So these things do need addressing. When something arises that 
requires our attention, we first acknowledge it as being an issue, and 
then investigate the extent of the problem it presents.

In some cases, we might not be able to fully see the problem or its 
cause, or we may suspect the problem is bigger than what we can fix 
ourselves. So with our Dhamma practice, when we have an inkling of 
that, it's important to enlist others for help, maybe a teacher who can 
look at the problem and help us work with it, undermine its cause, and 
eventually master it. That is how we can prevent a little leak in our 
practice from causing a large amount of suffering.

We can remember this talk as \emph{The Lesser Discourse on the Hole in 
the Ceiling}.

