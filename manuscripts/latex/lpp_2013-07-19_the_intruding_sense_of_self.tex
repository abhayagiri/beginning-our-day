\mychapter{The Intruding Sense of Self}{Luang Por Pasanno}{July 2013}

It is uncanny how the sense of self tends to intrude on everything, 
whether it is in our meditation, when we are working, doing chores, in 
a group or by ourselves. That sense of self keeps rearing its ugly head 
and creates suffering and a burden that we carry with us. Sometimes we 
think that coming to live and practice in a monastery is solely about 
relinquishing the world, but the heart of relinquishment is giving up 
of the self and the perception of self. Upon very close examination, 
the feeling, the projection, and the perception of self is nothing more 
than smoke and mirrors and does not have any real substance.

We have this self-referential obsession: comparing ourselves to others, 
worrying about how we are doing, how others perceive us or how we 
perceive ourselves in relationship to the past, present and future, and 
it becomes incredibly convoluted. If we take a good look we can 
appreciate how much of a burden all of this is. I believe this sense of 
self-obsession has never been as strong as it is now in modern culture. 
Before, most people's sense of self had to do with 
relationships---one's relationship to family, village or tribe. But in 
modern society everything ends up being about \emph{me}.

In our practice, we are trying to recognize this perception of self, 
seeing how we keep buying into it, giving weight to that feeling or 
perception, and recognizing that each building block of self is simply 
an impermanent phenomenon. It isn't a matter of trying to get rid of 
the self or annihilating it in any way, because it's not something that 
is solid or substantial anyway. That's why we reflect on the five 
\emph{khandas}---form, feeling, perception, mental formations and 
consciousness ---seeing them as impermanent, unsatisfactory and 
not-self. When we see the five khandas for what they are, we gain 
insight into their true nature and let go of our investment in 
them---which is not a rejection or annihilation, but a clear seeing.

To be free of the world and worldly tendencies, we need to take leave 
of the world. Taking leave of the world means taking leave of the 
self-making, I-making habit. In the Buddha's idiom, when we buy into 
the notion of self, it is an affirmative action we are doing: 
\emph{ahaṅkāra-mamaṅkāra-mānānusayā,} the I-making, my-making, 
based on the underlying tendency of conceit. It is the action of 
getting ourselves wrapped up in the perceptions of self---and believing 
in them.

As we go about our day, we can reflect, investigate, and challenge 
these assumptions about the self and be willing to work with them over 
time. These assumptions are deeply ingrained in the mind, so examining 
them, making this process conscious, and being aware of that self 
perspective is a central part of our practice.

