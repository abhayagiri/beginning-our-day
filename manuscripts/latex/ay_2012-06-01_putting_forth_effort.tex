\mychapter{Putting Forth Effort}{Ajahn Yatiko}{June 2012}

For both monastics and laypeople visiting the monastery, it is helpful 
to reflect on sustenance, what it is that sustains us materially. 
Laypeople offer food to the monastery, and we eat this food. They work 
some eight hours a day, five days a week or more, at a job that can 
often be unpleasant. It's hard work, and they call it work because it 
\emph{is} work. For most people the mind inclines toward not doing this 
work. People would rather be relaxing, sitting out under the sun and 
taking it easy, but they can't do that. They have to work, because they 
have to support themselves and sustain their lives. Many people come to 
this monastery on a regular basis with food they've bought using the 
fruits of their own labor. We eat and depend on this offered food every 
day at the monastery, so we have a very direct relationship with the 
work laypeople do.

And as a result, we have a responsibility to practice; our practice is 
why we are being offered food. Laypeople want to support the monastery 
so that we will grow in the Dhamma and lessen our greed, hatred, and 
delusion. Ajahn Dtun once said that we should be meditating at least 
eight hours a day, because that's the amount of time laypeople put into 
their workday---and for many of them it's even more than that. I think 
that at Abhayagiri eight hours a day of formal practice would be 
difficult, because we have a lot happening, especially in the mornings 
with our work period and chores. However, we can still think of this 
eight hours in the sense of making sure that mindfulness is present for 
that amount of time---actually, for the entire day. During this time 
we're not simply following our moods and opinions, we are going against 
the grain and putting forth effort to decrease the defilements.

Some of us may feel dispirited when we hear about putting forth effort, 
because it sounds heavy to us and we may not want to put forth a lot of 
exertion. To be straightforward, whether we like it or not, it's 
something we have to do; it's simply part of the deal, part of what 
monastic life is all about. If putting forth effort is something we 
don't want to do, then monastic life might not be for us. It is a life 
of effort, and it requires resolution and struggle. We should keep that 
in mind.

But the situation is not bleak. In the beginning stage of making an 
effort, there is a hump we have to get over, but then it gets easier. 
There are three phases of effort. In the first phase, energy needs to 
be aroused, and this requires discipline and exertion. In the second 
phase, once energy is aroused and established, the effort maintains 
itself to a certain extent, because of that established energy. The 
third phase is known as being unshakable, where nothing can stop the 
effort and energy until the goal is achieved. The first phase is the 
most difficult, because it takes a lot of strength and resolution 
simply to get things going. It's like a rocket that's leaving the 
atmosphere; it takes an enormous amount of fuel to overcome the pull of 
Earth's gravity, but once it gets into space then it can coast for 
awhile. It's a bit the same with the energy and effort we put into our 
practice.

Through this entire process, especially the first phase, it is 
extremely helpful to hold the attitude and perception that making an 
effort is actually \emph{pleasant}. If effort is exercised properly, 
the experience of effort is enjoyable, invigorating---something we can 
learn to delight in. If we don't take delight in putting forth effort, 
then it is quite an unpleasant experience. So we need to learn how to 
experience the pleasant side of effort. We can start with a simple and 
direct reflection: \emph{In what way can I put forth and sustain effort 
so that it is enjoyable, fulfilling, and nurturing to both my practice 
and to my heart?}

