\mychapter{Containing the Chicken}{Luang Por Pasanno}{August 2012}

The theme we used for \emph{Upāsikā} Day yesterday was brightening 
the mind and focusing on the ten anussati---the Ten Recollections: 
Buddha; Dhamma; Saṅgha; virtue; generosity; celestial beings; 
mindfulness of the body; mindfulness of death; mindfulness of 
breathing; and Nibbana---peace. Several people expressed their 
appreciation for being taught about the Ten Recollections in that it 
gave them more freedom and flexibility to work with the mind.

Meditation can often become mechanical or method oriented, or focus 
solely on concentration and trying to get the mind to stop thinking. 
Sometimes when we practice like this, there tends to be a lot of effort 
expended, and this can agitate the mind. However, the Buddha taught 
ways of using the thought process so that it supports the cultivation 
of wholesome and bright states of mind. Thinking in this way naturally 
leads the mind to settle down. The mind naturally settles when it feels 
good, when there's a quality of well-being. We encourage this by 
bringing to mind directed thought and attention and evaluating the 
result of that. When we do this, there can be a continuity of wholesome 
and skillful thoughts, and the mind can become settled and clear.

Ajahn Chah said that meditation is like containing a chicken. In a 
village, when a person has a chicken and is concerned that it's going 
to run off, he gets a loosely woven large bamboo basket and plunks it 
over the chicken. Rice can be put in the basket, and the chicken stays 
in the container. One doesn't have to tie the chicken down or keep it 
from taking a step anywhere. Having a container for the chicken is 
sufficient. In the same way, when the mind is imbued with a sphere of 
awareness and intention, and attention is being directed on a wholesome 
theme, then the mind will stay within that container. It doesn't mean 
that it's forced to stay there, but it's content within that sphere. 
And these recollections are a skillful means of directing attention, 
bringing wholesome recollections to the mind so it will settle within 
that container.

Both in our formal meditation practice and throughout the day we can 
bring to mind and recollect the Buddha, the Dhamma, the Saṅgha, 
virtue, generosity, celestial beings, the body, death, in- and 
out-breathing, and Nibbana---peace. Another helpful theme that is not 
on this list is the quality of saṃvega, spiritual urgency or a 
chastening of worldly tendencies. This can help remind us of death and 
the limitations of the body. A complementary theme for saṃvega is to 
set up the intention to put forth effort, to realize what the Buddha 
taught and the example that he set. Another important quality that 
comes to mind is pasāda, which is a sense of serene confidence and 
clarity of well-being; it has a joyful sense to it.

We want to encourage these ten recollections in our practice as well as 
these other skillful qualities so that we create a wholesome foundation 
for the mind. We have a lot of space to experiment and work with these 
themes. As we get to know them, we can see what works and what helps 
the mind direct itself toward the Dhamma.

