\mychapter{What Do We Take With Us?}{Ajahn Karuṇadhammo}{August 2013}

We can take this moment to let the momentum of planning for the morning 
settle before launching into the day. Watch the mind that's leaping 
forward to go and accomplish the tasks, to start the workday, to do 
what needs to be done. Watch the momentum of becoming---identifying 
with the various activities we're about to perform and the roles that 
we have---the ideas, thoughts, views, opinions, perceptions, 
conceptions. Watch how we identify with the activities and emotions of 
the mind.

A couple days ago I was visiting a woman who is in the process of 
dying. She probably doesn't have too much longer to go. She's someone 
who has always wanted to live life fully, being involved and engaged 
with many different good things. Like most of us, she's strongly 
identified with that and does not want to look closely at the dying 
process, at death, or at what that means. When I saw her, she was at a 
point where her body was not responding to her wishes to keep it going. 
She wanted to continue living life fully engaged, doing all the things 
that she was finding a lot of gratification with in a very wholesome 
way, but her body was going counter to that. It was saying to her: 
``I'm packing it in'' and it was no longer going to be a viable vehicle 
for living life fully. She is realizing all of this now.

When we have such a strong impulse toward becoming, identifying closely 
with the body, and then all of a sudden we're faced with the dying 
process---the body going in the opposite direction, the direction of 
non-becoming---then we can experience a very sudden wake-up call. The 
momentum of the mind toward using the body as a vehicle for 
identification, for becoming, for gratification is ripped out from 
underneath us when the dying process begins. What a lesson this can be 
for all of us.

This is why the Buddha suggested the contemplation of death as a 
necessary reflection for us. It helps uproot that sense of 
identification and impulse toward becoming what it is we think we are, 
who we think we are, what we identify with, whether it's with our 
bodies, our activities, or our roles in the monastery. In the 
ordination process here, when we're a long-term lay resident in the 
monastery, we look forward to becoming an anāgārika. When we're an 
anāgārika, we look very forward to becoming a sāmaṇera. When we're 
a sāmaṇera we look forward to becoming a bhikkhu. Then, when we're a 
bhikkhu, we look forward to becoming an ajahn. And then … we're 
almost dead.

What is it that we can take with us? We can't take any of those 
identities or roles. We can't take any of the accomplishments with us 
or our failures. We can't take any of the praise that we've received 
for things that we've done well or the blame that we've received for 
things that we haven't done well. Whatever we have gained or lost 
doesn't go with us when we die. All of the status that we've 
accumulated, all of our particular views about the way that things 
should be run, all of the times we were able to be alone, wanting to 
escape from being around others, or all of the times we sought the 
company of others, being surrounded by family and friends---we don't 
take any of those with us; they all disappear when we die. All we take 
with us are the impulses and the tendencies we have toward either 
skillful states of mind like generosity, morality, and patience (three 
of the pāramī) or the more unskillful impulses of greed, hatred, and 
delusion. These tendencies, in as much as we develop them, are taken 
with us, but we don't take any of the other things that we generally 
hold to as so important or permanent.

So throughout the day, what is it that we want to develop? If we're 
going to die tonight, what qualities do we want to take with us? Do we 
want to take the becoming tendencies, the activities, the momentum of 
tasks, identities and roles, how we think we are, how we want to be 
seen by other people, or how we present ourselves to the world? Do we 
want to work on those? Or do we want to work on developing the 
essential aspects of our practice: service, kindness, impulses of 
skillfulness, wisdom, discernment, patience, energy? Those are the 
qualities we can take with us. The choice is up to us. We can choose 
those skillful tendencies leading us toward seeing this process of 
becoming, seeing how we identify with the body, and then learning how 
to let go. Or, we can go in the opposite direction. It all depends on 
where we focus our attention as we move throughout the day.

