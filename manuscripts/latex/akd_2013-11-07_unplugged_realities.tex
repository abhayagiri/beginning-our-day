\mychapter{Unplugged Realities}{Ajahn Karuṇadhammo}{November 2013}

I recently came back from a one-week visit to Tisarana Monastery in 
Ottawa, Canada to help them celebrate their kathina ceremony. During 
this time I did a lot of traveling and spent some time in airports. 
While I was en route I noticed the alarming amount of people who were 
plugged into to one or another type of mobile device, either a 
smartphone, tablet, or laptop. It was interesting to observe them as 
each person seemed to be plugged into their own separate spheres or 
realities. They weren't interacting with others around them and seemed 
quite oblivious to what was happening in their surroundings.

One man I observed was conducting his business before boarding the 
plane. I watched him making several calls and doing business over the 
phone. He would talk, conclude a conversation, and begin another one. 
It occurred to me how internally caught up he was in this reality of 
his, this sphere of a world he created. As the announcements for the 
plane departure began, it seemed that there was this subtle attention 
off in one corner of his mind and he was able to be in touch with the 
fact that he was about to board the plane. When it was his turn to 
board, he rose up but continued to speak on the phone conducting 
business up until the very last point that he sat down on the plane and 
had to put his phone away.

I reflected upon this and thought about how, without cell phones or 
portable computers, we do much of the same thing here in the monastery. 
It is so easy to get caught up into our own realities and dwell in 
these separate spheres, absorbed in what we think is some permanent 
condition of the mind. We can get fixated upon some idea, mind state or 
concern we have and not really see that it is changing phenomena. It 
could be something like a health concern or some interaction we have 
with someone, and we believe this reality is all that exists for us. 
Our thoughts can revolve around this world to the degree that we are 
unable to see outside of this closed perspective or view.

Then, without even noticing it, some action outside of us or something 
someone says, completely distracts us from this world we have become so 
absorbed in, and we are reborn into this new world without even 
remembering what we were so obsessed with just moments ago. This new 
reality becomes the focus of attention in a way that can prevent us 
from seeing this entire process. It's as if we are jumping from one 
world to the next with no fixed point of reference. We can be blown 
around by our emotions, thoughts, and preoccupations about these 
insubstantial experiences.

Even right now, as I end this talk, we are all going to get up, begin 
cleaning dishes, and prepare for the work period. This will be a 
completely different experience from what we are having right now. So 
if we can be aware of this changing of worlds and not get so caught up 
with it, then we have the opportunity to see this blinding process and 
wake up to the understanding there is something outside of our very 
fixed perspectives. We can learn to expand the view and have a 
spaciousness that can encompass our experiences from a much broader 
context. By doing this, we allow ourselves to see things quite 
differently from a less absorbed and self-focused point of view. When 
we expand the view like this, we have the opportunity to see how our 
minds are moving from one reality to the next, creating world after 
world. We can step back from that process and watch it from a centered 
but detached point of view. When this occurs, there is the possibility 
of stopping, ending this cycle of world hopping, and opening ourselves 
up to the reality of our present experiences.

