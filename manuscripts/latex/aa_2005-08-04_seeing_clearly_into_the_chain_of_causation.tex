\mychaptertoc{Seeing Clearly Into the Chain of Causation}
{Seeing Clearly Into the\\Chain of Causation}
{Ajahn Amaro}{August 2005}

Last week at the Spirit Rock Family Retreat, we saw many small, young, 
human beings surrounded by wholesome structures and examples offered in 
the way of skillful guidance. Seeing the good results of that in just a 
few days made me reflect on the idea that if we can catch things early 
and have an influence at the beginning---as something is setting out 
and taking shape---then, even if the influence is small, it can go a 
very long way. The lessons we learn and examples that are internalized 
early on can affect us quite deeply.

Similarly, this works with how we apply the teachings. In particular, 
when we reflect on the cycle or chain of dependent origination---the 
laws of causality that govern our experience and the arising and 
ceasing of \emph{dukkha}---we can see that the earlier in the chain 
that we catch this process of causality, the less work we have to do to 
uproot suffering. When we look at the way we handle the worldly 
winds---happiness and unhappiness, praise and criticism, success and 
failure, gain and loss---the mind can be observed reacting to and 
chasing after gain, running away from loss, identifying with, seizing 
hold of and cherishing praise, rejecting criticism, and so forth. The 
sooner these reactive habits are seen and known, the less sorrow, 
lamentation, pain, grief, and despair we experience.

We live very much in our everyday world of perception and feeling. We 
see sights and hear sounds, touch objects, make decisions, engage with 
our bodies in the material world and with the 10,000 thoughts, moods, 
and ideas that arise from those various forms of sense contact. The 
more we internalize and make use of the teachings on dependent 
origination, the more we are aware that this is merely sense 
contact---\emph{phassa}. Sense contact gives rise to pleasant, 
unpleasant, and neutral feelings---this is beautiful, this is ugly, 
this is ordinary. From that launch-pad of feeling, in the ordinary flow 
of our experience, \emph{taṇhā} or craving arises---\emph{vedanā 
paccaya taṇhā}---feeling conditions craving, which then leads to 
sorrow, lamentation, pain, grief, and despair. When we know this chain 
of causation, we are able to see that a feeling can easily turn into 
thoughts of, \emph{I can't stand this. This is wrong. This is bad. I 
don't like this. I have to get rid of it.} Or the opposite, \emph{I 
want it. This is good. This is mine. I have to keep it, hold onto it, 
or own it.} When this cycle begins with possessiveness, we can see how 
that sense of ownership causes a feeling of loss, and then dukkha 
ensues.

Again, if we influence a three-year-old in a wholesome manner, those 
influences can have an effect for a lifetime. Similarly, if we 
mindfully catch the process right at the point where vedanā is 
conditioning taṇhā---feeling conditioning craving---the cycle can be 
broken right there. We can live with a heart completely at peace in the 
realm of feeling, reflecting: \emph{This vedanā is present. I don't 
have to own it. Praise is sweet; criticism is bitter. Gain is sweet; 
loss is bitter. This is how it is. I don't need to make anything more 
of it than just that. This is the mind that likes sweetness, this is 
the mind that dislikes bitterness. That's all. It's empty. There's 
nothing there. It doesn't belong to me or anyone else. It's merely one 
of the attributes of nature coming into being, taking shape, and 
dissolving. That's all it is.} So the heart remains at peace, even 
though there's full engagement in the world of seeing, hearing, 
smelling, tasting, and feeling, as well as perceiving, doing, and 
acting. There's full engagement, but it is free of confusion and there 
is no identification with it.

If we catch the process even earlier, way down at the deep-tissue 
level, and mindfulness is sustained acutely, then ignorance---the whole 
duality of me as a person experiencing the world out there, of me here 
going somewhere else, the subject being the knower of the object---is 
not given any strength or substance. If we don't catch it, however, 
that is when \emph{avijjā paccaya} \emph{saṅkhāra---}ignorance 
conditions mental formations. When there is ignorance, then that 
duality of sankhara or compoundedness arises. But if there is full 
awareness, full knowing, full mindfulness, then even that subtle degree 
of ignorance or delusion does not arise and is not given credence.

Discourses on dependent origination often state, ``With the cessation 
of ignorance there is the cessation of mental formations, and with the 
cessation of mental formations, there is the cessation of 
consciousness,'' and so forth. Phra Payutto points out in his book 
\emph{Dependent Origination} that \emph{cessation} is not meant to be 
taken only in terms of something beginning and then ending; the Pali 
word is \emph{nirodha}, which also implies \emph{non-arising}. When 
there's no ignorance, then mental formations do not arise; the mind 
does not create the world of \emph{thingness}, or 
\emph{this-and-thatness}. This is the realization of Dhamma. It is 
simply Dhamma. It is not fabricated. It is not divided. It is not 
created. It is not split into me being here and the world out there---a 
subject separated from an object. This is the peace and clarity of 
realizing the Dhamma here and now.

