\mychapter{A Mango Tree Was My Teacher}{Luang Por Pasanno}{May 2013}

As we practice, one of the qualities that we need to cultivate or 
attend to is learning how to pay attention to our circumstances, what 
is happening around us. So often we wait for instruction, or the right 
circumstances, or a sign that we are doing the right thing. For us to 
really take advantage of the practice, we need to learn from our 
circumstances and pay attention to the natural processes around us.

There's a story that Ajahn Chah used to use in his teachings. It is a 
Jataka tale about a king who takes his retinue off to do some business. 
As they pass by an area in the forest, the king sees a wild mango tree 
full of mangoes and thinks to himself, \emph{Those mangoes look really 
good. At the end of the day when we're going back to the palace, I'm 
going to stop by this tree and have some of those mangoes. That will be 
really refreshing.} Then he moves on. However, the people toward the 
back of the retinue have a different idea, and they start to bash the 
tree and shake the tree, knocking the branches down, taking the fallen 
fruit for themselves.

Evening comes and the king returns to the same spot. There, before him, 
is this poor, bashed up, barren tree where he believed all of the 
mangoes would be hanging. He is disappointed and thinks to himself, 
\emph{This is really sad. This big tree's been beaten up and abused.} 
He looks around and sees another mango tree that is not very full and 
does not have many mangoes on it. Reflecting to himself he thinks, 
\emph{That tree seems to be the same as it was before. There is a real 
problem with having a full tree---and it's the same with my life: I 
have many duties, responsibilities, and many people around me. Maybe I 
should transform my life to be more like this scrawny mango tree and 
step out of my duties and responsibilities.} That was his impetus to go 
forth on the spiritual path to become a religious seeker. In the 
future, whenever he was asked who his teacher was, he would respond: 
``A mango tree was my teacher.''

So when we consider the circumstances, the events, and the natural 
processes around us, we see that truth is being displayed all the time. 
The problem is that we don't pay attention, or if we do pay attention, 
it is not in a way that is reflective---we don't internalize or 
consider the true meaning. When this happens we don't derive the 
benefit of the experience; instead, we wait for some kind of overt 
teaching or instruction. With wise reflection, we learn how to 
internalize the teachings by observing the events and circumstances 
around us.

