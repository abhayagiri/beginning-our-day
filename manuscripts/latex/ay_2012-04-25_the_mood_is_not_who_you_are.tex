\mychapter{The Mood Is Not Who You Are}{Ajahn Yatiko}{April 2012}

There's always a mood present in our experience. It's amazing to think 
how the presence of a mood so completely shapes and conditions both our 
attitude and the way we see things. It's important to have straight 
vision---some sense of what our life is about, what it's for, and what 
it is we aspire to. This vision or aspiration provides a compass when 
moods arise that tear us apart and sometimes throw us into turmoil, 
irritability, anger, depression, or frustration. Various moods and 
emotions, often conditioned by relatively minor events, can arise and 
push us in a direction that's quite different than the major direction 
of our vision, our life, what we actually trust and what makes sense to 
us.

One of the lay residents here at Abhayagiri once talked about an 
interesting juxtaposition. One day, while feeling irritable, he came 
down from the mountain and went into the kitchen. There was a guest 
there who didn't seem to be pulling his weight, which affected this 
resident's mood. Later on that day, he got word that this guest was 
struggling to digest the news that his brother had been shot and 
killed. Juxtaposing reality and perception---the guest's shock and the 
resident's irritation, for instance---can put things into perspective 
and reveal how petty we can be.

It's human to get irritated when things aren't going as smoothly as 
we'd like or when we're feeling misunderstood. But we need to recognize 
that indulging in such moods puts our present and future well-being at 
risk. We have this life, and it's not a game. If we let our moods take 
control and spur us to act on impulse, then we wind up doing things 
that can damage our long-term interests and the well-being of others.

So when these unwelcome moods arise, it's important to do everything we 
can to gain perspective on them and remember that they come and go. We 
shouldn't blindly delight in good moods either. It's okay to enjoy a 
good mood, but if we get lost in it then we'll get lost when a bad mood 
comes around as well. There's no way around that. We can't 
realistically say, \emph{I'll take the good moods and forget the bad 
moods}. So we train ourselves in meditation with any mood that comes 
up. We take stock of it and remember that the mood is not who we are, 
as we chant in the Anāttalakkhaṇa Sutta. When a mood arises, we do 
our best to recognize its existence and then investigate how it may be 
pushing us to act in ways that aren't helpful to our welfare, or the 
welfare of others.

