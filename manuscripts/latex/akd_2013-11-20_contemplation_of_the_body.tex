\mychapter{Contemplation of the Body}{Ajahn Karuṇadhammo}{November 
2013}

The age-old themes of the body and contemplation of the body are again 
present for me. There is such a strong attachment to the body as self. 
I see this with my own experience and with so many of us here. There 
are those of us right now who are going in for physical exams, those 
with ongoing health problems, the issues with our friend Iris who is 
dealing with cancer, a number of people who need to see certain 
practitioners, doctors, and therapists for body issues, and six people 
going in for blood work tomorrow. We chanted the ``Reflection on the 
Thirty-Two Parts of the Body'' this morning. We do this reflection and 
examine the body because we are so strongly attached to it as either 
who I am, something of which I am in possession of, or something over 
which I should have control. It forms a major part of who we think we 
are, and because of that strong attachment, when things go wrong we 
suffer greatly. That is why a large part of the practice is to try, 
over and over again, to alter that belief of the body being self or 
possessed by a self. If we can lessen that attachment over time, little 
by little, when things go wrong, we do what we need to do to take care 
of the health of our bodies, but we don't do it with a sense of strong 
attachment, identification, clinging, pain, sorrow, lamentation, or 
grief.

We can practice this on an intellectual level and that's where it has 
to begin---bringing it in through the use of words and active 
reflection in a conceptual way. Yet how many of us let it reach a deep 
level, an emotional level, and possibly even further than that? Often, 
this level of deep understanding doesn't become apparent until 
something very serious happens. During those times we may come to 
understand that we have been dealing with body reflection in a shallow 
way. All of a sudden, the talks we have heard, the reflections we have 
done, all of the many times we have chanted the ``Thirty-Two Parts of 
the Body'' or some other body reflection, do not come into play. Where 
are they now?

The attachment to the body is so strong that we have to repeatedly 
bring it into consciousness at a very deep level. We do this by 
accepting that these bodies of ours are not things over which we have 
any control. We reflect on the process of aging, sickness, and death, 
let it sink in and at the same time, we look carefully at what is 
happening around us. For a person who is going through the process of 
catastrophic illness, the death process is not just something that's 
out there, but something that is happening right inside that person's 
body. It could be the proliferation of cancer cells reaching different 
parts of the body, obstructing tissues, and airways and spreading to 
the bones and the brain. I can recognize that all of that is something 
that could very likely happen to me. Or, some other disease will happen 
in this very body, to these internal organs, to all the parts we've 
been talking about, visualizing, and reciting. Each part will come to 
an end. They will all, at some point, cease to function, either one by 
one or at the same time. These parts of the body are subject to 
dissolution, decay, and decomposition and will return to the elements: 
earth, air, water, and fire, nothing more than that. Yet it all seems 
so real, so personal, and so ``me,'' which is where the suffering comes 
in.

Bringing this reflection to heart is a significant part of daily 
meditation practice, both on the cushion and off. To do this we need to 
look around and bring this reflection into as deep a level of the 
consciousness as possible. This can seem like a monumental practice but 
it is possible. Over and over again we can gradually whittle away at 
the identification we have with the physical form and eventually gain 
insight into the impermanent and not-self nature of the body.

