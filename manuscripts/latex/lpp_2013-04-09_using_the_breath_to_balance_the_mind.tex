\mychaptertoc{Using the Breath to Balance the Mind}
{Using the Breath to\\Balance the Mind}
{Luang Por Pasanno}{April 2013}

It has been nine days since we ended our three-month retreat, which 
seems like a good time to remind everyone: Don't forget to connect with 
mindfulness of the body and mindfulness of the breath. With the body, 
use the simple rhythms of moving from one place to another. Attend to 
that. Make it an anchor for developing mindfulness and awareness.

As for the breath, Ajahn Chah would often say, ``People complain that 
they don't have time to meditate, but do they have time to breathe?'' 
Well, of course we have time to breathe. But if we don't direct our 
attention to our breathing, then our attention slides by, and we end up 
absorbed in proliferation, expectation, irritation or whatever. So we 
must remember to consciously bring attention to the breath. Then we can 
learn how to \emph{use} the breath.

To begin, notice that breathing in has a different effect on the mind 
and body than breathing out. When we breathe in, there's an expansion 
of the body---we are bringing in the life force, expanding and 
energizing what one might call the system of our existence. That's 
different from breathing out, which is a release, a settling, a 
stopping. These distinct qualities have different effects on the mind. 
By knowing that, and by paying close attention to what's happening, we 
can use the different effects of our in-breaths and out-breaths to work 
with the mind when it's out of balance.

For instance, if the mind is leaning towards dullness and wanting not 
to deal with things, we can consciously attend to the in-breath and the 
energy it brings---enlivening, expanding and brightening the mind. At 
other times, the mind may be leaning towards distraction, doing and 
becoming; then we need to pay more attention to the out-breath, 
allowing the mind to settle down. By working with the breath in such 
ways, we can balance the mind, attuning ourselves with each in-breath 
and out-breath.

When we engage in any activity, whether sitting in front of a computer 
or out on the trail with a rake, we are still breathing, still 
occupying a physical body within the world around us. Each activity is 
an opportunity to inquire of ourselves, \emph{Where is my attention 
going? Am I aware of the body, of my in-breaths and out-breaths? What 
is needed to find a place of balance and clarity? Does the mind need 
lifting up, or settling down?} By doing this, we are caring for our 
practice throughout the day, not just during morning and evening 
meditation. We are developing the continuity of practice that is needed 
for the most beneficial results.

