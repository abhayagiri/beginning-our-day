\mychaptertoc{The Importance of Koṇḍañña's Insight}
{The Importance of\\Koṇḍañña's Insight}
{Ajahn Amaro}{November 2008}

In the Dhammacakkapavatana Sutta---the Discourse on the Turning of the 
Wheel of Dhamma---the Buddha addressed the five \emph{samanas}, 
including one named Kondañña, who were his practice companions before 
he became a Buddha. At the end of this talk, the Buddha recognizes that 
Kondañña has, with deep insight, understood the Dhamma. He says, 
``\emph{Annnyassivatabho Kondañño, annyassivatabho 
Kondaññoti}---Kondañña understands, Kondañña understands.'' 
Kondañña had awakened to the nature of Dhamma: 
``\emph{Dhammacakkhuṃ udapādi}---the eye of Dhamma arose.'' Then the 
sutta describes exactly what it was that Kondañña saw. It wasn't a 
spectacular vision of the heavens opening up, a bewildering light 
display that he experienced, or streams of deities beaming down from 
the heavens. Rather, his vision of the Dhamma is stated simply in the 
phrase, ``\emph{Yaṅkiñci samudaya dhammaṃ sabbantaṃ nirodha 
dhammanti}---whatever is subject to arising is subject to ceasing.''

On a worldly level this seems completely unremarkable. That which 
begins, ends. Whatever goes up, must come down. If it's born, it dies. 
\emph{No big deal,} we might think. But it's worth our reflection: This 
seemingly simple insight, which makes all the difference in the life of 
Kondañña, which enables him to awaken to the Dhamma, enter into the 
stream of Dhamma, thus making full enlightenment absolutely certain 
from that point on---why should that be the most profound insight?

It's helpful to reflect on this and use it as a theme of meditation. 
\emph{Why would it change my life to see that all that is subject to 
arising is subject to ceasing? If that's truly known and understood by 
me, why should that mean that full enlightenment is inevitable, that in 
a certain amount of time it will ripen in my complete and total 
liberation? How can that understanding be so important?} We can 
practice picking up that theme, applying it, and seeing it clearly for 
ourselves. It's not simply a collection of words that we hear, \emph{Oh 
yeah, ``All that's subject to arising is subject to ceasing.'' Yeah, I 
know all that.} We can practice exploring it and applying it, moment by 
moment, throughout the day. \emph{Why should that be so liberating? Why 
should that be so significant? Why should that be the change of vision 
that alters my whole way of life, my whole way of seeing who and what I 
am?} This is for all of us to investigate.

When we apply this insight to everything---to what we think, to what we 
feel, to the pleasant experiences, the painful experiences, the 
beautiful, the ugly, the emotionally pleasing, the emotionally 
distressing---it awakens a sense of the nature of experience itself. We 
begin to see that when we aren't judging life in worldly terms---good, 
bad, right, wrong, inside, outside, me, you, beautiful, ugly---we're 
seeing everything in terms of its nature rather than its content or 
whether we like it or we don't like it, whether we call it inside or 
outside. We awaken to that intuition within us that everything internal 
and external, mental and physical, is part of a natural order. It's not 
self. It's not who or what we are. It's not personal. It's not alien. 
It's just Dhamma. It's nature itself. This understanding undercuts the 
way we see ourselves as, ``Me in here, the world out there.'' We no 
longer believe that those perceptions reflect the way things truly are.

This insight seems like nothing, doesn't it? Our ego-centered thinking 
reacts like it's oxygen: \emph{Big deal. There's a lot of oxygen about, 
so what?} But when oxygen is denied us, then it quickly becomes 
apparent how crucial it is. In a similar way, when we apply this 
insight to how we experience our thoughts, feelings, and perceptions, 
then it changes the whole way we hold things. And we can notice that 
change within the heart. \emph{Oh right, this is a changing thing. I 
can call it good or bad. I can call it success or failure. But its 
primary quality is that it's changing. It came into being and now it's 
ending.} \emph{That's the most important thing about it.} We notice 
that shift in the heart, that shift that happens within us when we 
apply Kondañña's insight. It changes everything because we're no 
longer obsessing---fixating on the content of experience. We're 
becoming aware of the process of experience itself.

That's fundamentally what insight, \emph{vipassanā}, is about. 
Vipassanā is insight into the nature of experience itself---that 
moment of clear seeing when there's a direct awareness of how 
experience works, what experience is. And the result of that is 
liberation. At that moment the heart knows, \emph{This is merely a 
pattern of nature, a coming, a going---changing. That's all it is. 
That's all it can be. Nothing to get excited about. Nothing to get 
alarmed about.} Right there is the moment of freedom in seeing clearly 
that the heart is unburdened. There is an unentangled knowing, at least 
to some small degree. That liberating quality is right here, in the 
midst of everyday activity, mundane thoughts, and feelings. Just like 
Kondañña, it is possible for us to access this liberating insight 
right here and right now.

