\mychapter{Remembering to Loosen Up}{Ajahn Amaro}{September 2008}

By reflecting on the theme of kindness and establishing the quality of 
\emph{mettā}, we gain insight into how mettā is developed. One of the 
not so obvious places where we create qualities of unkindness is in the 
way that we relate to our own attitudes, opinions, and pet projects 
through self-identification. \emph{My responsibility, my job, my role.} 
Many of these aspects of our lives slip into existence and claim a 
substantial reality without our noticing. \emph{Of course this is what 
I am. This is my job. I'm the water monk; I'm the co-abbot; I'm the 
head cook; I'm the driver. This is what I am.} The mind takes hold of 
these particular attitudes and creates a false substantiality around 
them. The more solid they become, the more we create the causes of 
friction.

If we want to sharpen a blade, we don't get a soft grindstone made of 
gelatin. We get a good, hard grindstone for the blade to rub against, 
something that's solid, unyielding and abrasive. The more the mind 
hardens around opinions, then the more we buy into them, creating false 
solidities and divisions between the apparent me and the apparent 
world. The more solid and unyielding these attitudes and opinions 
become, then, just like an abrasive grindstone, the more friction there 
is and the more sparks that will fly.

So developing the quality of \emph{mettā}---in terms of non-contention 
in the way that we behave and relate to others---is an important 
dimension to notice. We don't often think of this in terms of 
\emph{mettā} practice, but it's helpful to bring a genuine 
attentiveness to the kind of fixedness, the sense of territoriality 
that the mind can have, or the fixedness of views that identifies with 
a particular role, position, or responsibility. \emph{This is my 
position. This is my job. You do your thing, and I'll do mine.}

I remember one of the nuns from Chithurst Monastery talking about her 
pre-nun life in the kitchen at Amaravati. She said, ``It's quite 
incredible---the opinions about the way to cut a carrot.'' She spent 
three years as an \emph{anagārikā} at Amaravati, which has a kitchen 
about three or four times the size of the kitchen here at 
Abhayagiri---and at least ten times the number of opinions. People were 
almost coming to blows on occasions and stomping out of the house. 
``Those carrots have been cut wrong---it's completely out of order, 
inappropriate, and contrary to the Dhamma to cut carrots like that!''

So we might not notice in the course of a day the 10,000 ways that the 
mind creates a sense of territory, priority, or position and seeks ways 
to establish them. We can be very polite, keep the precepts, and do 
things in the appropriate way, but at the same time sending out signals 
that are based on identifying with a particular position. \emph{I'm 
right. This is my job. You don't matter. What I'm doing is important. 
My views are correct. What you do is totally insignificant.} It's all 
the habits of self creating the \emph{I} and \emph{my.} We can take the 
ordinary, utterly innocent activities of our monastic lives---helping 
out with the work tasks, washing the dishes, putting things away in our 
cupboards, or whatever it might be---and use them to feed those 
I-creating, my-creating habits. It's useful to bring attention to that, 
to see where we create a fixedness of views and a false solidity. Then 
we can challenge our own opinions, habits, and preferences by training 
the heart into not buying into them, not going along with that kind of 
subtle grasping. And remember to loosen up! This is a genuine act of 
kindness. It's a kindness to ourselves not to create that sort of 
stress and fearfulness within our own hearts. It's also a great 
kindness to the people around us; they are much more able to harmonize 
with us when we're able to harmonize with them. There's less of the 
alienation that comes from clinging to self-view and self-creating 
habits. We are no longer like a hard unyielding grindstone.

There are many different dimensions to developing \emph{mettā}. We can 
see how much of a difference it makes when we notice within ourselves a 
hardening of the heart or a clinging to an opinion, belief, or 
preference, and then we change direction by relaxing around our views 
and loosening up our identification with those opinions and 
preferences. We experience for ourselves the blessings that come from 
the more easeful and pleasant world we live in because we're not vying 
for positions, judging each other, fearing being judged, or positioning 
ourselves against this or that person. Notice how delightful and 
wonderful it is to have none of that being created. We realize, 
\emph{Oh look at that. If I don't create it, then it's not there. What 
an amazing surprise!} So much of the stress and difficulty of our 
personal worlds is generated from our own \emph{cittas}, our own minds 
and hearts. And so, if we ourselves stop creating that division and 
stress---that friction and heat from the grindstone---then we find the 
citta becomes cool, because the division and tension simply aren't 
there.

