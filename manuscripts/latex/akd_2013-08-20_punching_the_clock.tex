\mychapter{Punching the Clock}{Ajahn Karuṇadhammo}{August 2013}

It's easy to slip into a preprogrammed mode, flowing through the day on 
automatic pilot, especially when we're organizing our tasks for the day 
and engaging during the work period. Whether it's in the office, the 
kitchen, or out in the forest, we usually know what our tasks are and 
can click into autopilot mode quite easily. We can become completely 
absorbed in our activities---identified with them, or with being the 
doer of particular tasks. This is when mindfulness, clear 
comprehension, attention, and alertness can all lapse. The goal of the 
practice and of the contemplative lifestyle in general isn't simply to 
go to work in the morning, punch the time clock, and mentally check 
out. Nor is it helpful to take the view that when the work is out of 
the way, then it's time to practice. Instead, as much as possible, we 
are looking for a continuity of mindfulness and clarity throughout all 
of our activities and that means bringing to mind a suitable object for 
focusing one's attention---that's what mindfulness is. We choose a 
wholesome object for this, one that will lead to skillful states of 
mind. Sometimes that's difficult, especially if we are doing work that 
requires us to be mentally focused. I think office work is among the 
most challenging in this regard, but it can be just as challenging out 
in the field if we find ourselves needing to think and plan for things. 
That's understandable and to be expected, but we try and attend to our 
duties with a sense of lightness and clarity. When tough moments arise, 
particularly if we are doing physical work, then we put effort into 
recollecting what we are doing in the moment. We can know our physical 
sensations clearly, and focus on something that will lead to a 
wholesome mind state, rather than a deluded one.

We keep the focus on the body, always here in the present moment. This 
is easier to do during physical work, as we can notice the very simple, 
basic, and uncomplicated position of our bodies, and remain attentive 
to the movements as they occur. This is described in the 
Satipaṭṭhāna Sutta: knowing whether we are sitting, standing, 
walking, or lying down. Noticing the basic posture of our bodies is 
something we can do right now. As we walk, we can pay closer attention 
to whether we're looking forward, looking back, extending or 
contracting our arms. We can also see movement from one position to 
another, squatting, kneeling, hammering, typing on a computer, or even 
turning to answer the phone. It's simply a practice of giving attention 
to the basic movements of the body as a way of grounding oneself in 
mindfulness. This way of practicing has the potential to lead to a 
sense of settledness in the mind and the body. It helps keep us from 
becoming completely, 100 percent swept away with what we are doing, and 
allows us to maintain a bit of objectivity moment by moment. We do this 
by building up a momentum of pauses, stopping every now and again and 
feeling the presence of the body. In the long run, if we are able to 
successfully practice with continuity and consistency, when the time 
has come to set down the task at hand, mindfulness and clarity come 
up---replacing the old habits of exhaustion, confusion, and the need to 
distract oneself in order to replenish drained energy. When we learn to 
sustain mindfulness for longer periods of time, we can have a sense of 
awareness and comprehension about what we are doing in the context of 
our Dhamma practice. We do this by taking the time to reflect on our 
experiences as they occur within this context. All of this arises from 
developing the skill of mindfully observing the body without allowing 
ourselves to drift off into our automatic, unconscious daily routines.

