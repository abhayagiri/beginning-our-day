\mychapter{Letting Go and Picking Up}{Ajahn Karuṇadhammo}{June 2013}

Many of the practices we hear and read about in our tradition are 
focused on the process of letting go---how we let go of our habits and 
tendencies, as well as objects of mind. We do this to dwell in and 
experience the pure state of awareness that comes from not grasping or 
holding onto anything. This is the ultimate practice on the path: 
letting go of negative tendencies and at the end of the practice, a 
complete release, a complete letting go of everything, including the 
path that has taken us there.

I think it is also true that a substantial amount of the practice---if 
not all of the practice before completely letting go---also involves 
picking up. We learn this skill by discerning what it is we need to let 
go of and what it is we need to pick up and engage with. All the 
obstructions and hindrances to meditation---negative thoughts, 
reactions out of anger, greed, or confusion---are habits that, with 
mindfulness we see arise in our minds and then let go of to the best of 
our abilities.

In the same process so much of the path is involved with skillfully 
picking up different positive qualities---the habits of generosity, 
renunciation, energy, patience, truthfulness, determination; the 
practices of virtue; the sublime states of mind: loving-kindness, 
compassion, sympathetic joy, equanimity---all of these are using an 
object that we are picking up, and at least temporarily, holding onto. 
Our learning consists of understanding which different types of mind 
states we can access, which ones we want to let go of, and which ones 
we want to develop further. Letting go and picking up often go 
hand-in-hand so that as we begin to recognize a difficult mind state, 
such as anger arising, we realize it's something we want to let go of 
while at the same time---even simultaneously---we recognize something 
like patient endurance that needs to be picked up. For example, if a 
difficult situation arises, such as a when a challenging person comes 
into our life, we can let go as we recognize the tendency to react out 
of anger while at the same time, we can pick up, develop, and nurture 
patience, the determination not to react, and kindness for the other 
person and for ourselves.

This morning I noticed one of the people in the meditation hall gently 
removing a spider from the room. How many people in the world will go 
to great pains to take a spider out of a room and set it free? Most of 
the time it's a quick stomping of the foot and that's it. This is 
something to pick up on, this type of sensitivity and caring for even 
the smallest of creatures. We can think of it as picking up a skillful 
mind habit. All of these actions have an impact on the mind and the 
heart. In our daily lives, we can explore what it is that's helpful for 
us to pick up, nurture, and engage with, as well as what habits we can 
develop that will support us in our practice.

