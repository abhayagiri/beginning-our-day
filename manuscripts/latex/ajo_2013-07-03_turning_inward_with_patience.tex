\mychapter{Turning Inward With Patience}{Ajahn Jotipālo}{July 2013}

I have been listening to a few of Bhikkhu Bodhi talks on 
loving-kindness and was surprised when he mentioned that the word 
\emph{khanti}, patience, is very closely related to the word 
\emph{mettā}, loving-kindness. I hadn't recognized that before.

He explained that in practice situations, the use of mettā is more of 
an external, outgoing energy, where we make a genuine wish for other 
people to be happy. Traditionally, the first stage of mettā practice 
is focused on wishing happiness for ourselves, which comes with a sense 
of generating kindness and gentleness that goes inward rather than out. 
But a good deal of the practice involves a desire for other beings to 
be happy. In this way, it is about applying an externally-directed 
energy.

I have given a few talks on loving-kindness. When I do, I often receive 
questions from people concerned with external circumstances, such as, 
``It's so painful to be with this person,'' or ``When I'm in this 
situation it's really difficult. How do you deal with that?'' Most of 
the questions are directed toward the practice of loving-kindness as a 
method for sending mettā outward. But we can also turn inward rather 
than outward. This is where Bhikkhu Bodhi says patience comes in. We 
can learn to turn toward the pain we feel---toward the \emph{dukkha} we 
are experiencing in these difficult circumstances---and to hold that 
dukkha with a quality of patience.

Ajahn Sucitto once said that we often think of patience as waiting for 
change. \emph{I will endure this situation, gritting my teeth, until it 
changes.} Certainly we might want a painful situation to change, but 
with true patience, according to Ajahn Sucitto, it's more like 
thinking, \emph{I will be with this situation, period.} In other words, 
there's no expectation that the situation will change or get better.

By learning to turn toward our suffering and simply be with it, we are 
staying at the level of feeling. We are not getting into the story, the 
proliferation, or creating a self around it. If someone says something 
to us and we become angry or feel uncomfortable, instead of going 
outward, as we typically do with mettā, we can go inward. So when we 
feel pain in a situation, we can first recognize it. Then, we move 
toward the painful feeling and explore it. If we can refrain from 
getting into the story behind the feeling, it will be that much easier 
to experience the feeling without wanting to change it. It's merely a 
physical sensation or a mental perception and we do not need to add 
anything to it or try to make it go away. When we stay with a painful 
feeling in this way, we are experiencing khanti, true patience.

