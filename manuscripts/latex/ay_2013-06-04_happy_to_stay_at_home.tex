\mychapter{Happy to Stay at Home}{Ajahn Yatiko}{June 2013}

One of the most important things for us to be doing here---as either 
monastics or as visiting laypeople living like monastics---is to 
develop our formal meditation practice. A key to this is learning how 
to delight in meditation---the freedom that comes from simply sitting, 
not becoming anything, resting in a state of mind that is able to put 
things down.

Yesterday I was reflecting on the word \emph{concentration}, which is 
the most common translation for \emph{samādhi}. I think that's a poor 
translation because \emph{concentration} has a strong sense of coercing 
the mind, forcing it to be a certain way. My understanding of samādhi 
is that it is a freeing experience, rather than one which squeezes the 
mind into a frozen or held state of being. It is much more an 
experience that allows us to let go of unwholesome states of mind, 
especially those we are obsessed with or attached to. What remains is a 
mind that is relaxed, peaceful, and stable. It's not going out to 
search for happiness, because it is quite happy to stay where it is, at 
home.

We often use the words \emph{cultivating concentration}, which is an 
important expression, but it can be misleading. Quite simply, it means 
developing the ability to put things down. We can recognize that the 
things which we attach ourselves to are endless, and that this 
attachment results in \emph{bhavataṇhā}, a sense of becoming---the 
incessant inclination to exist, plan, create an identity, and attach to 
some idea of ourselves as some thing. We're caught up in this becoming 
like an animal in a trap. Either we see that it's going to lead to 
suffering or we don't; but either way, if there's bhavataṇhā, 
suffering will be the result. So one of the main tasks in our 
meditation practice is to learn how to put down this becoming. Once we 
do, we will begin to experience a profound sense of freedom. As 
Sāriputta once commented to Ānanda, ``Nibbana is the cessation of 
becoming.''

So we need to use our time here to not only develop service and 
generosity---which is so important to communal living and harmony---but 
also to pay attention to our formal meditation practice---sitting and 
walking---and the ability to put things down.

