\mychapter{Not Taking Refuge in the Weather}{Ajahn Jotipālo}{December 
2012}

On mornings like this---when it's pouring down rain, when it's not 
comfortably warm, and we have been assigned a job working outside which 
will be wet and inconvenient---in this situation, the mind may rebel or 
complain.

It was quite a cold morning and pouring down rain during the first work 
meeting I attended at Abhayagiri. Everybody was going to be working 
outside because of last minute preparations for the winter retreat. 
Before the work began we were gathered together listening to Ajahn 
Pasanno's morning reflection. The mood in the room was pretty glum when 
Ajahn said to us, ``Well, it's a good thing we don't take refuge in the 
weather.'' When he said that, my mood immediately changed. We were only 
going to get wet, that's all. We had places to dry our clothes and we 
would be outside for only a couple hours. Plus it was going to be 
really good work, a substantial service to the monastery.

Ajahn Pasanno's comment has stuck with me over time, and has encouraged 
me to question myself, \emph{What am I taking refuge in? Is it the work 
I'm doing? The relationships I'm cultivating? Am I taking refuge in 
wanting to feel good and not being inconvenienced in body or mind?} 
After reflecting like that it can be easy to set aside my aversion to 
rain. On days when there are tasks I don't want to do, I can look at my 
perceptions and ask myself, \emph{What am I taking refuge in? What 
assumptions am I making? How can I see this situation from a different 
perspective so that I might incline my mind towards a brighter state?} 
When I reflect in this way, the work period or the task I'm attending 
to can be quite enjoyable.

