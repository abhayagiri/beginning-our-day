\mychapter{An Internal Articulation of Dhamma}{Luang Por Pasanno}{July 
2012}

I've been noticing something in my speech that I find irritating. It's 
my use of filler words such as ``like,'' ``sort of,'' ``um.'' I think 
to myself, \emph{Am I getting more and more inarticulate and hopeless 
with age?} Then I listen to other people, and it's pretty much the 
same. To break that habit, we can learn to internally compose what we 
want to say before we speak, rather than fumbling around with ``um,'' 
``ah,'' ``sort of,'' ``like,'' ``er.''

Even worse is the tendency to use speech that is imprecise. This is 
particularly problematic when it comes to speaking about the Dhamma. 
Being imprecise about the Dhamma does not benefit the Dhamma. The 
Dhamma itself is extremely clear, and we should reflect that clarity in 
our speech. Again, one way to do that is by focusing internally on what 
we're about to say before we say it. Not only will this help us compose 
our thoughts so they are clear, it will also help us to uncover any 
defects that exist in our understanding.

We can encourage ourselves or even encourage each other---if people are 
open to it---to highlight our speech patterns as a tool for clarifying 
our own thoughts and for seeing how we express and communicate our 
ideas to others. The benefit is that internal clarity arises as well as 
the internal articulation of Dhamma.

