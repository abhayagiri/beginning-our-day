\mychaptertoc{Mindfulness With Moods and Defilements}
{Mindfulness With Moods\\and Defilements}
{Ajahn Yatiko}{June 2013}

Ajahn Chah once said that if we have three seconds without mindfulness, 
it's like three seconds of madness. Without mindfulness we can get lost 
in moods---happiness, despair, depression, elation, whatever. 
Mindfulness is not about getting rid of our moods. It's about being 
able to observe them clearly, to step back from them and recognize, 
\emph{This is simply what I'm experiencing}. It's being able to see 
moods as a flow, like clouds moving through the sky.

During group meditation, it can appear as if nobody is experiencing a 
mood except us; everyone else is sitting there with faces that seem to 
look bored, while we're trying to be with some intense emotion. This is 
an experience some of us have. Others of us don't have a lot of 
emotional content or aren't really aware of it. Some people can be very 
much in their heads and rational, absorbed in thought, analysis or 
speculation. Others can be absorbed in some mood, positive or negative. 
But mindfulness is that which is in the background containing all of 
those experiences.

That's what we want to cultivate---that capacity to observe whatever it 
is we're experiencing. To be able to watch our experiences, moods, 
biases, rationalizations, and justifications is one of the most 
important skills that we can develop as practitioners. This process is 
connected with recognizing and accepting who we are. We're all doing 
our best in a monastery. We don't have to be anything we're not.

On the other hand, we also need to question ourselves about how a 
defilement manifests and to recognize when a defilement has arisen in 
the mind. When a defilement \emph{has} arisen, it's unhelpful to 
rationalize or justify it. The defilements are not going to disappear 
on their own. They are something we look at, recognize and see clearly. 
And we try to understand how these defilements are leading to 
suffering. Attaching to our desires, justifying, rationalizing, 
insisting on, and following our defilements all lead to suffering. And 
mindfulness is that which leads us out of suffering.

So whether we're investigating our moods or working with the 
defilements, mindfulness is key. It's what replaces madness with sanity.

