\mychapter{The Simplicity of Buddho}{Luang Por Pasanno}{June 2005}

These last two weeks were supposed to be a time of retreat for me, but 
it didn't quite end up that way. I had to catch up on work and help out 
with some of the construction. While I was working, something came to 
mind that I found very fruitful:

To work with the busyness---the activity that I was involved in---I 
returned to the very simple practice of repeating the word 
``\emph{buddho}.'' It's a basic, foundational practice that is used 
throughout Thailand, particularly in the northeast. I used that short 
mantra in conjunction with the breath. On the in-breath, ``Bud,'' on 
the out-breath, ``dho.'' Buddho is the name of the Buddha. It means to 
awaken or to be awake.

The mind is so easily attracted to proliferation and embellishment: 
\emph{Give me a story, a drama, something to bite into---give me 
anything, so long as I can play the starring role.} When we recite 
Buddho, we are not feeding, encouraging or supporting the embellishment 
and dramatization of experience.

On the in-breath, ``bud,'' on the out-breath, ``dho,''---it's simple 
and helpful. Because it is the name of the Buddha, it's a reminder of 
bringing the presence of the Buddha into each in-breath and each 
out-breath. It's bringing the recollection of the Buddha---both as the 
Teacher and as the archetype of the enlightened being---into the heart, 
the mind and consciousness.

Buddho is a powerful presence because it cuts through our tendency to 
fill the mind with worries, concerns, desires, fantasies, 
proliferations, ideas, ideals, views and opinions. All this clutter is 
the antithesis of the enlightened being. We can remember that by using 
this simple word. On the in-breath, ``Bud.'' On the out-breath, ``dho.''

