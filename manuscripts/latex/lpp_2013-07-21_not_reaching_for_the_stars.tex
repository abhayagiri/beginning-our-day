\mychapter{Not Reaching for the Stars}{Luang Por Pasanno}{July 2013}

In an ideal world, a summer day like this would be 80 degrees with a 
gentle breeze. In reality, it's 108 degrees, and the air is not moving 
at all. How do we handle the contrast between what's ideal and what's 
real? We can moan and whinge, or retreat into fantasy and desire, but 
from the Buddha's perspective, it's always about establishing a sense 
of clarity and equanimity within the reality of the present moment.

With something like the weather, it's fairly easy for us to see our 
unrealistic hopes clearly and accept them for what they are. But in 
other areas, it's more difficult. For instance, living as we do in a 
spiritual community, we may hold to a utopian notion that everyone here 
should be mindful, peaceful, contented, and harmonious. And it's 
true---we should be. But we can keep in mind that this ``should'' is 
based on an ideal. The reality is oftentimes quite different. So it's 
necessary to learn how to navigate those times when we, or others, 
aren't living up to the elevated principles that we hold. To do that, 
we can reflect on whether it's realistic to expect everybody to live up 
to our lofty notions simply because these notions are what we would 
like or prefer. We can reflect on how well we personally adapt and 
respond to the realities of our existence and see that it is often 
difficult to live up to our \emph{own} ideas of perfection.

This doesn't mean we should throw out all the ideals we have and tell 
ourselves to forget it. Holding to wholesome ideals can be a skillful 
thing to do. But we need to be very cautious of measuring ourselves 
against those high values all the time. Doing that exacerbates 
dissatisfaction, discontent, and a sense of suffering.

In former times, navigation at sea depended on frequently checking the 
stars with a sextant to set and maintain a realistic course of 
direction. It wasn't about trying to reach the stars! In a similar way, 
as practitioners we merely need to set ourselves on a wholesome, 
skillful course; we don't need to constantly frustrate ourselves or 
others by trying to comply with unrealistic notions of perfection. 
Usually our ideals---even wholesome ones---are filled with wishful 
thinking and fantasy. So it's important to ground our elevated 
standards in reality. By doing that, we can find a realistic course of 
action that supports our living skillfully. Applying the Dhamma in our 
day-to-day lives, we can learn to live with each other and with 
ourselves in a beneficial, wholesome way.

