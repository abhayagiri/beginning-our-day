\mychapter{If It Doesn't Die, Make It Good}{Ajahn Ñāṇiko}{November 
2013}

It's important to reflect on the habits we have, to ask ourselves, 
\emph{Does this habit help to make the mind more peaceful, or does it 
tend to make the mind more agitated?} The Thai \emph{kruba ajahns} 
often use the phrase \emph{plien nisai}, which means to change our 
habits. In the Thai Forest Tradition there is a strong emphasis on 
changing one's unskillful habits. Our chanting includes the \emph{Ten 
Subjects for Frequent Recollection by Monastics}, where we recite, ``I 
will strive to abandon my former habits. This should be reflected upon 
again and again by one who has gone forth.''

Habits run deep and they're very difficult to change. But if we have a 
habit and notice that it's causing us to be agitated, or if we are 
habitually irritated by something, we can use the Buddha's teachings 
any time of the day to change our direction.

I remember Ajahn Jayasāro telling a group of us in Thailand, ``If you 
are keeping \emph{sīla}, then no matter what lifestyle you are 
leading---even if you are not meditating much---all day is an 
opportunity to be letting go of defilement and training the mind.'' So 
whether we are working, meditating, or whatever, there's always an 
opening for us to change our obstructive habits. It takes mindfulness 
to see our habits, and it takes effort and patience to change those 
habits, but it is possible---and vital---to do.

As Ajahn Chah said: ``If it isn't good, let it die. If it doesn't die, 
make it good.'' Sometimes we have a habit we can't let go of. We try to 
let it die, we don't feed it, but it's so strong that we can't let go. 
In cases like that, we need to ``make it good.'' We need to steer it in 
a different direction. As we go about our day we can attend to our 
habits and little by little steer these habits in the direction of the 
skillful and wholesome.

