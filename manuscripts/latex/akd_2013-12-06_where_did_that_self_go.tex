\mychapter{Where Did That Self Go?}{Ajahn Karuṇadhammo}{December 2013}

This morning we chanted the recollection of the ``Thirty Two Parts of 
the Body,'' as well as the Anattalakkhaṇa Sutta, which mentions 
thoughts, perceptions, and feelings. They can both be used to reflect 
on aspects of not-self.

The ``Thirty-Two Parts of the Body'' is a reflection that can be 
utilized to reduce sensual desire when there is a strong attachment to 
the human body as an object of attraction. We can also bring this 
reflection up as a way of dividing up what we usually consider a unit, 
a whole, something to attach a sense of self to. Oftentimes we identify 
with this body as either me or mine, as something I possess or 
something that is actually who I am. The ``Thirty-Two Parts of the 
Body'' contemplation helps us tease the body apart so that we can 
reflect on how this body is just a collection of parts, any one of 
which we would be hard-pressed to say is who I am or ``me.'' We can ask 
ourselves, \emph{What would happen if I lost an arm or a leg or had a 
body part that had to be taken out because of disease?} \emph{If I take 
something away from this body then what is left is different. So is 
this body any less me than it was before? Or is it a different me? Or 
perhaps it is not me at all?} That is one way that we can reflect on 
the ``Thirty-Two Parts of the Body''.

Our friend Iris's cancer has moved to her brain. Although she is having 
treatment, her perceptual world seems to have been altered in some 
ways. She perceives the world around her differently than she normally 
did or how we normally do. When we look at things in a different way or 
see different aspects or angles of physical and mental phenomena, it 
can help us see how much stock we put into the continuity of the 
perceptual realm. As the Anattalakkhaṇa Sutta suggests, we can see 
how we attach to our perceptions a sense of me, myself, and who I am. 
This continuity of thoughts, perceptions, and feelings weaves a web of 
self and sometimes gets changed or altered or diminished, \emph{Where 
did that ``me'' go to? Where did that self go? Is it a different one? 
Is it the same one but looked at from a different angle? Or is it 
possibly just another perception that I have weaved together to give a 
sense of security and stability?}

It is good to reflect on the change and instability of what we normally 
call me or mine, both in the body and in the mind. We can take this up 
as a contemplation and examination, and ask ourselves, \emph{Am I 
really any of these?}

