\mychapter{Following Rules: What's the Point?}{Ajahn Yatiko}{October 
2012}

Yesterday in \emph{Vinaya} class, we were talking about how we relate 
to rules. As monastics, we have so many rules that are a part of our 
lifestyle. There are countless rules that define the way we live and 
the way we do things. It's interesting to see how Western monks like us 
often relate to the rules in a fearful way. There's a sense of all 
these different rules in place, and we're trying to control and force 
ourselves to live within these different constraints. It can be 
daunting.

We've heard from various teachers that following rules is not an end in 
itself. Ajahn Sumedho speaks about Ajahn Chah as being a monk who was 
very scrupulous with the Vinaya rules, but who didn't look or act like 
a limited person. Certainly when I listen to Ajahn Chah's Dhamma 
teachings, they have a vitality that's incredibly fresh and vibrant. 
This seems related to how he followed the monastic discipline and rules 
while not being attached to those rules. For ourselves, however, we 
need to be careful and discerning in this particular area of 
non-attachment.

We often hear the phrase, ``Don't get attached to rules.'' I remember 
when I first came across Buddhism, I found teachers who weren't 
``attached to rules'' and sometimes ended up making a real mess out of 
their lives. It seemed their disciples suffered in many ways as well, 
because of this confused relationship with following a moral 
discipline. As a result of witnessing that, whenever I hear the phrase, 
``Don't be attached to rules,'' a red flag goes up for me, as well as a 
bit of fear. It's important to recognize the potential for 
self-deception or self-delusion when we take on this attitude and say 
to ourselves, \emph{I'm not going to be attached to the rules.} We want 
to be cautious about this kind of attitude, because it can easily slip 
into following or not following some moral guideline simply based on 
our likes or dislikes. We can then throw out the rules we don't like 
anytime we feel they're getting in the way of satisfying a particular 
craving we have or avoiding some aversion. We go ahead and do as we 
please, regardless of the consequences it may have for us or others, 
based on the excuse, \emph{I'm not attached to these rules.}

At the same time, it's important to remember that we're here to be free 
from suffering. Although these guidelines help us, we're not here to 
live by a bunch of rules that are going to force us into conforming, 
keeping us nice and safe, or making us so bland that we don't have any 
problems. That's not the Buddha's path to freedom. The Buddha's path is 
to find contentment within limitations. We have limitations, forms, and 
practices that serve as containers: the rules and standards of the 
Vinaya. These aren't conventions that we have to feel limited or 
constrained by. We can learn to relax around them, finding a sense of 
inner contentment as we follow them and keep them in mind.

We can start by approaching the present moment with a mindset of 
self-acceptance, freedom, and contentment---a paradigm with which we 
don't have to be defined by rules or anything else. From that mindset, 
we can see that all these rules we're living by are simply conventions. 
They don't have any ultimate reality that has to define us. If we can 
see that, contemplate that, and have a sense of expansiveness and 
openness around these conventions, then perhaps we can experience 
contentment within limitation. We can be free of the sense of 
constraint while living within constraint. We don't have to be held 
down by these rules and structures, because a feeling of real freedom 
is present within us. And in that freedom, quite spontaneously, there 
can be a sense for us that, \emph{It's not a problem to keep to these 
rules}. And we might think, \emph{This is quite all right. I'm very 
happy eating one meal a day. I don't mind wearing these robes like 
everyone else here, and being celibate is beneficial for my practice 
and my mind.}

This mindset is quite different from feeling that we need to tightly 
control our behavior and constantly look around to see if anyone has 
spotted us making a mistake. When we have that sort of attitude, it can 
feel like there's no room to breathe. We might be able to live like 
this for a while, but I'm not convinced it's sustainable. It's a 
balancing act, and it can be quite tricky. We need to learn how to 
reflect on all this in a skillful way, recognizing the capacity for 
self-deception in the area of non-attachment to rules. Then again, we 
need to recognize that we haven't taken up the monastic life merely to 
perfect a litany of codes and standards. It's not about standards. 
That's not the point. It's about learning to find contentment within 
the context of conventions.

There may be a danger that what I've said this morning will be 
misunderstood. Do not think I'm suggesting that the rules should be 
tossed out or not respected. I'm simply offering these reflections for 
contemplation. How can we loosen our tight grip on the rules, while at 
the same time continuing to follow them, understanding that they are an 
essential part of the monastic path? Contemplating in this way takes 
intelligence and ingenuity---plus circumspection, to ensure that our 
reflections bring positive results---results that enrich and enliven 
our practice.

