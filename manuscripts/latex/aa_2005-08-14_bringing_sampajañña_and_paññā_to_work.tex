\mychaptertoc{Bringing Sampajañña and Paññā to Work}
{Bringing Sampajañña\\and Paññā to Work}
{Ajahn Amaro}{August 2005}

We have these community work days to create opportunities for the 
broader community to help contribute to the material fabric of the 
monastery and to support the life and energy of the Saṅgha. Along 
with emphasizing that, monasteries provide a context for individual 
spiritual development and cultivation. Our teacher Ajahn Chah strongly 
emphasized the essential quality of Saṅgha or community---learning 
how to collaborate and fit in with each other. So when we have these 
work days together, it is very much in the spirit of developing 
Saṅgha---cultivating a capacity to put forth effort and energy in a 
collaborative way. It requires a lot of attention to do it skillfully.

When we talk about \emph{sati}, mindfulness, we often think about it in 
terms of being careful with the task or job we're doing. However, along 
with mindfulness, the other elements of clear 
comprehension---\emph{sampajañña}---and wisdom---\emph{pañña}---are 
crucial. Clear comprehension means being aware of the context of the 
activity in which we are engaged, for instance, being mindful of the 
people with whom we are working. An example of this would be making 
sure the sharp end of the mattock we are swinging isn't about to go 
through the jaw of our work companion, or when we are turning around 
with a running chain saw in our hands, making sure there isn't a soft 
fleshy object immediately behind us. We want to avoid ``The Redwood 
Valley Chainsaw Massacre,'' so we attend to our physical surroundings 
and also recognize that we are working with others.

Another quality we can be aware of is sharing. We are looking to do our 
part of the work and to include others as well. For example, sometimes 
we might become absorbed in a job that is really enjoyable, and so we 
completely forget or don't notice that there is another person in the 
group, standing there feeling like a spare part, waiting for us. Or we 
might do the opposite. It could be really hot outside, but other people 
seem to be enjoying the work, so we let them do it all while we sit and 
watch. With clear comprehension, we attend to the energetic feeling of 
the people we are working with. We pay attention to how we're feeling, 
how everyone else is faring, and we notice that, if it's hot and dry, 
maybe the people working need a drink of water.

With sati-sampajañña, mindfulness combined with clear comprehension, 
there's an overall attending to and caring for not only our own needs, 
but the needs of the people around us. Our minds are attuned to making 
the time we spend working with others an opportunity to develop that 
collaborative spirit. It's not only an effort to benefit the 
monastery---creating more dwelling spaces, maintaining the finished 
buildings, digging a trench, clearing away brush, sorting out the 
stores, and so forth---it's also an effort to fit in and learn how to 
support others. And it's about letting others support us as well, which 
can sometimes be more difficult than the other way round.

Another element we combine with sati is paññā, wisdom. Paññā is 
recognizing that all of this is impermanent, not-self, and 
unsatisfactory. So when we're putting something together or attending 
to some project, there's an overarching recognition that eventually 
it's all going to fall apart and disperse. So there can be no inherent 
satisfaction in the projects we undertake if that satisfaction is based 
on a sense of permanence with these projects. And with this insight 
into impermanence we can also recognize that these activities we are 
engaging in are not truly ours. This recognition brings a lightness to 
the tasks we engage in. When we are doing our work projects, we are 
developing expedient and useful skills. But there's a relinquishing of 
what we do as well. There is not an ownership or claiming of the things 
we bring into existence.

Another teaching that Ajahn Chah would stress over and over was to 
recognize that ``the cup is already broken.'' The objects of the 
material world have within them an impermanent nature. They are 
destined to fall apart and disintegrate. The building we are currently 
constructing is already broken; it will fall apart or be demolished 
someday, so the efforts we're making are not toward a permanent end. We 
still continue to do what we do carefully, attentively, as well as we 
can, but there's a lightness, a spaciousness to the way it's held.

If we realize that the cup is already broken, then when the day comes 
that it does indeed break, our heart doesn't break with it. The cup 
physically smashes, and we realize, \emph{Oh, this is its }anicca. 
That's always been a part of it. Its anicca is simply ripening at this 
moment. Nothing has gone wrong; nothing bad has happened.* We are ready 
for that. There's no sense of loss or diminution in that regard.

When we hold things with wisdom in this way, then there's a 
heedfulness, a fullness that we experience with every act we do. We're 
not trying to make it be more than it actually is, and we're not adding 
anything onto it. So there's purity, simplicity, and a good-heartedness 
we can bring to everything we do.

