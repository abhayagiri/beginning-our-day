\mychapter{Choosing the Pāramīs}{Luang Por Pasanno}{July 2013}

Several nights ago I gave a talk on the ten \emph{pāramīs}, qualities 
that are helpful for cultivation and development: generosity, virtue, 
renunciation, wisdom, effort, truthfulness, resolution, 
loving-kindness, patience, and equanimity. Each is helpful to bring up 
as an alternative to a particular difficult or obstructive state 
present in the mind.

As a preliminary step, we can ask ourselves questions that pertain to 
these pāramīs in a general way. For example, \emph{How do I want to 
orient myself and my practice? What principles of training do I feel 
need attending to?} When we think in terms of the pāramīs, the 
underlying connecting thread is making a commitment to or leaning 
toward what is wholesome and skillful, uplifting and brightening. We 
don't need to put a name on it or have a label for it. We can orient 
ourselves with these fundamental principles of wholesomeness and 
skillfulness and encourage ourselves toward these brightening qualities 
of mind. In a very elemental way, the function of these principles of 
goodness is to alleviate a tendency towards suffering.

When we're engaged in work or duties or have to deal with different 
personalities in our living situations, we have an opportunity to bring 
to mind these principles of training and to incline ourselves toward 
that which is wholesome and skillful. It's so easy to rationalize 
negativity, aversion, frustration, and a sense of, \emph{Oh, woe is 
me}, and really miss the point that we always have an option to choose. 
We can make an effort to keep exercising that option of choosing and 
inclining ourselves toward those principles of goodness. And the 
pāramīs are ways of articulating those principles as specific 
qualities we can develop. We can also recognize the underlying impulse 
to opt for these principles of goodness, of wholesomeness, rather than 
resigning ourselves to attitudes of negativity and the things that 
thwart our spiritual aspirations. These are moment-to-moment choices, 
moment-to-moment opportunities for establishing a particular direction 
for our minds and our hearts. Ultimately, this direction will help us 
choose the pāramīs rather than something else.

