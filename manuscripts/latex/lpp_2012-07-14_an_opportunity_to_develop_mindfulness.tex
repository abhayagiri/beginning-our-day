\mychaptertoc{An Opportunity to Develop Mindfulness}
{An Opportunity to\\Develop Mindfulness}
{Luang Por Pasanno}{July 2012}

The Buddha spoke about the different postures for meditation–walking, 
sitting, standing and lying down–basically, every position of the 
body. He also spoke about the ``foundations of mindfulness,'' that is, 
the body, feelings, the mind, and objects of mind. These are all 
supports we can use to develop attention. So we have the opportunity to 
cultivate mindfulness and attention with everything we do. 
Specifically, we can use work and engagement in activity as a way to 
help with that cultivation.

As we develop mindfulness around our work chores, we can see that the 
functions and benefits of mindfulness are more than merely internal. 
For example, mindfulness enhances our ability to act more appropriately 
when in proximity with others who are also working. If we are painting 
next to somebody standing on a ladder, how do we maintain mindfulness 
so we don't knock them over? How do we engage in a conversation while 
wielding a paintbrush without splattering the person next to us? This 
sounds basic, but I've seen ``mindful meditators'' get into all sorts 
of trouble.

We can cultivate attention and mindfulness by anchoring our work 
meditation in the body. One way to do this is to ask ourselves, 
\emph{How am I feeling? Am I relaxed and grounded? Am I in contact with 
things around me? Am I in contact with attention and awareness?}

In terms of other people nearby, we want to stay mindful of the need we 
all have for personal space. It takes a certain degree of empathy to 
understand that other people have their own comfort zones and to 
mindfully account for that. And while not wanting to be encroached 
upon, most people also have a wish to be recognized. This gives us a 
purpose and an opportunity to extend kind attention when relating to 
each other through our actions and speech. During the work periods, we 
can offer each other support and respect as human beings. That's how we 
engender loving-kindness and one of the many ways we cultivate 
mindfulness.

It is important to recognize the practical applications of mindfulness 
and how mindfulness affects our attitudes and perspectives, both in how 
we relate to others and to the world around us. From something very 
simple like being mindful of one breath at a time, one step at a time, 
we can start to develop a foundation in clarity and learning, which 
brings much benefit into our lives.

