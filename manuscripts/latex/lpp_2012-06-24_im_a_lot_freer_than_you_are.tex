\mychapter{I'm a Lot Freer Than You Are}{Luang Por Pasanno}{June 2012}

Yesterday, a group of us went over to the City of Ten Thousand Buddhas 
for a commemoration of Master Xuan Hua's 50th anniversary of coming to 
America. This morning while I was sitting, I remembered something said 
by Doug Powers, a longtime disciple of Master Hua. He had mentioned 
yesterday that Master Hua would both tease and challenge the students 
who came to study with him in the sixties and seventies, because at 
that time there was a strong ethos of people seeking freedom in many 
different ways. Master Hua would say, ``You think you're free, but I'm 
a lot freer than you are.'' Our ways of trying to measure 
freedom---trying to do what we want, when we want, how we want and 
where we want---assumes that these ways will result in experiencing 
freedom. In putting forth that challenge, Master Hua was questioning 
how people can measure freedom when they are still following their 
greed, hatred and delusion. How can they be free when that's the case?

For all of our attempts to be free or peaceful or whatever, we still 
let greed, hatred, delusion, clinging, craving and confusion, not only 
follow us along, but push us from behind, conditioning how we react, 
respond, and interact with the world around us. If we spend our time 
scattered, impulsive, and compulsive, and then sit in meditation a few 
times a day hoping to be peaceful or free, what we find is that the 
whole juggernaut of our daily actions dogs us around, and we get 
completely swamped with the flow of our habits and reactions in the 
mind.

By using the Buddha's path of practice, bringing mindfulness into our 
day-to-day activities, we encourage a certain clarity, precision, and 
circumspection with the things that we do. By doing this, we bring some 
measure of freedom from greed, hatred, and delusion into the little 
activities we engage in everyday. For example, when we clean up after 
the meal or after tea time, what is the first impulse? Sometimes it's 
simply to get up and move on to the next thing, engage in conversation 
or whatever. We get swept up in our habits of mind. We need to create a 
container for our habits, so that we can see them more clearly and 
understand how we can circumvent or relinquish those tendencies. The 
tendencies of our trying to get what we want, following our views and 
opinions, reacting out of irritation or aversion, all of these things 
are lacking in any solidity. But as long as we keep reinforcing them, 
believing in them and following them, then these unwholesome habits 
grow and sustain themselves. So we need to learn how to use the skills 
of mindfulness and wisdom to put a wedge into the wheel of our habitual 
unskillful tendencies, to slow them down, see them clearly, and 
recognize that we have options. We have a choice and we can either 
relinquish those habits or go against their grain.

