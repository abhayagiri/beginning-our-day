\mychapter{Escaping the Infinite Rut}{Ajahn Amaro}{August 2008}

We begin another day, reflecting on the tasks that we all have, the 
lists of things to do, the never-ending need to attend to the kitchen, 
the various work projects, guest-monk duties, construction, and so 
forth. There's a sense of continuity, of things carrying on from where 
we were before, going on and on. Ajahn Chah once came up with a 
wonderful simile: ``\emph{Saṃsāra} is like the ruts left behind the 
wheels of a cart. As the wheels turn, the trail that's left behind them 
goes on and on.'' There's a sense of repetition, a continuous extension 
through time. But the wheel itself is very stable, it simply pivots 
around its axle. Even though the wheel is finite in length, it leaves 
an infinite trail behind it.

Often we can focus on the infinite trail of the things we've done or 
need to do. We get caught up by the sense of past and future, endlessly 
stretching out behind and before us. But the essential aspect is that 
one point where the wheel meets the ground. If we establish a quality 
of real attention, then that's all there is. It's simply this one 
moment as the edge of the wheel meets the ground, this one particular 
space of experience, this place where we know the qualities of sight, 
sound, smell, taste, touch, memory, and planning. It all happens here, 
in this moment, in the space of this mind. \emph{Right now, I'm sitting 
here, listening to these words of the Ajahn, having tea in silence, the 
weight of the body on the ground, feelings of coolness, warmth, 
tiredness.} It's all happening here in the space of our awareness. This 
is the wheel meeting the ground. There's a quality of great simplicity, 
stillness, and stability to this.

So even though there's a dynamism, a flow of experience, and 
perceptions come and go, there's also a quality of stillness. Luang Por 
Chah phrased it well: ``It's like still, flowing water.'' The mind that 
is aware and knows is still, but the perceptions, thoughts, feelings, 
memories, ideas, and plans have a quality of flowing, of continually 
and unrelentingly moving. It's so easy for us to be caught up in the 
urgencies of what's gone before or what we have to do, and then to 
create a sense of self around that in the present moment. It's 
important to question this process. When the urgencies of the mind come 
up\emph{---I did this, and I have to do that, and I need to get this 
done, and so-and-so is going to call, I have to, I have to---}it's good 
to get out of the rut, as it were, to stop obsessing on the ``have 
done's'' and the ``have to's,'' and question the feeling of urgency, 
obligation, and entanglement with past and future. We can ask 
ourselves, \emph{Is this endless trail of doing and being done the 
whole story? Is this the only way to perceive things?} In this way, we 
can challenge our presumptuous habits of judging the past, judging the 
future, and judging the present. \emph{Is that so? Is that really the 
case? Is that the whole story? Oh, it's good.---Is that so? Oh, it's 
terrible.---Really? Is that so?}

Practicing the Dhamma doesn't take much. It doesn't take a lot of 
complicated activity, but it does take application. It takes 
remembrance, recollection, and mindfulness---\emph{sati}. The word sati 
means remembering to pay attention. If we don't remember, if we don't 
bring attention to the practice, then that shift of 
perspective---escaping the rut of past and future---doesn't happen. But 
if we do remember, then we're able to bring to mind the quality of 
recollection. \emph{What's the hurry? Where do I think I'm going? Oh, 
is that so? This is great; this is terrible. Is that really so?} It 
only takes the tiniest suggestion, the briefest recollection, to 
catalyze that realization in the heart: \emph{Of course, how did I 
believe that was the whole story? Oh right, it's just a judgment. It's 
merely a perception. It's only a plan.}

