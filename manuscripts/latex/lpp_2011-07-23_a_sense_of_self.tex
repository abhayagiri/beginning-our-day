\mychapter{A Sense of Self}{Luang Por Pasanno}{July 2011}

Student: I'm trying to overcome a sense of self, but my spiritual quest 
seems to revolve around me trying to attain something. The thought of 
dropping my self seems like a snake swallowing its tail.

Luang Por Pasanno: That's the way it is in the beginning. It's a 
process of feeling it out. You have to play with it and work it out 
over time, as you get better and better at it. It's important in the 
beginning to be in the right container and have good spiritual friends. 
These serve as supports. In the beginning you flail around a lot, but 
it's not in vain.

S: Before the first stage of awakening, the stream-entry experience, 
are there places where a practitioner can experience the dropping away 
of the sense of self?

LPP: Sure: wherever it arises! Just like any other conditioned 
phenomenon, the sense of self arises and passes away. It's simply that 
we're so focused on the next thing that we don't notice the cessation.

S: During the stream-entry experience, are the three lower 
fetters---personality view, attachment to rights and practices, and 
skeptical doubt---all dropped at once, or in stages?

LPP: All at once.

S: Do the results of \emph{kamma} speed up when you are practicing?

LPP: Not necessarily. Kamma has its own way of playing itself out. 
There's a saying that kamma is going to do what kamma is going to do.

S: The Buddha said that kamma can't be understood by a normal mind. Are 
there practitioners who have good enough concentration that they can 
understand parts of it?

LPP: Oh, sure. There are meditators in Thailand with good concentration 
who can see parts of it. But only the Buddha could understand it 
completely.

