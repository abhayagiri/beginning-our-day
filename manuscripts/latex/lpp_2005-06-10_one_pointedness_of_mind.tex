\mychapter{One-Pointedness of Mind}{Luang Por Pasanno}{June 2005}

Yesterday, I listened to a talk by Ajahn Munindo. He spoke of a 
conversation he had with Luang Por Tate when Luang Por was about 
ninety-four years old. He asked Luang Por Tate, ``What is the heart of 
Buddhism?'' Luang Por answered simply, ``One-pointedness of mind.'' 
That statement may be simple, but it has many implications. The quality 
of one-pointedness is needed to understand anything, to see and act 
clearly rather than getting swept up by reactions or habits. So we must 
try to focus and sustain one-pointedness by bringing attention back to 
what we are doing.

It's so easy to get distracted by what's going on around us externally, 
especially when doing chores or engaging with other people. Or, we can 
get distracted by the internal wanderings of the mind---thinking about 
how things could be, should be, reactions, worries or 
doubt---proliferations in various shades and forms. We forget to 
establish a base of solidity---a foundation for the mind.

For that reason, we need to remember the importance of one-pointedness 
and to find skillful means to sustain it. In doing this, one-pointed 
mindfulness of the body can be an important anchor. That doesn't mean 
blocking out things around us, but rather, having a focal point. We 
maintain one-pointed attention, returning to the posture, physical 
sensations, movements, and touch. Whether we're going from here to 
there or standing in one place painting, we can bring attention to our 
body and cut through the mind's tendency to drift along.

If the mind is constantly drifting and dispersed, it will be unable to 
approach things in a clear fashion. But as we maintain one-pointed 
attention, a natural, energetic clarity arises. When it comes time to 
think, consider and decide things, we will have the energy to do so. 
The mind will be clear and able to perform its task well.

So again, we need to cultivate one-pointedness as our foundation. This 
includes keeping precepts, composing the senses, and settling the mind, 
while reflecting and investigating with wisdom. We cultivate this 
foundation and let the practice grow from there.

