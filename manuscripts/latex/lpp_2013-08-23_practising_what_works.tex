\mychapter{Practising What Works}{Luang Por Pasanno}{August 2013}

When we reflect on the best way to practice, it's good to focus on what 
works for us. We may read about various techniques and methods and 
wonder which is the right one for us. We can ask ourselves, \emph{What 
have I done that works? What has helped the mind relinquish its 
attachments and defilements? What has helped the mind become more 
peaceful, settled, and clear?}

As we practice, we come to realize what works for us will change, 
depending on conditions. Simply because something worked today doesn't 
mean it's going to work tomorrow, \emph{and} what didn't work in the 
past \emph{may} work for us now. This makes it necessary to adapt and 
experiment. Ajahn Chah used to repeat a quote from one of his teachers, 
Ajahn Tong Rat, who taught that the practice is very straightforward 
and easy: ``If the defilements come high, then duck; if they come low, 
jump.'' Within the bounds of \emph{sīla}, morality, we practice with 
whatever the situation demands---whatever works---as long as it has a 
wholesome outcome. This entails asking ourselves, \emph{How might I 
work with this particular situation?} Once we have a sense of what 
might be a proper approach, we put it into practice and evaluate the 
results.

This points to an ongoing, evolving relationship between how we 
practice and how the mind works within this practice. It takes time to 
discover skillful ways of engaging with that relationship; it's a 
learning process. But by sticking with this process, by taking a 
genuine interest in it, we can develop a good sense of what practices 
are truly beneficial---what truly works for us.

