\mychapter{Comfortable With Uncertainty}{Luang Por Pasanno}{June 2013}

As we reflect on the traditional explanation of \emph{aniccā}---how 
things are impermanent, inconstant, always changing---it is especially 
useful to also reflect on aniccā as a sense of uncertainty, or as 
Ajahn Chah would say, ``It's not a sure thing.'' We tend to deny or 
gloss over the fact that we don't know things for sure. We feel 
uncomfortable with uncertainty or uncomfortable with not knowing 
something. It can be intimidating. But reflecting on aniccā helps 
bring us back to the awareness of not knowing, of not being certain. 
And we can be aware of the feeling that arises within us when we're in 
touch with that uncertainty.

When we are out of touch with our awareness of uncertainty, needless 
stress and suffering can occur. Take, for instance, the way it feels to 
express some view that we later learn is wrong. If we ask ourselves 
what it feels like when we express something that is wrong, we might 
say that it feels awful, embarrassing, or uncomfortable. But actually 
that is not really true. At the time that we're speaking, if we don't 
know that we are wrong and there's no way to discern that what we have 
said \emph{is} wrong, then it feels just the same as when we are right, 
because at that point we think that we \emph{are} correct and believe 
that what we have said is true.

As soon as we learn of our mistake, however, we likely will feel 
embarrassed---but only if, at the time we expressed our view, we hadn't 
been open to the real possibility that our view might be wrong in the 
first place. In other words, if we have lost touch with the fact that, 
like all views and opinions, ``It's not a sure thing,'' then we may 
likely feel ashamed or uncomfortable. But if we keep this notion of 
uncertainty with all that we do and say then when we do make a 
mistake---an honest mistake---based on some assumption rather than a 
deliberate lie, then it's not so much of a problem. It's uncertain, and 
we don't need to take it personally, thinking that, \emph{I'm such an 
awful person, I never get it right, I should have known better.} When 
we hold our thoughts and words with an understanding that anything we 
say could be incorrect, it tends to hurt a lot less. If we happen to be 
wrong, then we can shrug our shoulders and say to ourselves, \emph{Oh, 
is that so? Okay, no problem.}

This is true with the way we relate to our moods as well. They're often 
screaming at us, \emph{This is the way it really is!} To save ourselves 
the needless suffering that comes from believing and acting on our 
moods, we challenge them by reflecting, \emph{Is this really true? Is 
it a sure thing that it is unchanging and permanent? Is it 
trustworthy?} Well, not for sure.

So, it's important to investigate and reflect on aniccā---to take it 
in. When we do this, we're able to apply a true form of wisdom, which 
doesn't require knowing a lot of things. This wisdom is the ability to 
abide in a place of stability, even as we stay aware of and feel the 
uncertainty of things.

