\mychapter{Stopping the Papañca Mill}{Ajahn Amaro}{August 2008}

\emph{Papañca}, a favorite theme of Ajahn Pasanno's, is that stream of 
incessant thinking, the flood of conceptual proliferation that the mind 
can so easily bring up. This produces an endless chatter of 
commentating, fretting, recollecting, re-creating, planning, and 
imagining. The Buddha pointed to three particular qualities which are 
the driving forces behind these processes: craving, conceit, and 
views---the \emph{papañca dhammas}. In the Anattalakkhana Sutta---the 
Discourse on the Characteristic of Not-Self---he outlined these 
qualities in very simple terms. \emph{Taṇhā} or craving is defined 
as \emph{etaṃ} \emph{mama:} \emph{This is mine. This belongs to me. 
This is my ladder, my chopping board, my space.} Next there's 
\emph{eso'hamasmi}: the conceit ``I am.'' \emph{This is what I am. I am 
a person, I am tall, I am short, I am a woman, I am a man, I am old, I 
am young, I am a monk, I am a layperson. I am.} And lastly, there are 
views: \emph{eso} \emph{me} \emph{attā: This is myself. This is me. 
This is my true nature.}

These are straightforward ways to define the forces that cause so much 
chaos in our lives and so many difficulties in our minds. The mind 
takes a simple perception with a given activity---walking along a path, 
picking up a length of pipe, chopping up some carrots, putting one's 
things away in a room, or any of the hundred little tasks of the 
day---and then runs with it: \emph{This is what I'm doing. This is 
mine. Those are not his. Who's moved my shoes?} Or it identifies with 
ideas about who we are, for instance that we're weaker than others: 
\emph{I'm pretty feeble, not nearly as strong as I used to be.} The 
mind takes simple feelings and perceptions, these habits of 
possessiveness and identification, and uses them to fuel what we see, 
hear, smell, taste, touch, and think, the objects of our six sense 
bases. Then---whoosh!---off it all goes. The papañca erupts.

I often think of it like a mill. The papañca dhammas, possessiveness 
and identification, are like the engine that keeps the mill churning. 
We simply feed into the mill the objects of the six sense bases. And 
out of it pours papañca, the endless flow of mental chatter. We need 
to pay attention to the engine, to turn it off or stop giving it fuel, 
rather than getting lost in the stream of mental chatter---the 
commenting, regretting, planning, and inflating that goes on. Then the 
whole thing gets a lot quieter and more peaceful. The mind can clearly 
be aware of the present without creating this whole welter of confusion 
around it.

So that's why it's very helpful that the Buddha pointed out these 
simple flags we can use to recognize when we're fueling the mill of 
papañca. If the mind is saying, \emph{This is mine, this belongs to 
me, and that's yours,} if it's saying, \emph{This is what I am, this is 
absolutely me and truly who and what I am,} that's the signal to let 
go, to not get caught in that process, and to bring more spaciousness 
to the mind. As soon as the mind says, \emph{That's mine, that's yours, 
I wish I had one of those, I'm like this, he's like that}---that's the 
time to get suspicious. See the draw, the pull of that. We use the 
quality of mindfulness to notice the me-mine-yours flags and choose not 
to buy into them, not to fuel the engine.

When we're mindfully carrying out our tasks for the day, engaging with 
each other, functioning together as a community, dealing skillfully 
with what we see, hear, smell, taste, touch, and think, then the 
papañca mill engine doesn't have any fuel. It has nothing to power it. 
It simply sits there and causes no trouble. If we make the simple 
resolution to be mindful and apply effort to see how these papañca 
dhammas operate, and if we train ourselves not to buy into the I-making 
my-making process, then we can save ourselves a great deal of 
difficulty. Life becomes far more harmonious and less complicated, and 
we can experience how pleasant and peaceful the mind can be.

