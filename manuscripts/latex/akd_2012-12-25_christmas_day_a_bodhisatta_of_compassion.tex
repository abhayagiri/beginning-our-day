\mychaptertoc{Christmas Day: A Bodhisatta of Compassion}
{Christmas Day: A Bodhisatta\\of Compassion}
{Ajahn Karuṇadhammo}{December 2012}

It's Christmas day. I don't know much about Christianity. Even though I 
grew up as a Christian, I wasn't very attentive to the religion. As 
Buddhists, we can sometimes have our own limited perspectives about 
Christianity. We may find ourselves making different judgments about 
the religion, or at least about how it's practiced in the world these 
days. But one of the more positive memories that I have from growing up 
as a Christian is Christ's essential message of compassion. Very 
similar to the Buddhist motivation for compassion, Christ expressed his 
wish for others to be free from suffering. We find compassion inherent 
within the Four Noble Truths---we're looking at the truth of suffering, 
and how to end it. One of the messages I remember from my early 
Christian upbringing was that Christ was there to help others find an 
end to suffering. As the teachings from Christianity suggest, Christ 
was so tuned into the suffering of others and had so much compassion 
for wanting to end it that he offered his life on the cross for all 
other beings, saying that he would willingly die so that others could 
be absolved of their sins. That's not something that we as Buddhists 
think is possible---that we could take somebody else's kamma away from 
them. But the notion of wanting to be able to do that, even if it's not 
possible, is quite a powerful contemplation. The sense of wanting 
others to be free of suffering so much that we are willing to die for 
it---that's, at least, in the Buddhist paradigm, the sign of a real 
Bodhisatta.

Whatever the reality is, having that kind of aspiration of wishing to 
be free from suffering and wishing to help others do the same is a 
useful message. As followers of the Buddha, we aim to figure out how we 
can realize that. It means focusing our attention on the First Noble 
Truth and recognizing the need to penetrate it. Before we can try to 
help others become truly liberated from their suffering, we have to 
face it within our own lives. We have to be willing to take a close 
look at this dukkha---to experience and know what it really means 
before we can move on to understanding its origin, letting it go, and 
developing the path. This entails honest self-appraisal, and it starts 
right at home inside ourselves. We begin to open up to this Noble Truth 
because we are willing to look at our own unskillful habits and the 
unwholesome ways that we've learned to operate in the world. For all of 
us, deep conditioning in some way or another causes us to continue our 
old habits of behavior which tend to bring on more difficulty and 
stress. This includes the way we view experience or view other people 
and what we try to get from the world or from our relationships with 
other people---the whole gamut of what it is that we do in our attempts 
to gain happiness and gratification. It takes bravery to go deep 
inside, look at the cause of suffering in our lives, have compassion 
for that, and begin the process of uprooting it. This requires honesty 
and integrity to truly look at ourselves. As we do that, we naturally 
experience more sensitivity, knowing that this same process is 
happening to everyone else too. Because we are willing to examine 
ourselves and understand our own suffering, this helps us to see that 
others are experiencing the same difficulties as us. We can be more 
accepting of other's foibles and bad habits because we realize that all 
of us make these same mistakes when we try and take care of our own 
needs. If we have compassion for ourselves, then we can have compassion 
for everybody else. To me that seems like a way we can understand and 
emulate some of the teachings that come from the Christian traditions.

