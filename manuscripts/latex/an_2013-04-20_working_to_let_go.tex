\mychapter{Working to Let Go}{Ajahn Ñāṇiko}{April 2013}

The work period is a good time for learning about how to let go. During 
these periods the mind is constantly thinking, \emph{I have to do this 
and I have to do that and this needs to get done before I do that.} But 
when the mind lets go of that it's malleable and, as the Buddha said, 
``fit for work.'' If we only meditate and don't challenge the mind with 
difficult and unpredictable situations, then the chances are the mind 
won't learn how to let go. What really trains us how to let go are the 
uncertain situations we are faced with. We might tell ourselves, 
\emph{Okay, today no matter what I do, no matter what anybody says to 
me, whether it's someone I don't like or someone I do, I'm going to let 
go no matter what.} And then, sure enough, as soon as we make that 
resolution, someone or something will come along and test us in a way 
we didn't think we would be tested and we're unable to let go.

It's not only about letting go, it's also about choosing the 
appropriate response. We are training ourselves to respond judiciously 
and to let go within the bounds of a suitable response. By doing this 
we learn to maintain internal and external harmony.

If we learn how to let go during or after activity, the mind will not 
be stressed, and our default mental condition will be one of quietude. 
Then the mind will also be quiet when it does not need to work. When 
the mind needs to think or figure something out, it will do that and 
then go quiet again.

We do not need to stress out over external things which may seem 
important but in reality are insignificant. The external world is a 
mere trifle compared to our internal development. The work period, or 
any time we are doing something, is for seeking that internal 
development, for seeking the Dhamma, and for letting go.

