\mychapter{The Khanti Pāramī}{Luang Por Pasanno}{April 2013}

It is helpful to contemplate how to use \emph{khanti}, patience, in our 
daily practice, and how we can cultivate it as a mental attitude during 
meditation. Patience is an underrated \emph{pāramī} and considered in 
different ways, sometimes even misinterpreted. I remember Varapañño 
Bhikkhu disparaging himself, saying: ``I just don't have any pāramīs 
of wisdom, meditation, loving-kindness or anything like that. But at 
least I can just put up with this. I can build some khanti pāramī.'' 
It had a sweet Eeyore type of attitude to it.

Khanti is actually a proactive engaging with experience in a way that's 
not getting caught up in or swayed by the reactions and impulses of 
either liking or disliking or of desire and aversion. It has an 
enduring quality to it. One of the phrases the Buddha uses to describe 
the impulses of mind is \emph{abhijjhā domanassa}, desire and aversion 
or a gladness-sadness type of impulse. We can try to have an enduring 
and patient attitude toward that so we're not reactive. When we're not 
reactive, then desire and aversion, liking and disliking go their 
natural way, and they cease. So when we're engaged in some sort of 
interaction, whether it's pleasant or unpleasant, the mind isn't 
getting swept up in it. Similarly, when we're engaged in some kind of 
task, chore or duty, and we have been cultivating patience, then what 
comes to mind is a kind of gravitas, a sense of weight, ballast or 
anchor in the mind that isn't pulled around.

I've never come across a satisfying translation of khanti: patient 
endurance doesn't quite get it. In general, I think it's important to 
reflect on it in terms of how it manifests for us: \emph{How do I turn 
the mind toward a quality that isn't swayed, pushed, or pulled?} It's a 
willingness to be present with experience and especially important in 
meditation. When any of the five hindrances come up, they feel quite 
compelling and true as moods. So we attempt to have an enduring quality 
of mind that is present with the hindrances and attends to them as they 
follow their natural cycle: arising, persisting, and ceasing. They come 
into being and pass away. One of the functions of patience is to help 
us refrain from feeding our defilements. When we're not feeding or 
nourishing our reactive moods, then there's a real steadiness there. 
With khanti present we're not trying to manipulate conditions around us 
so that they suit our preferences or to manipulate people to make them 
pleasing to be around. Neither worldly conditions nor people are going 
to fulfill our preferences or desires. But when we have khanti to 
receive our experiences, then we're not shaken by them.

As part of the ethos and flavor of Ajahn Chah's training, the 
development of the khanti pāramī was strongly encouraged so we would 
have a good, solid foundation in our practice---a foundation that 
helped to steady us despite the changing circumstances and conditions 
of our lives.

