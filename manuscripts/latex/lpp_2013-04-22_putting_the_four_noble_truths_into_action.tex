\mychaptertoc{Putting the Four Noble Truths Into Action}
{Putting the Four Noble\\Truths Into Action}
{Luang Por Pasanno}{April 2013}

As we bring the practice into our daily lives, it's immensely 
beneficial to use the Four Noble Truths when viewing experience---in 
our formal mediation, interactions with others, and engagements with 
various duties. This is not something to save for later---after 
studying the suttas, developing all the \emph{jhānas} and formless 
absorptions, all the psychic powers, we finally contemplate the Noble 
Truths and become enlightened---it's not like that. The Four Noble 
Truths are to be put into practice right now.

To do this, we need to establish a habit of reflecting and 
investigating in a particular way: First, by examining phenomena 
carefully---the quality of experience, and the results that come from 
interacting with experience in the usual way---we find there's always 
some kind of \emph{dukkha}, some kind of suffering, stress or 
discontent. Then we reflect on that, investigating its cause, and how 
to bring about its cessation. When applying these steps, we're not 
viewing the world through the lens of self, of me and my problems, me 
and my accomplishments, me and my preferences, me and my views and 
opinions. We're looking instead from the perspective of dukkha, its 
cause, its cessation and the path leading to its cessation. When we 
view experience like that, life becomes very simple and straightforward.

So we lift up that perspective in the mind and practice with it. We're 
not waiting patiently for some intuitive insight to arise 
spontaneously. We need to deliberately train the mind by frequently 
redirecting our attention and reflecting on experience in that way.

The Buddha gives us the practice method of using the Four Noble Truths 
as a lens to depersonalize experience. And when we use the right 
method, he says, beneficial results are bound to follow; if we use the 
wrong method, however, our efforts will be futile. To illustrate these 
points, the Buddha offers some humorous similes. For example, he says 
that when we're wrongly seeking the fruits of practice, we're like a 
man seeking milk: knowing that milk comes from cows, he approaches a 
nursing cow, twists her horn, and then wonders why he doesn't get any 
milk. The reason, of course, is that he's using the wrong method. But 
if he were to use the right method by pulling on the cow's udder, then 
the milk would surely flow. In the same way, even if we understand the 
Four Noble Truths in principle, nothing significant will result if we 
keep those teachings in our heads as mere abstractions. That would be 
using the wrong method, because the most essential bit is missing: 
bringing the Noble Truths into our practice. On the other hand, when we 
do put them into practice and apply them skillfully---that is, when we 
use the right method---then we are bound to realize beneficial results.

Paying attention to the method and its application, putting it into 
practice, bringing the Four Noble Truths into our experience, 
reflecting and applying these Truths: this is how we can reap the 
fruits of practice.

