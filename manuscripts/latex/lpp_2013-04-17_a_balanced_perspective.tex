\mychapter{A Balanced Perspective}{Luang Por Pasanno}{April 2013}

As most of us know, when bringing the practice into our daily lives, 
it's necessary to apply mindfulness. But it's also necessary to ensure 
that our mindfulness is operating under an appropriate and beneficial 
view or perspective. If we are mindful, but our view is misguided, then 
it's likely that we're mindfully following some sort of bias or 
obsession.

In order to keep on the right track, we need to question the views we 
superimpose on our experience. One way to expose those views is to 
notice the way we react to experience with comments and assumptions 
such as, \emph{This is really good---I like this. This is awful---I 
don't like this.} Reactions like those are habitual and need to be 
questioned.

When the mind assumes it likes an experience, it tends to block out the 
negative aspects in order to prop up the view, \emph{I like this.} So 
we need to deliberately bring up the negative side for examination, 
asking ourselves, \emph{What are the drawbacks? What are the pitfalls 
in this? How might I get hooked by it?} In the same way, when the mind 
experiences something it assumes is undesirable or challenging, we need 
to look at the situation from another perspective and ask ourselves, 
\emph{What's the beneficial side of this experience? What can I learn 
from it?}

When we examine our experience from both sides in this way, it makes us 
more flexible with regard to the views we hold. Instead of habitually 
seeing things as all black or white, we can open to a more balanced 
perspective from which we can see that experience is never just black 
or white---it's always a mixture. Without this balanced perspective, we 
tend to go through cycles of elation and depression, excitement and 
frustration, and then wonder why that's happening. Well, in all 
likelihood, it's happening because we've attached ourselves to some 
sort of all or nothing view.

Ajahn Chah encouraged us to question our views of experience by having 
us ask ourselves, \emph{Is this for sure? Is this really what's 
happening?} By doing that, we can better adapt to whatever circumstance 
we find ourselves in, because we're no longer preoccupied with trying 
desperately to force circumstances to be the way we like them. And it 
becomes easier to be mindful and present with experience as it unfolds.

So whether we're attending to our duties, engaged in some kind of 
interaction or alone, we need to question the habitual views and 
reactions that come up for us. Otherwise, those views and reactions 
will toss us around like a feather in the wind---wherever the wind 
blows, that's where we'll land.

