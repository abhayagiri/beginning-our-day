\mychapter{Slowing Down the Busy Mind}{Luang Por Pasanno}{June 2013}

Even when we live in a monastery, the mind tends toward busyness and 
proliferation. This is a natural habit of the human mind. We can make 
ourselves conscious of that---not through a force of will, trying to 
squelch or annihilate it---but through understanding the mind's natural 
habits and the tendencies we carry with us. And we can work with them 
in a skillful way. For instance, the mind tends towards speeding up and 
gaining momentum. When that happens, we can consciously slow down and 
pay attention. Sometimes this means physically slowing down, not to the 
point of irritating everybody around us, but also, not getting pulled 
into a sense that, \emph{This is important. There's not enough time. 
This has to be done now.} When we slow down, we can notice what 
\emph{happens} when we slow down: What happens is that we see things 
more clearly.

I've ridden in a car up and down Tomki Road to and from the monastery 
countless times. These last few months I've been walking on the road 
every day. It's a very different experience than riding in a car. When 
we ride in a car, we might think we see or know Tomki Road quite well, 
but because we're going so fast, we don't really see things clearly. By 
slowing down and walking, I've been experiencing many more of the 
nuances and details of the road, and there's more clarity as well. It's 
similar with the mind. We can learn how to slow it down, so that when 
we're engaged in activity, we can better attend to what we're doing. We 
slow down the impulse of getting swept up in the mind when it's 
worried, when it's proliferating, or simply chattering away. Of course, 
it's not \emph{necessary} for the mind to chatter away like this, but 
as long as we're able to distract ourselves with that sort of thing, we 
feel it's okay. This is not a great program for a practitioner!

Instead, we encourage ourselves to slow down. In particular, when we're 
not engaged with work or duties, we can slow down by using walking and 
sitting meditation. We consciously slow down by bringing attention and 
awareness to the nuances of sight, smell, taste, touch, and mental 
objects, and to the nuances of body, feeling, perception, mental 
formations, and consciousness. Those things---the six kinds of sense 
contact and the Five Aggregates---are what we use to create the sense 
of self and ``me'' and to solidify our moods and impressions into 
habitual tendencies. So we slow down enough to see them clearly. 
Learning how to slow down is quite simple, and it provides us with many 
direct and beneficial effects.

