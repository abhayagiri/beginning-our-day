\mychapter{Physical Therapy for the Mind}{Ajahn Karuṇadhammo}{July 
2013}

Recently I've made visits to a physical therapist because I have some 
ongoing muscle issues that have plagued me for the last twenty years. 
Often, this type of situation originates with a small abnormality that 
causes pain and many people will subconsciously allow the body to 
adjust to it or slump in a certain way to relieve that pain. Although 
this gives temporary relief, it turns out that people have adjusted 
their posture in a way that ends up perpetuating the problem. Then they 
adjust a small amount more to relieve more pain when it returns and, 
not too long after, they find themselves misaligned. All of the small 
adjustments they've made while seeking temporary relief simply do not 
take care of the condition. They're left with a posture that is 
unbalanced and it places additional stress on the bones, muscles, and 
connective tissues that are responsible for good alignment. The only 
way to correct the condition is to incorporate appropriate physical 
therapy and exercise to address all of the changes that have taken 
place over the years. The body is so accustomed to coping with the 
condition in a particular manner that they have to unlearn those coping 
strategies and go through some conscious discomfort to begin achieving 
the goal of long-term healing.

I thought about this in line with how the mind works, the way we 
usually buy into our moods, both the positive and the negative ones. We 
find ourselves swinging back and forth in a yo-yo-like manner being 
drawn to and believing in the salvation of our positive moods and then, 
when they fade, reacting quite aversively to the negative ones. We can 
end up getting lost in the entire process. Each time we respond by 
moving toward an enjoyable mood or away from a disagreeable one, we're 
seeking a temporary solution to a long-term problem. The solution is 
having a sense of equanimity so that we are not constantly buying into 
and reacting to these different moods that pass through the mind.

This is similar to the way habits develop in the body. In the mind, an 
event happens, it triggers a perception---a habitual way of looking at 
an experience. That reminds us of something similar in the past and we 
react in the same way through either aversion or attraction. Something 
can be unpleasant---a difficult situation that causes discomfort, 
unpleasantness, or aversion---and if we react to it automatically based 
on a past perception, it then reinforces the tendency to buy into old 
ways of reacting negatively. On the other hand, if it's something we 
are desirous of or excited about, that reinforces the tendency to go 
for it. Over time, we develop specific temporary coping strategies and 
react automatically in certain ways. We say, do, or think something, or 
we may internalize the experience with anger, blame, self-criticism, 
greed, fear, or confusion. Every time these coping strategies arise, 
they reinforce the original pattern that began the process in the first 
place. They seem to give some temporary relief for a period of time, 
but in the long run, they simply don't do the trick to relieve us of 
long-term suffering and pain.

Before those automatic responses come into play, we can spend time at 
the level of perception and feeling, using mindfulness and clear 
comprehension to observe the response as it is occurring. We do this by 
allowing ourselves to experience the discomfort of an unskillful habit 
as we get to know and examine our reactions. This helps us refrain from 
automatically repeating the same old pattern. It also gives us time to 
respond with more wisdom and skillfulness based on having seen and 
understood the reaction clearly. As it holds true with physical therapy 
for the body, we can unlearn habits that have caused long-term 
unwholesome reactions in the mind. We just need to be willing to pause 
and observe the space around our uncomfortable experiences.

