\mychapter{A Superior Resolve}{Luang Por Pasanno}{August 2013}

Yesterday, four senior monks from Abhayagiri participated in the 
ordination at the City of 10,000 Buddhas. The preceptor was Reverend 
Heng Sure. As he was instructing the candidates, he kept using a 
certain refrain: ``There is inferior resolve, medium resolve, and 
superior resolve.'' The examples he gave of inferior and medium resolve 
were humorous, so as to encourage the prospective monks to take on 
superior resolve.

It's helpful to reflect on what it means to make a resolution and to 
understand that the way we resolve to do something is going to 
condition the result. Whether it's about ordaining or simply helping 
with the dishes, we're constantly resolving to undertake specific 
activities. We set up in our minds a firm resolve to do the thing we 
want done, and then pay attention to the result that resolve has on our 
actions. When we don't do that, the mind tends to wander, drift, and 
get lost and scattered; we forget about our resolve to stick with 
whatever it is that we're doing.

It is important to develop the ability of attending to and following 
through on our resolutions. That's because, after setting our resolve 
concerning some activity, we often find ourselves experiencing 
restlessness or boredom when actually engaged in that 
activity---whether it's meditation or some mundane task. When that 
happens, we end up replacing what we're doing with something else, and 
the cycle begins again.

There's the idiom that nature abhors a vacuum. When we leave a vacuum 
in the mind, it tends to fill up with habits that aren't very useful. 
We can help prevent that from happening by filling the vacuum with the 
sense of resolve. We can bring up specific resolutions and follow 
through on them. We can investigate the very nature of resolutions, 
asking ourselves, \emph{What am I undertaking? Why am I undertaking it? 
What are the results of my intentions?}

In essence, we are resolving to take an interest in what we are 
doing---to be interested in the process of being present and applying 
ourselves to the activities at hand. That kind of resolve allows the 
mind to be buoyant and uplifted. And if we sustain this practice, the 
mind will become easily settled and clear. It's all about learning to 
bring about superior resolve and holding that resolve with 
understanding and discernment.

