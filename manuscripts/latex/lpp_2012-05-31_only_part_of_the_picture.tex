\mychapter{Only Part of the Picture}{Luang Por Pasanno}{May 2012}

As we did with conducting the recent ordination---thinking through 
things a bit, planning a little---we're developing a sense of 
circumspection as we attend to whatever circumstances we're in. We do 
this by asking ourselves, \emph{How can I fit into this situation? How 
can I be skillful, effective and composed?} Sometimes people can 
misunderstand how to apply the Buddha's teachings on being present in 
the moment, and they may think, \emph{All I need to do is be present in 
this moment and everything will be okay.} But that's only part of the 
picture.

The Buddha always encouraged his followers to recognize and pay 
attention to the causal nature of actions. What we've done in the past 
affects the present, and what we're doing in the present affects the 
future. Having seen and understood how those principles play out over 
time, we shape our behavior accordingly.

Sometimes there's an emphasis on having a nice, warm and fuzzy, 
be-here-now kind of feeling. But in reality, when people are guided by 
that sort of feeling, they're usually not very well connected with 
themselves or the circumstances around them. The Buddha didn't advocate 
that. Instead, he encouraged us to learn how to connect with the world 
skillfully. And so we practice being attentive and effective with our 
tasks, duties and interactions with other people.

When we learn to do that, we can build a firm foundation of clarity and 
clear comprehension---\emph{sampajañña}. The Buddha spoke of 
sampajañña as clearly comprehending the circumstances we're in, the 
people we're with, and the effects we're having on the world around us. 
With this quality we can anchor our actions in our own non-delusion, 
non-confusion. And when we apply sampajañña to our present-moment 
circumstances, it allows us to see a much bigger picture.

