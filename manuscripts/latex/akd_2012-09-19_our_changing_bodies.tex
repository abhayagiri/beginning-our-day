\mychapter{Our Changing Bodies}{Ajahn Karuṇadhammo}{September 2012}

Some people here have recently experienced the death of someone close 
to them. Also, a number of the community members here have experienced 
minor injuries to joints, feet, and knees, as well as other illnesses 
and maladies. When we reflect on the nature of our human bodies, we can 
see these bodies aren't really under our control. They don't obey our 
wishes, our wants, or our desires to be healthy and always comfortable. 
They have their own quirks. Each one of our bodies has a particular set 
of constituents, elements, and predispositions through kamma, biology, 
and genetics. Each body is more or less following its own life, its own 
course, and there is not a huge amount of control that we have over it. 
We can try to influence it and try to give it proper nourishment and 
support when we are ill, but by and large there is nothing we can do to 
control its ultimate outcome. It will age. Along the way, it will 
experience periods of health, periods of sickness, and eventually the 
body will die. It is simply a part of nature. That's the way nature 
works.

Look around at what is happening with plants and animals. All creatures 
have their birth, their time of life, and their passing away. It's so 
easy to get caught up in this process, to get caught up in the fear of 
illness or injury. There can often be a lot of anxiety about how we 
protect and care for the human body, not to mention the fear, anxiety, 
and difficulties we experience from the death and dying process.

The hallmark of the Buddha's practice is to contemplate this reality so 
that we can genuinely see, understand, and accept that it's all simply 
a part of nature. If we look around we can see that there's nothing 
that escapes this process. Gently, over time, we can apply this 
contemplation to ourselves and to what we experience throughout the 
day. By gaining insight into this process, we are able to live without 
a constant sense of protection and anxiety in regard to this body, 
which is merely following its own course. We look after ourselves in 
reasonable ways but without attachment or clinging to a sense that 
\emph{This is who I am; this is myself}. The body is basically a set of 
elements that is constantly, changing, moving, evolving, and 
transforming just like the world around us.

