\mychapter{The Reason We're Here}{Ajahn Yatiko}{December 2012}

What is the whole point of our being here? Where do we expect the 
cessation of suffering to be? Is it some sort of project we're working 
on? Is that where the cessation of suffering is going to be found? When 
the fault-finding mind obsesses over other people's behavior, or when 
we find fault with ourselves, is that where suffering ceases? Take a 
look at where the mind goes and becomes contracted. Ask yourself, 
\emph{Is this where the cessation of suffering is going to be found?} 
Because that's the reason we're here---to understand that.

So be really clear and try to bring that to mind. I often remind myself 
of this. Whatever problem or issue it is that I'm obsessing over, I say 
to myself, \emph{I didn't become a monk to solve a bunch of mundane, 
external problems.} What we need to focus on in our practice is much 
more important than that. When we reflect on the real reason we're 
here, the mind relaxes and steps back from the things it obsesses over. 
That's the state of letting go. We learn to apply it to all the 
different obsessions that come up in the mind, all the different 
activities and projects we focus on and become obsessed with.

Of course, projects do have to be done, and we do have to plan for 
things. But when we get caught up in all of it, that's what leads to 
suffering. We've lost our balance; we've temporarily lost contact with 
the whole purpose we're here. At the same time, it isn't helpful to get 
down on ourselves for losing the plot---losing the plot happens to 
everyone. But in order to snap out of it, we need to be reminded of our 
purpose, which is to end suffering. We need to know that this is a 
possibility, and we have to bring our mindfulness to that issue so that 
we can loosen our grip around these different obsessions of the mind 
and stop wasting precious time. By doing this over and over again, we 
can learn to let go and realize there's no reason to suffer.

