\mychapter{The Development of Equanimity}{Ajahn Amaro}{November 2008}

When we do the chanting on the four divine abodes, the 
\emph{Brahmavihāras}---\emph{mettā}, \emph{karuna}, \emph{mudita}, 
\emph{upekkha;} loving-kindness, compassion, gladness, 
equanimity---going through each one, it's important to notice the 
development of equanimity around reflections on \emph{kamma,} actions 
of cause and effect. ``I am the owner of my kamma, heir to my kamma, 
born of my kamma, related to my kamma, abide supported by my kamma. 
Whatever kamma I shall do, for good or for ill, of that I will be the 
heir.'' The other Brahmavihāras---mettā, karuṇā and muditā---have 
an emotional tone, a quality of wishing well to ourselves and others, 
of wishing to be free from suffering. But the fourth one, upekkhā, 
equanimity, is developed through a conscious reflection on \emph{kamma} 
\emph{vipāka,} kamma and its results. That may not sound terribly 
interesting, and so it can easily be pushed aside. But equanimity is 
the most refined of the Brahmavihāras. It's quite a significant mental 
quality and is difficult to establish.

When we consciously reflect on cause and effect, we can distinguish one 
from the other in our direct experience, \emph{Ah, this is the cause of 
that result}. Or with greater detail, \emph{Because I've been generous 
and kind and I've been keeping the precepts well, then an inner 
lightness and confidence in my own goodness has developed.} Or 
\emph{Because I've been selfish, deceitful, clumsy, casual, or sloppy, 
there's this feeling of regret, self-criticism, and a negative 
emotional tone.} This is a very conscious way of letting go of the 
content of happiness or unhappiness and seeing clearly for ourselves. 
\emph{Because of this good action, there's a pleasant result. Because 
of this unskillful action, there's an unpleasant result. I see.} It's 
the law of nature---because of this, there is that. It's the Dhamma in 
action. It's extraordinarily simple, and when we see it clearly, it 
takes a huge amount of the alienation and burden away from our 
experiences. This is not a small thing---it's essential to our practice.

As soon as the mind sees things in terms of right and wrong, when it 
buys into those kinds of judgments, equanimity is lost. \emph{This is 
right. This is the right way to do it. That's wrong. That's the wrong 
way to do it. He shouldn't be doing it that way. He should be doing it 
this way}. But by cultivating equanimity, we can reflect and see more 
clearly, \emph{I'm calling that ``right'' because of what? Because of 
the conventions of Theravada Buddhism?} Or, \emph{I'm calling this 
``right'' because of my understanding of the mechanics of the hillside? 
Or how the drainage channels work?} We can more clearly see how our 
conditioning leaps onto those judgments, saying, \emph{This is 
absolutely right, that is absolutely wrong. It should be this way. It 
shouldn't be that way.} We believe it over and over again, a thousand 
times a day. We believe in our judgments and take them to be absolutely 
real. But with some equanimity, the mind has the ability to recognize 
that this is called ``right'' because of a reason, and it's called 
``wrong'' because of a reason. There's a cause, and ``right'' and 
``wrong'' are simply effects.

Consider reflecting on a simple thing like eating. Because we have 
human bodies, we become hungry and need to eat. Needing to keep the 
body going to sustain life is the cause, needing to eat is the effect. 
When we eat we choose what to use to keep the body going. This effect 
appears as likes and dislikes, rights and wrongs. The mind chatters 
away, \emph{That's great food. That's terrible food. He's really 
brilliant at cooking this. Oh no, he's ruined that again.} It's only 
because we need to eat food that those judgments arise. It's because of 
having a human body---needing to put stuff into these holes in our 
faces, chomp it up and swallow it down to sustain life---that we have 
any relationship at all to these plants and animals in this way. If we 
didn't have human bodies, if we were all in the \emph{arūpa} realm 
where they have no bodies, then we wouldn't relate to fruit, 
vegetables, or meat in this way. None of that would be perceived as 
food---there wouldn't even be a word for physical food. It simply 
wouldn't concern us.

So this is an encouragement for all of us---particularly with the rainy 
weather, working outside in the rain, trickles of chilly water running 
down our necks and into our socks---to notice when the mind says 
``like'' or ``dislike,'' ``right'' or ``wrong.'' Because it's the rainy 
season, these feelings might arise. This is the cause, and this is the 
effect. See how that small reflection can take us a step back from the 
notion that right, wrong, like, and dislike are objective or lasting 
realities. And with that stepping back and seeing things more clearly, 
what we draw closer to is the Brahmavihāra of equanimity.

