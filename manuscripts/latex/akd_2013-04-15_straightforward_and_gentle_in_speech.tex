\mychaptertoc{Straightforward and Gentle in Speech}
{Straightforward and\\Gentle in Speech}
{Ajahn Karuṇadhammo}{April 2013}

Many of our teachers, especially in the West, speak about the need to 
adapt to modern circumstances while continuing to adhere to principles. 
One of these principles is concerned with wholesome and unwholesome 
courses of action---right speech. I was thinking about the need for us 
to adapt to different circumstances and cultures when we interact and 
converse. An example of this might be seen in the communication 
differences between monastics in a new Western monastery and in a 
monastery in Thailand. We have a high level of engagement here at 
Abhayagiri, due at least partially to the nature of building a 
monastery and trying to form a community. There tends to be more 
interaction and speech during work periods and sometimes this requires 
quick communication while we are attending to our chores. With this 
type of communication there is a necessity for being mindful with our 
speech. We need to be careful how we apply the guidelines of right 
speech and interact with each other in thoughtful and caring ways.

A line that comes to mind is in the Mettā Sutta, ``straightforward and 
gentle in speech''---it's a combination of two different actions. We 
aren't just straightforward but also gentle, nor are we merely gentle; 
we need to be straightforward. It's both qualities at the same time. 
This is often not an easy task---it takes a tremendous amount of skill 
and mindfulness to be straightforward and gentle simultaneously. It's 
easy to become straightforward when there is something we think is 
wrong, something we think should be done in a particular way, when 
we're in a rush to finish, or a previous interaction has led to some 
negativity. We can easily be straightforward in those circumstances and 
not consider the more gentle approach. During those times we can blurt 
something out that is a bit too aggressive, demanding, harsh, or simply 
not considerate. When we are speaking, it's important to monitor our 
energy so that we are communicating what needs to be communicated in a 
straightforward, clear manner while also remaining aware of the 
energetic impact of such speech---how what we say might affect the 
people around us. We can consider the impact a gentle quality has on 
communication, thinking, \emph{How would I like this message delivered 
to me if I were on the receiving end?}

Conversely, people can be overly concerned about how something is 
communicated because they do not want to upset anyone, cause any 
discord, or are afraid of someone's reaction. When this occurs, the 
message may be delivered in a way that is indirect, uncertain, or not 
straightforward at all and therefore, the content of what is being said 
and its meaning is misunderstood. Perhaps one is communicating out of 
fear---afraid to speak up in a situation where someone needs a clear 
statement. We can end up being imprecise or ambiguous, thinking that 
we've communicated a message, but the other person has not heard a word 
of what needed to be said.

Being able to combine clarity, honesty, gentleness, and kindness to 
what we say is a skill that requires a lot of attention and work. We 
are learning to consider both how our speech is delivered and how it 
will be received. I find for myself that it's an ongoing exercise of 
making mistakes and learning, going back and apologizing, and if 
necessary, re-clarifying what it was I was trying to say in the first 
place. It's a meaningful and significant part of the practice and 
something that I believe is at the core of our daily lives.

