\mychapter{Breathing Into Busy Activity}{Luang Por Pasanno}{June 2005}

There has been a lot of busyness here these days. Many things need to 
be done, and we're doing them. But it's important to be careful about 
how much we get swept up into the busyness of our lives. Those are two 
different things–the doing of a task and the frantic, busy, scattered 
energy we may bring to the task. We can try to watch and reflect on the 
feeling behind what we're doing. What is the energy behind it? 
Recognize where the feeling of agitation comes from.

Much depends upon staying with the breathing–breathing into the 
activity of what we're doing. Sometimes it helps to step back, relax, 
and slow down. That doesn't mean we accomplish less. Oftentimes, the 
more frantic we become, the less we accomplish. In addition, being 
frantic obstructs the enjoyment of what we're doing and may compromise 
our harmony with others. So pay attention to breathing and relaxing.

We also need to be careful not to take on too many things at once. 
However, it's more the attitude we carry in our mind that's the 
problem, because we can really only do one thing at a time. We carry 
around in our minds all the things we think we have to do, and that 
stirs up this frantic energy.

We deal with this by breathing into what we are doing, being with it, 
and not getting too swept up. Make the breath a force for settling. It 
can be very satisfying paying attention to the breath, the body, and 
the actions involved in the task we're doing. The more we attend to the 
body while we are engaged, the more we can generate a sense of focus 
and well-being in the midst of activity.

